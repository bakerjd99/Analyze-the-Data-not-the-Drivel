%% bm.pdf preamble - material merged from previous preamble and current pandoc preamable output
% NOTE: float placement required changes to the source files referenced by bm.tex
% May 28, 2020
%
% Use lualatex to compile - test with MiKTeX 2.9

% uncomment to list all files in log
%\listfiles

\documentclass[12pt]{report}


\usepackage{fontspec}

%\setmainfont[Scale=MatchLowercase]{Lucida Bright}
%\setmonofont{FreeMono}
%\setmonofont{Source Code Pro}
\setmonofont[Scale=MatchLowercase]{Ubuntu Mono}

% short snippets of asian languages
\newfontfamily\myAsian{Noto Serif TC Medium}

\usepackage[headings]{fullpage}

% national use characters 
%\usepackage{inputenc}

% ams mathematical symbols
\usepackage{amsmath,amssymb}

% added to support pandoc highlighting
\usepackage{microtype}

\usepackage{makeidx}

% add index and bibliographies to table of contents
\usepackage[nottoc]{tocbibind}

% postscript courier and times in place of cm fonts
%\usepackage{courier}
%\usepackage{times}

% extended coloring
\usepackage{color}
\usepackage[table,dvipsnames]{xcolor}
\usepackage{colortbl}

% advanced date formating
\usepackage{datetime}

%support pandoc code highlighting
\usepackage{fancyvrb}

% \DefineShortVerb[commandchars=\\\{\}]{\|}
% \DefineVerbatimEnvironment{Highlighting}{Verbatim}{commandchars=\\\{\}}
% % Add ',fontsize=\small' for more characters per line

% tango style colors
% \usepackage{framed}
% \definecolor{shadecolor}{RGB}{255,255,255}
% \newenvironment{Shaded}{\begin{snugshade}}{\end{snugshade}}
% \newcommand{\KeywordTok}[1]{\textcolor[rgb]{0.13,0.29,0.53}{\textbf{{#1}}}}
% \newcommand{\DataTypeTok}[1]{\textcolor[rgb]{0.13,0.29,0.53}{{#1}}}
% \newcommand{\DecValTok}[1]{\textcolor[rgb]{0.00,0.00,0.81}{{#1}}}
% \newcommand{\BaseNTok}[1]{\textcolor[rgb]{0.00,0.00,0.81}{{#1}}}
% \newcommand{\FloatTok}[1]{\textcolor[rgb]{0.00,0.00,0.81}{{#1}}}
% \newcommand{\CharTok}[1]{\textcolor[rgb]{0.31,0.60,0.02}{{#1}}}
% \newcommand{\StringTok}[1]{\textcolor[rgb]{0.31,0.60,0.02}{{#1}}}
% \newcommand{\CommentTok}[1]{\textcolor[rgb]{0.56,0.35,0.01}{\textit{{#1}}}}
% \newcommand{\OtherTok}[1]{\textcolor[rgb]{0.56,0.35,0.01}{{#1}}}
% \newcommand{\AlertTok}[1]{\textcolor[rgb]{0.94,0.16,0.16}{{#1}}}
% \newcommand{\FunctionTok}[1]{\textcolor[rgb]{0.00,0.00,0.00}{{#1}}}
% \newcommand{\RegionMarkerTok}[1]{{#1}}
% \newcommand{\ErrorTok}[1]{\textbf{{#1}}}
% \newcommand{\NormalTok}[1]{{#1}}

% %espresso style colors
% \usepackage{framed}
% \definecolor{shadecolor}{RGB}{42,33,28}
% \newenvironment{Shaded}{\begin{snugshade}}{\end{snugshade}}
% \newcommand{\KeywordTok}[1]{\textcolor[rgb]{0.26,0.66,0.93}{\textbf{{#1}}}}
% \newcommand{\DataTypeTok}[1]{\textcolor[rgb]{0.74,0.68,0.62}{\underline{{#1}}}}
% \newcommand{\DecValTok}[1]{\textcolor[rgb]{0.27,0.67,0.26}{{#1}}}
% \newcommand{\BaseNTok}[1]{\textcolor[rgb]{0.27,0.67,0.26}{{#1}}}
% \newcommand{\FloatTok}[1]{\textcolor[rgb]{0.27,0.67,0.26}{{#1}}}
% \newcommand{\CharTok}[1]{\textcolor[rgb]{0.02,0.61,0.04}{{#1}}}
% \newcommand{\StringTok}[1]{\textcolor[rgb]{0.02,0.61,0.04}{{#1}}}
% \newcommand{\CommentTok}[1]{\textcolor[rgb]{0.00,0.40,1.00}{\textit{{#1}}}}
% \newcommand{\OtherTok}[1]{\textcolor[rgb]{0.74,0.68,0.62}{{#1}}}
% \newcommand{\AlertTok}[1]{\textcolor[rgb]{1.00,1.00,0.00}{{#1}}}
% \newcommand{\FunctionTok}[1]{\textcolor[rgb]{1.00,0.58,0.35}{\textbf{{#1}}}}
% \newcommand{\RegionMarkerTok}[1]{\textcolor[rgb]{0.74,0.68,0.62}{{#1}}}
% \newcommand{\ErrorTok}[1]{\textcolor[rgb]{0.74,0.68,0.62}{\textbf{{#1}}}}
% \newcommand{\NormalTok}[1]{\textcolor[rgb]{0.74,0.68,0.62}{{#1}}}

% %kete style colors
% \newenvironment{Shaded}{}{}
% \newcommand{\KeywordTok}[1]{\textbf{{#1}}}
% \newcommand{\DataTypeTok}[1]{\textcolor[rgb]{0.50,0.00,0.00}{{#1}}}
% \newcommand{\DecValTok}[1]{\textcolor[rgb]{0.00,0.00,1.00}{{#1}}}
% \newcommand{\BaseNTok}[1]{\textcolor[rgb]{0.00,0.00,1.00}{{#1}}}
% \newcommand{\FloatTok}[1]{\textcolor[rgb]{0.50,0.00,0.50}{{#1}}}
% \newcommand{\CharTok}[1]{\textcolor[rgb]{1.00,0.00,1.00}{{#1}}}
% \newcommand{\StringTok}[1]{\textcolor[rgb]{0.87,0.00,0.00}{{#1}}}
% \newcommand{\CommentTok}[1]{\textcolor[rgb]{0.50,0.50,0.50}{\textit{{#1}}}}
% \newcommand{\OtherTok}[1]{{#1}}
% \newcommand{\AlertTok}[1]{\textcolor[rgb]{0.00,1.00,0.00}{\textbf{{#1}}}}
% \newcommand{\FunctionTok}[1]{\textcolor[rgb]{0.00,0.00,0.50}{{#1}}}
% \newcommand{\RegionMarkerTok}[1]{{#1}}
% \newcommand{\ErrorTok}[1]{\textcolor[rgb]{1.00,0.00,0.00}{\textbf{{#1}}}}
% \newcommand{\NormalTok}[1]{{#1}}
% %end pandoc code hacks

% jodliterate colors
\usepackage{color}
\definecolor{shadecolor}{RGB}{248,248,248}
% j control structures 
\definecolor{keywcolor}{rgb}{0.13,0.29,0.53}
% j explicit arguments x y m n u v
\definecolor{datacolor}{rgb}{0.13,0.29,0.53}
% j numbers - all types see j.xml
\definecolor{decvcolor}{rgb}{0.00,0.00,0.81}
\definecolor{basencolor}{rgb}{0.00,0.00,0.81}
\definecolor{floatcolor}{rgb}{0.00,0.00,0.81}
% j local assignments
\definecolor{charcolor}{rgb}{0.31,0.60,0.02}
\definecolor{stringcolor}{rgb}{0.31,0.60,0.02}
\definecolor{commentcolor}{rgb}{0.56,0.35,0.01}
% primitive adverbs and conjunctions
%\definecolor{othercolor}{rgb}{0.56,0.35,0.01}   
\definecolor{othercolor}{RGB}{0,0,255}
% global assignments
\definecolor{alertcolor}{rgb}{0.94,0.16,0.16}
% primitive J verbs and noun names
\definecolor{funccolor}{rgb}{0.00,0.00,0.00}

% custom colors
\definecolor{CodeBackGround}{cmyk}{0.0,0.0,0,0.05}    % light gray
\definecolor{CodeComment}{rgb}{0,0.50,0.00}           % dark green {0,0.45,0.08}
\definecolor{TableStripes}{gray}{0.9}                 % odd/even background in tables

% Colors for the hyperref package
\definecolor{urlcolor}{rgb}{0,.145,.698}
\definecolor{linkcolor}{rgb}{.71,0.21,0.01}
\definecolor{citecolor}{rgb}{.12,.54,.11}

% % Exact colors from NB
\definecolor{incolor}{HTML}{303F9F}
\definecolor{outcolor}{HTML}{D84315}
\definecolor{cellborder}{HTML}{CFCFCF}
\definecolor{cellbackground}{HTML}{F7F7F7}

% % ANSI colors
\definecolor{ansi-black}{HTML}{3E424D}
\definecolor{ansi-black-intense}{HTML}{282C36}
\definecolor{ansi-red}{HTML}{E75C58}
\definecolor{ansi-red-intense}{HTML}{B22B31}
\definecolor{ansi-green}{HTML}{00A250}
\definecolor{ansi-green-intense}{HTML}{007427}
\definecolor{ansi-yellow}{HTML}{DDB62B}
\definecolor{ansi-yellow-intense}{HTML}{B27D12}
\definecolor{ansi-blue}{HTML}{208FFB}
\definecolor{ansi-blue-intense}{HTML}{0065CA}
\definecolor{ansi-magenta}{HTML}{D160C4}
\definecolor{ansi-magenta-intense}{HTML}{A03196}
\definecolor{ansi-cyan}{HTML}{60C6C8}
\definecolor{ansi-cyan-intense}{HTML}{258F8F}
\definecolor{ansi-white}{HTML}{C5C1B4}
\definecolor{ansi-white-intense}{HTML}{A1A6B2}
\definecolor{ansi-default-inverse-fg}{HTML}{FFFFFF}
\definecolor{ansi-default-inverse-bg}{HTML}{000000}
    

% \usepackage{framed}
% \newenvironment{Shaded}{}{}
% \newcommand{\KeywordTok}[1]{\textcolor{keywcolor}{\textbf{{#1}}}}
% \newcommand{\DataTypeTok}[1]{\textcolor{datacolor}{{#1}}}
% %\newcommand{\DecValTok}[1]{\textcolor{decvcolor}{{#1}}}
% \newcommand{\DecValTok}[1]{{#1}} 
% \newcommand{\BaseNTok}[1]{\textcolor{basencolor}{{#1}}}
% \newcommand{\FloatTok}[1]{\textcolor{floatcolor}{{#1}}}
% \newcommand{\CharTok}[1]{\textcolor{charcolor}{\textbf{{#1}}}}
% \newcommand{\StringTok}[1]{\textcolor{stringcolor}{{#1}}}
% \newcommand{\CommentTok}[1]{\textcolor{commentcolor}{\textit{{#1}}}}
% \newcommand{\OtherTok}[1]{\textcolor{othercolor}{{#1}}} 
% \newcommand{\AlertTok}[1]{\textcolor{alertcolor}{\textbf{{#1}}}}
% %\newcommand{\FunctionTok}[1]{\textcolor{funccolor}{{#1}}}
% \newcommand{\FunctionTok}[1]{{#1}}
% \newcommand{\RegionMarkerTok}[1]{{#1}}
% \newcommand{\ErrorTok}[1]{\textbf{{#1}}}
% \newcommand{\NormalTok}[1]{{#1}}

% The default LaTeX title has an obnoxious amount of whitespace. By default,
% titling removes some of it. It also provides customization options.
\usepackage{titling}

% headers and footers
\usepackage{fancyhdr}
%\pagestyle{fancy}
\pagestyle{plain}

\fancyhead{}
\fancyfoot{}

%\fancyhead[LE,RO]{\slshape \rightmark}
%\fancyhead[LO,RE]{\slshape \leftmark}
\fancyfoot[C]{\thepage}
%\headrulewidth 0.4pt
%\footrulewidth 0 pt

%\addtolength{\headheight}{\baselineskip}

%\lfoot{\emph{Analyze the Data not the Drivel}}
%\rfoot{\emph{\today}}

% subfigure handles figures that contain subfigures
%\usepackage{color,graphicx,subfigure,sidecap}
\usepackage{graphicx,sidecap}
\usepackage{subfigure}
\graphicspath{{./inclusions/}}

% floatflt provides for text wrapping around small figures and tables
\usepackage{floatflt}

% tweak caption formats 
\usepackage{caption} 
\usepackage{sidecap}
%\usepackage{subcaption} % not compatible with subfigure

\usepackage{rotating} % flip tables sideways

% complex footnotes
%\usepackage{bigfoot}

% weird logos \XeLaTeX
\usepackage{metalogo}

\newcommand{\HRule}{\rule{\linewidth}{0.5mm}}

\usepackage[breakable]{tcolorbox}

\usepackage{parskip} % Stop auto-indenting (to mimic markdown behaviour)
    
% Basic figure setup, for now with no caption control since it's done
% automatically by Pandoc (which extracts ![](path) syntax from Markdown).
\usepackage{graphicx}

%\DeclareCaptionFormat{nocaption}{}
%\captionsetup{format=nocaption,aboveskip=0pt,belowskip=0pt}

\usepackage[Export]{adjustbox} % Used to constrain images to a maximum size
\adjustboxset{max size={0.9\linewidth}{0.9\paperheight}}
\usepackage{float}

%\floatplacement{figure}{H} % forces figures to be placed at the correct location

\usepackage{xcolor} % Allow colors to be defined
\usepackage{enumerate} % Needed for markdown enumerations to work
\usepackage{geometry} % Used to adjust the document margins

%\usepackage{amsmath} % Equations
%\usepackage{amssymb} % Equations

\usepackage{textcomp} % defines textquotesingle

% Hack from http://tex.stackexchange.com/a/47451/13684:
\AtBeginDocument{%
	\def\PYZsq{\textquotesingle}% Upright quotes in Pygmentized code
}

\usepackage{upquote} % Upright quotes for verbatim code
\usepackage{eurosym} % defines \euro
\usepackage[mathletters]{ucs} % Extended unicode (utf-8) support

%\usepackage{fancyvrb} % verbatim replacement that allows latex

\usepackage{grffile} % extends the file name processing of package graphics 
					 % to support a larger range
					 
\makeatletter % fix for grffile with XeLaTeX
\def\Gread@@xetex#1{%
  \IfFileExists{"\Gin@base".bb}%
  {\Gread@eps{\Gin@base.bb}}%
  {\Gread@@xetex@aux#1}%
}
\makeatother

% The hyperref package gives us a pdf with properly built
% internal navigation ('pdf bookmarks' for the table of contents,
% internal cross-reference links, web links for URLs, etc.)
\usepackage{hyperref}
% The default LaTeX title has an obnoxious amount of whitespace. By default,
% titling removes some of it. It also provides customization options.
\usepackage{titling}
\usepackage{longtable} % longtable support required by pandoc >1.10
\usepackage{booktabs}  % table support for pandoc > 1.12.2
\usepackage[inline]{enumitem} % IRkernel/repr support (it uses the enumerate* environment)
\usepackage[normalem]{ulem} % ulem is needed to support strikethroughs (\sout)
							% normalem makes italics be italics, not underlines
\usepackage{mathrsfs}

% commands and environments needed by pandoc snippets
% extracted from the output of `pandoc -s`
\providecommand{\tightlist}{%
  \setlength{\itemsep}{0pt}\setlength{\parskip}{0pt}}
  
\DefineVerbatimEnvironment{Highlighting}{Verbatim}{commandchars=\\\{\}}
% Add ',fontsize=\small' for more characters per line
\newenvironment{Shaded}{}{}
\newcommand{\KeywordTok}[1]{\textcolor[rgb]{0.00,0.44,0.13}{\textbf{{#1}}}}
\newcommand{\DataTypeTok}[1]{\textcolor[rgb]{0.56,0.13,0.00}{{#1}}}
\newcommand{\DecValTok}[1]{\textcolor[rgb]{0.25,0.63,0.44}{{#1}}}
\newcommand{\BaseNTok}[1]{\textcolor[rgb]{0.25,0.63,0.44}{{#1}}}
\newcommand{\FloatTok}[1]{\textcolor[rgb]{0.25,0.63,0.44}{{#1}}}
\newcommand{\CharTok}[1]{\textcolor[rgb]{0.25,0.44,0.63}{{#1}}}
\newcommand{\StringTok}[1]{\textcolor[rgb]{0.25,0.44,0.63}{{#1}}}
\newcommand{\CommentTok}[1]{\textcolor[rgb]{0.38,0.63,0.69}{\textit{{#1}}}}
\newcommand{\OtherTok}[1]{\textcolor[rgb]{0.00,0.44,0.13}{{#1}}}
\newcommand{\AlertTok}[1]{\textcolor[rgb]{1.00,0.00,0.00}{\textbf{{#1}}}}
\newcommand{\FunctionTok}[1]{\textcolor[rgb]{0.02,0.16,0.49}{{#1}}}
\newcommand{\RegionMarkerTok}[1]{{#1}}
\newcommand{\ErrorTok}[1]{\textcolor[rgb]{1.00,0.00,0.00}{\textbf{{#1}}}}
\newcommand{\NormalTok}[1]{{#1}}

% Additional commands for more recent versions of Pandoc
\newcommand{\ConstantTok}[1]{\textcolor[rgb]{0.53,0.00,0.00}{{#1}}}
\newcommand{\SpecialCharTok}[1]{\textcolor[rgb]{0.25,0.44,0.63}{{#1}}}
\newcommand{\VerbatimStringTok}[1]{\textcolor[rgb]{0.25,0.44,0.63}{{#1}}}
\newcommand{\SpecialStringTok}[1]{\textcolor[rgb]{0.73,0.40,0.53}{{#1}}}
\newcommand{\ImportTok}[1]{{#1}}
\newcommand{\DocumentationTok}[1]{\textcolor[rgb]{0.73,0.13,0.13}{\textit{{#1}}}}
\newcommand{\AnnotationTok}[1]{\textcolor[rgb]{0.38,0.63,0.69}{\textbf{\textit{{#1}}}}}
\newcommand{\CommentVarTok}[1]{\textcolor[rgb]{0.38,0.63,0.69}{\textbf{\textit{{#1}}}}}
\newcommand{\VariableTok}[1]{\textcolor[rgb]{0.10,0.09,0.49}{{#1}}}
\newcommand{\ControlFlowTok}[1]{\textcolor[rgb]{0.00,0.44,0.13}{\textbf{{#1}}}}
\newcommand{\OperatorTok}[1]{\textcolor[rgb]{0.40,0.40,0.40}{{#1}}}
\newcommand{\BuiltInTok}[1]{{#1}}
\newcommand{\ExtensionTok}[1]{{#1}}
\newcommand{\PreprocessorTok}[1]{\textcolor[rgb]{0.74,0.48,0.00}{{#1}}}
\newcommand{\AttributeTok}[1]{\textcolor[rgb]{0.49,0.56,0.16}{{#1}}}
\newcommand{\InformationTok}[1]{\textcolor[rgb]{0.38,0.63,0.69}{\textbf{\textit{{#1}}}}}
\newcommand{\WarningTok}[1]{\textcolor[rgb]{0.38,0.63,0.69}{\textbf{\textit{{#1}}}}}

% Define a nice break command that doesn't care if a line doesn't already exist.
\def\br{\hspace*{\fill} \\* }
% Math Jax compatibility definitions
\def\gt{>}
\def\lt{<}
\let\Oldtex\TeX
\let\Oldlatex\LaTeX
\renewcommand{\TeX}{\textrm{\Oldtex}}
\renewcommand{\LaTeX}{\textrm{\Oldlatex}}
 
% Pygments definitions
\makeatletter
\def\PY@reset{\let\PY@it=\relax \let\PY@bf=\relax%
    \let\PY@ul=\relax \let\PY@tc=\relax%
    \let\PY@bc=\relax \let\PY@ff=\relax}
\def\PY@tok#1{\csname PY@tok@#1\endcsname}
\def\PY@toks#1+{\ifx\relax#1\empty\else%
    \PY@tok{#1}\expandafter\PY@toks\fi}
\def\PY@do#1{\PY@bc{\PY@tc{\PY@ul{%
    \PY@it{\PY@bf{\PY@ff{#1}}}}}}}
\def\PY#1#2{\PY@reset\PY@toks#1+\relax+\PY@do{#2}}

\expandafter\def\csname PY@tok@w\endcsname{\def\PY@tc##1{\textcolor[rgb]{0.73,0.73,0.73}{##1}}}
\expandafter\def\csname PY@tok@c\endcsname{\let\PY@it=\textit\def\PY@tc##1{\textcolor[rgb]{0.25,0.50,0.50}{##1}}}
\expandafter\def\csname PY@tok@cp\endcsname{\def\PY@tc##1{\textcolor[rgb]{0.74,0.48,0.00}{##1}}}
\expandafter\def\csname PY@tok@k\endcsname{\let\PY@bf=\textbf\def\PY@tc##1{\textcolor[rgb]{0.00,0.50,0.00}{##1}}}
\expandafter\def\csname PY@tok@kp\endcsname{\def\PY@tc##1{\textcolor[rgb]{0.00,0.50,0.00}{##1}}}
\expandafter\def\csname PY@tok@kt\endcsname{\def\PY@tc##1{\textcolor[rgb]{0.69,0.00,0.25}{##1}}}
\expandafter\def\csname PY@tok@o\endcsname{\def\PY@tc##1{\textcolor[rgb]{0.40,0.40,0.40}{##1}}}
\expandafter\def\csname PY@tok@ow\endcsname{\let\PY@bf=\textbf\def\PY@tc##1{\textcolor[rgb]{0.67,0.13,1.00}{##1}}}
\expandafter\def\csname PY@tok@nb\endcsname{\def\PY@tc##1{\textcolor[rgb]{0.00,0.50,0.00}{##1}}}
\expandafter\def\csname PY@tok@nf\endcsname{\def\PY@tc##1{\textcolor[rgb]{0.00,0.00,1.00}{##1}}}
\expandafter\def\csname PY@tok@nc\endcsname{\let\PY@bf=\textbf\def\PY@tc##1{\textcolor[rgb]{0.00,0.00,1.00}{##1}}}
\expandafter\def\csname PY@tok@nn\endcsname{\let\PY@bf=\textbf\def\PY@tc##1{\textcolor[rgb]{0.00,0.00,1.00}{##1}}}
\expandafter\def\csname PY@tok@ne\endcsname{\let\PY@bf=\textbf\def\PY@tc##1{\textcolor[rgb]{0.82,0.25,0.23}{##1}}}
\expandafter\def\csname PY@tok@nv\endcsname{\def\PY@tc##1{\textcolor[rgb]{0.10,0.09,0.49}{##1}}}
\expandafter\def\csname PY@tok@no\endcsname{\def\PY@tc##1{\textcolor[rgb]{0.53,0.00,0.00}{##1}}}
\expandafter\def\csname PY@tok@nl\endcsname{\def\PY@tc##1{\textcolor[rgb]{0.63,0.63,0.00}{##1}}}
\expandafter\def\csname PY@tok@ni\endcsname{\let\PY@bf=\textbf\def\PY@tc##1{\textcolor[rgb]{0.60,0.60,0.60}{##1}}}
\expandafter\def\csname PY@tok@na\endcsname{\def\PY@tc##1{\textcolor[rgb]{0.49,0.56,0.16}{##1}}}
\expandafter\def\csname PY@tok@nt\endcsname{\let\PY@bf=\textbf\def\PY@tc##1{\textcolor[rgb]{0.00,0.50,0.00}{##1}}}
\expandafter\def\csname PY@tok@nd\endcsname{\def\PY@tc##1{\textcolor[rgb]{0.67,0.13,1.00}{##1}}}
\expandafter\def\csname PY@tok@s\endcsname{\def\PY@tc##1{\textcolor[rgb]{0.73,0.13,0.13}{##1}}}
\expandafter\def\csname PY@tok@sd\endcsname{\let\PY@it=\textit\def\PY@tc##1{\textcolor[rgb]{0.73,0.13,0.13}{##1}}}
\expandafter\def\csname PY@tok@si\endcsname{\let\PY@bf=\textbf\def\PY@tc##1{\textcolor[rgb]{0.73,0.40,0.53}{##1}}}
\expandafter\def\csname PY@tok@se\endcsname{\let\PY@bf=\textbf\def\PY@tc##1{\textcolor[rgb]{0.73,0.40,0.13}{##1}}}
\expandafter\def\csname PY@tok@sr\endcsname{\def\PY@tc##1{\textcolor[rgb]{0.73,0.40,0.53}{##1}}}
\expandafter\def\csname PY@tok@ss\endcsname{\def\PY@tc##1{\textcolor[rgb]{0.10,0.09,0.49}{##1}}}
\expandafter\def\csname PY@tok@sx\endcsname{\def\PY@tc##1{\textcolor[rgb]{0.00,0.50,0.00}{##1}}}
\expandafter\def\csname PY@tok@m\endcsname{\def\PY@tc##1{\textcolor[rgb]{0.40,0.40,0.40}{##1}}}
\expandafter\def\csname PY@tok@gh\endcsname{\let\PY@bf=\textbf\def\PY@tc##1{\textcolor[rgb]{0.00,0.00,0.50}{##1}}}
\expandafter\def\csname PY@tok@gu\endcsname{\let\PY@bf=\textbf\def\PY@tc##1{\textcolor[rgb]{0.50,0.00,0.50}{##1}}}
\expandafter\def\csname PY@tok@gd\endcsname{\def\PY@tc##1{\textcolor[rgb]{0.63,0.00,0.00}{##1}}}
\expandafter\def\csname PY@tok@gi\endcsname{\def\PY@tc##1{\textcolor[rgb]{0.00,0.63,0.00}{##1}}}
\expandafter\def\csname PY@tok@gr\endcsname{\def\PY@tc##1{\textcolor[rgb]{1.00,0.00,0.00}{##1}}}
\expandafter\def\csname PY@tok@ge\endcsname{\let\PY@it=\textit}
\expandafter\def\csname PY@tok@gs\endcsname{\let\PY@bf=\textbf}
\expandafter\def\csname PY@tok@gp\endcsname{\let\PY@bf=\textbf\def\PY@tc##1{\textcolor[rgb]{0.00,0.00,0.50}{##1}}}
\expandafter\def\csname PY@tok@go\endcsname{\def\PY@tc##1{\textcolor[rgb]{0.53,0.53,0.53}{##1}}}
\expandafter\def\csname PY@tok@gt\endcsname{\def\PY@tc##1{\textcolor[rgb]{0.00,0.27,0.87}{##1}}}
\expandafter\def\csname PY@tok@err\endcsname{\def\PY@bc##1{\setlength{\fboxsep}{0pt}\fcolorbox[rgb]{1.00,0.00,0.00}{1,1,1}{\strut ##1}}}
\expandafter\def\csname PY@tok@kc\endcsname{\let\PY@bf=\textbf\def\PY@tc##1{\textcolor[rgb]{0.00,0.50,0.00}{##1}}}
\expandafter\def\csname PY@tok@kd\endcsname{\let\PY@bf=\textbf\def\PY@tc##1{\textcolor[rgb]{0.00,0.50,0.00}{##1}}}
\expandafter\def\csname PY@tok@kn\endcsname{\let\PY@bf=\textbf\def\PY@tc##1{\textcolor[rgb]{0.00,0.50,0.00}{##1}}}
\expandafter\def\csname PY@tok@kr\endcsname{\let\PY@bf=\textbf\def\PY@tc##1{\textcolor[rgb]{0.00,0.50,0.00}{##1}}}
\expandafter\def\csname PY@tok@bp\endcsname{\def\PY@tc##1{\textcolor[rgb]{0.00,0.50,0.00}{##1}}}
\expandafter\def\csname PY@tok@fm\endcsname{\def\PY@tc##1{\textcolor[rgb]{0.00,0.00,1.00}{##1}}}
\expandafter\def\csname PY@tok@vc\endcsname{\def\PY@tc##1{\textcolor[rgb]{0.10,0.09,0.49}{##1}}}
\expandafter\def\csname PY@tok@vg\endcsname{\def\PY@tc##1{\textcolor[rgb]{0.10,0.09,0.49}{##1}}}
\expandafter\def\csname PY@tok@vi\endcsname{\def\PY@tc##1{\textcolor[rgb]{0.10,0.09,0.49}{##1}}}
\expandafter\def\csname PY@tok@vm\endcsname{\def\PY@tc##1{\textcolor[rgb]{0.10,0.09,0.49}{##1}}}
\expandafter\def\csname PY@tok@sa\endcsname{\def\PY@tc##1{\textcolor[rgb]{0.73,0.13,0.13}{##1}}}
\expandafter\def\csname PY@tok@sb\endcsname{\def\PY@tc##1{\textcolor[rgb]{0.73,0.13,0.13}{##1}}}
\expandafter\def\csname PY@tok@sc\endcsname{\def\PY@tc##1{\textcolor[rgb]{0.73,0.13,0.13}{##1}}}
\expandafter\def\csname PY@tok@dl\endcsname{\def\PY@tc##1{\textcolor[rgb]{0.73,0.13,0.13}{##1}}}
\expandafter\def\csname PY@tok@s2\endcsname{\def\PY@tc##1{\textcolor[rgb]{0.73,0.13,0.13}{##1}}}
\expandafter\def\csname PY@tok@sh\endcsname{\def\PY@tc##1{\textcolor[rgb]{0.73,0.13,0.13}{##1}}}
\expandafter\def\csname PY@tok@s1\endcsname{\def\PY@tc##1{\textcolor[rgb]{0.73,0.13,0.13}{##1}}}
\expandafter\def\csname PY@tok@mb\endcsname{\def\PY@tc##1{\textcolor[rgb]{0.40,0.40,0.40}{##1}}}
\expandafter\def\csname PY@tok@mf\endcsname{\def\PY@tc##1{\textcolor[rgb]{0.40,0.40,0.40}{##1}}}
\expandafter\def\csname PY@tok@mh\endcsname{\def\PY@tc##1{\textcolor[rgb]{0.40,0.40,0.40}{##1}}}
\expandafter\def\csname PY@tok@mi\endcsname{\def\PY@tc##1{\textcolor[rgb]{0.40,0.40,0.40}{##1}}}
\expandafter\def\csname PY@tok@il\endcsname{\def\PY@tc##1{\textcolor[rgb]{0.40,0.40,0.40}{##1}}}
\expandafter\def\csname PY@tok@mo\endcsname{\def\PY@tc##1{\textcolor[rgb]{0.40,0.40,0.40}{##1}}}
\expandafter\def\csname PY@tok@ch\endcsname{\let\PY@it=\textit\def\PY@tc##1{\textcolor[rgb]{0.25,0.50,0.50}{##1}}}
\expandafter\def\csname PY@tok@cm\endcsname{\let\PY@it=\textit\def\PY@tc##1{\textcolor[rgb]{0.25,0.50,0.50}{##1}}}
\expandafter\def\csname PY@tok@cpf\endcsname{\let\PY@it=\textit\def\PY@tc##1{\textcolor[rgb]{0.25,0.50,0.50}{##1}}}
\expandafter\def\csname PY@tok@c1\endcsname{\let\PY@it=\textit\def\PY@tc##1{\textcolor[rgb]{0.25,0.50,0.50}{##1}}}
\expandafter\def\csname PY@tok@cs\endcsname{\let\PY@it=\textit\def\PY@tc##1{\textcolor[rgb]{0.25,0.50,0.50}{##1}}}

\def\PYZbs{\char`\\}
\def\PYZus{\char`\_}
\def\PYZob{\char`\{}
\def\PYZcb{\char`\}}
\def\PYZca{\char`\^}
\def\PYZam{\char`\&}
\def\PYZlt{\char`\<}
\def\PYZgt{\char`\>}
\def\PYZsh{\char`\#}
\def\PYZpc{\char`\%}
\def\PYZdl{\char`\$}
\def\PYZhy{\char`\-}
\def\PYZsq{\char`\'}
\def\PYZdq{\char`\"}
\def\PYZti{\char`\~}
% for compatibility with earlier versions
\def\PYZat{@}
\def\PYZlb{[}
\def\PYZrb{]}
\makeatother

% For linebreaks inside Verbatim environment from package fancyvrb. 
\makeatletter
	\newbox\Wrappedcontinuationbox 
	\newbox\Wrappedvisiblespacebox 
	\newcommand*\Wrappedvisiblespace {\textcolor{red}{\textvisiblespace}} 
	\newcommand*\Wrappedcontinuationsymbol {\textcolor{red}{\llap{\tiny$\m@th\hookrightarrow$}}} 
	\newcommand*\Wrappedcontinuationindent {3ex } 
	\newcommand*\Wrappedafterbreak {\kern\Wrappedcontinuationindent\copy\Wrappedcontinuationbox} 
	% Take advantage of the already applied Pygments mark-up to insert 
	% potential linebreaks for TeX processing. 
	%        {, <, #, %, $, ' and ": go to next line. 
	%        _, }, ^, &, >, - and ~: stay at end of broken line. 
	% Use of \textquotesingle for straight quote. 
	\newcommand*\Wrappedbreaksatspecials {% 
		\def\PYGZus{\discretionary{\char`\_}{\Wrappedafterbreak}{\char`\_}}% 
		\def\PYGZob{\discretionary{}{\Wrappedafterbreak\char`\{}{\char`\{}}% 
		\def\PYGZcb{\discretionary{\char`\}}{\Wrappedafterbreak}{\char`\}}}% 
		\def\PYGZca{\discretionary{\char`\^}{\Wrappedafterbreak}{\char`\^}}% 
		\def\PYGZam{\discretionary{\char`\&}{\Wrappedafterbreak}{\char`\&}}% 
		\def\PYGZlt{\discretionary{}{\Wrappedafterbreak\char`\<}{\char`\<}}% 
		\def\PYGZgt{\discretionary{\char`\>}{\Wrappedafterbreak}{\char`\>}}% 
		\def\PYGZsh{\discretionary{}{\Wrappedafterbreak\char`\#}{\char`\#}}% 
		\def\PYGZpc{\discretionary{}{\Wrappedafterbreak\char`\%}{\char`\%}}% 
		\def\PYGZdl{\discretionary{}{\Wrappedafterbreak\char`\$}{\char`\$}}% 
		\def\PYGZhy{\discretionary{\char`\-}{\Wrappedafterbreak}{\char`\-}}% 
		\def\PYGZsq{\discretionary{}{\Wrappedafterbreak\textquotesingle}{\textquotesingle}}% 
		\def\PYGZdq{\discretionary{}{\Wrappedafterbreak\char`\"}{\char`\"}}% 
		\def\PYGZti{\discretionary{\char`\~}{\Wrappedafterbreak}{\char`\~}}% 
	} 
	% Some characters . , ; ? ! / are not pygmentized. 
	% This macro makes them "active" and they will insert potential linebreaks 
	\newcommand*\Wrappedbreaksatpunct {% 
		\lccode`\~`\.\lowercase{\def~}{\discretionary{\hbox{\char`\.}}{\Wrappedafterbreak}{\hbox{\char`\.}}}% 
		\lccode`\~`\,\lowercase{\def~}{\discretionary{\hbox{\char`\,}}{\Wrappedafterbreak}{\hbox{\char`\,}}}% 
		\lccode`\~`\;\lowercase{\def~}{\discretionary{\hbox{\char`\;}}{\Wrappedafterbreak}{\hbox{\char`\;}}}% 
		\lccode`\~`\:\lowercase{\def~}{\discretionary{\hbox{\char`\:}}{\Wrappedafterbreak}{\hbox{\char`\:}}}% 
		\lccode`\~`\?\lowercase{\def~}{\discretionary{\hbox{\char`\?}}{\Wrappedafterbreak}{\hbox{\char`\?}}}% 
		\lccode`\~`\!\lowercase{\def~}{\discretionary{\hbox{\char`\!}}{\Wrappedafterbreak}{\hbox{\char`\!}}}% 
		\lccode`\~`\/\lowercase{\def~}{\discretionary{\hbox{\char`\/}}{\Wrappedafterbreak}{\hbox{\char`\/}}}% 
		\catcode`\.\active
		\catcode`\,\active 
		\catcode`\;\active
		\catcode`\:\active
		\catcode`\?\active
		\catcode`\!\active
		\catcode`\/\active 
		\lccode`\~`\~ 	
	}
\makeatother

\let\OriginalVerbatim=\Verbatim
\makeatletter
\renewcommand{\Verbatim}[1][1]{%
	%\parskip\z@skip
	\sbox\Wrappedcontinuationbox {\Wrappedcontinuationsymbol}%
	\sbox\Wrappedvisiblespacebox {\FV@SetupFont\Wrappedvisiblespace}%
	\def\FancyVerbFormatLine ##1{\hsize\linewidth
		\vtop{\raggedright\hyphenpenalty\z@\exhyphenpenalty\z@
			\doublehyphendemerits\z@\finalhyphendemerits\z@
			\strut ##1\strut}%
	}%
	% If the linebreak is at a space, the latter will be displayed as visible
	% space at end of first line, and a continuation symbol starts next line.
	% Stretch/shrink are however usually zero for typewriter font.
	\def\FV@Space {%
		\nobreak\hskip\z@ plus\fontdimen3\font minus\fontdimen4\font
		\discretionary{\copy\Wrappedvisiblespacebox}{\Wrappedafterbreak}
		{\kern\fontdimen2\font}%
	}%
	
	% Allow breaks at special characters using \PYG... macros.
	\Wrappedbreaksatspecials
	% Breaks at punctuation characters . , ; ? ! and / need catcode=\active 	
	\OriginalVerbatim[#1,codes*=\Wrappedbreaksatpunct]%
}
\makeatother


% prompt
\makeatletter
\newcommand{\boxspacing}{\kern\kvtcb@left@rule\kern\kvtcb@boxsep}
\makeatother
\newcommand{\prompt}[4]{
	\ttfamily\llap{{\color{#2}[#3]:\hspace{3pt}#4}}\vspace{-\baselineskip}
}
    

% Prevent overflowing lines due to hard-to-break entities
\sloppy 

% Setup hyperref package
\hypersetup{
  breaklinks=true,  % so long urls are correctly broken across lines
  colorlinks=true,
  urlcolor=urlcolor,
  linkcolor=linkcolor,
  citecolor=citecolor,
  pdfauthor={John D. Baker},
  pdftitle={Analyze the Data not the Drivel},
  pdfsubject={Blog},
  pdfcreator={MikTeX+LaTeXe},
  pdfkeywords={blog,wordpress},
  }
  
% Slightly bigger margins than the latex defaults
% \geometry{verbose,tmargin=1in,bmargin=1in,lmargin=1in,rmargin=1in}  

%\usepackage{wrapfig}

% source code listings
\usepackage{listings}

\lstdefinelanguage{bat}
{morekeywords={echo,title,pushd,popd,setlocal,endlocal,off,if,not,exist,set,goto,pause},
sensitive=True,
morecomment=[l]{rem}
}

\lstdefinelanguage{jdoc}
{
morekeywords={},
otherkeywords={assert.,break.,continue.,for.,do.,if.,else.,elseif.,return.,select.,end.
,while.,whilst.,throw.,catch.,catchd.,catcht.,try.,case.,fcase.},
sensitive=True,
morecomment=[l]{NB.},
morestring=[b]',
morestring=[d]',
}

% latex size ordering - can never remember it
% \tiny
% \scriptsize
% \footnotesize
% \small
% \normalsize
% \large
% \Large
% \LARGE
% \huge
% \Huge
 
% listings package settings  
\lstset{%
  language=jdoc,                                % j document settings
  basicstyle=\ttfamily\footnotesize,            
  keywordstyle=\bfseries\color{keywcolor}\footnotesize,
  identifierstyle=\color{black},
  commentstyle=\slshape\color{CodeComment},     % colored slanted comments
  stringstyle=\color{red}\ttfamily,
  showstringspaces=false,                       
  %backgroundcolor=\color{CodeBackGround},       
  frame=single,                                
  framesep=1pt,                                 
  framerule=0.8pt,                             
  rulecolor=\color{CodeBackGround},   
  showspaces=false,
  %columns=fullflexible,
  %numbers=left,
  %numberstyle=\footnotesize,
  %numbersep=9pt,
  tabsize=2,
  showtabs=false,
  captionpos=b
  breaklines=true,                              
  breakindent=5pt                              
}

\lstdefinelanguage{JavaScript}{
  keywords={typeof, new, true, false, catch, function, return, null, catch, switch, var, if, in, while, do, else, case, break},
  ndkeywords={class, export, boolean, throw, implements, import, this},
  ndkeywordstyle=\color{darkgray}\bfseries,
  sensitive=false,
  comment=[l]{//},
  morecomment=[s]{/*}{*/},
  morestring=[b]',
  morestring=[b]"
}

% C# settings
\lstdefinestyle{sharpc}{
language=[Sharp]C,
basicstyle=\ttfamily\scriptsize, 
keywordstyle=\bfseries\color{keywcolor}\scriptsize,
framerule=0pt
}

% for source code listing longer than two use smaller font
\lstdefinestyle{smallersource}{
basicstyle=\ttfamily\scriptsize, 
keywordstyle=\bfseries\color{keywcolor}\scriptsize,
framerule=0pt
}

\lstdefinestyle{resetdefaults}{
language=jdoc,
basicstyle=\ttfamily\footnotesize,  
keywordstyle=\bfseries\color{keywcolor}\footnotesize,                                                               
framerule=0.8pt 
}

% APL UTF8 code points listed for lstlisting processing
\makeatletter
\lst@InputCatcodes
\def\lst@DefEC{%
 \lst@CCECUse \lst@ProcessLetter
  ^^80^^81^^82^^83^^84^^85^^86^^87^^88^^89^^8a^^8b^^8c^^8d^^8e^^8f%
  ^^90^^91^^92^^93^^94^^95^^96^^97^^98^^99^^9a^^9b^^9c^^9d^^9e^^9f%
  ^^a0^^a1^^a2^^a3^^a4^^a5^^a6^^a7^^a8^^a9^^aa^^ab^^ac^^ad^^ae^^af%
  ^^b0^^b1^^b2^^b3^^b4^^b5^^b6^^b7^^b8^^b9^^ba^^bb^^bc^^bd^^be^^bf%
  ^^c0^^c1^^c2^^c3^^c4^^c5^^c6^^c7^^c8^^c9^^ca^^cb^^cc^^cd^^ce^^cf%
  ^^d0^^d1^^d2^^d3^^d4^^d5^^d6^^d7^^d8^^d9^^da^^db^^dc^^dd^^de^^df%
  ^^e0^^e1^^e2^^e3^^e4^^e5^^e6^^e7^^e8^^e9^^ea^^eb^^ec^^ed^^ee^^ef%
  ^^f0^^f1^^f2^^f3^^f4^^f5^^f6^^f7^^f8^^f9^^fa^^fb^^fc^^fd^^fe^^ff%
  ^^^^20ac^^^^0153^^^^0152%
  ^^^^20a7^^^^2190^^^^2191^^^^2192^^^^2193^^^^2206^^^^2207^^^^220a%
  ^^^^2218^^^^2228^^^^2229^^^^222a^^^^2235^^^^223c^^^^2260^^^^2261%
  ^^^^2262^^^^2264^^^^2265^^^^2282^^^^2283^^^^2296^^^^22a2^^^^22a3%
  ^^^^22a4^^^^22a5^^^^22c4^^^^2308^^^^230a^^^^2336^^^^2337^^^^2339%
  ^^^^233b^^^^233d^^^^233f^^^^2340^^^^2342^^^^2347^^^^2348^^^^2349%
  ^^^^234b^^^^234e^^^^2350^^^^2352^^^^2355^^^^2357^^^^2359^^^^235d%
  ^^^^235e^^^^235f^^^^2361^^^^2362^^^^2363^^^^2364^^^^2365^^^^2368%
  ^^^^236a^^^^236b^^^^236c^^^^2371^^^^2372^^^^2373^^^^2374^^^^2375%
  ^^^^2377^^^^2378^^^^237a^^^^2395^^^^25af^^^^25ca^^^^25cb%  
  ^^00}
\lst@RestoreCatcodes
\makeatother

% custom lengths used within minipages
\newcommand{\minindent}{17pt}

\makeindex

\begin{document}

\subsection*{\href{http://analyzethedatanotthedrivel.org/2023/11/28/the-book-of-terms/}{The Book of Terms}}
\addcontentsline{toc}{subsection}{The Book of Terms}


\noindent\emph{Posted: 28 Nov 2023 18:43:06}
\vspace{6pt}

A tale by \emph{Kline Leopold Hedrös} ©2023

\hypertarget{arthur-2024}{%
\subsection*{­Arthur -- 2024}\label{arthur-2024}}

The first time Arthur saw the old wooden box, he wondered if it was
worth anything. It was a recurring thought. A week before, he had
learned of the death of his grandmother, Alice, while zooming with his
AI dataset group in Tacoma. During a discussion about hashing rapidly
changing token networks, an intrusive call bypassed his spam filters and
rang his \emph{Charge of the Light Brigade} ringtone. He was annoyed and
relieved to take the call. Alice was his favorite and only grandmother.
She had raised him in Pulaski, New York after his mother and father died
in a grisly northern New York winter car accident. Black ice and
lake-effect snow kill often. Arthur never really knew his parents; he
was eight months old when the accident happened, so he didn't grieve his
parents, but he missed them. He would be looking over some dumbass bit
of database code and then want to talk to a dad he couldn't remember.
The mind is weird. Alice was a good and loving grandmother. She did her
best with Arthur, and despite her limited means, Arthur, a bright child,
moved from success to success: a full scholarship to Columbia, then
lucrative West Coast job offers, followed by a quick ascent of corporate
career ladders. When the call announcing Alice's death interrupted his
Zoom session, Arthur was a senior manager at a small company that
specialized in preparing training pipelines for AI systems.

Arthur couldn't afford time off to deal with his grandmother's estate,
but his wife Beth wouldn't have it. She reminded him that Alice deserved
better and he better ``man up.'' So, Arthur found himself in Alice's
tiny Pulaski home, sifting through a lifetime of debris. Oddly, the more
he sorted through Alice's belongings, the more he enjoyed it. Alice
wasn't a hoarder, but her house was filled with interesting archaic
stuff: old records, magazines, out-of-print books, family pictures,
kitschy ceramic dog figurines, and mysterious kitchen tools. Arthur
enjoyed Alice's effects because her affairs were in good order. She had
a clearly written will, no outstanding debts or pending taxes, and no
heirs except Arthur. Settling the estate and selling the house would be
a smooth, trouble-free chore. It was the last of Alice's gifts. He gave
away as much as he could to Alice's Pulaski neighbors but opted to sell
the house as is, contents included. It was a lazy, fast, nerdy first
approximation to estate processing. He disposed of everything with one
exception: the wooden box.

It wasn't really a box; it was an old traveling portable desk. The type
of container Alexander Hamilton wrote George Washington's letters on
during the Revolutionary War. The wood, like all finished 19th-century
woodwork, was high quality; modern boxes are crap by comparison. Even
the box's brass hinges gleamed. Arthur didn't know how old it was, but
it was definitely an antique. When he picked it up, he was surprised by
its weight and the muted thumping sound of its contents.

Placing the box on his grandmother's kitchen counter, he opened it to
find around two hundred unbound sheets of tan paper. An orange sticky
note with two words: ``Grandma Martha,'' was attached to the first page.
It was in Alice's bad handwriting; Alice's handwriting had rapidly
declined in her final years. Arthur and Beth could barely decipher her
last Christmas card.

Each tan sheet looked like it had been laid against a gravestone or
another hard surface and then gone over with charcoal. The charcoal
highlighted evenly spaced, not quite hieroglyphic, not quite Greek, and
not quite Arabic symbols. Arthur was intrigued. His AI unit was involved
with a cuneiform translation project. Due to the sheer drudgery needed,
most of the world's surviving cuneiform tablets have never been
translated. Maybe a trained AI could help. Alice's strange charcoal
sheets might be useful test materials.

\hypertarget{martha-1880}{%
\subsection*{Martha -- 1880}\label{martha-1880}}

Martha was a precocious child, a superb student, and a worry to her
mother. At nine years old, it was clear to anyone who dealt with her
that she was the brightest person they had ever met. Her mother feared
she would never get married. Men in rural 1880s Illinois didn't marry
smart girls; it would have been better, her mother thought, if Martha
was pretty. Then Martha could get married, have children, and have a
normal life.

Martha was not on the normal life path. She would spend years working as
a calculator in Harvard's esteemed stellar catalog group, but that's in
Martha's future. Today, she is in her one-room schoolhouse with her
teacher, Miss Wright. The other students were outside playing and being
the semi-literate Illinois backwoods hicks that they were.

Martha was trying to interest Miss Wright in something she had noticed
yesterday. If you draw three circles with different diameters on a piece
of paper and then draw tangents to the three pairs of circles, you'll
see that the three tangent pair crossings always line up. Martha drew
half a dozen diagrams illustrating her discovery, but Miss Wright seemed
unimpressed. It was Martha being Martha. It's not true that all teachers
are looking for bright students. Many teachers prefer quiet,
unchallenging students who do as they're told. Smart kids are work. And
Miss Wright was only teaching until she could get married, have
children, and lead a normal life. Still, Miss Wright looked over
Martha's circle drawings as the child waved them in front of her, and
then she remembered the old Frenchman.

\hypertarget{antoine-1824}{%
\subsection*{Antoine -- 1824}\label{antoine-1824}}

Most days, Antoine found it hard to believe he was a Mississippi river
man. How could an \emph{École Polytechnique} student and an enthusiastic
supporter of Napoleon's efforts to reform French education end up eking
out a living manhandling riverboat cargo? As Antoine lamented daily, the
chain of events that led from Paris to the Mississippi was improbable
but not impossible.

In April of 1823 (the same year Joseph Smith claimed the Angel Moroni
visited him), Antoine and his friends were carousing on the South Bank.
Everyone was drunk and misbehaving. While crossing a Seine bridge near
Notre Dame, his party ran into an equally drunk party of young army
recruits. A minor brawl broke out, and Antoine pushed a recruit over the
bridge during the melee. He watched as the recruit tumbled in the air
and head-first-thunked into a bridge abutment. His body instantly
relaxed and fell smoothly into the water. Antoine knew he had died the
instant his head struck the bridge. Heads were severed in France for
less. So, as Antoine watched the body float downstream, he ran.

Assuming the alias of a young seminarian named Ancetin, he made it to
Bordeaux and secured passage to New Orleans. Four weeks later, at
nineteen, he arrived in New Orleans. Napoleonic law still reigned in New
Orleans, so Antoine inconspicuously worked north, going up the
Mississippi and Ohio rivers and trying out his new career as an
itinerant schoolteacher and laborer. Antoine found that early
19th-century America was not interested in Napoleonic rigor or Yankee
sloth. Most villages didn't have schools, and when they did, teachers
were poorly paid, and even if schools existed and teachers were paid,
students often showed up drunk on beer or cider. Inebriated kids
attended school haphazardly, learned little, and couldn't wait to be
elsewhere. In a dozen towns, Antoine encountered a handful of diligent
students and no parents who believed in ``book learning.'' It was
depressing but better than the guillotine.

\hypertarget{arthur-2024-1}{%
\subsection*{Arthur -- 2024}\label{arthur-2024-1}}

On returning to Tacoma, Arthur showed Beth Alice's odd etching-filled
wooden box. He was expecting she would share his curiosity, but not so.
She didn't want more ``old junk'' in their apartment. She was going
through one of her organizing moods. Ok. He took the box into his office
and placed it on a metal cabinet filled with laser printer refills.

Arthur was lucky to have an office. In the 2020s, even senior managers
were cast into the open office abyss. One of Arthur's cynical peers
described the open office as a corporate conspiracy to save money on
furniture. Arthur only objected to the ``conspiracy'' bit. And people
wonder why it's hard to get post-COVID programmers back into the office.
How about putting us in offices with doors and making it worthwhile?
Nah! Arthur's team seldom met in ``the office''; most were remote, they
weren't in Tacoma or North America, and there was nothing the suits
could do about it. Try ``managing'' top-notch, 21st-century machine
learning programmers with 19th-century office methods. You'd be
surprised how fast nerds can move when necessary.

Still annoyed with Beth's box banishment, Arthur logged into his
morning's one-on-one with Dan late. Dan often \emph{Starlinked} in from
his four-by-four. Currently, Dan was on the back roads of Canyonlands
National Park. He delighted in zooming over beautiful landscapes while
his home and office bond compatriots looked over their sad bedroom Nerf
collections and generic, always out-of-date, office whiteboards.

``Programmers be sad,'' Dan teased.

The camera on Arthur's monitor showed Alice's box in the background. Dan
noticed and asked about it. Arthur told Dan what he knew about the box.
He even opened it up and showed Dan some of the etchings.

``Are all the etchings the same size?'' Dan asked.

Arthur said the paper sheets were about the same size, but he hadn't
looked at all the pages; the charcoal marks might vary. Dan was the lead
on the cuneiform project; he saw language puzzles everywhere. He asked
Arthur to scan the sheets and load them into Jones. Jones, after
\emph{Indiana Jones} -- because programmers -- was the company's
language AI. It had been trained on hundreds of millions of pages of
text from all curated modern and extinct human languages. Jones knows,
in the inexpressible AI way, more about the structure of human languages
than any bipedal meat puppet. Arthur didn't feel like doing any real
work, so he grabbed Alice's box and headed down the hall to wage war
with that goddam pain in the fricking-ass printer-scanner that always
needed unjamming and new printer cartridges. One of these days, Arthur
was going to go full \emph{Office Space} on that damn device.

\hypertarget{martha---1880}{%
\subsection*{Martha - 1880}\label{martha---1880}}

Martha wasn't scared by the old Frenchman, but she wasn't entirely at
ease with the ancient rocking-chaired figure before her either. The old
Frenchman lived in Banner's boarding house. Every day, he sat beside the
boarding house front door in his rocking chair and called out arrivals
and departures. Nobody knew how this paid for his room and board, but
the Banners didn't seem to mind his presence. Miss Wright told Martha
that the old Frenchman used to tutor the governor's children and
encouraged Abraham Lincoln when the president studied his figures. Maybe
he would be interested in Martha's circles. When Martha heard about the
old Frenchman, she told Miss Wright she wanted to see him -- now! Then
Martha abruptly got up and walked out of school. Miss Wright watched her
go but then, prudently, sent one of her many underachieving students to
tell Martha's mother where the child was heading.

When Martha arrived on the boarding house porch, the old Frenchman
called out her arrival. He suffered from large cataracts in both eyes
and couldn't see her clearly, but he knew a small person was present.

He asked, ``Who is it?''

Martha said her teacher said he could help her with figures. Then, she
enthusiastically waved her circle diagrams before his cataract-fogged
eyes and started describing her discovery.

``I'm sorry,'' the old Frenchman said, ``My eyes are very bad. You will
have to speak slowly so I can picture it in here,'' he said, tapping his
head. ``Also, Miss \ldots''

``Martha,'' Martha answered. She had forgotten her manners. She was
always to introduce herself and ask how others were doing. ``Hello,
what's your name?'' she said.

``Antoine Legault, young lady,'' the old Frenchman replied.

With introductions out of the way, Martha described her circle discovery
to Antoine, and of course, Antoine knew precisely what she was talking
about. Martha had rediscovered Gaspard Monge's lovely little circle
theorem. Antoine didn't tell Martha she had found something already
known. In his seventy-six years, he had taught long enough to know that
self-discovery is the essence of education. You don't stamp it out with
academic pedigrees; you encourage it. Antoine asked Martha how she knew
the crossing points lined up. It was the start of Martha's first big
friendship and Antoine's last. Martha often said that she was fortunate
to have met the old Frenchman; he was the best teacher she had ever had,
and it's unlikely she would have ended up a Harvard calculator without
him.

\hypertarget{antoine-1825}{%
\subsection*{Antoine -- 1825}\label{antoine-1825}}

Teaching school in frontier America wasn't lucrative. It didn't take an
\emph{École Polytechnique} certificate to figure that out. Antoine
taught school when possible but was forced to pursue other ways of
keeping his belly full. It was a relentless struggle to pay for room and
board when he wasn't on the Mississippi or Ohio rivers, but he got lucky
one afternoon on the Saint Louis docks. An argument had broken out
between an old, short, well-dressed townsman and a boat filled with
fur-coated trappers. It only caught his ear because the trappers were
screaming in a guttural French accent, probably French Canadian. Without
thinking, he weighed in with his smooth Parisian French. The townsman
and the trappers both shut up and looked at him. Then, both parties
started begging him to help settle the issue. It turned out that Antoine
had just the skills Saint Louis traders needed: mathematical training to
manage accounts, facility with many languages, particularly French for
the Missouri river trapper boats, and a knack for negotiation. The old
townsman, Mr.~Knoles, offered Antoine work on the dock. In the next two
years, Old Knoles, as Antoine affectionately called him, became almost a
second father to Antoine. Both men were lonely; Old Knoles was a widower
without surviving children. He had buried eight consecutive girls and
then his wife. He knew his days were ending, and despite some success
with his ``Fur and Mercantile'' business, he had no heirs or trustworthy
friends. Antoine was a blessing to Knoles, and Antoine felt the same
about Old Knoles.

\hypertarget{arthur-2024-2}{%
\subsection*{Arthur -- 2024}\label{arthur-2024-2}}

Scanning Alice's sheets took far less time than Arthur expected. The
fricking printer-scanner just did its job. It was a technological
miracle. As he was feeding sheets through the device, he was impressed
by the quality of the paper. Modern pulp paper, like modern music,
modern art, and modern shit in general, is mostly garbage. Once
something is ``commodified'' in our modern world, its production follows
a predictable course of suit-driven cost-cutting and
``enshitification,'' to use the precise term. As Arthur wiped charcoal
off his fingers, he realized the paper might tell him how old the
etchings were. It only took a few minutes to look up antique book and
stationary dealers on his phone. The nearest was just north of Tacoma,
almost in Seattle. He still didn't feel like doing any real work, so he
gathered the etchings, put them back in Alice's box, and headed to the
stationary dealer. Before starting his car, he texted Dan a link to the
etching scan data. Dan would have to load Jones himself.

\hypertarget{martha-1885}{%
\subsection*{Martha -- 1885}\label{martha-1885}}

Monsieur Legault's health started failing as he turned eighty-one. For
five years, Martha had visited the Old Frenchman almost daily. Most of
the time, he directed her post-school schooling. Martha's mother
demanded that she quit school at twelve, devote her energies to the
family farm, and master all the skills needed to make a good village
wife. It was like a death sentence for the mind. Martha relented but
nagged her mother into letting her continue visiting the Old Frenchman.
Martha's mother didn't consider the old rocking chair-bound man a threat
to her daughter's reputation, and having the girl pass through town
often might attract suitors. The hour or two Martha spent with Monsieur
Legault gave her perhaps the best education available in western
Illinois. Both relished their sessions. Martha noticed her mentor's
decline and tried to help Monsieur Legault as much as she could, but she
also did her best not to think about what life would be like if Monsieur
Legault died. So, she was taken aback when Monsieur Legault said he had
wanted her to take something before he died.

Monsieur Legault started their conversation with the oddest question,
``Do you know about Joseph Smith?''

Of course, Martha knew Joseph Smith. Smith and his crazy followers had
left an indelible impression on the people of western Illinois. Her
mother told her stories about how Smith's followers, the Mormons, as
they called themselves, would leave nearby Nauvoo, steal cattle, sheep,
horses, and whatever else they could get their hands on, and then
retreat to Nauvoo, where Smith would refuse to hand them over when
petitioned by neighboring towns. Smith ran Nauvoo like a little
caliphate and largely ignored Illinois state laws. He got away with it
for years because his followers voted as a block. If you wanted to get
elected to anything in Illinois, you had to cut deals with Smith. He
would then tell his zombies to vote for you or your opponent. It worked
until Smith crossed both sides in an election.

Consequently, when Smith was gunned down and the Mormons were driven
out, a lot of people, Martha's mother in particular, felt they had it
coming. Martha had also heard the tale of the Golden Plates. It was a
great story. As a Lutheran, she didn't believe a word of it, but it was
still a great story. How many people have glowing white angels show up
and tell them where to dig for buried treasure?

\hypertarget{antoine-1827}{%
\subsection*{Antoine -- 1827}\label{antoine-1827}}

Antoine worked with Old Knoles for two years out of his tiny Saint Louis
Fifth Street "Fur and Mercantile." It was a good time for Antoine.
Mr.~Knoles was delighted with his work and told Antoine he was planning
to leave the business to him; he was getting too old to keep haggling
with river traders, and Antoine was better at it anyway. Antoine was
also settling into Saint Louis's embryonic society. He had made some
friends, and his elevated status as a trader made him attractive to
young women. The mercenary nature of women did not surprise Antoine:
highly educated Frenchmen consider it their defining trait. Still, the
female attention was nice. Antoine would have stayed in Saint Louis,
lived out his life as an accomplished townsman, and ended up under an
elegant tombstone in the Cathedral Catholic Cemetery on Walnut. At least
that was the fantasy; then Jules appeared on a New Orleans steamer.
Jules saw Antoine push the army recruit off the Seine bridge four years
ago. Antoine's running wasn't done.

Antoine told Old Knoles about what happened on the bridge. The old man
wasn't entirely surprised or particularly bothered. Men of Antoine's
class did not work the rivers unless compelled. And Antoine was hardly
the only inhabitant of Saint Louis with past shame. Old Knoles would
have happily overlooked the bridge incident, but he recognized the
younger man's peril. People wouldn't prosecute Antoine. Nobody in Saint
Louis cared about a drunken Parisian brawl, but the gossip, "He just let
the poor boy drown," would slowly poison his business reputation, and
who knows how far the gossip would spread. Bounty hunters trolled up and
down the rivers; all it would take was a small reward on Antoine's head,
and then, instead of attracting young women, another more violent
section of society would take notice. Antoine had to flee again.

Before leaving Saint Louis, Old Knoles sat him down in the tiny "Fur and
Mercantile" shop they had shared and placed his prized traveling desk in
Antoine's hands. The traveling desk was a small, hinged wooden box. Old
Knoles said it had belonged to his father, who claimed Lewis and Clark
had used it. Knoles didn't believe the Lewis and Clark story, but the
desk reminded him of his father every time he used it. Antoine had
recently filled the desk with a Lexington Fayette Mill paper shipment.
It had been their largest expense in the prior month. Paper was
expensive. Antoine started taking sheets out, but Old Knoles stopped
him. Both knew the sheets could be traded for anything on the rivers.
Blank paper sheets were oddly more valuable than many banknotes.

\hypertarget{arthur-2024-3}{%
\subsection*{Arthur -- 2024}\label{arthur-2024-3}}

Antique stationery goes in and out of style. Unbeknownst to Arthur, the
social media addicts in Seattle were enjoying another rich white people
fad. They would buy genuine, expensive 19th-century paper stock, write
tweets with quills dipped in iron gall ink, and then pay Uber drivers to
hand-deliver their missives. It was the 21st century cosplaying the
19th.

So, when Arthur showed Alice's etched sheets to David, an outlandishly
effeminate antique book dealer, you'd have sworn the guy creamed his
underwear. David immediately offered to pay three figures for the paper.
It was a bad move. Arthur now knew, as he had first suspected, that the
box was worth something. If the sheets had been blank, Arthur probably
would have unloaded the box and its contents at whatever ridiculous
price he could get, but the strange markings interested him.

David knew more about two-hundred-year-old paper than any bipedal meat
puppet should have. He pointed out the faint (unless you have a
microscope with polarized light) watermarks on the pages. The paper came
from a paper mill that operated in Lexington, Kentucky, from 1811 until
sometime in the 1830s. It was classic hand-pressed wove rag paper,
durable, and ridiculously high quality by modern standards. David wasn't
interested in the charcoal etchings; they were a waste of good old
paper.

Arthur considered what he heard and asked, ``Just to be clear, these
etchings are around two hundred years old?''

David replied, ``Well, not conclusively. The etchings could have been
done anytime during the last two centuries, but the paper is two hundred
years old. If you're interested, it's possible to date the Carbon in the
etchings. You won't get much Carbon off the etchings, but with particle
acceleration methods, even traces of Carbon can be dated to decades.
There are labs in Stanford, and I believe Irvine, of all places, that do
this.''

\hypertarget{antoine-1828}{%
\subsection*{Antoine -- 1828}\label{antoine-1828}}

Antoine's years in Saint Louis spoiled him: sleeping in a warm bed and
not worrying about his next meal made his return to river work harder
than his flight from Paris. Everywhere he went, he looked for
opportunities. Gold digging presented as a peculiar opportunity. When
Antoine heard of gold digging, he confused it with gold mining. Gold
digging has nothing to do with gold mining. Gold digging is a con. You
scour the countryside with seer stones, dowsing sticks, amulets, and
Indian bones in search of buried treasure. Antoine knew that magic did
not reveal treasure, but that wasn't a problem because you supplied your
own treasure. Americans may have been poor geometry students, but they
excelled in do-it-yourself ancient artifact forgeries. The trick with
gold digging was finding someone stupid enough to trade for the
worthless objects you hid. It was fraudulent, criminal, duplicitous, and
dangerous. Cheating violent men takes gall and glib, and Antoine was
lucky enough to learn from one of history's finest con artists: Joseph
Smith.

Before founding his religion, Joseph Smith was a gold digger. He
traveled the countryside with his magic seer stone, uncovering treasure
that he sold at inflated prices to the gullible. As long as the gullible
stayed gullible, gold digging remained lucrative, but people weren't as
stupid as conmen would like, so gold diggers needed to keep moving.
Antoine encountered Smith in western New York near the Pennsylvania
border. Both men were digging. Antoine was digging out of necessity, but
Smith seemed made for it. Smith was tall, blue-eyed, and spoke with
impressive disarming sincerity.

One gold digger said, ``Smith could pee in a mug and sell it as beer.''

Antoine joined the diggers orbiting Smith and followed them around
middle America. For Antoine, digging was barely better than handling
boat cargo, but he kept at it more for the company than anything else.
Diggers were aggressive, unkempt, and deluded, but thoroughly
entertaining people. Antoine never joined Smith's new church, but he
wasn't surprised that Smith managed to don the mantle of a prophet
without being laughed out of town; \emph{his disarming sincerity was
mesmerizing.} As Smith's cult grew and his enraptured followers became
more fanatical and dangerous, Antoine had the sense to keep his
distance. He'd seen unthinking loyalty as a boy in Napoleonic France,
and though Smith was a poor Napoleon, both inspired the worst in some of
their followers.

When Antoine tracked Smith to Harmony, Smith's zealous, dangerous
followers were still years away. He had come to Harmony with other
disgruntled diggers because Smith had cheated them, and they were
looking to settle scores. Their grievances were bolstered by wild rumors
that Smith had found a real treasure. Antoine had heard all this before
but had to admit things were different in Harmony. Since their last
encounter, Smith had married (who would marry him), he dressed better,
and the townspeople deferred to him. Smith was now playing the part of a
prosperous townsman. Antoine and his fellow gold diggers discreetly
staked out Smith's properties and waited for an opportunity to grab his
rumored treasure.

\hypertarget{arthur-2024-4}{%
\subsection*{Arthur -- 2024}\label{arthur-2024-4}}

His consultation with David about Alice's old etchings piqued Arthur's
curiosity. Before driving back to Tacoma, he tracked down the Stanford
Lab that David had mentioned. What did people do before smartphones? He
talked to a young Gen-X lab tech while illegally parked near David's
stationary store. The tech explained that even trace amounts of Carbon
could be accurately dated, but the sample would be destroyed. The tech
offered to run a quick test if Arthur could send him a page or two.
Arthur fetched the box beside him, grabbed the last sheet, and then,
leaving his car illegally parked, he found a store that managed UPS
deliveries. Before slipping the sheet into an overnight UPS envelope, he
attached a note with his name, email, phone number, and business card.
Arthur seldom passed out business cards. Programmers consider business
cards archaic suit-people talismans. However, he thought the lab might
take him more seriously if they saw he was a senior manager in a hot AI
company instead of a random crank. He paid the inflated overnight UPS
rate and then rushed to his illegally parked car.

\hypertarget{martha-1903}{%
\subsection*{Martha -- 1903}\label{martha-1903}}

Martha, like every citizen of the United States, had run into Mormon
missionaries. An eager pair of young men had once wasted a few hours
trying to convert her years ago. She couldn't remember why she had ever
talked to them; maybe they were cute. Mormons found few converts in
Boston. The city had its own brand of religious crazy that had cornered
the market since Puritan times, and Boston's universities, especially
Harvard, offered thin missionary pickings. But repeated failure didn't
stop the missionaries; failure is part of the missionary process. It
reinforces believers of all creeds that they are custodians of special
wisdom that a corrupt and fallen world must reject. The two men standing
before Martha didn't fit the Mormon missionary style. They were much
older and looked more like lawyers or process servers than harbingers of
the one true church. They politely informed Martha that they were Mormon
church historians gathering information about the early days of Smith's
church.

At the dawn of the 20th century, the world suffered a plague of
historical retrospectives. It was a new century; all that had gone
before was dead history. People felt it was important to document the
best parts of the past and forget the rest. Smith's church responded by
tracking down first-hand accounts of the church's early years, buying up
properties important to Mormon history, and cranking out quasi-scholarly
apologetics that smoothed over questionable aspects -- like polygamy --
of church doctrine. It reminded Martha of early Christendom's Councils
of Nicaea when Catholic leaders got together and established the
standard Biblical canon. Evidently, the inerrant, unalterable, and holy
word of God requires extensive copy editing. Catholics weren't the only
eager editors; Muslims indulged themselves three centuries later. The
\emph{Koran} was assembled from the sayings of Mohammed that nomadic
tribesmen had embossed on their saddles. One day, Mohammed's saddle
sayings were sorted into two big piles. One pile was burned, and the
other mutated into the \emph{Koran}. Mormon apologists were following
well-established religious precedents.

Martha knew all this, so she responded coyly when the meatier of the
process servers asked her, ``Did you know Antoine Legault?''

Many years ago, Monsieur Legault had warned her to keep his box away
from the Mormons. It was an odd, memorable warning, but Martha, looking
at the obsequious process servers before her, decided to take Monsieur
Legault's advice. And, knowing that the best way to bend the truth is to
tell only parts of it, Martha honestly testified that Mr.~Legault had
been one of her childhood teachers.

The meatier process server asked, ``Did Mr.~Legault ever give you
anything?''

Martha slyly deflected with, ``You mean other than school lessons?"

\hypertarget{antoine-1828-1}{%
\subsection*{Antoine -- 1828}\label{antoine-1828-1}}

Watching Smith's Harmony properties turned into a boring chore. Many of
Antoine's fellow gold diggers gave up watching Smith and spent most of
their time drinking, whoring, or running side digs. Antoine persisted.
Not because he thought he would recover treasure but because he was
falling under Smith's spell. It frankly annoyed him that Smith was doing
better than he was. Smith lived in a decent house, was married to an
honorable woman, and was prospering. How? After a few weeks of closely
watching people come and go, Antoine saw few opportunities to sneak onto
Smith's grounds and search for treasure. Smith's wife Emma, his acolyte
Oliver Cowdery, Smith himself, or all three were constantly present. He
had almost exhausted his patience when Emma stormed out of Smith's house
one afternoon with her father. They climbed onto a small wagon and
abruptly left. Antoine surmised that Emma might have learned of Smith's
dalliances with other women.

With Emma gone, Smith and Cowdery's routine changed. Most evenings,
Smith left the house and wasn't seen until the next day. Antoine
suspected he was seeing other women. Cowdery usually stayed on the
property, but both left the house on some nights. On one fortuitously
moonlit night, Smith hauled something wrapped in a dark cloth into a
small outbuilding beside his house. The outbuilding served as a stable
and storage shed. Antoine, concealed on the edge of Smith's property,
was too far away to tell what it was, but Smith had clearly carried
something into the shed and emerged empty-handed a few minutes later.
Antoine watched as Smith carefully looked around in the moonlight before
calling Cowdery, who was still in the house. Cowdery exited the house,
locked the door, and then, a few minutes later, both men left, leaving
the property unguarded.

Antoine was immediately anxious and fearful. This was his chance. If he
didn't act, it might be weeks before he would get another opportunity.
He crawled out from behind the cluster of bushes he had been hiding
behind and climbed over the wooden fence surrounding Smith's house. He
went straight to the shed. Cowdery had locked the house, but the shed
was open. There were no horses or other animals to make noise. The shed
was mostly empty except for a few large wooden barrels stacked against
the back wall. Antoine knew exactly where to start looking. Gold diggers
get a lot of practice hiding their fake treasures. Hiding something is
the inverse of looking for something. The ironic symmetry appealed to
Antoine's \emph{École Polytechnique-trained} mind. He opened the
hardest-to-reach barrel. It was filled with beans. He took notice of the
bean level in the barrel so he could reset the beans without leaving a
trace. Then, he started probing in the beans with his bare fingers.
About eight inches under the beans, he felt something like a cloth bag
filled with something flat and solid. Pushing the beans out of the way,
he exposed the wrapped object. Before unwrapping it, he noted how the
cloth was twisted so he could rewrap it without detection, and then he
unwrapped whatever Smith had hidden.

It wasn't what he or any of the gold diggers expected. It was a small,
heavy stack of what looked like thin metal plates. The plates were held
together by three long horseshoe-shaped pins that pierced each plate in
perfect circular holes. In the dim shed moonlight, Antoine could see the
plates were decorated with extremely regular and precise odd markings.
Without thinking, he lifted one of the plates. It was astonishingly thin
and incredibly smooth, and the markings extruded from both sides. The
plate, despite being extremely thin, did not bend or warp. It remained a
perfectly flat little Euclidean rectangle as it moved on the pins. Even
stranger, the metal did not warm to the touch. It seemed to absorb
whatever heat your fingers imparted. It wasn't like any metal Antoine
had ever seen. He estimated there were about three or four hundred
double-sided plates. About a third of the plates had been turned from
one side of the horseshoe pins to the other. Smith was going through the
plates. Antoine released the plate. It eerily moved without friction or
sound back into position. Antoine didn't know what the plates were, but
he had seen enough gold digger fakes to recognize that Smith had
something real. Then, in a rare fit of hubristic self-importance,
Antoine decided to decode the plates. Smith was probably trying to do
the same; surely, he could do better. Knowing he couldn't steal the
plates without attracting attention, he carefully wrapped and reburied
them in the beans. The next time he handled the plates, he would come
prepared.

\hypertarget{arthur-2024}{%
\subsection*{Arthur -- 2024}\label{arthur-2024}}

The next morning, before Arthur drove to his office, Dan
\emph{Starlinked} in again from the wilds of Canyonlands.

``You know I'm out here to image the sky,'' Dan said.

Dan had an expensive amateur astronomy habit that required getting away
from city lights. Canyonlands was the darkest, high elevation,
clear-skyed region within easy driving distance of Dan's Grand Junction
home. When he wasn't teasing his workmates with live scenic Zoom
backgrounds, he showed off gorgeous deep sky images he'd captured with
his \emph{Celestron} 11" Rowe-Ackermann Schmidt astrograph.

``Bad news, I blew off a perfect new moon to work on your etchings last
night. Good news, you'll love this! After loading your scans into Jones.
I asked if the marks were a human language. Jones said no. Then I asked
if the marks were an unknown human language. Jones said no again. So,
hail-Marying, I asked if the marks were structured, random, or resembled
anything in Jones's training data. This is what Jones came up with.''

Dan shared his screen with Arthur, and they both read through the AI's
synopsis. The etchings contained 191 distinct orthographic symbols. Even
more interesting, the mark size statistics indicated a higher degree of
uniformity than modern typefaces. Whatever the marks were, they weren't
hand-drawn or even printed. The precise alphabet was only the start of
the surprises. The etching language was more highly structured and
regular than any natural human language. Jones reported that it had
statistics like formal mathematical languages and closely resembled
fully elaborated terms in type theories.

``What the Hell?'' Arthur asked.

``Yeah, I didn't know what an elaborated term was either,'' Dan replied.
``But I know people.''

``I Facetimed an old buddy in the U of U Computer Science faculty, and
he pointed me to a Zulip Group of young mathematicians researching proof
assistants. I joined the group, uploaded a few etching images, and told
them Jones said the etchings were elaborated terms.''

``How did that go?'' Arthur asked.

``I was surprised as hell that they all knew Jones. Our language AI is
famous, Arthur! Anyway, they quickly confirmed the statistics. I had
Jones generate a custom Unicode font from the mark orthographies,
transcribe the entire etching dataset, and then upload it to the
Zulipers to hack. I hope that's OK. This can't fall under any bullshit
NDAs, can it?''

Arthur didn't know how to respond. Was somebody punking him? If he
hadn't picked Alice's etching box out of her belongings, he would have
suspected an elaborate hoax. Who would go to the trouble to get ahold of
two-hundred-year-old paper, invent an unknown alphabet, encode some
``elaborated terms'' -- still unclear what they are -- and then hide
everything in his deceased grandmother's Pulaski house?

``No, this isn't NDA shit; this is far weirder,'' Arthur replied.

Before signing off from Canyonlands, Dan said the Zulip group was all
worked up and had some ideas.

\hypertarget{martha-1937}{%
\subsection*{Martha -- 1937}\label{martha-1937}}

When Elsie stopped by to introduce Martha to her new baby granddaughter,
Alice, Martha thought that her mother would have been pleased. She had
always wanted Martha to grow up, get married, and lead a normal life. It
took her longer than most, but she did just that.

After years of laboring as a Harvard calculator in Henrietta Swan
Leavitt's stellar catalog group, Eric, an insurance actuary, tried to
recruit her. Boston insurance firms often approached Harvard
calculators; their arithmetic prowess was well known, and actuaries
always looked for people to crunch numbers. Martha didn't leave Harvard,
but she and Eric fell in love, married, and had one daughter, Elsie.
Martha was thirty-four when Elsie was born, and Eric was forty-nine.
They were old, competent parents who raised a dutiful and loving
daughter. Eric had died of lung cancer five years ago, and Martha,
getting old herself, had decided to sell their home and move into an
apartment close to Elsie and baby Alice.

While preparing to move, Martha came across Monsieur Legault's old box.
She had stored it in Eric's linen chest decades ago. This was one item
she was not going to sell or throw away. Monsieur Legault felt that the
etchings, still in the box, were important. As Elsie nursed Alice beside
her, Martha decided to give her the box. Elsie had always been dutiful
and conscientious. If Martha asked her to do something, she always did
it. A few days after meeting baby Alice, Martha handed Elsie the box,
told her what it meant, and then asked Elsie to pass it on to Alice one
day.

\hypertarget{antoine-1828-2}{%
\subsection*{Antoine -- 1828}\label{antoine-1828-2}}

It had been a month since Antoine had seen Smith's hidden plates. He
hadn't wasted the time. While keeping a close eye on Smith's house, he
got ready to copy the plates. He decided to rub the plates. He would
place a sheet of paper on the extruded marks, apply pressure with a flat
board, lift the sheet off, and then dust the indented paper with
heelball, a fine mixture of wax, and ground charcoal. It was the same
technique used to make woodcuts, and if the marks on the plates were as
hard as he hoped they were, it wouldn't take long to make impressions.

While he waited for another chance to handle the plates, he ground
charcoal with a mortar, sifted out the finest grains, and stored the
pitch-black powder in beer mugs. He also melted and filtered candle wax.
You need clean wax for heelball. It took a few tries to get the mixture
of charcoal and wax right, but in a few weeks, Antoine had four large
mugs ready. He tested his mixture on coins, leaves, and even dragonfly
wings. It was gratifying to press a dragonfly wing into paper, dust it
with heelball, blow it off, and then see the tiny wing veins perfectly
recorded.

On June 27, 1828, a full moon night, Antoine watched from his hiding
spot as Smith smuggled the plates into the outbuilding and emerged
without them. Like before, he called for Cowdery, who always remained in
the house. Smith was hiding the plates from Cowdery, too. Cowdery left
the house, locked the door, and both men left the grounds.

Antoine quickly snuck into the outbuilding and extracted the plates from
the bean barrel. Smith had arrogantly used the same barrel. Antoine laid
out the bundle he was carrying. It was filled with the old box Knoles
had given him, his mugs of heelball, a short flat board, and some
brushes. He spread the bundle blanket beside the plates to catch the
heelball that would spill off the paper as he dusted. When all was
ready, he started rubbing the plates.

It went far better than he had hoped. The marks on the plates were
astonishingly hard; they cut into the paper like little metal diamonds
and left beautiful marks that stood out after heelballing even in the
dim moonlight of the shed. Antoine quickly pressed and dusted plate
after plate. As he worked, he was again impressed by the rigidity of the
plates. The plates never bent or flexed, even when squeezed with his
pressure board. Whatever they were made of, it was perfect for rubbing.
His rubbing went so well that he ran out of paper before the plates were
exhausted, and about a quarter of the plates remained when he dusted his
last sheet. Antoine considered reusing some pages but quickly rejected
the idea. It would have to do. He packed all the etchings in Knoles's
box and rewrapped his bundle. Then, he delicately replaced the plates in
the dark cloth and reburied them in the beans.

Antoine felt triumphant when he snuck out of the shed with his bundle.
He didn't know where the plates came from, what they were made of, or
what the markings might mean, but he knew they were the most amazing
thing he had ever touched. Imagining the glory he'd reap after decoding
the plates, Antoine's attention lapsed just long enough to miss the lone
man walking toward Smith's house. When Antoine saw the man, he was sure
that the man had seen him. They were too far apart to recognize each
other in the full moonlight, but Antoine knew he was in trouble. He ran.
The man gave chase.

He jumped over the fence near his hiding spot and fled into the brush.
Antoine was familiar with the terrain from months of sneaking up on
Smith's house and managed to lose his follower, but he wasn't out of
trouble. Harmony was a small town. If Smith got word that somebody was
poking around on his property, he would immediately suspect his old
gold-digging accomplices. Antoine had to get away. Without waiting, he
hurried to a stable on the outskirts of town and roused the muleskinner.
The muleskinner wasn't thrilled with Antoine waking him and drove a hard
bargain when Antoine tried to purchase a mule. He eventually relented,
but Antoine had to trade away more than eighty of his newly etched
sheets. Only one side was etched; the back was perfect, valuable paper.
Antoine knew that when the muleskinner started using the sheets around
town, some would likely get back to Smith, alerting him that somebody
had traced his plates. Harmony was a small town; if Antoine disappeared,
it would be a small leap to implicate him. It wasn't ideal, but neither
was fleeing from Paris years ago.

\hypertarget{arthur-2024-4}{%
\subsection*{Arthur -- 2024}\label{arthur-2024-4}}

Since going down the charcoal etching rabbit hole, Arthur couldn't
concentrate on his day job. Unfortunately, his job wouldn't back off.
Clients were pressuring his team, and Jones went down unexpectedly.
Major software outages always reinforced Arthur's admiration of
biological brains; they can run without outages for a century. AIs need
to get more reliable before disposing of bipedal meat puppets.

Arthur hadn't followed up on the sheet he had sent for Carbon dating and
was surprised when the young lab tech he had talked to called. When
Gen-X'ers call, instead of texting or DM'ing, shit is getting super
real. The Carbon on the sheet could be dated from 1815 to 1845 with Six
Sigma confidence. Jesus! The Carbon was also two hundred years old.
Faking the etchings just got a lot harder. It would take an extremely
talented forger to make the sheets. Maybe somebody in the early 19th
century was into elaborated terms more than a century before they were
created.

The tech added, ``We had to destroy the sheet, but I scanned both sides
before testing. Did you know that the back side had some handwriting on
it? I just sent you an email with scans of both sides.''

Arthur had overlooked the backside text. He zoomed Dan to pass on the
Carbon dating results. Dan had been offline for a few days. There wasn't
much point in \emph{Starlinking} in with Jones down. Dan used the outage
to image the glorious dark skies of Canyonlands. He was rested and
relaxed when Arthur's Zoom came in. The news about the Carbon reawakened
his interest in the etchings.

``We should check with the Zulipers,'' Dan said. ``Here, I'll share my
screen.''

Within minutes, Dan and Arthur were in a big Zoom call with around
twenty young proof assistant researchers. Judging from the agitation and
excitement levels, you'd have sworn they were all in a burning building.
Everybody tried to talk at once, and, like every other damn Zoom Arthur
had ever attended, everyone had to mute their device microphones and
take turns. When all the little zoom-head boxes went quiet, one
zoom-head box labeled ``R. Yang'' spoke for the Zulipers.

``Before we get started,'' Arthur said, ``Do you mind if we dial in
Jones? We've found that Jones provides excellent summaries of large Zoom
meetings. And, people tend to stay on point when AIs record everything
they say or type.''

R. Yang agreed to Jones's inclusion, and within seconds, another
zoom-head box appeared on the screen. Not having an actual face or body,
Jones enjoyed filling his zoom-head square with AI art. Today, he was
going with an animated version of Sauron's Mordor tower, complete with
an ominous rotating orange eye. Very comforting. Who said AIs don't have
a sense of humor? In the side chat channels, some Zulipers welcomed
their new AI overlord, and others posted links to \emph{Wayne's World},
``We are not Worthy'' YouTubes. This is how AIs get their warped senses
of humor.

With Jones plugged in, the meeting began. R. Yang told Arthur, Dan, and
Jones that after looking at the etching data, some Zulipers guessed bits
might be encoded formal Peano arithmetic. They developed a rudimentary
mapping from the etching symbols to LEAN programming language terms. It
didn't take long to match the symbols to LEAN versions of basic number
properties, like commutativity, associativity, etc. Using this as a
basis; several abstract etching grammars were tried until about half of
the etching symbols could be translated as syntactically correct LEAN
terms. Pleased with their progress, the Zulipers spent the next three
days expanding their mapping into several thousand lines of Python that
reliably translated most of BOT to LEAN.

``BOT?'' Arthur asked.

R. Yang replied, ``Book of Terms after Erdős's Book.''

Paul Erdős was a peripatetic 20th-century Hungarian mathematical genius
who roamed the world with his mother. He was famously homeless and would
appear on colleagues' doorsteps and say, ``My brain is open.'' If Erdős
found a proof beautiful, deep, satisfying, and impossible to improve, he
would say it was from \emph{The Book}. \emph{The Book} is an imaginary
nerd heaven book containing the \emph{best} mathematical proofs. To
create a book-worthy proof is the goal of many mathematicians. Erdős
collaborated with so many mathematicians that they started sorting
themselves by Erdős number. Erdős himself is number 0. Mathematicians
that directly worked with Erdős get number 1. Mathematicians who worked
with a direct Erdős collaborator get number 2, and so on. An entire
Wikipedia page ranks world mathematicians by Erdős number, and the guy
has been dead for almost three decades! A low Erdős number is the
mathematical equivalent of driving around with a vanity license plate
that says, ``WELL HUNG.''

After developing a usable BOT to LEAN converter, the Zulipers attacked
the bulk of the BOT etching text. Preliminary token parsing indicated
the BOT text was incomplete. Big chunks were missing, but what remained
was astonishing. Some sequences corresponded to lists of basic group
properties; another encoded a recurrence relation that generated an
integer sequence that could not be found in OEIS (The Online
Encyclopedia of Integer Sequences), and, topping that, another BOT
statement asserted the infinitude of primes. The corresponding BOT term
did not match the LEAN equivalent, but it wasn't difficult to permute
both until a perfect match was obtained. At this point, the Zulipers
lost it because the BOT proof was a formal version of Euclid's famous
proof of the infinitude of primes. A proof that Erdős himself said was
straight from \emph{The Book}. Calling the etchings, \emph{The Book of
Terms} suddenly seemed serious.

The revelations continued. After finding Euclid's prime proof, they
attempted to decode as much BOT as possible. The Python translator
program couldn't handle everything. Still, they extracted one large
term, roughly half of the BOT text, that corresponded to a statement
that there is an infinite collection of pairs of primes that differ by
no more than five. This is a much tighter constraint than the best-known
result of 246. The Zulipers now suspected somebody was punking them
hard, but enjoying the hack, they converted the BOT to LEAN, and after
fixing some memory leaks in LEAN itself, the term checked out! It was a
valid formal proof of a new, deep, previously unknown theorem. They
still didn't believe it. Maybe LEAN was broken. They dumped the term in
check format and ran it through three independent validation programs.
The result held. \emph{Formal proof cannot be faked!} That's pretty much
the whole point of formal proof.

R. Yang then asked Arthur and Dan who proved the new theorem. About half
the Zulipers already had draft papers ready to upload to the arxiv.org
preprint server, but they needed to credit this new result. Who did
this?

Unmuting his Zoom microphone, Arthur told the Zulipers, ``I don't know.
I found The Book,'' why not call it `The Book' he thought to himself,''
in my grandmother's house.''

``Was your grandmother a number theorist?'' a Zuliper typed into the
chat window.

Dan unmuted, ``Guys, guys, this a lot to digest. Can you give us some
time to absorb this? We'll get back to you.''

The Zulipers dropped off the Zoom call until only Arthur, Dan, and Jones
were left.

``What the Hell,'' Arthur exclaimed.

``Did you find this in your grandmother's house?''

``Dan, do you think I could have faked this? It's two hundred years old,
man! I failed my second differential equation course. There is no way in
hell I would ever come up with a new (finger air quotes) deep theorem.''

Dan wasn't used to Arthur acting out, but he agreed. He asked, ``Is
there anything else? The guys on the call think parts are missing.''

``I scanned all the pages. They have everything I found.'' Arthur
replied. Then he remembered the backside of the sheet he'd sent for
Carbon dating.

``There's a tiny bit more. There was a postscript or something on the
last page.'' He looked up the email from the lab tech and forwarded the
image attachments to Dan.

``Dan, I gotta deal with the Microsoft mess. They want their pipelines
reset, and Legal is giving me shit about forwarding the etchings. Who
knew that crap in my grandmother's house would run afoul of company
NDAs. God, the world would be so much better off if all the suits just
fucked off and died.''

``We can hope.'' Dan chimed as Arthur left the call.

\hypertarget{dan-and-jones-2024}{%
\subsection*{Dan and Jones -- 2024}\label{dan-and-jones-2024}}

Dan enlarged Arthur's last-page image attachment. The small postscript
was written in cursive with an elegant hand. It wasn't BOT; it used a
Latin alphabet. Dan couldn't read the scrambled text, so he passed it to
Jones and asked the AI to analyze the script.

Jones typed, ``The text is most probably Caesar ciphered French.''

Dan typed into the chatbot window, ``What does it say?''

Jones instantly translated, ``Rubbed from bound plates found in Joseph
Smith's shed. Harmony, June 27, 1828. A. Legault.''

Dan didn't know how Arthur pulled this off, but it couldn't be real. Dan
was a lapsed Mormon; he had left the one true church shortly after
rejecting the very idea that there could be a ``one true church.'' Like
all Mormons and many uncouth infidels, he knew all about the Golden
Plates. The tale was pure Red Sea parting, Christ rising from the grave,
and Mohammed beaming up from the Dome of the Rock, bullshit. It's a
great story, but come on! How can we be expected to take such sky-fairy
nonsense seriously? If only the Golden Plates hadn't conveniently
disappeared. Maybe a bigfoot piloting a UFO took them. Dan completely
understood why people wanted to move to Mars. Planet moron is exhausting
for hard-ass skeptics. Believe nothing and verify everything.

Dan was a hard apostate, but unlike many of his peers, he \emph{had}
seriously considered what a \emph{credible divine revelation} might be.
Old Testament bromides about not coveting and buggering your neighbors'
goats wouldn't cut it. Do you know what might? How about a complete
Periodic Table with accurate transuranium element masses? Find that
carved into a Bronze Age stela; even hard-ass skeptics might pay
attention. Maybe valid, verifiable formal proofs were even better.
\emph{Formal proof cannot be faked.} You may question the source of the
``revelation,'' but you cannot dismiss the content.

Annoyed and amused, Dan asked Jones, ``What do you think?''

Jones politely typed, ``Perhaps the Angel Moroni attempted to reveal
Erdős's book of perfect mathematical proofs to Joseph Smith, but Smith
didn't understand it and made up \emph{The Book of Mormon} instead.''



%\end{document}
 

