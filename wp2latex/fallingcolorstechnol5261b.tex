% uncomment to list all files in log
%\listfiles

\documentclass[12pt]{report}

\usepackage{fontspec}

%\setmainfont[Scale=MatchLowercase]{Lucida Bright}
%\setmonofont{FreeMono}
%\setmonofont{Source Code Pro}
\setmonofont[Scale=MatchLowercase]{Ubuntu Mono}

\usepackage[headings]{fullpage}

% national use characters 
%\usepackage{inputenc}

% ams mathematical symbols
\usepackage{amsmath,amssymb}

% added to support pandoc highlighting
\usepackage{microtype}

\usepackage{makeidx}

% add index and bibliographies to table of contents
\usepackage[nottoc]{tocbibind}

% postscript courier and times in place of cm fonts
%\usepackage{courier}
%\usepackage{times}

% extended coloring
\usepackage{color}
\usepackage[table,dvipsnames]{xcolor}
\usepackage{colortbl}

% advanced date formating
\usepackage{datetime}

%support pandoc code highlighting
\usepackage{fancyvrb}
\DefineShortVerb[commandchars=\\\{\}]{\|}
\DefineVerbatimEnvironment{Highlighting}{Verbatim}{commandchars=\\\{\}}
% Add ',fontsize=\small' for more characters per line

%tango style colors
% \usepackage{framed}
% \definecolor{shadecolor}{RGB}{255,255,255}
% \newenvironment{Shaded}{\begin{snugshade}}{\end{snugshade}}
% \newcommand{\KeywordTok}[1]{\textcolor[rgb]{0.13,0.29,0.53}{\textbf{{#1}}}}
% \newcommand{\DataTypeTok}[1]{\textcolor[rgb]{0.13,0.29,0.53}{{#1}}}
% \newcommand{\DecValTok}[1]{\textcolor[rgb]{0.00,0.00,0.81}{{#1}}}
% \newcommand{\BaseNTok}[1]{\textcolor[rgb]{0.00,0.00,0.81}{{#1}}}
% \newcommand{\FloatTok}[1]{\textcolor[rgb]{0.00,0.00,0.81}{{#1}}}
% \newcommand{\CharTok}[1]{\textcolor[rgb]{0.31,0.60,0.02}{{#1}}}
% \newcommand{\StringTok}[1]{\textcolor[rgb]{0.31,0.60,0.02}{{#1}}}
% \newcommand{\CommentTok}[1]{\textcolor[rgb]{0.56,0.35,0.01}{\textit{{#1}}}}
% \newcommand{\OtherTok}[1]{\textcolor[rgb]{0.56,0.35,0.01}{{#1}}}
% \newcommand{\AlertTok}[1]{\textcolor[rgb]{0.94,0.16,0.16}{{#1}}}
% \newcommand{\FunctionTok}[1]{\textcolor[rgb]{0.00,0.00,0.00}{{#1}}}
% \newcommand{\RegionMarkerTok}[1]{{#1}}
% \newcommand{\ErrorTok}[1]{\textbf{{#1}}}
% \newcommand{\NormalTok}[1]{{#1}}

%espresso style colors
% \usepackage{framed}
% \definecolor{shadecolor}{RGB}{42,33,28}
% \newenvironment{Shaded}{\begin{snugshade}}{\end{snugshade}}
% \newcommand{\KeywordTok}[1]{\textcolor[rgb]{0.26,0.66,0.93}{\textbf{{#1}}}}
% \newcommand{\DataTypeTok}[1]{\textcolor[rgb]{0.74,0.68,0.62}{\underline{{#1}}}}
% \newcommand{\DecValTok}[1]{\textcolor[rgb]{0.27,0.67,0.26}{{#1}}}
% \newcommand{\BaseNTok}[1]{\textcolor[rgb]{0.27,0.67,0.26}{{#1}}}
% \newcommand{\FloatTok}[1]{\textcolor[rgb]{0.27,0.67,0.26}{{#1}}}
% \newcommand{\CharTok}[1]{\textcolor[rgb]{0.02,0.61,0.04}{{#1}}}
% \newcommand{\StringTok}[1]{\textcolor[rgb]{0.02,0.61,0.04}{{#1}}}
% \newcommand{\CommentTok}[1]{\textcolor[rgb]{0.00,0.40,1.00}{\textit{{#1}}}}
% \newcommand{\OtherTok}[1]{\textcolor[rgb]{0.74,0.68,0.62}{{#1}}}
% \newcommand{\AlertTok}[1]{\textcolor[rgb]{1.00,1.00,0.00}{{#1}}}
% \newcommand{\FunctionTok}[1]{\textcolor[rgb]{1.00,0.58,0.35}{\textbf{{#1}}}}
% \newcommand{\RegionMarkerTok}[1]{\textcolor[rgb]{0.74,0.68,0.62}{{#1}}}
% \newcommand{\ErrorTok}[1]{\textcolor[rgb]{0.74,0.68,0.62}{\textbf{{#1}}}}
% \newcommand{\NormalTok}[1]{\textcolor[rgb]{0.74,0.68,0.62}{{#1}}}

%kete style colors
% \newenvironment{Shaded}{}{}
% \newcommand{\KeywordTok}[1]{\textbf{{#1}}}
% \newcommand{\DataTypeTok}[1]{\textcolor[rgb]{0.50,0.00,0.00}{{#1}}}
% \newcommand{\DecValTok}[1]{\textcolor[rgb]{0.00,0.00,1.00}{{#1}}}
% \newcommand{\BaseNTok}[1]{\textcolor[rgb]{0.00,0.00,1.00}{{#1}}}
% \newcommand{\FloatTok}[1]{\textcolor[rgb]{0.50,0.00,0.50}{{#1}}}
% \newcommand{\CharTok}[1]{\textcolor[rgb]{1.00,0.00,1.00}{{#1}}}
% \newcommand{\StringTok}[1]{\textcolor[rgb]{0.87,0.00,0.00}{{#1}}}
% \newcommand{\CommentTok}[1]{\textcolor[rgb]{0.50,0.50,0.50}{\textit{{#1}}}}
% \newcommand{\OtherTok}[1]{{#1}}
% \newcommand{\AlertTok}[1]{\textcolor[rgb]{0.00,1.00,0.00}{\textbf{{#1}}}}
% \newcommand{\FunctionTok}[1]{\textcolor[rgb]{0.00,0.00,0.50}{{#1}}}
% \newcommand{\RegionMarkerTok}[1]{{#1}}
% \newcommand{\ErrorTok}[1]{\textcolor[rgb]{1.00,0.00,0.00}{\textbf{{#1}}}}
% \newcommand{\NormalTok}[1]{{#1}}
%end pandoc code hacks

% jodliterate colors
\usepackage{color}
\definecolor{shadecolor}{RGB}{248,248,248}
% j control structures 
\definecolor{keywcolor}{rgb}{0.13,0.29,0.53}
% j explicit arguments x y m n u v
\definecolor{datacolor}{rgb}{0.13,0.29,0.53}
% j numbers - all types see j.xml
\definecolor{decvcolor}{rgb}{0.00,0.00,0.81}
\definecolor{basencolor}{rgb}{0.00,0.00,0.81}
\definecolor{floatcolor}{rgb}{0.00,0.00,0.81}
% j local assignments
\definecolor{charcolor}{rgb}{0.31,0.60,0.02}
\definecolor{stringcolor}{rgb}{0.31,0.60,0.02}
\definecolor{commentcolor}{rgb}{0.56,0.35,0.01}
% primitive adverbs and conjunctions
%\definecolor{othercolor}{rgb}{0.56,0.35,0.01}   
\definecolor{othercolor}{RGB}{0,0,255}
% global assignments
\definecolor{alertcolor}{rgb}{0.94,0.16,0.16}
% primitive J verbs and noun names
\definecolor{funccolor}{rgb}{0.00,0.00,0.00}    

\usepackage{framed}
\newenvironment{Shaded}{}{}
\newcommand{\KeywordTok}[1]{\textcolor{keywcolor}{\textbf{{#1}}}}
\newcommand{\DataTypeTok}[1]{\textcolor{datacolor}{{#1}}}
%\newcommand{\DecValTok}[1]{\textcolor{decvcolor}{{#1}}}
\newcommand{\DecValTok}[1]{{#1}} 
\newcommand{\BaseNTok}[1]{\textcolor{basencolor}{{#1}}}
\newcommand{\FloatTok}[1]{\textcolor{floatcolor}{{#1}}}
\newcommand{\CharTok}[1]{\textcolor{charcolor}{\textbf{{#1}}}}
\newcommand{\StringTok}[1]{\textcolor{stringcolor}{{#1}}}
\newcommand{\CommentTok}[1]{\textcolor{commentcolor}{\textit{{#1}}}}
\newcommand{\OtherTok}[1]{\textcolor{othercolor}{{#1}}} 
\newcommand{\AlertTok}[1]{\textcolor{alertcolor}{\textbf{{#1}}}}
%\newcommand{\FunctionTok}[1]{\textcolor{funccolor}{{#1}}}
\newcommand{\FunctionTok}[1]{{#1}}
\newcommand{\RegionMarkerTok}[1]{{#1}}
\newcommand{\ErrorTok}[1]{\textbf{{#1}}}
\newcommand{\NormalTok}[1]{{#1}}

% headers and footers
\usepackage{fancyhdr}
\pagestyle{fancy}

\fancyhead{}
\fancyfoot{}

%\fancyhead[LE,RO]{\slshape \rightmark}
%\fancyhead[LO,RE]{\slshape \leftmark}
\fancyfoot[C]{\thepage}
%\headrulewidth 0.4pt
%\footrulewidth 0 pt

%\addtolength{\headheight}{\baselineskip}

%\lfoot{\emph{Analyze the Data not the Drivel}}
%\rfoot{\emph{\today}}

% subfigure handles figures that contain subfigures
%\usepackage{color,graphicx,subfigure,sidecap}
\usepackage{graphicx,sidecap}
\usepackage{subfigure}
\graphicspath{{./inclusions/}}

% floatflt provides for text wrapping around small figures and tables
\usepackage{floatflt}

% tweak caption formats 
\usepackage{caption} 
\usepackage{sidecap}
%\usepackage{subcaption} % not compatible with subfigure

\usepackage{rotating} % flip tables sideways

% complex footnotes
%\usepackage{bigfoot}

% weird logos \XeLaTeX
\usepackage{metalogo}

% source code listings
\usepackage{listings}

% long tables
% \usepackage{longtable}

\newcommand{\HRule}{\rule{\linewidth}{0.5mm}}

% map LaTeX cross references into PDF cross references
\usepackage[
            %dvips,
            colorlinks,
            linkcolor=blue,
            citecolor=blue,
            urlcolor=blue,   % magenta, cyan default        
            pdfauthor={John D. Baker},
            pdftitle={Analyze the Data not the Drivel},
            pdfsubject={Blog},
            pdfcreator={MikTeX+LaTeXe with hyperref package},
            pdfkeywords={blog,wordpress},
            ]{hyperref}
           
% custom colors
\definecolor{CodeBackGround}{cmyk}{0.0,0.0,0,0.05}    % light gray
\definecolor{CodeComment}{rgb}{0,0.50,0.00}           % dark green {0,0.45,0.08}
\definecolor{TableStripes}{gray}{0.9}                 % odd/even background in tables

\lstdefinelanguage{bat}
{morekeywords={echo,title,pushd,popd,setlocal,endlocal,off,if,not,exist,set,goto,pause},
sensitive=True,
morecomment=[l]{rem}
}

\lstdefinelanguage{jdoc}
{
morekeywords={},
otherkeywords={assert.,break.,continue.,for.,do.,if.,else.,elseif.,return.,select.,end.
,while.,whilst.,throw.,catch.,catchd.,catcht.,try.,case.,fcase.},
sensitive=True,
morecomment=[l]{NB.},
morestring=[b]',
morestring=[d]',
}

% latex size ordering - can never remember it
% \tiny
% \scriptsize
% \footnotesize
% \small
% \normalsize
% \large
% \Large
% \LARGE
% \huge
% \Huge
 
% listings package settings  
\lstset{%
  language=jdoc,                                % j document settings
  basicstyle=\ttfamily\footnotesize,            
  keywordstyle=\bfseries\color{keywcolor}\footnotesize,
  identifierstyle=\color{black},
  commentstyle=\slshape\color{CodeComment},     % colored slanted comments
  stringstyle=\color{red}\ttfamily,
  showstringspaces=false,                       
  %backgroundcolor=\color{CodeBackGround},       
  frame=single,                                
  framesep=1pt,                                 
  framerule=0.8pt,                             
  rulecolor=\color{CodeBackGround},   
  showspaces=false,
  %columns=fullflexible,
  %numbers=left,
  %numberstyle=\footnotesize,
  %numbersep=9pt,
  tabsize=2,
  showtabs=false,
  captionpos=b
  breaklines=true,                              
  breakindent=5pt                              
}

\lstdefinelanguage{JavaScript}{
  keywords={typeof, new, true, false, catch, function, return, null, catch, switch, var, if, in, while, do, else, case, break},
  ndkeywords={class, export, boolean, throw, implements, import, this},
  ndkeywordstyle=\color{darkgray}\bfseries,
  sensitive=false,
  comment=[l]{//},
  morecomment=[s]{/*}{*/},
  morestring=[b]',
  morestring=[b]"
}

% C# settings
\lstdefinestyle{sharpc}{
language=[Sharp]C,
basicstyle=\ttfamily\scriptsize, 
keywordstyle=\bfseries\color{keywcolor}\scriptsize,
framerule=0pt
}

% for source code listing longer than two use smaller font
\lstdefinestyle{smallersource}{
basicstyle=\ttfamily\scriptsize, 
keywordstyle=\bfseries\color{keywcolor}\scriptsize,
framerule=0pt
}

\lstdefinestyle{resetdefaults}{
language=jdoc,
basicstyle=\ttfamily\footnotesize,  
keywordstyle=\bfseries\color{keywcolor}\footnotesize,                                                               
framerule=0.8pt 
}

% APL UTF8 code points listed for lstlisting processing
\makeatletter
\lst@InputCatcodes
\def\lst@DefEC{%
 \lst@CCECUse \lst@ProcessLetter
  ^^80^^81^^82^^83^^84^^85^^86^^87^^88^^89^^8a^^8b^^8c^^8d^^8e^^8f%
  ^^90^^91^^92^^93^^94^^95^^96^^97^^98^^99^^9a^^9b^^9c^^9d^^9e^^9f%
  ^^a0^^a1^^a2^^a3^^a4^^a5^^a6^^a7^^a8^^a9^^aa^^ab^^ac^^ad^^ae^^af%
  ^^b0^^b1^^b2^^b3^^b4^^b5^^b6^^b7^^b8^^b9^^ba^^bb^^bc^^bd^^be^^bf%
  ^^c0^^c1^^c2^^c3^^c4^^c5^^c6^^c7^^c8^^c9^^ca^^cb^^cc^^cd^^ce^^cf%
  ^^d0^^d1^^d2^^d3^^d4^^d5^^d6^^d7^^d8^^d9^^da^^db^^dc^^dd^^de^^df%
  ^^e0^^e1^^e2^^e3^^e4^^e5^^e6^^e7^^e8^^e9^^ea^^eb^^ec^^ed^^ee^^ef%
  ^^f0^^f1^^f2^^f3^^f4^^f5^^f6^^f7^^f8^^f9^^fa^^fb^^fc^^fd^^fe^^ff%
  ^^^^20ac^^^^0153^^^^0152%
  ^^^^20a7^^^^2190^^^^2191^^^^2192^^^^2193^^^^2206^^^^2207^^^^220a%
  ^^^^2218^^^^2228^^^^2229^^^^222a^^^^2235^^^^223c^^^^2260^^^^2261%
  ^^^^2262^^^^2264^^^^2265^^^^2282^^^^2283^^^^2296^^^^22a2^^^^22a3%
  ^^^^22a4^^^^22a5^^^^22c4^^^^2308^^^^230a^^^^2336^^^^2337^^^^2339%
  ^^^^233b^^^^233d^^^^233f^^^^2340^^^^2342^^^^2347^^^^2348^^^^2349%
  ^^^^234b^^^^234e^^^^2350^^^^2352^^^^2355^^^^2357^^^^2359^^^^235d%
  ^^^^235e^^^^235f^^^^2361^^^^2362^^^^2363^^^^2364^^^^2365^^^^2368%
  ^^^^236a^^^^236b^^^^236c^^^^2371^^^^2372^^^^2373^^^^2374^^^^2375%
  ^^^^2377^^^^2378^^^^237a^^^^2395^^^^25af^^^^25ca^^^^25cb%  
  ^^00}
\lst@RestoreCatcodes
\makeatother

% custom lengths used within minipages
\newcommand{\minindent}{17pt}


\makeindex

\begin{document}

\subsection*{\href{https://analyzethedatanotthedrivel.org/2016/07/05/falling-colors-technology-a-bhsd-crony-that-needs-competition/}{Falling Colors Technology a BHSD Crony that needs Competition}}
\addcontentsline{toc}{subsection}{Falling Colors Technology a BHSD Crony that needs Competition}


\noindent\emph{Posted: 05 Jul 2016 20:13:36}
\vspace{6pt}

BHSD
(\href{http://www.hsd.state.nm.us/Behavioral_health_services_division.aspx}{Behavioral
Health Services Division}) is a New Mexico state agency that doles out
federal and state funds to a variety of small, ostensibly health
related, programs. For example, in the state of New Mexico, BHSD runs a
program called
\href{http://www.samhsa.gov/synar/about}{Synar}\protect\hyperlink{fn1}{\textsuperscript{1}}
that attempts to cut down on merchants selling cigarettes to minors. One
Falling Colors employee characterized the program as ``mostly stick and
no carrots.'' Synar funds a stable of ambush inspectors that descend on
merchants hoping to catch them selling to minors. It's a standard bit of
well-intentioned government coercion. If you are wondering what's in it
for the merchants stop wondering: it's all stick. They lose sales and
face fines. If you are wondering why the state of New Mexico supports
Synar follow the money. Depending on dubious
statistics\protect\hyperlink{fn2}{\textsuperscript{2}} compiled by Synar
administrators the state could lose millions of dollars of federal
grants if the percentage of offending merchants exceeds an arbitrary
threshold. Synar, in the twisted minds of state bureaucrats, ``generates
revenue.''

In addition to saving the state from the unconscionable scourge of
teenage smoking BHSD also funds a mishmash of programs to prevent drug
addiction, help the mentally ill, subsidize methadone treatments and
reimburse psychologists, psychiatrists and other health professionals
for counseling and other services. BHSD's budget for all these
operations is, according to Mindy Hale, roughly fifty million dollars
per year. In the greater wasteful schemes of government this is small
beans, even for New Mexico, but, it's still fifty million public dollars
so it's not out-of-bounds to ask, what are the taxpayers getting for
their money?

If you are naïve enough to think the intended \emph{clients} of BHSD's
largesse, the teenage smokers, the mentally ill, and the drug addicts,
garner the lion's share of that fifty million dollars you're probably a
statist or a moron, but
\href{http://www.twainquotes.com/Congress.html}{I repeat myself}. Many
years ago a wise old wag, when badgered about the high cost of landing a
man on the moon, chirped, ``None of that money was spent on the moon!''
While some of BHSD's fifty million is directed to clients, \emph{the
moon}, the lion's share goes to contractors, service providers, and BHSD
internals. Whether the state of New Mexico and the federal government
are getting good value for their money is debatable; what's not
debatable is that some IT service providers are doing very well from
themselves.

Two IT providers consume a significant share of BHSD IT funds:
\href{https://www.optumhealthnewmexico.com/}{Optum New Mexico} and
\href{https://www.linkedin.com/company/falling-colors-technology}{Falling
Colors Technology}. The founders of Falling Colors, Mindy Hale and
Pamela Koster\protect\hyperlink{fn3}{\textsuperscript{3}}, claim Optum
bills the state of New Mexico roughly four million dollars per year for
the onerous job of cutting checks. It's important to understand that
Optum is not dispensing their own money. They are simply managing a pool
of funds that are replenished by state and federal tax dollars. Yes, it
takes money to manage money. You have to pay auditors, comptrollers, and
other financial professionals to make sure the funds are not redirected
into questionable pockets. Surely you don't think New Mexico's
corruption free government would abscond with unwatched dollars?


%{[}caption id=``attachment\_5264'' align=``alignright''  width=``200''{]}
\begin{SCfigure}[30]
 \centering
  %\includegraphics[width=0.30\textwidth]{fctlogo.png}
  \includegraphics[width=1.25in,height=1.25in]{fctlogo.png}
  \caption{The Falling Colors Technology Logo. This logo was designed by a
competent graphic designer. I've observed an inverse relationship
between the quality of company logos and the products and services they
offer. Usually the better the logo the worse the
offerings.}
  \label{fig:5261X1}
\end{SCfigure}
%\href{http://analyzethedatanotthedrivel.org/fctlogo/}{\includegraphics[width=2.08333in,height=2.08333in]{fctlogo.png}}
%The Falling Colors Technology Logo. This logo was designed by a
%competent graphic designer. I've observed an inverse relationship
%between the quality of company logos and the products and services they
%offer. Usually the better the logo the worse the
%offerings.
%{[}/caption{]}


Still, four million seems a bit steep for providing a routine service
that any experienced financial entity like a bank could do, and to the
state's credit, they have recognized this are in the process of
renegotiating Optum's four million dollar fee. Optum has responded with
a ``this isn't worth the damn hassle'' attitude. If they cannot get
their four million they're threatening to pull out of the state and cede
the check cutting business to others. How much of this is hardball
negotiating, corporate whining, or even \emph{the truth}, is hard to
determine. The only thing that seems certain is that there is a business
opportunity for an IT provider if Optum makes good on their threat and
leaves New Mexico.

Falling Colors Technology, a little company that is already extracting
about one million dollars per year from the state, is angling to take
over Optum's fund disbursement role. \emph{In standard insider crony
fashion, they hope to keep this transfer quiet and elude potential
competitors.} Why go through all that messy inefficient public bidding?
There's only one problem with their \emph{business plan}. \emph{Falling
Colors has absolutely no experience managing funds.} There is nobody on
their staff that could be considered a financial professional. They are
planning to hire staff, but I have to wonder why BHSD, and the state of
New Mexico, are considering flushing millions of dollars through an
entity that has no financial expertise and has already received a formal
letter of warning for shoddy IT work.

Instead of branching out into lines of business that they have no
experience with Falling Colors efforts would be better invested in
fixing their core problems and they have lots of core problems. Let's
look at what nearly one million dollars or public funds per year buys
from Falling Colors Technology.

Your one million is buying a few unreliable, crash prone, insecure, low
volume websites geared towards BHSD staff and service providers. When I
first ran the following SQL query on the database that backs many
Falling Colors websites I was alarmed at the results.

\small
\begin{Shaded}
\begin{Highlighting}[]
\KeywordTok{SELECT}  \NormalTok{iq.WeekNumber ,}
        \FunctionTok{AVG}\NormalTok{(iq.ErrorCount) }\KeywordTok{AS} \NormalTok{AvgWeekErrors ,}
        \FunctionTok{MIN}\NormalTok{(iq.ErrorCount) MinWeekErrors ,}
        \FunctionTok{MAX}\NormalTok{(iq.ErrorCount) }\KeywordTok{AS} \NormalTok{MaxWeekErrors ,}
        \NormalTok{STDEV(iq.ErrorCount) }\KeywordTok{AS} \NormalTok{StdDevWeekErrors}
\KeywordTok{FROM}    \NormalTok{( }\KeywordTok{SELECT}    \FunctionTok{CAST}\NormalTok{(}\FunctionTok{CONVERT}\NormalTok{(}\DataTypeTok{VARCHAR}\NormalTok{(}\DecValTok{8}\NormalTok{), TimeUtc, }\DecValTok{112}\NormalTok{) }\KeywordTok{AS} \DataTypeTok{INTEGER}\NormalTok{) }\KeywordTok{AS} \NormalTok{DayNumber ,}
                    \FunctionTok{COUNT}\NormalTok{(}\DecValTok{1}\NormalTok{) }\KeywordTok{AS} \NormalTok{ErrorCount ,}
                    \FunctionTok{MIN}\NormalTok{(DATEPART(iso_week, TimeUtc)) }\KeywordTok{AS} \NormalTok{WeekNumber}
          \KeywordTok{FROM}      \NormalTok{dbo.ELMAH_Error}
          \KeywordTok{GROUP} \KeywordTok{BY}  \FunctionTok{CAST}\NormalTok{(}\FunctionTok{CONVERT}\NormalTok{(}\DataTypeTok{VARCHAR}\NormalTok{(}\DecValTok{8}\NormalTok{), TimeUtc, }\DecValTok{112}\NormalTok{) }\KeywordTok{AS} \DataTypeTok{INTEGER}\NormalTok{)}
        \NormalTok{) iq}
\KeywordTok{GROUP} \KeywordTok{BY} \NormalTok{iq.WeekNumber}
\end{Highlighting}
\end{Shaded}
\normalsize

Falling Colors websites were crashing about twenty times per day. On
some days the crash count exceeded fifty. I thought to myself, ``If this
doesn't dramatically improve this little company is doomed.'' I've
worked with lots of bug infested software over my long career but twenty
to fifty crashes per day, distributed over a few dozen users, was an
entirely new level of unreliability.

Why is it so bad? The developers at Falling Colors, like developers
everywhere, bitched about ``inherited code.'' Basically, this means
they're working with code that they didn't entirely write themselves.
Developers complaining about inherited code is so common that software
managers rightly label it whining. Software developers bitching about
inherited code is like civil engineers griping about inherited bridges.
The world is not created fresh every day. The inherited code base is a
source of problems but the main reason Falling Colors exhibits such a
high crash rate is simply a lack of formal quality control.

Testing at Falling Colors is mostly performed by one beleaguered
Business Analyst. She runs through a series of basic web page checks
after significant new releases. This is a very low standard of testing
for modern software development. Falling Colors does not practice many
common quality control techniques. For example, most development
environments support a variety of internal testing tools. Falling Colors
is a Visual Studio shop and Visual Studio has built-in unit testing
tools and supports a host of third-party add-ons. Developers focused on
quality, spend as much time implementing internal units test as they do
writing production code. There is an entire coding regime known as
\href{http://agiledata.org/essays/tdd.html}{TDD} that strongly promotes
writing tests before you write software to pass the tests. At the end of
June 2016, there were precisely zero internal unit tests in Falling
Color's code base. In addition to missing internal unit tests, there
were no repeatable or scripted tests, no large case tests, and no stress
tests. Lack of formal testing combined with misplaced developer optimism
is a recipe for high error rates and Falling Colors is really boiling
that pot.

Buggy insecure low volume websites are a dime a dozen. There's a lot of
crap out there. If Falling Colors cranked out standard public websites
we would \emph{click on} and ignore their rubbish. Unfortunately, being
intertwined with BHSD, the users of Falling Colors websites do not have
the option of \emph{clicking on}. Making things worse, Falling Colors
hosts a substantial amount of
\href{http://www.hhs.gov/hipaa/for-professionals/privacy/}{HIPAA}
protected information.

HIPAA is a set of federal guidelines that outline how health providers
and their contractors must protect information that might be used to
uniquely identify people. HIPAA penalties, for both providers and
individuals, are severe if protected information is either accidentally
or willfully disclosed. You can go to jail for exposing HIPAA protected
information.

HIPAA guidelines
\href{https://en.wikipedia.org/wiki/Protected_health_information}{list
common data elements} that must be protected. There is only one way to
properly protect these elements: \emph{full element encryption.} Every
single data element should be encrypted and the keys should be
rigorously guarded by a small number of individuals. Even developers,
\emph{especially developers}, should never see the unencrypted
information. This is the way things should work, but, if you have
followed the news about an unending stream of website hacks and data
breaches, you're probably aware that this is not how it works in the big
nasty world.

It's certainly not the way things are working at Falling Colors. With
the exception of website passwords, which were only hashed in the last
year,\protect\hyperlink{fn4}{\textsuperscript{4}} HIPAA data is stored
in plain, ready to hack, text. If I were an IT savvy methadone user in
the state of New Mexico I would be reluctant to disclose personal
information to \emph{CareLink}, \emph{TreatFirst}, \emph{Prevention}, or
any of the Falling Colors managed programs. One HIPAA breach and your
methadone habit is on Facebook.

Falling Colors is fully cognizant of their shabby security and are
planning to \emph{eventually} fix it. They're taking steps to harden
their websites and tighten up their loose databases but they are not, as
of the end of June 2016, pursuing a full element encryption regime.
Anything short of full element encryption is just putting lipstick on
the security pig. Currently, Falling Colors is a HIPAA breach in
waiting. \emph{BHSD would be well advised to insist on an immediate and
independent full security audit of Falling Colors systems!}

\emph{BHSD should also demand a fair and public RFP (Request for
Proposal) process when seeking IT contracting services.} Currently, some
individuals in BHSD, in connivance with Falling Colors, are delicately
crafting RFPs that are designed to exclude Falling Colors competitors.
This is a blatant abuse of the public RFP process and the perpetrators
should be ashamed of themselves. Crony state contracting may be business
as usual in New Mexico but it is not in the interests of the pubic,
BHSD, or even Falling Colors. Cronies without competition invariably
turn into parasites and BHSD, which recently suffered a bedbug outbreak
in their Santa Fe offices, has enough of those.

\begin{center}\rule{0.5\linewidth}{\linethickness}\end{center}

\begin{enumerate}
\item
  \hypertarget{fn1}{}

  The Synar program is named after Congressman Mike Synar of Oklahoma.
  How many tax dollars would be saved if it was illegal to name things
  after politicians?\protect\hyperlink{fnref1}{↩}
\item
  \hypertarget{fn2}{}

  File a Freedom of Information~request~to check raw program
  data.\protect\hyperlink{fnref2}{↩}
\item
  \hypertarget{fn3}{}

  The founders of Falling Colors are questionable sources; their claims
  should be subjected to a high standard of
  scrutiny.\protect\hyperlink{fnref3}{↩}
\item
  \hypertarget{fn4}{}

  Yes, incredibly, user passwords were stored as~plain text for years.
  This is monumentally inept!\protect\hyperlink{fnref4}{↩}
\end{enumerate}



\end{document}