%% uncomment to list all files in log
%\listfiles

\documentclass[12pt]{report}

\usepackage{fontspec}

%\setmainfont[Scale=MatchLowercase]{Lucida Bright}
%\setmonofont{FreeMono}
%\setmonofont{Source Code Pro}
\setmonofont[Scale=MatchLowercase]{Ubuntu Mono}

\usepackage[headings]{fullpage}

% national use characters 
%\usepackage{inputenc}

% ams mathematical symbols
\usepackage{amsmath,amssymb}

% added to support pandoc highlighting
\usepackage{microtype}

\usepackage{makeidx}

% add index and bibliographies to table of contents
\usepackage[nottoc]{tocbibind}

% postscript courier and times in place of cm fonts
%\usepackage{courier}
%\usepackage{times}

% extended coloring
\usepackage{color}
\usepackage[table,dvipsnames]{xcolor}
\usepackage{colortbl}

% advanced date formating
\usepackage{datetime}

%support pandoc code highlighting
\usepackage{fancyvrb}
\DefineShortVerb[commandchars=\\\{\}]{\|}
\DefineVerbatimEnvironment{Highlighting}{Verbatim}{commandchars=\\\{\}}
% Add ',fontsize=\small' for more characters per line

%tango style colors
% \usepackage{framed}
% \definecolor{shadecolor}{RGB}{255,255,255}
% \newenvironment{Shaded}{\begin{snugshade}}{\end{snugshade}}
% \newcommand{\KeywordTok}[1]{\textcolor[rgb]{0.13,0.29,0.53}{\textbf{{#1}}}}
% \newcommand{\DataTypeTok}[1]{\textcolor[rgb]{0.13,0.29,0.53}{{#1}}}
% \newcommand{\DecValTok}[1]{\textcolor[rgb]{0.00,0.00,0.81}{{#1}}}
% \newcommand{\BaseNTok}[1]{\textcolor[rgb]{0.00,0.00,0.81}{{#1}}}
% \newcommand{\FloatTok}[1]{\textcolor[rgb]{0.00,0.00,0.81}{{#1}}}
% \newcommand{\CharTok}[1]{\textcolor[rgb]{0.31,0.60,0.02}{{#1}}}
% \newcommand{\StringTok}[1]{\textcolor[rgb]{0.31,0.60,0.02}{{#1}}}
% \newcommand{\CommentTok}[1]{\textcolor[rgb]{0.56,0.35,0.01}{\textit{{#1}}}}
% \newcommand{\OtherTok}[1]{\textcolor[rgb]{0.56,0.35,0.01}{{#1}}}
% \newcommand{\AlertTok}[1]{\textcolor[rgb]{0.94,0.16,0.16}{{#1}}}
% \newcommand{\FunctionTok}[1]{\textcolor[rgb]{0.00,0.00,0.00}{{#1}}}
% \newcommand{\RegionMarkerTok}[1]{{#1}}
% \newcommand{\ErrorTok}[1]{\textbf{{#1}}}
% \newcommand{\NormalTok}[1]{{#1}}

%espresso style colors
% \usepackage{framed}
% \definecolor{shadecolor}{RGB}{42,33,28}
% \newenvironment{Shaded}{\begin{snugshade}}{\end{snugshade}}
% \newcommand{\KeywordTok}[1]{\textcolor[rgb]{0.26,0.66,0.93}{\textbf{{#1}}}}
% \newcommand{\DataTypeTok}[1]{\textcolor[rgb]{0.74,0.68,0.62}{\underline{{#1}}}}
% \newcommand{\DecValTok}[1]{\textcolor[rgb]{0.27,0.67,0.26}{{#1}}}
% \newcommand{\BaseNTok}[1]{\textcolor[rgb]{0.27,0.67,0.26}{{#1}}}
% \newcommand{\FloatTok}[1]{\textcolor[rgb]{0.27,0.67,0.26}{{#1}}}
% \newcommand{\CharTok}[1]{\textcolor[rgb]{0.02,0.61,0.04}{{#1}}}
% \newcommand{\StringTok}[1]{\textcolor[rgb]{0.02,0.61,0.04}{{#1}}}
% \newcommand{\CommentTok}[1]{\textcolor[rgb]{0.00,0.40,1.00}{\textit{{#1}}}}
% \newcommand{\OtherTok}[1]{\textcolor[rgb]{0.74,0.68,0.62}{{#1}}}
% \newcommand{\AlertTok}[1]{\textcolor[rgb]{1.00,1.00,0.00}{{#1}}}
% \newcommand{\FunctionTok}[1]{\textcolor[rgb]{1.00,0.58,0.35}{\textbf{{#1}}}}
% \newcommand{\RegionMarkerTok}[1]{\textcolor[rgb]{0.74,0.68,0.62}{{#1}}}
% \newcommand{\ErrorTok}[1]{\textcolor[rgb]{0.74,0.68,0.62}{\textbf{{#1}}}}
% \newcommand{\NormalTok}[1]{\textcolor[rgb]{0.74,0.68,0.62}{{#1}}}

%kete style colors
% \newenvironment{Shaded}{}{}
% \newcommand{\KeywordTok}[1]{\textbf{{#1}}}
% \newcommand{\DataTypeTok}[1]{\textcolor[rgb]{0.50,0.00,0.00}{{#1}}}
% \newcommand{\DecValTok}[1]{\textcolor[rgb]{0.00,0.00,1.00}{{#1}}}
% \newcommand{\BaseNTok}[1]{\textcolor[rgb]{0.00,0.00,1.00}{{#1}}}
% \newcommand{\FloatTok}[1]{\textcolor[rgb]{0.50,0.00,0.50}{{#1}}}
% \newcommand{\CharTok}[1]{\textcolor[rgb]{1.00,0.00,1.00}{{#1}}}
% \newcommand{\StringTok}[1]{\textcolor[rgb]{0.87,0.00,0.00}{{#1}}}
% \newcommand{\CommentTok}[1]{\textcolor[rgb]{0.50,0.50,0.50}{\textit{{#1}}}}
% \newcommand{\OtherTok}[1]{{#1}}
% \newcommand{\AlertTok}[1]{\textcolor[rgb]{0.00,1.00,0.00}{\textbf{{#1}}}}
% \newcommand{\FunctionTok}[1]{\textcolor[rgb]{0.00,0.00,0.50}{{#1}}}
% \newcommand{\RegionMarkerTok}[1]{{#1}}
% \newcommand{\ErrorTok}[1]{\textcolor[rgb]{1.00,0.00,0.00}{\textbf{{#1}}}}
% \newcommand{\NormalTok}[1]{{#1}}
%end pandoc code hacks

% jodliterate colors
\usepackage{color}
\definecolor{shadecolor}{RGB}{248,248,248}
% j control structures 
\definecolor{keywcolor}{rgb}{0.13,0.29,0.53}
% j explicit arguments x y m n u v
\definecolor{datacolor}{rgb}{0.13,0.29,0.53}
% j numbers - all types see j.xml
\definecolor{decvcolor}{rgb}{0.00,0.00,0.81}
\definecolor{basencolor}{rgb}{0.00,0.00,0.81}
\definecolor{floatcolor}{rgb}{0.00,0.00,0.81}
% j local assignments
\definecolor{charcolor}{rgb}{0.31,0.60,0.02}
\definecolor{stringcolor}{rgb}{0.31,0.60,0.02}
\definecolor{commentcolor}{rgb}{0.56,0.35,0.01}
% primitive adverbs and conjunctions
%\definecolor{othercolor}{rgb}{0.56,0.35,0.01}   
\definecolor{othercolor}{RGB}{0,0,255}
% global assignments
\definecolor{alertcolor}{rgb}{0.94,0.16,0.16}
% primitive J verbs and noun names
\definecolor{funccolor}{rgb}{0.00,0.00,0.00}    

\usepackage{framed}
\newenvironment{Shaded}{}{}
\newcommand{\KeywordTok}[1]{\textcolor{keywcolor}{\textbf{{#1}}}}
\newcommand{\DataTypeTok}[1]{\textcolor{datacolor}{{#1}}}
%\newcommand{\DecValTok}[1]{\textcolor{decvcolor}{{#1}}}
\newcommand{\DecValTok}[1]{{#1}} 
\newcommand{\BaseNTok}[1]{\textcolor{basencolor}{{#1}}}
\newcommand{\FloatTok}[1]{\textcolor{floatcolor}{{#1}}}
\newcommand{\CharTok}[1]{\textcolor{charcolor}{\textbf{{#1}}}}
\newcommand{\StringTok}[1]{\textcolor{stringcolor}{{#1}}}
\newcommand{\CommentTok}[1]{\textcolor{commentcolor}{\textit{{#1}}}}
\newcommand{\OtherTok}[1]{\textcolor{othercolor}{{#1}}} 
\newcommand{\AlertTok}[1]{\textcolor{alertcolor}{\textbf{{#1}}}}
%\newcommand{\FunctionTok}[1]{\textcolor{funccolor}{{#1}}}
\newcommand{\FunctionTok}[1]{{#1}}
\newcommand{\RegionMarkerTok}[1]{{#1}}
\newcommand{\ErrorTok}[1]{\textbf{{#1}}}
\newcommand{\NormalTok}[1]{{#1}}

% headers and footers
\usepackage{fancyhdr}
\pagestyle{fancy}

\fancyhead{}
\fancyfoot{}

%\fancyhead[LE,RO]{\slshape \rightmark}
%\fancyhead[LO,RE]{\slshape \leftmark}
\fancyfoot[C]{\thepage}
%\headrulewidth 0.4pt
%\footrulewidth 0 pt

%\addtolength{\headheight}{\baselineskip}

%\lfoot{\emph{Analyze the Data not the Drivel}}
%\rfoot{\emph{\today}}

% subfigure handles figures that contain subfigures
%\usepackage{color,graphicx,subfigure,sidecap}
\usepackage{graphicx,sidecap}
\usepackage{subfigure}
\graphicspath{{./inclusions/}}

% floatflt provides for text wrapping around small figures and tables
\usepackage{floatflt}

% tweak caption formats 
\usepackage{caption} 
\usepackage{sidecap}
%\usepackage{subcaption} % not compatible with subfigure

\usepackage{rotating} % flip tables sideways

% complex footnotes
%\usepackage{bigfoot}

% weird logos \XeLaTeX
\usepackage{metalogo}

% source code listings
\usepackage{listings}

% long tables
% \usepackage{longtable}

\newcommand{\HRule}{\rule{\linewidth}{0.5mm}}

% map LaTeX cross references into PDF cross references
\usepackage[
            %dvips,
            colorlinks,
            linkcolor=blue,
            citecolor=blue,
            urlcolor=blue,   % magenta, cyan default        
            pdfauthor={John D. Baker},
            pdftitle={Analyze the Data not the Drivel},
            pdfsubject={Blog},
            pdfcreator={MikTeX+LaTeXe with hyperref package},
            pdfkeywords={blog,wordpress},
            ]{hyperref}
           
% custom colors
\definecolor{CodeBackGround}{cmyk}{0.0,0.0,0,0.05}    % light gray
\definecolor{CodeComment}{rgb}{0,0.50,0.00}           % dark green {0,0.45,0.08}
\definecolor{TableStripes}{gray}{0.9}                 % odd/even background in tables

\lstdefinelanguage{bat}
{morekeywords={echo,title,pushd,popd,setlocal,endlocal,off,if,not,exist,set,goto,pause},
sensitive=True,
morecomment=[l]{rem}
}

\lstdefinelanguage{jdoc}
{
morekeywords={},
otherkeywords={assert.,break.,continue.,for.,do.,if.,else.,elseif.,return.,select.,end.
,while.,whilst.,throw.,catch.,catchd.,catcht.,try.,case.,fcase.},
sensitive=True,
morecomment=[l]{NB.},
morestring=[b]',
morestring=[d]',
}

% latex size ordering - can never remember it
% \tiny
% \scriptsize
% \footnotesize
% \small
% \normalsize
% \large
% \Large
% \LARGE
% \huge
% \Huge
 
% listings package settings  
\lstset{%
  language=jdoc,                                % j document settings
  basicstyle=\ttfamily\footnotesize,            
  keywordstyle=\bfseries\color{keywcolor}\footnotesize,
  identifierstyle=\color{black},
  commentstyle=\slshape\color{CodeComment},     % colored slanted comments
  stringstyle=\color{red}\ttfamily,
  showstringspaces=false,                       
  %backgroundcolor=\color{CodeBackGround},       
  frame=single,                                
  framesep=1pt,                                 
  framerule=0.8pt,                             
  rulecolor=\color{CodeBackGround},   
  showspaces=false,
  %columns=fullflexible,
  %numbers=left,
  %numberstyle=\footnotesize,
  %numbersep=9pt,
  tabsize=2,
  showtabs=false,
  captionpos=b
  breaklines=true,                              
  breakindent=5pt                              
}

\lstdefinelanguage{JavaScript}{
  keywords={typeof, new, true, false, catch, function, return, null, catch, switch, var, if, in, while, do, else, case, break},
  ndkeywords={class, export, boolean, throw, implements, import, this},
  ndkeywordstyle=\color{darkgray}\bfseries,
  sensitive=false,
  comment=[l]{//},
  morecomment=[s]{/*}{*/},
  morestring=[b]',
  morestring=[b]"
}

% C# settings
\lstdefinestyle{sharpc}{
language=[Sharp]C,
basicstyle=\ttfamily\scriptsize, 
keywordstyle=\bfseries\color{keywcolor}\scriptsize,
framerule=0pt
}

% for source code listing longer than two use smaller font
\lstdefinestyle{smallersource}{
basicstyle=\ttfamily\scriptsize, 
keywordstyle=\bfseries\color{keywcolor}\scriptsize,
framerule=0pt
}

\lstdefinestyle{resetdefaults}{
language=jdoc,
basicstyle=\ttfamily\footnotesize,  
keywordstyle=\bfseries\color{keywcolor}\footnotesize,                                                               
framerule=0.8pt 
}

% APL UTF8 code points listed for lstlisting processing
\makeatletter
\lst@InputCatcodes
\def\lst@DefEC{%
 \lst@CCECUse \lst@ProcessLetter
  ^^80^^81^^82^^83^^84^^85^^86^^87^^88^^89^^8a^^8b^^8c^^8d^^8e^^8f%
  ^^90^^91^^92^^93^^94^^95^^96^^97^^98^^99^^9a^^9b^^9c^^9d^^9e^^9f%
  ^^a0^^a1^^a2^^a3^^a4^^a5^^a6^^a7^^a8^^a9^^aa^^ab^^ac^^ad^^ae^^af%
  ^^b0^^b1^^b2^^b3^^b4^^b5^^b6^^b7^^b8^^b9^^ba^^bb^^bc^^bd^^be^^bf%
  ^^c0^^c1^^c2^^c3^^c4^^c5^^c6^^c7^^c8^^c9^^ca^^cb^^cc^^cd^^ce^^cf%
  ^^d0^^d1^^d2^^d3^^d4^^d5^^d6^^d7^^d8^^d9^^da^^db^^dc^^dd^^de^^df%
  ^^e0^^e1^^e2^^e3^^e4^^e5^^e6^^e7^^e8^^e9^^ea^^eb^^ec^^ed^^ee^^ef%
  ^^f0^^f1^^f2^^f3^^f4^^f5^^f6^^f7^^f8^^f9^^fa^^fb^^fc^^fd^^fe^^ff%
  ^^^^20ac^^^^0153^^^^0152%
  ^^^^20a7^^^^2190^^^^2191^^^^2192^^^^2193^^^^2206^^^^2207^^^^220a%
  ^^^^2218^^^^2228^^^^2229^^^^222a^^^^2235^^^^223c^^^^2260^^^^2261%
  ^^^^2262^^^^2264^^^^2265^^^^2282^^^^2283^^^^2296^^^^22a2^^^^22a3%
  ^^^^22a4^^^^22a5^^^^22c4^^^^2308^^^^230a^^^^2336^^^^2337^^^^2339%
  ^^^^233b^^^^233d^^^^233f^^^^2340^^^^2342^^^^2347^^^^2348^^^^2349%
  ^^^^234b^^^^234e^^^^2350^^^^2352^^^^2355^^^^2357^^^^2359^^^^235d%
  ^^^^235e^^^^235f^^^^2361^^^^2362^^^^2363^^^^2364^^^^2365^^^^2368%
  ^^^^236a^^^^236b^^^^236c^^^^2371^^^^2372^^^^2373^^^^2374^^^^2375%
  ^^^^2377^^^^2378^^^^237a^^^^2395^^^^25af^^^^25ca^^^^25cb%  
  ^^00}
\lst@RestoreCatcodes
\makeatother

% custom lengths used within minipages
\newcommand{\minindent}{17pt}


\makeindex

\begin{document}

\subsection*{\href{http://bakerjd99.wordpress.com/2014/05/10/marcus-aurelius-tunes-my-rss-feeds/}{Marcus Aurelius tunes my RSS Feeds}}
\addcontentsline{toc}{subsection}{Marcus Aurelius tunes my RSS Feeds}


\noindent\emph{Posted: 10 May 2014 16:08:13}
\vspace{6pt}

%\href{http://www.barnesandnoble.com/w/emperors-handbook-marcus-aurelius/1115032739?ean=9780743233835}{\includegraphics{emperors-handbook.jpg}}

\captionsetup[floatingfigure]{labelformat=empty}
\begin{floatingfigure}[l]{0.205\textwidth}
\centering
\href{http://www.barnesandnoble.com/w/emperors-handbook-marcus-aurelius/1115032739?ean=9780743233835}{\includegraphics[width=0.18\textwidth]{emperors-handbook.jpg}}
\label{fig:4682X0}
\end{floatingfigure} \href{http://www.amazon.com/The-Emperors-Handbook-Translation-Meditations/dp/0743233832}{The
Emperor's Handbook} is a new translation of Marcus Aurelius' classic
\emph{The Meditations}. Marcus Aurelius was a second century Roman
emperor and stoic philosopher. You probably know him as the old guy
(Richard Harris) that chose Maximus (Russell Crowe) as his successor in
\href{http://en.wikipedia.org/wiki/Gladiator_(2000_film)}{Gladiator}.
Marcus is counted among the ``five good Roman emperors''\footnote{
Most Roman emperors were brutal corrupt monsters.
} and his
\emph{Meditations} has been hailed as the single best book ever written
by a major ruler. Nowadays every
\href{http://www.amazon.com/Decision-Points-George-W-Bush/dp/0307590631}{semi-literate
hack} that's held office
dumps \href{http://en.wikipedia.org/wiki/My_Life_(Bill_Clinton_autobiography)}{memoirs}.
The more vacuous
\href{http://www.goodreads.com/book/show/88061.Dreams_from_My_Father}{excrete
before holding office}! While political autobiography is usually the
vilest form of pornography and begs the question; is book burning all
bad? There are exceptions and The Meditations is a magnificent example.

\medskip
The Emperors Handbook is a sequence of short notes. Some are sentences
like:

\begin{quote}
Not knowing what other people are thinking is not the cause of much
human misery, but failing to understand the workings of one's own mind
is bound to lead to unhappiness

How shameful and absurd it is for the spirit to surrender when the body
is able to fight on!

What is useless for the hive is of no use to bee.
\end{quote}

Others take a page or two. It's not clear that Marcus intended to
``publish'' his notes and this may partly account for their frank and
honest elegance. I don't read ancient Greek, the language Marcus used to
compose his notes, so I cannot judge his original style but the Hicks
brother's English translation is an absolute delight. I often found
myself rereading passages aloud to fully savior Marcus's phrases; they
ring like poetry and tell like prose. This guy would crush modern
Internet trolls!

The most striking thing about Marcus's passages is their stark
modernity. If you ignore the allusions to multiple gods and references
to 2\textsuperscript{nd} century contemporaries many of Marcus's 1,800
year old notes might~have been composed yesterday. The following would
not be out-of-place in the preamble of any modern mathematical logic
text.

\begin{quote}
Reason and logic are governed by their own laws and employ their own
methods. They launch themselves at will, and they head straight for
their target. This is why we call actions that seem to us reasonable and
logical ``right,'' because they are right on target.
\end{quote}

Of course the real measure of any work is: does it change the way you
think and act? A lot of Marcus's stoic advice will be hard for us. He
repeatedly stresses the importance of playing your part in the greater
scheme of things. In his view the universe is either meaningless atoms
smashing together or it's arranged by providence. If the first case
holds then we should play our part because we are social beings and
need~the cooperation and support of others to fully prosper. If the
second holds then we should strive to find our designated purpose and
execute it to the best of our abilities. Either way we should do our
duty without whining. A stoic man's
\href{https://www.goodreads.com/quotes/148061-a-man-s-got-to-do-what-a-man-s-got-to}{``got
to do what a man's got to do.''}

I'm more of a skeptic than Marcus, and I have the benefit of 1,800 more
years of history and science to drawn on, so until there is
overwhelming, ultra-hard, fully repeatable, and independently verified
scientific evidence to the contrary, it's almost certain that life is
meaningless and random! The ``atoms'' that smash together in the
21\textsuperscript{st} century have
\href{http://scienceblogs.com/startswithabang/2010/09/27/the-new-nu-news/}{a
richer taxonomy} than their hypothetical 2\textsuperscript{nd} century
antecedents but they are just as meaningless. The notion that we have a
duty or purpose is ludicrous. We are, as Richard Dawkins wrote in the
Selfish Gene long ago, robots evolved to propagate~genes. And, as many
have noted, it's not clear that ``intelligent robots'' are ideal for
gene propagation: bacteria and ants are~doing a better job. I cannot
accept the unsubstantiated notion of duty so I will not play my alleged
part. The greater scheme will have to manage without me.

Though I reject ``duty'' I still find much of use in The Meditations. Of
great value is Marcus's long view.

\begin{quote}
Camillus, Caeso, Volesus, Dentatus and to a lesser degree Scipio and
Cato, and yes, even Augustus, Hadrian and Antoninus are less spoken of
now than they were in their own days. For all things fade away, become
the stuff of legend, and are soon buried by oblivion. Mind you, this is
true only for those who blazed once like bright stars in the firmament,
but for the rest, as soon a few clods of earth cover their corpses, they
are ``out of sight, out of mind.''
\end{quote}

That pretty much sums up my approach to the Obama administration.~Marcus
also has sound advice on Internet filtering.

\begin{quote}
Bear in mind that the measure of a man is the worth of the things he
cares about.
\end{quote}

If you find something predictable and shrill do a quick calculation.
Does the signal justify enduring the noise? I went through my long list
of \href{http://feedly.com/}{Feedly} RSS feeds and deleted two prominent
sources of noise: \href{http://www.rawstory.com/}{The Raw Story} and
\href{http://www.breitbart.com/}{Breitbart}. These sites are loud
practitioners of \emph{look at this idiot journalism}. One side is
predictably far left-wing and the other is predictably far right-wing.
Occasionally they cough up something worth a look but usually their
articles, and attending moronic troll infested commentary, is a complete
waste of time. Is such drivel worthy of my gaze? Applying Marcus's rule
I had to cut them loose. Thank you Marcus for pruning my RSS feeds.


%{[}1{]}
% Most Roman emperors were brutal corrupt monsters.

%\captionsetup[floatingfigure]{labelformat=empty}
%\begin{figure}[htbp]
%\begin{floatingfigure}[l]{0.25\textwidth}
%\centering
%\includegraphics[width=0.23\textwidth]{emperors-handbook.jpg}
%\caption{~~~IMCAPTION~~~}
%\label{fig:4682X0}
%\end{floatingfigure}
%\end{figure}



%\end{document}