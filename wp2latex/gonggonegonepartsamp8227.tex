%% bm.pdf preamble - material merged from previous preamble and current pandoc preamable output
% NOTE: float placement required changes to the source files referenced by bm.tex
% May 28, 2020
%
% Use lualatex to compile - test with MiKTeX 2.9

% uncomment to list all files in log
%\listfiles

\documentclass[12pt]{report}


\usepackage{fontspec}

%\setmainfont[Scale=MatchLowercase]{Lucida Bright}
%\setmonofont{FreeMono}
%\setmonofont{Source Code Pro}
\setmonofont[Scale=MatchLowercase]{Ubuntu Mono}

% short snippets of asian languages
\newfontfamily\myAsian{Noto Serif TC Medium}

\usepackage[headings]{fullpage}

% national use characters 
%\usepackage{inputenc}

% ams mathematical symbols
\usepackage{amsmath,amssymb}

% added to support pandoc highlighting
\usepackage{microtype}

\usepackage{makeidx}

% add index and bibliographies to table of contents
\usepackage[nottoc]{tocbibind}

% postscript courier and times in place of cm fonts
%\usepackage{courier}
%\usepackage{times}

% extended coloring
\usepackage{color}
\usepackage[table,dvipsnames]{xcolor}
\usepackage{colortbl}

% advanced date formating
\usepackage{datetime}

%support pandoc code highlighting
\usepackage{fancyvrb}

% \DefineShortVerb[commandchars=\\\{\}]{\|}
% \DefineVerbatimEnvironment{Highlighting}{Verbatim}{commandchars=\\\{\}}
% % Add ',fontsize=\small' for more characters per line

% tango style colors
% \usepackage{framed}
% \definecolor{shadecolor}{RGB}{255,255,255}
% \newenvironment{Shaded}{\begin{snugshade}}{\end{snugshade}}
% \newcommand{\KeywordTok}[1]{\textcolor[rgb]{0.13,0.29,0.53}{\textbf{{#1}}}}
% \newcommand{\DataTypeTok}[1]{\textcolor[rgb]{0.13,0.29,0.53}{{#1}}}
% \newcommand{\DecValTok}[1]{\textcolor[rgb]{0.00,0.00,0.81}{{#1}}}
% \newcommand{\BaseNTok}[1]{\textcolor[rgb]{0.00,0.00,0.81}{{#1}}}
% \newcommand{\FloatTok}[1]{\textcolor[rgb]{0.00,0.00,0.81}{{#1}}}
% \newcommand{\CharTok}[1]{\textcolor[rgb]{0.31,0.60,0.02}{{#1}}}
% \newcommand{\StringTok}[1]{\textcolor[rgb]{0.31,0.60,0.02}{{#1}}}
% \newcommand{\CommentTok}[1]{\textcolor[rgb]{0.56,0.35,0.01}{\textit{{#1}}}}
% \newcommand{\OtherTok}[1]{\textcolor[rgb]{0.56,0.35,0.01}{{#1}}}
% \newcommand{\AlertTok}[1]{\textcolor[rgb]{0.94,0.16,0.16}{{#1}}}
% \newcommand{\FunctionTok}[1]{\textcolor[rgb]{0.00,0.00,0.00}{{#1}}}
% \newcommand{\RegionMarkerTok}[1]{{#1}}
% \newcommand{\ErrorTok}[1]{\textbf{{#1}}}
% \newcommand{\NormalTok}[1]{{#1}}

% %espresso style colors
% \usepackage{framed}
% \definecolor{shadecolor}{RGB}{42,33,28}
% \newenvironment{Shaded}{\begin{snugshade}}{\end{snugshade}}
% \newcommand{\KeywordTok}[1]{\textcolor[rgb]{0.26,0.66,0.93}{\textbf{{#1}}}}
% \newcommand{\DataTypeTok}[1]{\textcolor[rgb]{0.74,0.68,0.62}{\underline{{#1}}}}
% \newcommand{\DecValTok}[1]{\textcolor[rgb]{0.27,0.67,0.26}{{#1}}}
% \newcommand{\BaseNTok}[1]{\textcolor[rgb]{0.27,0.67,0.26}{{#1}}}
% \newcommand{\FloatTok}[1]{\textcolor[rgb]{0.27,0.67,0.26}{{#1}}}
% \newcommand{\CharTok}[1]{\textcolor[rgb]{0.02,0.61,0.04}{{#1}}}
% \newcommand{\StringTok}[1]{\textcolor[rgb]{0.02,0.61,0.04}{{#1}}}
% \newcommand{\CommentTok}[1]{\textcolor[rgb]{0.00,0.40,1.00}{\textit{{#1}}}}
% \newcommand{\OtherTok}[1]{\textcolor[rgb]{0.74,0.68,0.62}{{#1}}}
% \newcommand{\AlertTok}[1]{\textcolor[rgb]{1.00,1.00,0.00}{{#1}}}
% \newcommand{\FunctionTok}[1]{\textcolor[rgb]{1.00,0.58,0.35}{\textbf{{#1}}}}
% \newcommand{\RegionMarkerTok}[1]{\textcolor[rgb]{0.74,0.68,0.62}{{#1}}}
% \newcommand{\ErrorTok}[1]{\textcolor[rgb]{0.74,0.68,0.62}{\textbf{{#1}}}}
% \newcommand{\NormalTok}[1]{\textcolor[rgb]{0.74,0.68,0.62}{{#1}}}

% %kete style colors
% \newenvironment{Shaded}{}{}
% \newcommand{\KeywordTok}[1]{\textbf{{#1}}}
% \newcommand{\DataTypeTok}[1]{\textcolor[rgb]{0.50,0.00,0.00}{{#1}}}
% \newcommand{\DecValTok}[1]{\textcolor[rgb]{0.00,0.00,1.00}{{#1}}}
% \newcommand{\BaseNTok}[1]{\textcolor[rgb]{0.00,0.00,1.00}{{#1}}}
% \newcommand{\FloatTok}[1]{\textcolor[rgb]{0.50,0.00,0.50}{{#1}}}
% \newcommand{\CharTok}[1]{\textcolor[rgb]{1.00,0.00,1.00}{{#1}}}
% \newcommand{\StringTok}[1]{\textcolor[rgb]{0.87,0.00,0.00}{{#1}}}
% \newcommand{\CommentTok}[1]{\textcolor[rgb]{0.50,0.50,0.50}{\textit{{#1}}}}
% \newcommand{\OtherTok}[1]{{#1}}
% \newcommand{\AlertTok}[1]{\textcolor[rgb]{0.00,1.00,0.00}{\textbf{{#1}}}}
% \newcommand{\FunctionTok}[1]{\textcolor[rgb]{0.00,0.00,0.50}{{#1}}}
% \newcommand{\RegionMarkerTok}[1]{{#1}}
% \newcommand{\ErrorTok}[1]{\textcolor[rgb]{1.00,0.00,0.00}{\textbf{{#1}}}}
% \newcommand{\NormalTok}[1]{{#1}}
% %end pandoc code hacks

% jodliterate colors
\usepackage{color}
\definecolor{shadecolor}{RGB}{248,248,248}
% j control structures 
\definecolor{keywcolor}{rgb}{0.13,0.29,0.53}
% j explicit arguments x y m n u v
\definecolor{datacolor}{rgb}{0.13,0.29,0.53}
% j numbers - all types see j.xml
\definecolor{decvcolor}{rgb}{0.00,0.00,0.81}
\definecolor{basencolor}{rgb}{0.00,0.00,0.81}
\definecolor{floatcolor}{rgb}{0.00,0.00,0.81}
% j local assignments
\definecolor{charcolor}{rgb}{0.31,0.60,0.02}
\definecolor{stringcolor}{rgb}{0.31,0.60,0.02}
\definecolor{commentcolor}{rgb}{0.56,0.35,0.01}
% primitive adverbs and conjunctions
%\definecolor{othercolor}{rgb}{0.56,0.35,0.01}   
\definecolor{othercolor}{RGB}{0,0,255}
% global assignments
\definecolor{alertcolor}{rgb}{0.94,0.16,0.16}
% primitive J verbs and noun names
\definecolor{funccolor}{rgb}{0.00,0.00,0.00}

% custom colors
\definecolor{CodeBackGround}{cmyk}{0.0,0.0,0,0.05}    % light gray
\definecolor{CodeComment}{rgb}{0,0.50,0.00}           % dark green {0,0.45,0.08}
\definecolor{TableStripes}{gray}{0.9}                 % odd/even background in tables

% Colors for the hyperref package
\definecolor{urlcolor}{rgb}{0,.145,.698}
\definecolor{linkcolor}{rgb}{.71,0.21,0.01}
\definecolor{citecolor}{rgb}{.12,.54,.11}

% % Exact colors from NB
\definecolor{incolor}{HTML}{303F9F}
\definecolor{outcolor}{HTML}{D84315}
\definecolor{cellborder}{HTML}{CFCFCF}
\definecolor{cellbackground}{HTML}{F7F7F7}

% % ANSI colors
\definecolor{ansi-black}{HTML}{3E424D}
\definecolor{ansi-black-intense}{HTML}{282C36}
\definecolor{ansi-red}{HTML}{E75C58}
\definecolor{ansi-red-intense}{HTML}{B22B31}
\definecolor{ansi-green}{HTML}{00A250}
\definecolor{ansi-green-intense}{HTML}{007427}
\definecolor{ansi-yellow}{HTML}{DDB62B}
\definecolor{ansi-yellow-intense}{HTML}{B27D12}
\definecolor{ansi-blue}{HTML}{208FFB}
\definecolor{ansi-blue-intense}{HTML}{0065CA}
\definecolor{ansi-magenta}{HTML}{D160C4}
\definecolor{ansi-magenta-intense}{HTML}{A03196}
\definecolor{ansi-cyan}{HTML}{60C6C8}
\definecolor{ansi-cyan-intense}{HTML}{258F8F}
\definecolor{ansi-white}{HTML}{C5C1B4}
\definecolor{ansi-white-intense}{HTML}{A1A6B2}
\definecolor{ansi-default-inverse-fg}{HTML}{FFFFFF}
\definecolor{ansi-default-inverse-bg}{HTML}{000000}
    

% \usepackage{framed}
% \newenvironment{Shaded}{}{}
% \newcommand{\KeywordTok}[1]{\textcolor{keywcolor}{\textbf{{#1}}}}
% \newcommand{\DataTypeTok}[1]{\textcolor{datacolor}{{#1}}}
% %\newcommand{\DecValTok}[1]{\textcolor{decvcolor}{{#1}}}
% \newcommand{\DecValTok}[1]{{#1}} 
% \newcommand{\BaseNTok}[1]{\textcolor{basencolor}{{#1}}}
% \newcommand{\FloatTok}[1]{\textcolor{floatcolor}{{#1}}}
% \newcommand{\CharTok}[1]{\textcolor{charcolor}{\textbf{{#1}}}}
% \newcommand{\StringTok}[1]{\textcolor{stringcolor}{{#1}}}
% \newcommand{\CommentTok}[1]{\textcolor{commentcolor}{\textit{{#1}}}}
% \newcommand{\OtherTok}[1]{\textcolor{othercolor}{{#1}}} 
% \newcommand{\AlertTok}[1]{\textcolor{alertcolor}{\textbf{{#1}}}}
% %\newcommand{\FunctionTok}[1]{\textcolor{funccolor}{{#1}}}
% \newcommand{\FunctionTok}[1]{{#1}}
% \newcommand{\RegionMarkerTok}[1]{{#1}}
% \newcommand{\ErrorTok}[1]{\textbf{{#1}}}
% \newcommand{\NormalTok}[1]{{#1}}

% The default LaTeX title has an obnoxious amount of whitespace. By default,
% titling removes some of it. It also provides customization options.
\usepackage{titling}

% headers and footers
\usepackage{fancyhdr}
%\pagestyle{fancy}
\pagestyle{plain}

\fancyhead{}
\fancyfoot{}

%\fancyhead[LE,RO]{\slshape \rightmark}
%\fancyhead[LO,RE]{\slshape \leftmark}
\fancyfoot[C]{\thepage}
%\headrulewidth 0.4pt
%\footrulewidth 0 pt

%\addtolength{\headheight}{\baselineskip}

%\lfoot{\emph{Analyze the Data not the Drivel}}
%\rfoot{\emph{\today}}

% subfigure handles figures that contain subfigures
%\usepackage{color,graphicx,subfigure,sidecap}
\usepackage{graphicx,sidecap}
\usepackage{subfigure}
\graphicspath{{./inclusions/}}

% floatflt provides for text wrapping around small figures and tables
\usepackage{floatflt}

% tweak caption formats 
\usepackage{caption} 
\usepackage{sidecap}
%\usepackage{subcaption} % not compatible with subfigure

\usepackage{rotating} % flip tables sideways

% complex footnotes
%\usepackage{bigfoot}

% weird logos \XeLaTeX
\usepackage{metalogo}

\newcommand{\HRule}{\rule{\linewidth}{0.5mm}}

\usepackage[breakable]{tcolorbox}

\usepackage{parskip} % Stop auto-indenting (to mimic markdown behaviour)
    
% Basic figure setup, for now with no caption control since it's done
% automatically by Pandoc (which extracts ![](path) syntax from Markdown).
\usepackage{graphicx}

%\DeclareCaptionFormat{nocaption}{}
%\captionsetup{format=nocaption,aboveskip=0pt,belowskip=0pt}

\usepackage[Export]{adjustbox} % Used to constrain images to a maximum size
\adjustboxset{max size={0.9\linewidth}{0.9\paperheight}}
\usepackage{float}

%\floatplacement{figure}{H} % forces figures to be placed at the correct location

\usepackage{xcolor} % Allow colors to be defined
\usepackage{enumerate} % Needed for markdown enumerations to work
\usepackage{geometry} % Used to adjust the document margins

%\usepackage{amsmath} % Equations
%\usepackage{amssymb} % Equations

\usepackage{textcomp} % defines textquotesingle

% Hack from http://tex.stackexchange.com/a/47451/13684:
\AtBeginDocument{%
	\def\PYZsq{\textquotesingle}% Upright quotes in Pygmentized code
}

\usepackage{upquote} % Upright quotes for verbatim code
\usepackage{eurosym} % defines \euro
\usepackage[mathletters]{ucs} % Extended unicode (utf-8) support

%\usepackage{fancyvrb} % verbatim replacement that allows latex

\usepackage{grffile} % extends the file name processing of package graphics 
					 % to support a larger range
					 
\makeatletter % fix for grffile with XeLaTeX
\def\Gread@@xetex#1{%
  \IfFileExists{"\Gin@base".bb}%
  {\Gread@eps{\Gin@base.bb}}%
  {\Gread@@xetex@aux#1}%
}
\makeatother

% The hyperref package gives us a pdf with properly built
% internal navigation ('pdf bookmarks' for the table of contents,
% internal cross-reference links, web links for URLs, etc.)
\usepackage{hyperref}
% The default LaTeX title has an obnoxious amount of whitespace. By default,
% titling removes some of it. It also provides customization options.
\usepackage{titling}
\usepackage{longtable} % longtable support required by pandoc >1.10
\usepackage{booktabs}  % table support for pandoc > 1.12.2
\usepackage[inline]{enumitem} % IRkernel/repr support (it uses the enumerate* environment)
\usepackage[normalem]{ulem} % ulem is needed to support strikethroughs (\sout)
							% normalem makes italics be italics, not underlines
\usepackage{mathrsfs}

% commands and environments needed by pandoc snippets
% extracted from the output of `pandoc -s`
\providecommand{\tightlist}{%
  \setlength{\itemsep}{0pt}\setlength{\parskip}{0pt}}
  
\DefineVerbatimEnvironment{Highlighting}{Verbatim}{commandchars=\\\{\}}
% Add ',fontsize=\small' for more characters per line
\newenvironment{Shaded}{}{}
\newcommand{\KeywordTok}[1]{\textcolor[rgb]{0.00,0.44,0.13}{\textbf{{#1}}}}
\newcommand{\DataTypeTok}[1]{\textcolor[rgb]{0.56,0.13,0.00}{{#1}}}
\newcommand{\DecValTok}[1]{\textcolor[rgb]{0.25,0.63,0.44}{{#1}}}
\newcommand{\BaseNTok}[1]{\textcolor[rgb]{0.25,0.63,0.44}{{#1}}}
\newcommand{\FloatTok}[1]{\textcolor[rgb]{0.25,0.63,0.44}{{#1}}}
\newcommand{\CharTok}[1]{\textcolor[rgb]{0.25,0.44,0.63}{{#1}}}
\newcommand{\StringTok}[1]{\textcolor[rgb]{0.25,0.44,0.63}{{#1}}}
\newcommand{\CommentTok}[1]{\textcolor[rgb]{0.38,0.63,0.69}{\textit{{#1}}}}
\newcommand{\OtherTok}[1]{\textcolor[rgb]{0.00,0.44,0.13}{{#1}}}
\newcommand{\AlertTok}[1]{\textcolor[rgb]{1.00,0.00,0.00}{\textbf{{#1}}}}
\newcommand{\FunctionTok}[1]{\textcolor[rgb]{0.02,0.16,0.49}{{#1}}}
\newcommand{\RegionMarkerTok}[1]{{#1}}
\newcommand{\ErrorTok}[1]{\textcolor[rgb]{1.00,0.00,0.00}{\textbf{{#1}}}}
\newcommand{\NormalTok}[1]{{#1}}

% Additional commands for more recent versions of Pandoc
\newcommand{\ConstantTok}[1]{\textcolor[rgb]{0.53,0.00,0.00}{{#1}}}
\newcommand{\SpecialCharTok}[1]{\textcolor[rgb]{0.25,0.44,0.63}{{#1}}}
\newcommand{\VerbatimStringTok}[1]{\textcolor[rgb]{0.25,0.44,0.63}{{#1}}}
\newcommand{\SpecialStringTok}[1]{\textcolor[rgb]{0.73,0.40,0.53}{{#1}}}
\newcommand{\ImportTok}[1]{{#1}}
\newcommand{\DocumentationTok}[1]{\textcolor[rgb]{0.73,0.13,0.13}{\textit{{#1}}}}
\newcommand{\AnnotationTok}[1]{\textcolor[rgb]{0.38,0.63,0.69}{\textbf{\textit{{#1}}}}}
\newcommand{\CommentVarTok}[1]{\textcolor[rgb]{0.38,0.63,0.69}{\textbf{\textit{{#1}}}}}
\newcommand{\VariableTok}[1]{\textcolor[rgb]{0.10,0.09,0.49}{{#1}}}
\newcommand{\ControlFlowTok}[1]{\textcolor[rgb]{0.00,0.44,0.13}{\textbf{{#1}}}}
\newcommand{\OperatorTok}[1]{\textcolor[rgb]{0.40,0.40,0.40}{{#1}}}
\newcommand{\BuiltInTok}[1]{{#1}}
\newcommand{\ExtensionTok}[1]{{#1}}
\newcommand{\PreprocessorTok}[1]{\textcolor[rgb]{0.74,0.48,0.00}{{#1}}}
\newcommand{\AttributeTok}[1]{\textcolor[rgb]{0.49,0.56,0.16}{{#1}}}
\newcommand{\InformationTok}[1]{\textcolor[rgb]{0.38,0.63,0.69}{\textbf{\textit{{#1}}}}}
\newcommand{\WarningTok}[1]{\textcolor[rgb]{0.38,0.63,0.69}{\textbf{\textit{{#1}}}}}

% Define a nice break command that doesn't care if a line doesn't already exist.
\def\br{\hspace*{\fill} \\* }
% Math Jax compatibility definitions
\def\gt{>}
\def\lt{<}
\let\Oldtex\TeX
\let\Oldlatex\LaTeX
\renewcommand{\TeX}{\textrm{\Oldtex}}
\renewcommand{\LaTeX}{\textrm{\Oldlatex}}
 
% Pygments definitions
\makeatletter
\def\PY@reset{\let\PY@it=\relax \let\PY@bf=\relax%
    \let\PY@ul=\relax \let\PY@tc=\relax%
    \let\PY@bc=\relax \let\PY@ff=\relax}
\def\PY@tok#1{\csname PY@tok@#1\endcsname}
\def\PY@toks#1+{\ifx\relax#1\empty\else%
    \PY@tok{#1}\expandafter\PY@toks\fi}
\def\PY@do#1{\PY@bc{\PY@tc{\PY@ul{%
    \PY@it{\PY@bf{\PY@ff{#1}}}}}}}
\def\PY#1#2{\PY@reset\PY@toks#1+\relax+\PY@do{#2}}

\expandafter\def\csname PY@tok@w\endcsname{\def\PY@tc##1{\textcolor[rgb]{0.73,0.73,0.73}{##1}}}
\expandafter\def\csname PY@tok@c\endcsname{\let\PY@it=\textit\def\PY@tc##1{\textcolor[rgb]{0.25,0.50,0.50}{##1}}}
\expandafter\def\csname PY@tok@cp\endcsname{\def\PY@tc##1{\textcolor[rgb]{0.74,0.48,0.00}{##1}}}
\expandafter\def\csname PY@tok@k\endcsname{\let\PY@bf=\textbf\def\PY@tc##1{\textcolor[rgb]{0.00,0.50,0.00}{##1}}}
\expandafter\def\csname PY@tok@kp\endcsname{\def\PY@tc##1{\textcolor[rgb]{0.00,0.50,0.00}{##1}}}
\expandafter\def\csname PY@tok@kt\endcsname{\def\PY@tc##1{\textcolor[rgb]{0.69,0.00,0.25}{##1}}}
\expandafter\def\csname PY@tok@o\endcsname{\def\PY@tc##1{\textcolor[rgb]{0.40,0.40,0.40}{##1}}}
\expandafter\def\csname PY@tok@ow\endcsname{\let\PY@bf=\textbf\def\PY@tc##1{\textcolor[rgb]{0.67,0.13,1.00}{##1}}}
\expandafter\def\csname PY@tok@nb\endcsname{\def\PY@tc##1{\textcolor[rgb]{0.00,0.50,0.00}{##1}}}
\expandafter\def\csname PY@tok@nf\endcsname{\def\PY@tc##1{\textcolor[rgb]{0.00,0.00,1.00}{##1}}}
\expandafter\def\csname PY@tok@nc\endcsname{\let\PY@bf=\textbf\def\PY@tc##1{\textcolor[rgb]{0.00,0.00,1.00}{##1}}}
\expandafter\def\csname PY@tok@nn\endcsname{\let\PY@bf=\textbf\def\PY@tc##1{\textcolor[rgb]{0.00,0.00,1.00}{##1}}}
\expandafter\def\csname PY@tok@ne\endcsname{\let\PY@bf=\textbf\def\PY@tc##1{\textcolor[rgb]{0.82,0.25,0.23}{##1}}}
\expandafter\def\csname PY@tok@nv\endcsname{\def\PY@tc##1{\textcolor[rgb]{0.10,0.09,0.49}{##1}}}
\expandafter\def\csname PY@tok@no\endcsname{\def\PY@tc##1{\textcolor[rgb]{0.53,0.00,0.00}{##1}}}
\expandafter\def\csname PY@tok@nl\endcsname{\def\PY@tc##1{\textcolor[rgb]{0.63,0.63,0.00}{##1}}}
\expandafter\def\csname PY@tok@ni\endcsname{\let\PY@bf=\textbf\def\PY@tc##1{\textcolor[rgb]{0.60,0.60,0.60}{##1}}}
\expandafter\def\csname PY@tok@na\endcsname{\def\PY@tc##1{\textcolor[rgb]{0.49,0.56,0.16}{##1}}}
\expandafter\def\csname PY@tok@nt\endcsname{\let\PY@bf=\textbf\def\PY@tc##1{\textcolor[rgb]{0.00,0.50,0.00}{##1}}}
\expandafter\def\csname PY@tok@nd\endcsname{\def\PY@tc##1{\textcolor[rgb]{0.67,0.13,1.00}{##1}}}
\expandafter\def\csname PY@tok@s\endcsname{\def\PY@tc##1{\textcolor[rgb]{0.73,0.13,0.13}{##1}}}
\expandafter\def\csname PY@tok@sd\endcsname{\let\PY@it=\textit\def\PY@tc##1{\textcolor[rgb]{0.73,0.13,0.13}{##1}}}
\expandafter\def\csname PY@tok@si\endcsname{\let\PY@bf=\textbf\def\PY@tc##1{\textcolor[rgb]{0.73,0.40,0.53}{##1}}}
\expandafter\def\csname PY@tok@se\endcsname{\let\PY@bf=\textbf\def\PY@tc##1{\textcolor[rgb]{0.73,0.40,0.13}{##1}}}
\expandafter\def\csname PY@tok@sr\endcsname{\def\PY@tc##1{\textcolor[rgb]{0.73,0.40,0.53}{##1}}}
\expandafter\def\csname PY@tok@ss\endcsname{\def\PY@tc##1{\textcolor[rgb]{0.10,0.09,0.49}{##1}}}
\expandafter\def\csname PY@tok@sx\endcsname{\def\PY@tc##1{\textcolor[rgb]{0.00,0.50,0.00}{##1}}}
\expandafter\def\csname PY@tok@m\endcsname{\def\PY@tc##1{\textcolor[rgb]{0.40,0.40,0.40}{##1}}}
\expandafter\def\csname PY@tok@gh\endcsname{\let\PY@bf=\textbf\def\PY@tc##1{\textcolor[rgb]{0.00,0.00,0.50}{##1}}}
\expandafter\def\csname PY@tok@gu\endcsname{\let\PY@bf=\textbf\def\PY@tc##1{\textcolor[rgb]{0.50,0.00,0.50}{##1}}}
\expandafter\def\csname PY@tok@gd\endcsname{\def\PY@tc##1{\textcolor[rgb]{0.63,0.00,0.00}{##1}}}
\expandafter\def\csname PY@tok@gi\endcsname{\def\PY@tc##1{\textcolor[rgb]{0.00,0.63,0.00}{##1}}}
\expandafter\def\csname PY@tok@gr\endcsname{\def\PY@tc##1{\textcolor[rgb]{1.00,0.00,0.00}{##1}}}
\expandafter\def\csname PY@tok@ge\endcsname{\let\PY@it=\textit}
\expandafter\def\csname PY@tok@gs\endcsname{\let\PY@bf=\textbf}
\expandafter\def\csname PY@tok@gp\endcsname{\let\PY@bf=\textbf\def\PY@tc##1{\textcolor[rgb]{0.00,0.00,0.50}{##1}}}
\expandafter\def\csname PY@tok@go\endcsname{\def\PY@tc##1{\textcolor[rgb]{0.53,0.53,0.53}{##1}}}
\expandafter\def\csname PY@tok@gt\endcsname{\def\PY@tc##1{\textcolor[rgb]{0.00,0.27,0.87}{##1}}}
\expandafter\def\csname PY@tok@err\endcsname{\def\PY@bc##1{\setlength{\fboxsep}{0pt}\fcolorbox[rgb]{1.00,0.00,0.00}{1,1,1}{\strut ##1}}}
\expandafter\def\csname PY@tok@kc\endcsname{\let\PY@bf=\textbf\def\PY@tc##1{\textcolor[rgb]{0.00,0.50,0.00}{##1}}}
\expandafter\def\csname PY@tok@kd\endcsname{\let\PY@bf=\textbf\def\PY@tc##1{\textcolor[rgb]{0.00,0.50,0.00}{##1}}}
\expandafter\def\csname PY@tok@kn\endcsname{\let\PY@bf=\textbf\def\PY@tc##1{\textcolor[rgb]{0.00,0.50,0.00}{##1}}}
\expandafter\def\csname PY@tok@kr\endcsname{\let\PY@bf=\textbf\def\PY@tc##1{\textcolor[rgb]{0.00,0.50,0.00}{##1}}}
\expandafter\def\csname PY@tok@bp\endcsname{\def\PY@tc##1{\textcolor[rgb]{0.00,0.50,0.00}{##1}}}
\expandafter\def\csname PY@tok@fm\endcsname{\def\PY@tc##1{\textcolor[rgb]{0.00,0.00,1.00}{##1}}}
\expandafter\def\csname PY@tok@vc\endcsname{\def\PY@tc##1{\textcolor[rgb]{0.10,0.09,0.49}{##1}}}
\expandafter\def\csname PY@tok@vg\endcsname{\def\PY@tc##1{\textcolor[rgb]{0.10,0.09,0.49}{##1}}}
\expandafter\def\csname PY@tok@vi\endcsname{\def\PY@tc##1{\textcolor[rgb]{0.10,0.09,0.49}{##1}}}
\expandafter\def\csname PY@tok@vm\endcsname{\def\PY@tc##1{\textcolor[rgb]{0.10,0.09,0.49}{##1}}}
\expandafter\def\csname PY@tok@sa\endcsname{\def\PY@tc##1{\textcolor[rgb]{0.73,0.13,0.13}{##1}}}
\expandafter\def\csname PY@tok@sb\endcsname{\def\PY@tc##1{\textcolor[rgb]{0.73,0.13,0.13}{##1}}}
\expandafter\def\csname PY@tok@sc\endcsname{\def\PY@tc##1{\textcolor[rgb]{0.73,0.13,0.13}{##1}}}
\expandafter\def\csname PY@tok@dl\endcsname{\def\PY@tc##1{\textcolor[rgb]{0.73,0.13,0.13}{##1}}}
\expandafter\def\csname PY@tok@s2\endcsname{\def\PY@tc##1{\textcolor[rgb]{0.73,0.13,0.13}{##1}}}
\expandafter\def\csname PY@tok@sh\endcsname{\def\PY@tc##1{\textcolor[rgb]{0.73,0.13,0.13}{##1}}}
\expandafter\def\csname PY@tok@s1\endcsname{\def\PY@tc##1{\textcolor[rgb]{0.73,0.13,0.13}{##1}}}
\expandafter\def\csname PY@tok@mb\endcsname{\def\PY@tc##1{\textcolor[rgb]{0.40,0.40,0.40}{##1}}}
\expandafter\def\csname PY@tok@mf\endcsname{\def\PY@tc##1{\textcolor[rgb]{0.40,0.40,0.40}{##1}}}
\expandafter\def\csname PY@tok@mh\endcsname{\def\PY@tc##1{\textcolor[rgb]{0.40,0.40,0.40}{##1}}}
\expandafter\def\csname PY@tok@mi\endcsname{\def\PY@tc##1{\textcolor[rgb]{0.40,0.40,0.40}{##1}}}
\expandafter\def\csname PY@tok@il\endcsname{\def\PY@tc##1{\textcolor[rgb]{0.40,0.40,0.40}{##1}}}
\expandafter\def\csname PY@tok@mo\endcsname{\def\PY@tc##1{\textcolor[rgb]{0.40,0.40,0.40}{##1}}}
\expandafter\def\csname PY@tok@ch\endcsname{\let\PY@it=\textit\def\PY@tc##1{\textcolor[rgb]{0.25,0.50,0.50}{##1}}}
\expandafter\def\csname PY@tok@cm\endcsname{\let\PY@it=\textit\def\PY@tc##1{\textcolor[rgb]{0.25,0.50,0.50}{##1}}}
\expandafter\def\csname PY@tok@cpf\endcsname{\let\PY@it=\textit\def\PY@tc##1{\textcolor[rgb]{0.25,0.50,0.50}{##1}}}
\expandafter\def\csname PY@tok@c1\endcsname{\let\PY@it=\textit\def\PY@tc##1{\textcolor[rgb]{0.25,0.50,0.50}{##1}}}
\expandafter\def\csname PY@tok@cs\endcsname{\let\PY@it=\textit\def\PY@tc##1{\textcolor[rgb]{0.25,0.50,0.50}{##1}}}

\def\PYZbs{\char`\\}
\def\PYZus{\char`\_}
\def\PYZob{\char`\{}
\def\PYZcb{\char`\}}
\def\PYZca{\char`\^}
\def\PYZam{\char`\&}
\def\PYZlt{\char`\<}
\def\PYZgt{\char`\>}
\def\PYZsh{\char`\#}
\def\PYZpc{\char`\%}
\def\PYZdl{\char`\$}
\def\PYZhy{\char`\-}
\def\PYZsq{\char`\'}
\def\PYZdq{\char`\"}
\def\PYZti{\char`\~}
% for compatibility with earlier versions
\def\PYZat{@}
\def\PYZlb{[}
\def\PYZrb{]}
\makeatother

% For linebreaks inside Verbatim environment from package fancyvrb. 
\makeatletter
	\newbox\Wrappedcontinuationbox 
	\newbox\Wrappedvisiblespacebox 
	\newcommand*\Wrappedvisiblespace {\textcolor{red}{\textvisiblespace}} 
	\newcommand*\Wrappedcontinuationsymbol {\textcolor{red}{\llap{\tiny$\m@th\hookrightarrow$}}} 
	\newcommand*\Wrappedcontinuationindent {3ex } 
	\newcommand*\Wrappedafterbreak {\kern\Wrappedcontinuationindent\copy\Wrappedcontinuationbox} 
	% Take advantage of the already applied Pygments mark-up to insert 
	% potential linebreaks for TeX processing. 
	%        {, <, #, %, $, ' and ": go to next line. 
	%        _, }, ^, &, >, - and ~: stay at end of broken line. 
	% Use of \textquotesingle for straight quote. 
	\newcommand*\Wrappedbreaksatspecials {% 
		\def\PYGZus{\discretionary{\char`\_}{\Wrappedafterbreak}{\char`\_}}% 
		\def\PYGZob{\discretionary{}{\Wrappedafterbreak\char`\{}{\char`\{}}% 
		\def\PYGZcb{\discretionary{\char`\}}{\Wrappedafterbreak}{\char`\}}}% 
		\def\PYGZca{\discretionary{\char`\^}{\Wrappedafterbreak}{\char`\^}}% 
		\def\PYGZam{\discretionary{\char`\&}{\Wrappedafterbreak}{\char`\&}}% 
		\def\PYGZlt{\discretionary{}{\Wrappedafterbreak\char`\<}{\char`\<}}% 
		\def\PYGZgt{\discretionary{\char`\>}{\Wrappedafterbreak}{\char`\>}}% 
		\def\PYGZsh{\discretionary{}{\Wrappedafterbreak\char`\#}{\char`\#}}% 
		\def\PYGZpc{\discretionary{}{\Wrappedafterbreak\char`\%}{\char`\%}}% 
		\def\PYGZdl{\discretionary{}{\Wrappedafterbreak\char`\$}{\char`\$}}% 
		\def\PYGZhy{\discretionary{\char`\-}{\Wrappedafterbreak}{\char`\-}}% 
		\def\PYGZsq{\discretionary{}{\Wrappedafterbreak\textquotesingle}{\textquotesingle}}% 
		\def\PYGZdq{\discretionary{}{\Wrappedafterbreak\char`\"}{\char`\"}}% 
		\def\PYGZti{\discretionary{\char`\~}{\Wrappedafterbreak}{\char`\~}}% 
	} 
	% Some characters . , ; ? ! / are not pygmentized. 
	% This macro makes them "active" and they will insert potential linebreaks 
	\newcommand*\Wrappedbreaksatpunct {% 
		\lccode`\~`\.\lowercase{\def~}{\discretionary{\hbox{\char`\.}}{\Wrappedafterbreak}{\hbox{\char`\.}}}% 
		\lccode`\~`\,\lowercase{\def~}{\discretionary{\hbox{\char`\,}}{\Wrappedafterbreak}{\hbox{\char`\,}}}% 
		\lccode`\~`\;\lowercase{\def~}{\discretionary{\hbox{\char`\;}}{\Wrappedafterbreak}{\hbox{\char`\;}}}% 
		\lccode`\~`\:\lowercase{\def~}{\discretionary{\hbox{\char`\:}}{\Wrappedafterbreak}{\hbox{\char`\:}}}% 
		\lccode`\~`\?\lowercase{\def~}{\discretionary{\hbox{\char`\?}}{\Wrappedafterbreak}{\hbox{\char`\?}}}% 
		\lccode`\~`\!\lowercase{\def~}{\discretionary{\hbox{\char`\!}}{\Wrappedafterbreak}{\hbox{\char`\!}}}% 
		\lccode`\~`\/\lowercase{\def~}{\discretionary{\hbox{\char`\/}}{\Wrappedafterbreak}{\hbox{\char`\/}}}% 
		\catcode`\.\active
		\catcode`\,\active 
		\catcode`\;\active
		\catcode`\:\active
		\catcode`\?\active
		\catcode`\!\active
		\catcode`\/\active 
		\lccode`\~`\~ 	
	}
\makeatother

\let\OriginalVerbatim=\Verbatim
\makeatletter
\renewcommand{\Verbatim}[1][1]{%
	%\parskip\z@skip
	\sbox\Wrappedcontinuationbox {\Wrappedcontinuationsymbol}%
	\sbox\Wrappedvisiblespacebox {\FV@SetupFont\Wrappedvisiblespace}%
	\def\FancyVerbFormatLine ##1{\hsize\linewidth
		\vtop{\raggedright\hyphenpenalty\z@\exhyphenpenalty\z@
			\doublehyphendemerits\z@\finalhyphendemerits\z@
			\strut ##1\strut}%
	}%
	% If the linebreak is at a space, the latter will be displayed as visible
	% space at end of first line, and a continuation symbol starts next line.
	% Stretch/shrink are however usually zero for typewriter font.
	\def\FV@Space {%
		\nobreak\hskip\z@ plus\fontdimen3\font minus\fontdimen4\font
		\discretionary{\copy\Wrappedvisiblespacebox}{\Wrappedafterbreak}
		{\kern\fontdimen2\font}%
	}%
	
	% Allow breaks at special characters using \PYG... macros.
	\Wrappedbreaksatspecials
	% Breaks at punctuation characters . , ; ? ! and / need catcode=\active 	
	\OriginalVerbatim[#1,codes*=\Wrappedbreaksatpunct]%
}
\makeatother


% prompt
\makeatletter
\newcommand{\boxspacing}{\kern\kvtcb@left@rule\kern\kvtcb@boxsep}
\makeatother
\newcommand{\prompt}[4]{
	\ttfamily\llap{{\color{#2}[#3]:\hspace{3pt}#4}}\vspace{-\baselineskip}
}
    

% Prevent overflowing lines due to hard-to-break entities
\sloppy 

% Setup hyperref package
\hypersetup{
  breaklinks=true,  % so long urls are correctly broken across lines
  colorlinks=true,
  urlcolor=urlcolor,
  linkcolor=linkcolor,
  citecolor=citecolor,
  pdfauthor={John D. Baker},
  pdftitle={Analyze the Data not the Drivel},
  pdfsubject={Blog},
  pdfcreator={MikTeX+LaTeXe},
  pdfkeywords={blog,wordpress},
  }
  
% Slightly bigger margins than the latex defaults
% \geometry{verbose,tmargin=1in,bmargin=1in,lmargin=1in,rmargin=1in}  

%\usepackage{wrapfig}

% source code listings
\usepackage{listings}

\lstdefinelanguage{bat}
{morekeywords={echo,title,pushd,popd,setlocal,endlocal,off,if,not,exist,set,goto,pause},
sensitive=True,
morecomment=[l]{rem}
}

\lstdefinelanguage{jdoc}
{
morekeywords={},
otherkeywords={assert.,break.,continue.,for.,do.,if.,else.,elseif.,return.,select.,end.
,while.,whilst.,throw.,catch.,catchd.,catcht.,try.,case.,fcase.},
sensitive=True,
morecomment=[l]{NB.},
morestring=[b]',
morestring=[d]',
}

% latex size ordering - can never remember it
% \tiny
% \scriptsize
% \footnotesize
% \small
% \normalsize
% \large
% \Large
% \LARGE
% \huge
% \Huge
 
% listings package settings  
\lstset{%
  language=jdoc,                                % j document settings
  basicstyle=\ttfamily\footnotesize,            
  keywordstyle=\bfseries\color{keywcolor}\footnotesize,
  identifierstyle=\color{black},
  commentstyle=\slshape\color{CodeComment},     % colored slanted comments
  stringstyle=\color{red}\ttfamily,
  showstringspaces=false,                       
  %backgroundcolor=\color{CodeBackGround},       
  frame=single,                                
  framesep=1pt,                                 
  framerule=0.8pt,                             
  rulecolor=\color{CodeBackGround},   
  showspaces=false,
  %columns=fullflexible,
  %numbers=left,
  %numberstyle=\footnotesize,
  %numbersep=9pt,
  tabsize=2,
  showtabs=false,
  captionpos=b
  breaklines=true,                              
  breakindent=5pt                              
}

\lstdefinelanguage{JavaScript}{
  keywords={typeof, new, true, false, catch, function, return, null, catch, switch, var, if, in, while, do, else, case, break},
  ndkeywords={class, export, boolean, throw, implements, import, this},
  ndkeywordstyle=\color{darkgray}\bfseries,
  sensitive=false,
  comment=[l]{//},
  morecomment=[s]{/*}{*/},
  morestring=[b]',
  morestring=[b]"
}

% C# settings
\lstdefinestyle{sharpc}{
language=[Sharp]C,
basicstyle=\ttfamily\scriptsize, 
keywordstyle=\bfseries\color{keywcolor}\scriptsize,
framerule=0pt
}

% for source code listing longer than two use smaller font
\lstdefinestyle{smallersource}{
basicstyle=\ttfamily\scriptsize, 
keywordstyle=\bfseries\color{keywcolor}\scriptsize,
framerule=0pt
}

\lstdefinestyle{resetdefaults}{
language=jdoc,
basicstyle=\ttfamily\footnotesize,  
keywordstyle=\bfseries\color{keywcolor}\footnotesize,                                                               
framerule=0.8pt 
}

% APL UTF8 code points listed for lstlisting processing
\makeatletter
\lst@InputCatcodes
\def\lst@DefEC{%
 \lst@CCECUse \lst@ProcessLetter
  ^^80^^81^^82^^83^^84^^85^^86^^87^^88^^89^^8a^^8b^^8c^^8d^^8e^^8f%
  ^^90^^91^^92^^93^^94^^95^^96^^97^^98^^99^^9a^^9b^^9c^^9d^^9e^^9f%
  ^^a0^^a1^^a2^^a3^^a4^^a5^^a6^^a7^^a8^^a9^^aa^^ab^^ac^^ad^^ae^^af%
  ^^b0^^b1^^b2^^b3^^b4^^b5^^b6^^b7^^b8^^b9^^ba^^bb^^bc^^bd^^be^^bf%
  ^^c0^^c1^^c2^^c3^^c4^^c5^^c6^^c7^^c8^^c9^^ca^^cb^^cc^^cd^^ce^^cf%
  ^^d0^^d1^^d2^^d3^^d4^^d5^^d6^^d7^^d8^^d9^^da^^db^^dc^^dd^^de^^df%
  ^^e0^^e1^^e2^^e3^^e4^^e5^^e6^^e7^^e8^^e9^^ea^^eb^^ec^^ed^^ee^^ef%
  ^^f0^^f1^^f2^^f3^^f4^^f5^^f6^^f7^^f8^^f9^^fa^^fb^^fc^^fd^^fe^^ff%
  ^^^^20ac^^^^0153^^^^0152%
  ^^^^20a7^^^^2190^^^^2191^^^^2192^^^^2193^^^^2206^^^^2207^^^^220a%
  ^^^^2218^^^^2228^^^^2229^^^^222a^^^^2235^^^^223c^^^^2260^^^^2261%
  ^^^^2262^^^^2264^^^^2265^^^^2282^^^^2283^^^^2296^^^^22a2^^^^22a3%
  ^^^^22a4^^^^22a5^^^^22c4^^^^2308^^^^230a^^^^2336^^^^2337^^^^2339%
  ^^^^233b^^^^233d^^^^233f^^^^2340^^^^2342^^^^2347^^^^2348^^^^2349%
  ^^^^234b^^^^234e^^^^2350^^^^2352^^^^2355^^^^2357^^^^2359^^^^235d%
  ^^^^235e^^^^235f^^^^2361^^^^2362^^^^2363^^^^2364^^^^2365^^^^2368%
  ^^^^236a^^^^236b^^^^236c^^^^2371^^^^2372^^^^2373^^^^2374^^^^2375%
  ^^^^2377^^^^2378^^^^237a^^^^2395^^^^25af^^^^25ca^^^^25cb%  
  ^^00}
\lst@RestoreCatcodes
\makeatother

% custom lengths used within minipages
\newcommand{\minindent}{17pt}

\makeindex

\begin{document}

\subsection*{\href{http://analyzethedatanotthedrivel.org/2025/03/16/gonggone-gone-parts-1-2/}{Gonggone Gone --- Parts 1 \& 2}}
\addcontentsline{toc}{subsection}{Gonggone Gone --- Parts 1 \& 2}


\noindent\emph{Posted: 16 Mar 2025 18:36:44}
\vspace{6pt}

I'm back, did you miss me?

If you're foolish enough to follow this blog and even more foolish to
care about its author, you may have wondered what I've been up to for
the last umpteen months --- even I've forgotten when I made my previous
entry. I've been busy indulging myself in writing \emph{long-form}
fiction.

Before the end of last year, I chanced upon, or rather the YouTube
algorithm shoved in my face, a cute little video about the Kuiper belt
object
\href{https://duckduckgo.com/?q=gonggong&t=brave&iax=videos&iai=https\%3A\%2F\%2Fwww.youtube.com\%2Fwatch\%3Fv\%3DNQUMmEMOk-c&ia=videos}{Gonggong}.
I think we can all agree ``Gonggong'' is the coolest dwarf planet name
ever. As I motor-mouthed Gonggong, \emph{Gonggong Gone} popped out.
Sweet alliteration, John; you should write a story. I had a sweet title
but nothing else. Then, maybe thinking about
\emph{\href{https://www.imdb.com/title/tt7605074/}{The Wandering
Earth,}} a ridiculous Chinese science fiction movie about building
millions of giant fusion thrusters on Earth, canceling Earth's rotation,
and propelling the entire planet out of the solar system (and you
thought
\href{https://www.imdb.com/title/tt1190080/?ref_=fn_all_ttl_1}{\emph{2012}}
sucked). I wondered what it would be like if Earth suddenly stopped
\emph{feeling} solar gravity. \emph{Gonggong Gone got going.}

While writing the story, I tried some new things. Most of my blog posts
start as barely legible longhand scribbles. Despite a lifetime of
writing, editing, and executing code online, I still write with pen and
paper; it helps me get things out without pursuing typographic chimeras.
I usually resort to longhand when I'm stuck, and the words will not
come, but whenever I ink, I know I'll soon face a big transcription and
editing job. For \emph{Gonggong Gone}, I wrote the whole thing, all
25,000 words, on my computer.

I use \emph{Word} on both Windows and Mac systems mostly because I know
it well enough to get on with writing and because \emph{Word} hosts a
variety of helpful writing aids. I found
\href{https://www.grammarly.com/}{Grammarly} and various AI chatbots
helpful. \emph{Grammarly} fixes spelling errors and does a good job of
punctuating sentences. I am a piss poor punctuator. \emph{Grammarly}
also suggests alternate ways of laying out the clauses in your
sentences. I sometimes take its suggestions and sometimes not.
\emph{Grammarly} doesn't seem to pay a lot of attention to paragraph
context.

AI chatbots are capable of spewing out entire stories when prompted.
Unfortunately, their stories are deeply derivative and don't sound like
you. If you're writing, use your own \emph{voice}. Chatbots shine when
responding to precise prompts like, ``Rewrite this paragraph without the
word `was.'\,'' They will take your text and often find good
context-sensitive alternatives, and even better when used surgically
like this, they don't change your \emph{voice}.

While \emph{Word} met most of my needs, \emph{Gonggong Gone's} length
quickly exposed some shortcomings. \emph{Grammarly} does not handle
large documents. \emph{Word} files, being binaries, do not ``diff'' and
version well. This was a problem. Fortunately, your fearless
correspondent is a longtime LaTeX user. LaTeX scales to absurdly huge
and complex documents. It's geared to academic and mathematical writing,
so handling vanilla text is almost beneath it. I fixed my scale and diff
problems with \href{https://pandoc.org/}{Pandoc},
\href{https://github.com/}{GitHub}, and
\href{https://wiki.jsoftware.com/wiki/Guides/GettingStarted}{some custom
J}, bash, and batch scripts.

While \emph{Gonggong Gone} turned out OK, I believe the best thing to
come out of this project is my template for composing and managing very
long text documents. I will use
\href{https://github.com/bakerjd99/jackshacks/commit/180557a189cf77b256d8540e0a964d261dd25953\#diff-5af772d764c0ebbfe2aca599489c07ab689f7d78f4c24c7375d93ba3101c44f6}{this
J script} to create working story directories.

Now, here are the first two sections of \emph{Gonggong Gone.} I ripped
off the idea of descriptive section titles from Cormac McCarthy's
\href{https://www.goodreads.com/book/show/394535.Blood_Meridian_or_the_Evening_Redness_in_the_West}{Blood
Meridian}. Always steal from the best.

\begin{center}\Large\textbf{Gonggong Gone}\normalsize\end{center}

\begin{center}\large\textbf{-- \emph{an occultation} --}\normalsize\end{center}

Alex Burdick arrived at his observing spot as the Sun dropped below the
horizon. Stepping out of his old SUV, he studied the deep orange and
wine-red sunset. Vivid horizon colors indicate superb air transparency,
low relative humidity, and dust-free winds. Such signs augur excellent
sky-watching conditions. He took a few minutes to enjoy the deepening
dusk. To the northeast, the ever-darkening Owyhee mountains blocked the
lights of Boise, Idaho, and its ever-expanding, light-polluting suburbs.
He stood only 120 kilometers from Boise, not far enough to escape the
city's lights. From here, on new moon nights, he could see Boise's light
dome creeping about five degrees over the crests of the Owyhee's. Boise
was the only nearby city. The next closest source of light pollution,
the small town of Winnemucca, Nevada, sat about two hundred kilometers
to the southwest.

Tonight, the weather looked promising, but the bright half-crescent Moon
did not. Alex usually skipped moonlit nights. He was a deep-sky guy.
Alex enjoyed hunting down faint galaxies with his Dobsonian reflector
telescope, and bright moonlit skies got in the way. Moonlight would not
be a problem tonight. He came to see the Moon block or \emph{occult},
the bright star Spica.

He'd been sky-watching for decades but had never seen a bright star
occultation. Unlike hunting for faint galaxies, bright star occultations
are no-brainers. Look for the Moon, duh, and find the bright star near
its disk. Wait and watch. If clouds don't get in the way, you'll see the
star wink out. If the wink is on the dark side of the crescent, it's
dramatic. If the weather holds, you'll see the same star wink back in.
You don't need any equipment. Your eyes will do. For tonight, Alex
didn't pack his Dobsonian; he brought two pairs of binoculars and his
iPhone with its sky chart app. This evening would be a low-energy
affair. He'd sit on his folding observing chair, watch the occultation,
wait for the Moon to set, bathe in Milky Way light, and sleep in his
car. In the morning, he would drive back to Boise to finalize his
divorce.

Alex dragged his folding chair into a small circular patch of bare
ground near his parked car. The patch was surrounded by tall sage
plants, filling the air with their perfect odor. In addition to
perfuming the air, the sage helped block car headlights. Alex shared his
observing spot with other members of the Boise Astronomical Society.
Most nights, he was alone, but sometimes, others would show up and blind
him with their headlights. The first rule of naked eye observing is to
protect your night-adapted eyes.

With his chair pointed toward the Moon, Alex returned to his car and dug
out his graphene-heated jacket. When he first heard about battery-heated
jackets, he thought they were scams, but online reviews were positive.
Some of his fellow amateur astronomers said they worked, so Alex forked
over a few hundred bucks and tried the best-reviewed one. It worked so
well he spent another hundred bucks to get a second jacket battery. Alex
had spent many long hours sitting in the dark watching stars. In Idaho,
even summer nights could get chilly. With snow on the ground sitting in
the dark beside his Dobsonian or astrograph could get damn unpleasant.
The heated graphene jacket, especially its hand-warming pockets, which
he used to warm eyepieces, turned cold nights into beach days. And, when
Alex learned mid-20th century Mount Wilson astronomers used electrically
heated flight jackets while sitting at the prime focus and manually
guiding the 100-inch telescope, he didn't even feel like a \emph{girly
man for pussying it out} in an electric jacket.

Slipping the jacket on, he returned to his chair and checked his
location on his phone.

The phone's GPS app coordinates did not match his previously recorded
observing position.

``What the hell?''

The GPS app was about ten kilometers off. He took several GPS positions,
and every single one of them differed from his stored position. The
worse readings had him about a hundred kilometers away. It was probably
Apple's recent iPhone upgrade. It only takes one pooched parameter to
wreck calculations. Somebody somewhere screwed up. No doubt Apple would
soon push an emergency software patch along with profuse apologies.
Let's hope planes don't fly into each other. Alex was tired of Apple's
bullshit\emph{,} and he was considering divorcing his iPhone.

Unable to get a consistent GPS reading, Alex used his previously stored
GPS coordinates to override the location settings on his phone's sky
chart app. He centered the app on the Moon and held up the dim red night
vision sky chart to the sky. Comparing the sky to the iPhone screen, it
seemed like Spica was farther from the Moon than it should have been.
Humans don't perceive small sky angles accurately. The apparent
discrepancy didn't bother Alex. It was two hours until the occultation.
He would keep an eye on the Moon and Spica while sitting and watching
the stars fill the sky.

While waiting, Alex traded text messages with \emph{The Occulters}.
\emph{The Occulters} were a haphazardly organized online group of
amateur astronomers trading real-time observations. Tonight, they were
scattered all over western North America and waiting for Spica to
disappear behind the Moon. Like Alex, many group members reported whacky
GPS readings. The GPS problem was widespread. Observers east of Alex
would see the event a few moments before him.

As the occultation time drew nearer, Alex texted, ``Spica is too far
from the Moon.''

Many seconded his observation. Nate, a well-equipped observer in
Cloudcroft, New Mexico, measured the angular separation from the center
of the Moon's disk to Spica. The angle was significantly greater than it
should have been. Alerted to the discrepancy, eastern observers prepared
to precisely time Spica's disappearance.

At the expected time of the occultation, Spica was still about half a
moon diameter from the Moon's limb. What?

Many observers sent frantic, ``Are you seeing this?'' messages.

\emph{The Occulters} watched in disbelief as the expected occultation
time came and went. \emph{The Moon never blocked Spica!} This wasn't a
software problem. Some checked printed versions of the \emph{RASC
Observers Handbook; sure} enough, Spica should have been blocked.
\emph{Something was out of whack with the heavens.}

Alex didn't know what to think. It was probably some longstanding
systematic error in occultation prediction math. Hell, it may be
Spica-specific! How else could astute editors of ephemeris tables miss
such a boner? There would be \emph{mea culpas} aplenty tomorrow, along
with numerous snide X posts about trusting the \emph{settled science}.
Others were equally puzzled, and some started checking the positions of
other objects. Nate, in Cloudcroft, claimed Mars's position was slightly
off. Oh, come on! Alex didn't want to get into it; he gave up and
retreated to a sleeping bag in his car.

Before going to sleep, his son Doug called.

``Hey, Dad, Mom called. She wanted me to remind you about the meeting
tomorrow.''

Annoyed, Alex said, ``I haven't forgotten.''

``How did the sky watching go?''

``The occultation didn't happen.''

``What? How's that possible? Did you get the date wrong?''

``No.~It wasn't just me. It's got to be some stupid mistake. It doesn't
matter. I'm going to turn off my phone. I don't want it pinging all
night long. I'll see you tomorrow.''

Alex worried about Doug. Life had been kicking them both around. Two
years ago, Alex's father died of prostate cancer. Alex sat by his side,
watching kids play on a cracked sidewalk outside his father's hospice
room, when his dad, doped up on strong painkillers but still in agony,
breathed his last. Six months after his dad's death, Alice, his wife of
twenty-nine years, left him because she was, in her words, ``tired of
his bullshit.'' While absorbing these shocks, the worst landed like a
trebuchet boulder. An illegal alien running a red light slammed into the
passenger side of Doug's car. Doug suffered minor injuries, but his
pregnant wife Ellen was severely injured. She lived long enough to ride
an expensive screaming ambulance to a nearby hospital, where she and her
unborn baby perished. The illegal alien who killed Ellen and her baby
didn't have a license or insurance. Immigration authorities arrested him
but inexplicably released him within days. He was never seen again.

Doug, numb with grief, exploded when unbelievable hospital and ambulance
bills arrived. He had started a new job, and his benefits had not
vested. Making things worse, Ellen had recently aged out of her parents'
useless ``Obongocare'' plan. Doug, a formidable 6-foot 5-inch muscular
man, foolishly confronted his new employer about his insurance. Alarmed,
they immediately fired him, citing his probationary status. Doug got
stuck paying many thousands of dollars in medical and funeral expenses.
Alex and Ellen's parents helped, but it was a major blow to all and
devastating for Doug.

Doug, depressed, unemployed, and bankrupt after paying Ellen's medical
and funeral expenses, moved back in with his father. For a few weeks
Doug looked for jobs but slowly gave up and spent his days sequestered
in his childhood bedroom playing X-Box games.

Alex encouraged him as much as he could, but he also boiled with
repressed rage. He was mad at Doug's shitty ex-employer, mad at
immigration authorities, mad at his ex-wife, mad at Doug for moping
about, and mad at himself for resenting the many thousands of dollars he
contributed to Doug's bills.

In the morning, Alex returned to Boise. Along the way, he saw many cars
parked in roadside turnouts. People were out of their cars and talking
to each other. Weird. Nobody willingly talks to strangers. At 10 Mile
and McMillan, the Walmart parking lot was almost filled. It was a
weekend, but it wasn't a holiday. He would have stopped, but he had a
meeting with his ex-wife and her dickhead lawyer, so he went home to
take a shower and get dressed. Alex didn't have to dress up and play the
role of \emph{responsible caregiver}. He would show up in a
ketchup-stained MAGA hoody because it annoyed his ex and her soy-boy
faggot California lawyer.

He wasn't home ten minutes before his ex-wife called and told him their
meeting was off. ``Turn on your TV,'' she commanded. Alex turned on the
TV. All the local channels were doing breaking news.

\emph{Tides had ceased worldwide!}

Numerous reporters from beaches, ports, and sea fronts all around the
globe were reporting tides didn't go out or come in as expected. The
most panicky report came from a frightened New Brunswick, Canada weather
lady. The famous Bay of Fundy tides had rolled out but only weakly
returned, only to immediately go out again and settle neither in nor
out.

Not only had the tides stopped, but communications with various
spacecraft were also lost. The James Webb Space Telescope, The Parker
Solar Probe, the Martian rovers, the Voyagers, and many others were
incommunicado. Even more damaging, all GPS satellite constellations were
reporting erroneous positions. The errors plagued American, Chinese,
Russian, and European systems. Also, internet satellites were
misbehaving. Ground connections were coming and going. TV news is
dominated by smug, supercilious imbeciles promenading their
holier-than-thou condescension to the world. Alex delighted in the white
fear on their faces.

``You dumb twats don't have a clue.''

He flipped over to YouTube and started doomscrolling X. Oh boy! His X
astronomy boys were hyped way up. The lost occultation had triggered a
tsunami of X posts. It wasn't just amateurs who noticed; many major
observatories also did. Reports asserted the positions of Venus and Mars
relative to Earth were off. Even more alarming, lunar laser ranging
experiments failed. Fortunately, some clever astronomers at the McDonald
Observatory in West Texas succeeded in retargeting their lunar ranging
laser after it stopped working. The new laser returns showed the Moon
rapidly receding from the Earth.

The \emph{McDonald Hack}, as it quickly became known on X, was simple.
Go into your favorite solar system simulator program and set the Earth's
mass to zero while maintaining its orbital speed and angular momentum.
Of course, this made no goddamn sense! It's arbitrarily flattening
spacetime around Earth and somehow insulating the entire planet from the
rest of the solar system. The net effect is deadly simple. It's like a
planet-slinging God, whipping the Earth around the Sun on an invisible
rope, had mysteriously let go.

Alex clicked around YouTube and saw Dr.~Rebecca had posted a short
video. Dr.~Rebecca, his favorite science YouTuber, was a young British
astrophysicist who cultivated an enormous YouTube following with her
excellent videos. Dr.~Rebecca's typically cheerful demeanor was very
subdued. Her evident concern alarmed Alex more than the predictable
unhinged prattle of mainstream media morons. Dr.~Rebecca quickly
summarized the situation. About sixteen hours ago, the Earth departed
from its solar orbit. The best-fixed star observations and repeated
radar ranging of the receding Moon indicated that the Earth no longer
orbited the Sun, and the Moon no longer orbited the Earth. Many
observers reported seeing previously hidden regions on the backside of
the Moon. And, as far as anyone could determine, Earth was flying
straight out of the solar system. Solar system gravity no longer
interacted with Earth. The gravity distortion extended at least as far
as geosynchronous satellites. Geosynchronous satellites were still
circling the Earth, but their orbits, as well as those of many other
satellites, were perturbed by the sudden absence of the Sun and Moon's
gravity. Dr.~Rebecca reiterated she and her colleagues had no
explanation for this unprecedented suspension of the laws of gravity.
Alex admired Dr.~Rebecca's analytic calm. He knew she knew how bad this
was. Alex's favorite \emph{X Meme}, of a stupid cartoon dog sitting in a
burning room, thinking, ``This is Fine,'' accurately summed things up.

After listening to Dr.~Rebecca, he flipped the TV back to the news.
Cable channels were, for once, \emph{correctly} obsessing on runaway
Earth. Fake news morons had a story worth covering. Runaway Earth
preempted everything, but even now, hours into the end of the world,
\emph{disparate impactor talking heads,} as he contemptuously called
them, were chasing the this is so much worse for gay, trans, black,
indigenous, bipolar, obese, and vegan folks. Incredibly, it was a
civilization ends and degenerates hardest hit, spin.

Thinking Doug needed to see this, he knocked on his son's bedroom door,
``Doug, get out here; you must see this!''

Doug grumbled but opened his bedroom door. Towering over his father,
Doug rubbed his hands through his long, unkempt blond hair and said,
``What?''

``Look at the tube.''

Doug watched the endlessly repeated news.

Two minutes later, he said, ``This can't be real; somebody is punking
hard.''

``I wouldn't have believed it either, but I've seen it. Last night's
occultation didn't happen \emph{because} the Earth wasn't where it
should have been.''

``That's impossible!''

``Apparently not.'

``If this is for real, what's going to happen?''

Alex hadn't \emph{seriously considered this} until Doug's question. ``If
things don't change, we'll all freeze to death.''

Saying it out loud settled Alex's mind; he knew what to do. This would
be his last and greatest observing session. He turned off the TV. You
don't listen to idiots in a crisis.

Turning to Doug, Alex said, ``We have to get out of here ---pronto.''

``Where would we go? Isn't the entire world screwed?''

``Remember the old mine in Grampa's valley? We can hole up there.''

The expression on Doug's face mixed calm alarm and suppressed
excitement, like he had been diagnosed with terminal cancer.

``What about mom?''

``She's made her choice. I'm asking you.''

Doug took a long look at his dad before saying, ``OK.''

``Good, we've got to pack as much as we can right away.''

They had to act quickly before the authorities made things worse. How
dipshits like FEMA could make things worse than a runaway planet would
require Herculean levels of \emph{fuck-ti-tude,} but Alex figured the
deep administrative state would rise to the occasion. Three things were
clear: farming was over, famine was coming, and it wouldn't be possible
to shelter everyone from the punishing cold. They rushed through the
house, grabbing suitcases and stuffing them with all their warm-weather
clothes. Looting closets, they packed their bedsheets, pillows,
blankets, and towels. Their clothes weren't up to cryogenic
temperatures, but it was too late for \emph{Amazon}. They'd grab more
clothes later. On the last run-through they packed the little electric
ceramic heater on Doug's bedroom windowsill; it might help where they
were going. They stashed their suitcases, camping equipment, kitchen
pots and pans, and toolboxes in the SUV.

While Doug loaded the car, Alex entered his mancave/office bedroom and
gathered all his blank notebooks, pens, and pencils. Alex kept diaries.
They were filled with inconsequential day-to-day drivel. For years, he
had daydreamed about writing a \emph{Pepys} diary: a \emph{magnum opus}
informing future generations about life in early 21st-century America.
Alex knew his odds of winning \emph{Power Ball} were better. Still,
diaries would get a lot more interesting with the world ending! Maybe
he'd write a Pepys diary after all. Too bad there would be no future
generations to read it.

It didn't matter now. Before doing anything else, Alex opened a blank
notebook, wrote the date and time, and quickly scribbled a list of
things to get. Knowing his list was incomplete, he left a few blank
pages before writing. \emph{Life is over for everyone now!}

Closing the notebook, Alex piled it with the others and fetched his
wheeled carry-on bag. He loaded the carry-on with notebooks, pens,
pencils, and his laptop. Running around, Alex added flashlights and
rechargeable batteries to his carry-on. After filling the carry-on, he
got out his small refracting telescope with its lightweight carbon fiber
tripod and its CCD astro cam. The small telescope was a high-quality
astrograph, a refractor optimized for imaging. It had cost thousands of
dollars, and he bought it right after his wife left. No more budgeting
for the bitch. The little telescope and its accessories fit in a neat
backpack. The sky would be fascinating as Earth died, and Alex resolved
to see as much of it as possible. He put his telescope backpack and
carry-on bag in the back seat of the SUV.

In a final flourish, they emptied the house of paper, mostly printer
paper, but, in a nod to past hysteria, a big bag of toilet paper, too.
In their frenzy, they almost forgot about toothbrushes, razor blades,
floss, and deodorant. Personal hygiene was over: more bad news for gay
hairdressers and lesbian snatch waxers. The last few things they took
were books. What do you read when the world freezes? It would have made
an excellent writer's group discussion topic, but there was no time.
Doug stuffed a pile of old Manga in the SUV. At the same time, Alex
packed \emph{The Ultimate Survival Medicine Guide,} an old \emph{CRC
Engineering Handbook}, \emph{Moby Dick}, Joyce's \emph{Ulysses}, two
cookbooks, his \emph{Tirion Sky Atlas 2000.0}, a stack of old
\emph{Royal Astronomical Society Observer's Handbooks}, and because he
was currently reading it, \emph{Selected Non-fictions} by Jorge Luis
Borges.

With the car loaded, Alex and Doug squeezed into the front seats.

``Now what?'' Doug asked.

``We need to do a little shopping.''

%\subsection{apocalypse shopping}\label{apocalypse-shopping}
\begin{center}\large\textbf{-- \emph{apocalypse shopping} --}\normalsize\end{center}

Alex and Doug drove straight to a truck rental dealer. Alex hoped the
astronomically ill-informed public had not connected the dots. Once they
did, he expected complete shopping bedlam. It wouldn't be like the
comical COVID toilet paper runs of a few years ago or typical Black
Friday ghetto riots. No, this would be serious, shoot-to-kill shit. Alex
figured they had about a week. After a week, the missing tides, the
runaway moon, and the noticeably smaller solar disk would drive it home
to even the dumbest shits on the planet; this wasn't a lefty,
right-wing, or Zionist plot. That billionaires weren't conniving to rob
you. That commies, nazis, transphobes, Trump voters, or World Economic
Forum stooges weren't responsible for the sudden annulment of solar
gravity. That everyone on Earth, every poor, rich, smart, or stupid
son-of-a-bitch was trapped on runaway spaceship Earth.

They rented a big king cab pickup and a large trailer to drag behind it
for a month. As a sick joke, Alex took out the deluxe rental insurance
package. Not only was farming over, but credit would soon follow. This
month, he wouldn't pay his credit card bills, and it was unlikely the
banks or anyone would care. By the end of the month, everyone would have
bigger problems. After hitching up the trailer and transferring their
gear from the car to the truck, Alex returned to the rental office and
asked the salesman if he could leave his car outside for the day. He
told the salesman they'd return and gave him his cell phone number. With
their rides sorted, they drove the truck and trailer straight to Alex's
bank near Fairview and Meridian Avenue.

Relieved to find the bank open, Alex withdrew twenty-three thousand
dollars in twenties and hundreds. He had to wipe out his savings and max
out his line of credit. As usual, the bank made you feel guilty for
cashing out your own damn money. An annoyance they compounded by
insisting he sign the idiotic IRS --- \emph{I am not in the minor sex
trafficking and fentanyl smuggling business} --- tax forms. Alex
despised the toady little deep-state dick-fellating fools in banks, but
they needed cash. How long would the credit-based economy last? Besides,
using cash for \emph{sensitive purchases} is always a good idea, and
today, everything is sensitive.

Before leaving the bank's parking lot, Alex amended the shopping list he
started in his bedroom and asked Doug to check it.

``Anything missing?''

``There are no sewing supplies. Sewing is an underappreciated survival
skill.''

``Really, OK. Trouble is I can't sew shit.''

``I can.''

``Since when?''

``Ellen taught me before \ldots'' Doug looked away, but Alex saw his
eyes tear up.

``OK then. We'll follow Ellen's lead here.''

Satisfied with their list, they drove to the restaurant supply
wholesaler, US Foods CHEF'STORE, on North Hickory Avenue. Wholesalers
were more likely to have bulk nonperishable foods than ordinary grocery
stores. They filled the truck bed with 50-pound bags of beans, rice,
flour, sugar, and cases of canned meats, vegetables, and fruits. At two
pounds per day per person, they had a four-year supply.

From US Foods, they drove to Scheels on Wayfinder.

Pulling into the parking lot, Alex handed Doug five thousand dollars and
said, ``I'm looking for a specialty item in the gun shop. Take this and
stock up on the warmest cold-weather outfits you can find. Get at least
three complete outfits for each of us: boots, parkas, mittens, heavily
insulated ski pants, super warm base layers, everything. Keeping dry is
easier if you have many outfits.''

Scheels has an excellent gun shop. Alex wasn't loading up on firearms
and ammo like brain-dead head-for-the-hills preppers. Preppers are
clueless dolts who think they'll prevail in firefights against roving
bands of post-apocalypse vampire scavengers because every dumb-ass
Netflix end-times series says they will. Some people are too stupid to
live, and soon, due to the Earth's sudden release, preppers wouldn't
have to.

Alex sought large silica gel canisters. He grabbed all the display
canisters and a small gauge hygrometer to measure humidity. Wanting
more, Alex asked a salesman if they had additional canisters in storage.
While the salesman fetched a large case of two dozen kilogram canisters,
Alex browsed rifle scopes. As an astronomical telescope man, they
impressed him with their edge-to-edge sharpness; hunters have high
refractor standards. He almost bought a rifle but knew it would be of
limited utility\emph{;} he wasn't planning on getting into prepper
firefights. He picked up two 9mm pistols and, almost as an afterthought,
\emph{just} enough bullets to deter \emph{maybe one post-apocalypse
zombie.}

To buy the pistols, Alex needed an instant Idaho background check.
Approvals were automatic if you didn't have outstanding warrants or
restraining orders. Both he and the salesman exchanged this is annoying
commie government bullshit eye rolls while waiting for the approval.
Once approved, he paid in cash and ferried the pistols and silica gel
canisters to the trailer, where he found Doug stacking his clothing
purchases.

``I got extra mittens, boot liners, and some ski masks. Oh, I also
grabbed half a dozen nylon down-filled comforters. With alterations, we
can wear them over parkas.''

``Looks like Ellen's sewing lessons will be handy. Good thinking.''

Returning to the store, they got six cold-weather sleeping bags, twelve
large four-liter steel thermos jugs, two propane camp stoves, and scores
of \emph{these might be useful} items. Among the items that might be
useful, Doug got four bicycle helmets.

When Alex gave him a \emph{what} look, Doug said, ``I'm not hitting my
head on my shafts.''

After storing their second haul, they returned and filled six shopping
carts with freeze-dried camping food, dozens of yoga mats, and
multicolored pool noodles. On the way to the checkout, Doug picked up
three trail cameras.

``Sweet!'' Alex approved.

The trail cameras recorded time-lapse video unless triggered by
movement. Movement switched them to normal speed. At normal speed, the
cameras also recorded sound. Nifty gadgets. The trail cameras would help
them keep tabs on their surroundings. The cameras came with memory
cards, but they also picked up spares. The checkout lady gave them a
wary look as she scanned their items.

After Scheels, they went to Walmart on Fairview. Before going in, they
rechecked Alex's list. In the store, they split up. Doug headed to the
grocery section and loaded four carts with canned and nonperishable food
items. Alex headed to hardware and grabbed hammers, saws, bump lights,
duct tape (lots and lots of duct tape), twelve strings of small
battery-powered fans, spools of polypropene rope, four plastic wash
tubs, matches, fire starters, a dozen Velcro rolls, garbage bags, a
dozen 100-foot heavy-duty extension cords, a few cases of aluminum foil,
Vitamin C and D (let's not die of scurvy or rickets before we freeze),
floor mops, brooms, two hand-held battery powered vacuum cleaners, two
dozen big rolls of bubble wrap, and twenty large plastic tarps.

Meeting at the truck, they packed their supplies in the trailer and
headed back in to clear out Walmart's large four-gallon water bottles.
By their fourth Walmart run, Alex saw they weren't alone in bulk buying.
Mob panic buying hadn't started yet, but others were clueing in. They
finished at Walmart by getting ``medical'' supplies: bandages, gauze,
rubbing alcohol, and painkillers. They also grabbed a few first-aid kits
without inspecting their contents.

Leaving the pharmacy aisles, Alex dumped eight face mask cartons in a
shopping cart. Doug couldn't resist teasing.

``I thought you were anti-mask.''

``Mines are dusty. And they'll help keep your face warm in -80 weather.
And, most importantly, they keep your breath off cold eyepieces.''

``You're enjoying this. Admit it.''

Next, they hit sporting goods and outdoor stores. They snapped up more
cold weather sleeping bags, water purification kits, two folding camping
cots, and a costly image-intensifying night vision monocular. At Home
Goods, they picked up electric blankets, three large turkey baster oven
pans, and --- honoring Ellen, a sewing machine with extra needles and a
few dozen large spools of thread. At Best Buy, they bought two
top-of-the-line short-wave radios, two pairs of long-range
walkie-talkies, and six infrared motion detector alarms. They also
purchased wired security webcams and a few spools of active USB cables.
In Dicks they picked up half a dozen axes, knives, drink coolers, more
large steel thermoses, small camping butane canisters, and two small
butane hiking stoves.

Checking and updating their shopping list again Alex and Doug went to
Lowes on Eagle and Ustick, where they bought two dozen cinder blocks,
three dozen large sheets of plywood, a pallet of two-by-fours, a big
stack of six-by-one fence planks, a dozen Styrofoam sheets, a power
staple gun, power drills, a power saw with half a dozen blades, and
plenty of nails, screws, bolts, staples, and drill hooks. They also
purchased two dozen eight-foot by 8-inch metal duct pipes, two hundred
feet of flexible PVC tubing, rolls of vapor barrier wrap, and as many
batts of rockwool insulation as they could pack in the trailer. Lastly,
they picked up half a dozen 200-foot spools of various grades of
electrical wiring along with electrician supplies: multimeters, propane
soldering torches, a handful of resistors, capacitors, switches, strings
of 12V DC lights, and a mixed bag of wire stripping and cutting tools.

Their last stop was at D\&B, a ranch supply store on Fairview. At D\&B,
they bought four XXL size coveralls, a dozen pairs of thick work gloves,
six sets of red \emph{Santa Claus} woolen underwear, metal pails,
shovels, picks, crowbars, sledgehammers, two dozen steel pike bars, a
propane chain saw, two propane-powered electrical generators, seventeen
30-pound propane tanks, and three 100-pound propane tanks. It was all
the propane they had. Alex also got a small wheelbarrow to ferry the
heavy propane tanks to the trailer. After securing the tanks in the
trailer, they returned and bought two large rechargeable 1500-watt-hour
powerpacks. The powerpacks had nifty programmable built-in timers to
turn things on and off. They were pricey, and typically, Alex would have
shopped around, but bargain hunting was over.

Their last purchase was perhaps the best of the entire shopping run: two
rugged stainless steel hot tent stoves. The stoves came with all the
vents needed to install them. When set up in a hot tent, you could
comfortably camp in -40 Celsius weather. No hot tents were on sale, but
Doug found an insulated winter tunnel tent made by the same outfit.

\emph{It was insane.} Who keeps end-of-the-world shopping lists? Even
the most die-hard preppers and billionaires in their luxury nuke-proof
bunkers weren't ready for this. Still, Alex had to smile. Who knew
apocalypse shopping would perfectly suit bitter and grieving men who had
given up.


%\end{document}

