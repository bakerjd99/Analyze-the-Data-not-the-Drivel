%% uncomment to list all files in log
%\listfiles

\documentclass[12pt]{report}

\usepackage{fontspec}

%\setmainfont[Scale=MatchLowercase]{Lucida Bright}
%\setmonofont{FreeMono}
%\setmonofont{Source Code Pro}
\setmonofont[Scale=MatchLowercase]{Ubuntu Mono}

\usepackage[headings]{fullpage}

% national use characters 
%\usepackage{inputenc}

% ams mathematical symbols
\usepackage{amsmath,amssymb}

% added to support pandoc highlighting
\usepackage{microtype}

\usepackage{makeidx}

% add index and bibliographies to table of contents
\usepackage[nottoc]{tocbibind}

% postscript courier and times in place of cm fonts
%\usepackage{courier}
%\usepackage{times}

% extended coloring
\usepackage{color}
\usepackage[table,dvipsnames]{xcolor}
\usepackage{colortbl}

% advanced date formating
\usepackage{datetime}

%support pandoc code highlighting
\usepackage{fancyvrb}
\DefineShortVerb[commandchars=\\\{\}]{\|}
\DefineVerbatimEnvironment{Highlighting}{Verbatim}{commandchars=\\\{\}}
% Add ',fontsize=\small' for more characters per line

%tango style colors
% \usepackage{framed}
% \definecolor{shadecolor}{RGB}{255,255,255}
% \newenvironment{Shaded}{\begin{snugshade}}{\end{snugshade}}
% \newcommand{\KeywordTok}[1]{\textcolor[rgb]{0.13,0.29,0.53}{\textbf{{#1}}}}
% \newcommand{\DataTypeTok}[1]{\textcolor[rgb]{0.13,0.29,0.53}{{#1}}}
% \newcommand{\DecValTok}[1]{\textcolor[rgb]{0.00,0.00,0.81}{{#1}}}
% \newcommand{\BaseNTok}[1]{\textcolor[rgb]{0.00,0.00,0.81}{{#1}}}
% \newcommand{\FloatTok}[1]{\textcolor[rgb]{0.00,0.00,0.81}{{#1}}}
% \newcommand{\CharTok}[1]{\textcolor[rgb]{0.31,0.60,0.02}{{#1}}}
% \newcommand{\StringTok}[1]{\textcolor[rgb]{0.31,0.60,0.02}{{#1}}}
% \newcommand{\CommentTok}[1]{\textcolor[rgb]{0.56,0.35,0.01}{\textit{{#1}}}}
% \newcommand{\OtherTok}[1]{\textcolor[rgb]{0.56,0.35,0.01}{{#1}}}
% \newcommand{\AlertTok}[1]{\textcolor[rgb]{0.94,0.16,0.16}{{#1}}}
% \newcommand{\FunctionTok}[1]{\textcolor[rgb]{0.00,0.00,0.00}{{#1}}}
% \newcommand{\RegionMarkerTok}[1]{{#1}}
% \newcommand{\ErrorTok}[1]{\textbf{{#1}}}
% \newcommand{\NormalTok}[1]{{#1}}

%espresso style colors
% \usepackage{framed}
% \definecolor{shadecolor}{RGB}{42,33,28}
% \newenvironment{Shaded}{\begin{snugshade}}{\end{snugshade}}
% \newcommand{\KeywordTok}[1]{\textcolor[rgb]{0.26,0.66,0.93}{\textbf{{#1}}}}
% \newcommand{\DataTypeTok}[1]{\textcolor[rgb]{0.74,0.68,0.62}{\underline{{#1}}}}
% \newcommand{\DecValTok}[1]{\textcolor[rgb]{0.27,0.67,0.26}{{#1}}}
% \newcommand{\BaseNTok}[1]{\textcolor[rgb]{0.27,0.67,0.26}{{#1}}}
% \newcommand{\FloatTok}[1]{\textcolor[rgb]{0.27,0.67,0.26}{{#1}}}
% \newcommand{\CharTok}[1]{\textcolor[rgb]{0.02,0.61,0.04}{{#1}}}
% \newcommand{\StringTok}[1]{\textcolor[rgb]{0.02,0.61,0.04}{{#1}}}
% \newcommand{\CommentTok}[1]{\textcolor[rgb]{0.00,0.40,1.00}{\textit{{#1}}}}
% \newcommand{\OtherTok}[1]{\textcolor[rgb]{0.74,0.68,0.62}{{#1}}}
% \newcommand{\AlertTok}[1]{\textcolor[rgb]{1.00,1.00,0.00}{{#1}}}
% \newcommand{\FunctionTok}[1]{\textcolor[rgb]{1.00,0.58,0.35}{\textbf{{#1}}}}
% \newcommand{\RegionMarkerTok}[1]{\textcolor[rgb]{0.74,0.68,0.62}{{#1}}}
% \newcommand{\ErrorTok}[1]{\textcolor[rgb]{0.74,0.68,0.62}{\textbf{{#1}}}}
% \newcommand{\NormalTok}[1]{\textcolor[rgb]{0.74,0.68,0.62}{{#1}}}

%kete style colors
% \newenvironment{Shaded}{}{}
% \newcommand{\KeywordTok}[1]{\textbf{{#1}}}
% \newcommand{\DataTypeTok}[1]{\textcolor[rgb]{0.50,0.00,0.00}{{#1}}}
% \newcommand{\DecValTok}[1]{\textcolor[rgb]{0.00,0.00,1.00}{{#1}}}
% \newcommand{\BaseNTok}[1]{\textcolor[rgb]{0.00,0.00,1.00}{{#1}}}
% \newcommand{\FloatTok}[1]{\textcolor[rgb]{0.50,0.00,0.50}{{#1}}}
% \newcommand{\CharTok}[1]{\textcolor[rgb]{1.00,0.00,1.00}{{#1}}}
% \newcommand{\StringTok}[1]{\textcolor[rgb]{0.87,0.00,0.00}{{#1}}}
% \newcommand{\CommentTok}[1]{\textcolor[rgb]{0.50,0.50,0.50}{\textit{{#1}}}}
% \newcommand{\OtherTok}[1]{{#1}}
% \newcommand{\AlertTok}[1]{\textcolor[rgb]{0.00,1.00,0.00}{\textbf{{#1}}}}
% \newcommand{\FunctionTok}[1]{\textcolor[rgb]{0.00,0.00,0.50}{{#1}}}
% \newcommand{\RegionMarkerTok}[1]{{#1}}
% \newcommand{\ErrorTok}[1]{\textcolor[rgb]{1.00,0.00,0.00}{\textbf{{#1}}}}
% \newcommand{\NormalTok}[1]{{#1}}
%end pandoc code hacks

% jodliterate colors
\usepackage{color}
\definecolor{shadecolor}{RGB}{248,248,248}
% j control structures 
\definecolor{keywcolor}{rgb}{0.13,0.29,0.53}
% j explicit arguments x y m n u v
\definecolor{datacolor}{rgb}{0.13,0.29,0.53}
% j numbers - all types see j.xml
\definecolor{decvcolor}{rgb}{0.00,0.00,0.81}
\definecolor{basencolor}{rgb}{0.00,0.00,0.81}
\definecolor{floatcolor}{rgb}{0.00,0.00,0.81}
% j local assignments
\definecolor{charcolor}{rgb}{0.31,0.60,0.02}
\definecolor{stringcolor}{rgb}{0.31,0.60,0.02}
\definecolor{commentcolor}{rgb}{0.56,0.35,0.01}
% primitive adverbs and conjunctions
%\definecolor{othercolor}{rgb}{0.56,0.35,0.01}   
\definecolor{othercolor}{RGB}{0,0,255}
% global assignments
\definecolor{alertcolor}{rgb}{0.94,0.16,0.16}
% primitive J verbs and noun names
\definecolor{funccolor}{rgb}{0.00,0.00,0.00}    

\usepackage{framed}
\newenvironment{Shaded}{}{}
\newcommand{\KeywordTok}[1]{\textcolor{keywcolor}{\textbf{{#1}}}}
\newcommand{\DataTypeTok}[1]{\textcolor{datacolor}{{#1}}}
%\newcommand{\DecValTok}[1]{\textcolor{decvcolor}{{#1}}}
\newcommand{\DecValTok}[1]{{#1}} 
\newcommand{\BaseNTok}[1]{\textcolor{basencolor}{{#1}}}
\newcommand{\FloatTok}[1]{\textcolor{floatcolor}{{#1}}}
\newcommand{\CharTok}[1]{\textcolor{charcolor}{\textbf{{#1}}}}
\newcommand{\StringTok}[1]{\textcolor{stringcolor}{{#1}}}
\newcommand{\CommentTok}[1]{\textcolor{commentcolor}{\textit{{#1}}}}
\newcommand{\OtherTok}[1]{\textcolor{othercolor}{{#1}}} 
\newcommand{\AlertTok}[1]{\textcolor{alertcolor}{\textbf{{#1}}}}
%\newcommand{\FunctionTok}[1]{\textcolor{funccolor}{{#1}}}
\newcommand{\FunctionTok}[1]{{#1}}
\newcommand{\RegionMarkerTok}[1]{{#1}}
\newcommand{\ErrorTok}[1]{\textbf{{#1}}}
\newcommand{\NormalTok}[1]{{#1}}

% headers and footers
\usepackage{fancyhdr}
\pagestyle{fancy}

\fancyhead{}
\fancyfoot{}

%\fancyhead[LE,RO]{\slshape \rightmark}
%\fancyhead[LO,RE]{\slshape \leftmark}
\fancyfoot[C]{\thepage}
%\headrulewidth 0.4pt
%\footrulewidth 0 pt

%\addtolength{\headheight}{\baselineskip}

%\lfoot{\emph{Analyze the Data not the Drivel}}
%\rfoot{\emph{\today}}

% subfigure handles figures that contain subfigures
%\usepackage{color,graphicx,subfigure,sidecap}
\usepackage{graphicx,sidecap}
\usepackage{subfigure}
\graphicspath{{./inclusions/}}

% floatflt provides for text wrapping around small figures and tables
\usepackage{floatflt}

% tweak caption formats 
\usepackage{caption} 
\usepackage{sidecap}
%\usepackage{subcaption} % not compatible with subfigure

\usepackage{rotating} % flip tables sideways

% complex footnotes
%\usepackage{bigfoot}

% weird logos \XeLaTeX
\usepackage{metalogo}

% source code listings
\usepackage{listings}

% long tables
% \usepackage{longtable}

\newcommand{\HRule}{\rule{\linewidth}{0.5mm}}

% map LaTeX cross references into PDF cross references
\usepackage[
            %dvips,
            colorlinks,
            linkcolor=blue,
            citecolor=blue,
            urlcolor=blue,   % magenta, cyan default        
            pdfauthor={John D. Baker},
            pdftitle={Analyze the Data not the Drivel},
            pdfsubject={Blog},
            pdfcreator={MikTeX+LaTeXe with hyperref package},
            pdfkeywords={blog,wordpress},
            ]{hyperref}
           
% custom colors
\definecolor{CodeBackGround}{cmyk}{0.0,0.0,0,0.05}    % light gray
\definecolor{CodeComment}{rgb}{0,0.50,0.00}           % dark green {0,0.45,0.08}
\definecolor{TableStripes}{gray}{0.9}                 % odd/even background in tables

\lstdefinelanguage{bat}
{morekeywords={echo,title,pushd,popd,setlocal,endlocal,off,if,not,exist,set,goto,pause},
sensitive=True,
morecomment=[l]{rem}
}

\lstdefinelanguage{jdoc}
{
morekeywords={},
otherkeywords={assert.,break.,continue.,for.,do.,if.,else.,elseif.,return.,select.,end.
,while.,whilst.,throw.,catch.,catchd.,catcht.,try.,case.,fcase.},
sensitive=True,
morecomment=[l]{NB.},
morestring=[b]',
morestring=[d]',
}

% latex size ordering - can never remember it
% \tiny
% \scriptsize
% \footnotesize
% \small
% \normalsize
% \large
% \Large
% \LARGE
% \huge
% \Huge
 
% listings package settings  
\lstset{%
  language=jdoc,                                % j document settings
  basicstyle=\ttfamily\footnotesize,            
  keywordstyle=\bfseries\color{keywcolor}\footnotesize,
  identifierstyle=\color{black},
  commentstyle=\slshape\color{CodeComment},     % colored slanted comments
  stringstyle=\color{red}\ttfamily,
  showstringspaces=false,                       
  %backgroundcolor=\color{CodeBackGround},       
  frame=single,                                
  framesep=1pt,                                 
  framerule=0.8pt,                             
  rulecolor=\color{CodeBackGround},   
  showspaces=false,
  %columns=fullflexible,
  %numbers=left,
  %numberstyle=\footnotesize,
  %numbersep=9pt,
  tabsize=2,
  showtabs=false,
  captionpos=b
  breaklines=true,                              
  breakindent=5pt                              
}

\lstdefinelanguage{JavaScript}{
  keywords={typeof, new, true, false, catch, function, return, null, catch, switch, var, if, in, while, do, else, case, break},
  ndkeywords={class, export, boolean, throw, implements, import, this},
  ndkeywordstyle=\color{darkgray}\bfseries,
  sensitive=false,
  comment=[l]{//},
  morecomment=[s]{/*}{*/},
  morestring=[b]',
  morestring=[b]"
}

% C# settings
\lstdefinestyle{sharpc}{
language=[Sharp]C,
basicstyle=\ttfamily\scriptsize, 
keywordstyle=\bfseries\color{keywcolor}\scriptsize,
framerule=0pt
}

% for source code listing longer than two use smaller font
\lstdefinestyle{smallersource}{
basicstyle=\ttfamily\scriptsize, 
keywordstyle=\bfseries\color{keywcolor}\scriptsize,
framerule=0pt
}

\lstdefinestyle{resetdefaults}{
language=jdoc,
basicstyle=\ttfamily\footnotesize,  
keywordstyle=\bfseries\color{keywcolor}\footnotesize,                                                               
framerule=0.8pt 
}

% APL UTF8 code points listed for lstlisting processing
\makeatletter
\lst@InputCatcodes
\def\lst@DefEC{%
 \lst@CCECUse \lst@ProcessLetter
  ^^80^^81^^82^^83^^84^^85^^86^^87^^88^^89^^8a^^8b^^8c^^8d^^8e^^8f%
  ^^90^^91^^92^^93^^94^^95^^96^^97^^98^^99^^9a^^9b^^9c^^9d^^9e^^9f%
  ^^a0^^a1^^a2^^a3^^a4^^a5^^a6^^a7^^a8^^a9^^aa^^ab^^ac^^ad^^ae^^af%
  ^^b0^^b1^^b2^^b3^^b4^^b5^^b6^^b7^^b8^^b9^^ba^^bb^^bc^^bd^^be^^bf%
  ^^c0^^c1^^c2^^c3^^c4^^c5^^c6^^c7^^c8^^c9^^ca^^cb^^cc^^cd^^ce^^cf%
  ^^d0^^d1^^d2^^d3^^d4^^d5^^d6^^d7^^d8^^d9^^da^^db^^dc^^dd^^de^^df%
  ^^e0^^e1^^e2^^e3^^e4^^e5^^e6^^e7^^e8^^e9^^ea^^eb^^ec^^ed^^ee^^ef%
  ^^f0^^f1^^f2^^f3^^f4^^f5^^f6^^f7^^f8^^f9^^fa^^fb^^fc^^fd^^fe^^ff%
  ^^^^20ac^^^^0153^^^^0152%
  ^^^^20a7^^^^2190^^^^2191^^^^2192^^^^2193^^^^2206^^^^2207^^^^220a%
  ^^^^2218^^^^2228^^^^2229^^^^222a^^^^2235^^^^223c^^^^2260^^^^2261%
  ^^^^2262^^^^2264^^^^2265^^^^2282^^^^2283^^^^2296^^^^22a2^^^^22a3%
  ^^^^22a4^^^^22a5^^^^22c4^^^^2308^^^^230a^^^^2336^^^^2337^^^^2339%
  ^^^^233b^^^^233d^^^^233f^^^^2340^^^^2342^^^^2347^^^^2348^^^^2349%
  ^^^^234b^^^^234e^^^^2350^^^^2352^^^^2355^^^^2357^^^^2359^^^^235d%
  ^^^^235e^^^^235f^^^^2361^^^^2362^^^^2363^^^^2364^^^^2365^^^^2368%
  ^^^^236a^^^^236b^^^^236c^^^^2371^^^^2372^^^^2373^^^^2374^^^^2375%
  ^^^^2377^^^^2378^^^^237a^^^^2395^^^^25af^^^^25ca^^^^25cb%  
  ^^00}
\lst@RestoreCatcodes
\makeatother

% custom lengths used within minipages
\newcommand{\minindent}{17pt}


\makeindex

\begin{document}

\subsection*{\href{http://bakerjd99.wordpress.com/2012/10/01/semi-literate-jod/}{Semi-Literate JOD}}
\addcontentsline{toc}{subsection}{Semi-Literate JOD}


\noindent\emph{Posted: 02 Oct 2012 04:08:52}
\vspace{6pt}


%{[}caption id=``attachment\_3312'' align=``alignleft''  width=``101''{]}
%\href{https://www.box.com/s/rhs9kfmim65vm6sv5p8y}{\includegraphics{jodlittlecube.png}}
%Click to view jodliterate.pdf
%{[}/caption{]}

\captionsetup[floatingfigure]{labelformat=empty}
\begin{floatingfigure}[r]{0.14\textwidth}
\centering
\href{http://www.box.com/s/rhs9kfmim65vm6sv5p8y}{\includegraphics[width=0.11\textwidth]{jodlittlecube.png}}
%\caption{Click to view \texttt{jodliterate.pdf}}
\label{fig:3303X0}
\end{floatingfigure} Despite seven decades of programming experience documenting software
remains a challenge. There are many reasons for this sorry state of
affairs with the most important being that programmers simply do not
agree on the \emph{need} for documentation. As pathetic as this sounds
it's not without merit. It all depends on what you call
``documentation.''

Writing technical documents for management, marketing or users usually
results in excruciating rounds of
\href{http://search.dilbert.com/comic/Documentation}{Dilbertian}
critiques. Everyone understands your code better than you do. If you
provide too much detail, you get complaints. If you use unfamiliar
words, you get complaints. If you point out limitations, assumptions or
caveats, you get complaints. If you assume basic 8th grade reading
levels, you get complaints. If you use nonstandard fonts or
\emph{unauthorized style templates,} you get complaints. No wonder many
programmers hate ``documentation'' and blow off the entire problem by
making ludicrous claims about ``self documenting code.'' The self
documenting cabal may have fooled management but they're not fooling the
rest of us. The need for \emph{illuminating} program documentation is as
pressing today as it was for
\href{http://inventors.about.com/od/estartinventions/a/Eniac.htm}{ENIAC}
coders in the 1940's and, when in it comes to illuminating documentation,
the best overall approach was pioneered by
\href{http://www-cs-faculty.stanford.edu/~uno/}{Donald Knuth} over
twenty-five years ago and goes by the moniker
\href{http://en.wikipedia.org/wiki/Literate\_programming}{\emph{literate
programming.}}

Providing basic literate programming support in JOD has been on my to-do
list for ages. I've held off until recently because I have never been
happy with my \href{http://en.wikipedia.org/wiki/Markup\_language}{mark
up} options. JOD directly supports simple J
\href{http://www.jsoftware.com/help/user/scriptdoc.htm}{scriptdoc}
compatible leading comment block formatting. For example many of my J
verbs start with a comment block like:

\small
\begin{Shaded}
\begin{Highlighting}[]
\NormalTok{betweenstrs}\AlertTok{=:}\NormalTok{4 }\OtherTok{:} \NormalTok{0}

\CommentTok{NB.*betweenstrs v-- select sublists between  nonnested delimiters}
\CommentTok{NB. discarding delimiters.}
\CommentTok{NB.}
\CommentTok{NB. dyad:  blcl =. (clStart;clEnd) betweenstrs cl}
\CommentTok{NB.        blnl =. (nlStart;nlEnd) betweenstrs nl}
\CommentTok{NB.}
\CommentTok{NB.   cstr=. ;20#<'start yada yada end boo hoo start ahh end '}
\CommentTok{NB.   ('start';'end') betweenstrs cstr}
\CommentTok{NB.}
\CommentTok{NB.   NB. also applies to numeric delimiters}
\CommentTok{NB.   (1 1;2 2) betweenstrs 1 1 66 666 2 2 7 87 1 1 0 2 2}

\StringTok{'s e'}\CharTok{=.} \DataTypeTok{x}
\NormalTok{llst}\CharTok{=.} \RegionMarkerTok{((}\FunctionTok{-#}\NormalTok{s}\RegionMarkerTok{)} \RegionMarkerTok{(}\FunctionTok{\textbar{}.}\OtherTok{!.}\NormalTok{0}\RegionMarkerTok{)} \NormalTok{s }\FunctionTok{E.} \DataTypeTok{y}\RegionMarkerTok{)} \FunctionTok{+.} \NormalTok{e }\FunctionTok{E.} \DataTypeTok{y}
\NormalTok{mask}\CharTok{=.} \FunctionTok{~:}\OtherTok{/\textbackslash{}} \NormalTok{llst}
\RegionMarkerTok{(}\NormalTok{mask}\FunctionTok{#}\NormalTok{llst}\RegionMarkerTok{)} \FunctionTok{<}\OtherTok{;.}\NormalTok{1 mask}\FunctionTok{#}\DataTypeTok{y}
\RegionMarkerTok{)}
\end{Highlighting}
\end{Shaded}
\normalsize

Even if you can't spell J I bet you have a good idea about what this
``program'' does and, if you doubt my claims, I've left you with some
examples to try the next time you find yourself in J. Stupid comments
may \href{http://php.dzone.com/news/comments-are-losers-2}{be for
losers} but telling comments, especially example laden ones, really
help! And, if you really find comments distracting, JOD has a deal for
you!

\small
\begin{Shaded}
\begin{Highlighting}[]
   \FunctionTok{;}\NormalTok{1}\FunctionTok{\{}\NormalTok{compj }\StringTok{'betweenstrs'} 
\NormalTok{betweenstrs}\AlertTok{=:}\NormalTok{4 }\OtherTok{:}\NormalTok{0}
\StringTok{'s e'}\CharTok{=.}\DataTypeTok{x}
\NormalTok{a}\CharTok{=.}\RegionMarkerTok{((}\FunctionTok{-#}\NormalTok{s }\RegionMarkerTok{)(}\FunctionTok{\textbar{}.}\OtherTok{!.}\NormalTok{0}\RegionMarkerTok{)}\NormalTok{s }\FunctionTok{E.}\DataTypeTok{y}\RegionMarkerTok{)}\FunctionTok{+.}\NormalTok{e }\FunctionTok{E.}\DataTypeTok{y}
\NormalTok{b}\CharTok{=.}\FunctionTok{~:}\OtherTok{/\textbackslash{}}\NormalTok{a}
\RegionMarkerTok{(}\NormalTok{b}\FunctionTok{#}\NormalTok{a}\RegionMarkerTok{)}\FunctionTok{<}\OtherTok{;.}\NormalTok{1 b}\FunctionTok{#}\DataTypeTok{y}
\RegionMarkerTok{)}
\end{Highlighting}
\end{Shaded}
\normalsize

\texttt{compj} purges pesky comments and reduces tedious long
identifiers like \texttt{mask} to pure compact J. Getting rid of
comments is trivial, putting them back in: not so much! JOD's simple
comment block formatting has been very effective but it's hardly
literate programming.

Literate programming requires more muscle. Knuth used his own \TeX. \TeX\
and \LaTeX\ are certainly up to the job, as are many HTML and XML
approaches. Unfortunately, all these mark up formats suffer from
``distracting taggyness.'' I can tolerate \LaTeX\ but HTML and XML drives
me nuts. Yes, there are perfectly fine editors for all these formats, but
remember, we are inserting the resulting text into code that we will be
looking at for the \emph{rest of our miserable coding lives!} We need a
mark up format that's stable, readable, versatile, easy to use and,
\emph{this is very important}, easy to ignore!
\href{http://daringfireball.net/projects/markdown/}{Markdown} is such a
format. It's almost ideal for program comments and is capable of much
more. I've started using markdown in JOD and it's already paying its
way.

\href{https://www.box.com/s/1fcil5s5zhvlxzzwpjqn}{\texttt{jodliterate.ijs}}
is a J utility script that can generate \emph{semi-literate} \LaTeX\
documents directly from JOD groups. It uses a version of pandoc with J
syntax highlighting, see
\href{http://bakerjd99.wordpress.com/2012/09/20/pandoc-based-j-syntax-highlighting/}{Pandoc
based J Syntax Highlighting} for details. I consider
\texttt{jodliterate} semi-literate because it's completely at the mercy
of the programmer. If you don't store coherent markdown text fragments
in JOD all you get is a nice syntax highlighted listing. But, if you
actually \emph{write} about your group, \texttt{jodliterate} can produce
essential documents.
\href{https://www.box.com/s/rhs9kfmim65vm6sv5p8y}{\texttt{jodliterate.pdf}}
is an example of this tool being used on itself. Self reference always
makes an excellent test case. \texttt{jodliterate} will be included in
the next JOD release. Until then you can download the J script from
\href{https://www.box.com/s/eb537470b25ab230a0ab}{this directory}. As always
referenced files are available in the files sidebar. Enjoy!

%\captionsetup[floatingfigure]{labelformat=empty}
%\begin{figure}[htbp]
%\begin{floatingfigure}[l]{0.25\textwidth}
%\centering
%\includegraphics[width=0.23\textwidth]{jodlittlecube.png}
%\caption{~~~IMCAPTION~~~}
%\label{fig:3303X0}
%\end{floatingfigure}
%\end{figure}



%\end{document}