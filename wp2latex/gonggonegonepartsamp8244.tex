%% bm.pdf preamble - material merged from previous preamble and current pandoc preamable output
% NOTE: float placement required changes to the source files referenced by bm.tex
% May 28, 2020
%
% Use lualatex to compile - test with MiKTeX 2.9

% uncomment to list all files in log
%\listfiles

\documentclass[12pt]{report}


\usepackage{fontspec}

%\setmainfont[Scale=MatchLowercase]{Lucida Bright}
%\setmonofont{FreeMono}
%\setmonofont{Source Code Pro}
\setmonofont[Scale=MatchLowercase]{Ubuntu Mono}

\usepackage[headings]{fullpage}

% national use characters 
%\usepackage{inputenc}

% ams mathematical symbols
\usepackage{amsmath,amssymb}

% added to support pandoc highlighting
\usepackage{microtype}

\usepackage{makeidx}

% add index and bibliographies to table of contents
\usepackage[nottoc]{tocbibind}

% postscript courier and times in place of cm fonts
%\usepackage{courier}
%\usepackage{times}

% extended coloring
\usepackage{color}
\usepackage[table,dvipsnames]{xcolor}
\usepackage{colortbl}

% advanced date formating
\usepackage{datetime}

%support pandoc code highlighting
\usepackage{fancyvrb}

% \DefineShortVerb[commandchars=\\\{\}]{\|}
% \DefineVerbatimEnvironment{Highlighting}{Verbatim}{commandchars=\\\{\}}
% % Add ',fontsize=\small' for more characters per line

% tango style colors
% \usepackage{framed}
% \definecolor{shadecolor}{RGB}{255,255,255}
% \newenvironment{Shaded}{\begin{snugshade}}{\end{snugshade}}
% \newcommand{\KeywordTok}[1]{\textcolor[rgb]{0.13,0.29,0.53}{\textbf{{#1}}}}
% \newcommand{\DataTypeTok}[1]{\textcolor[rgb]{0.13,0.29,0.53}{{#1}}}
% \newcommand{\DecValTok}[1]{\textcolor[rgb]{0.00,0.00,0.81}{{#1}}}
% \newcommand{\BaseNTok}[1]{\textcolor[rgb]{0.00,0.00,0.81}{{#1}}}
% \newcommand{\FloatTok}[1]{\textcolor[rgb]{0.00,0.00,0.81}{{#1}}}
% \newcommand{\CharTok}[1]{\textcolor[rgb]{0.31,0.60,0.02}{{#1}}}
% \newcommand{\StringTok}[1]{\textcolor[rgb]{0.31,0.60,0.02}{{#1}}}
% \newcommand{\CommentTok}[1]{\textcolor[rgb]{0.56,0.35,0.01}{\textit{{#1}}}}
% \newcommand{\OtherTok}[1]{\textcolor[rgb]{0.56,0.35,0.01}{{#1}}}
% \newcommand{\AlertTok}[1]{\textcolor[rgb]{0.94,0.16,0.16}{{#1}}}
% \newcommand{\FunctionTok}[1]{\textcolor[rgb]{0.00,0.00,0.00}{{#1}}}
% \newcommand{\RegionMarkerTok}[1]{{#1}}
% \newcommand{\ErrorTok}[1]{\textbf{{#1}}}
% \newcommand{\NormalTok}[1]{{#1}}

% %espresso style colors
% \usepackage{framed}
% \definecolor{shadecolor}{RGB}{42,33,28}
% \newenvironment{Shaded}{\begin{snugshade}}{\end{snugshade}}
% \newcommand{\KeywordTok}[1]{\textcolor[rgb]{0.26,0.66,0.93}{\textbf{{#1}}}}
% \newcommand{\DataTypeTok}[1]{\textcolor[rgb]{0.74,0.68,0.62}{\underline{{#1}}}}
% \newcommand{\DecValTok}[1]{\textcolor[rgb]{0.27,0.67,0.26}{{#1}}}
% \newcommand{\BaseNTok}[1]{\textcolor[rgb]{0.27,0.67,0.26}{{#1}}}
% \newcommand{\FloatTok}[1]{\textcolor[rgb]{0.27,0.67,0.26}{{#1}}}
% \newcommand{\CharTok}[1]{\textcolor[rgb]{0.02,0.61,0.04}{{#1}}}
% \newcommand{\StringTok}[1]{\textcolor[rgb]{0.02,0.61,0.04}{{#1}}}
% \newcommand{\CommentTok}[1]{\textcolor[rgb]{0.00,0.40,1.00}{\textit{{#1}}}}
% \newcommand{\OtherTok}[1]{\textcolor[rgb]{0.74,0.68,0.62}{{#1}}}
% \newcommand{\AlertTok}[1]{\textcolor[rgb]{1.00,1.00,0.00}{{#1}}}
% \newcommand{\FunctionTok}[1]{\textcolor[rgb]{1.00,0.58,0.35}{\textbf{{#1}}}}
% \newcommand{\RegionMarkerTok}[1]{\textcolor[rgb]{0.74,0.68,0.62}{{#1}}}
% \newcommand{\ErrorTok}[1]{\textcolor[rgb]{0.74,0.68,0.62}{\textbf{{#1}}}}
% \newcommand{\NormalTok}[1]{\textcolor[rgb]{0.74,0.68,0.62}{{#1}}}

% %kete style colors
% \newenvironment{Shaded}{}{}
% \newcommand{\KeywordTok}[1]{\textbf{{#1}}}
% \newcommand{\DataTypeTok}[1]{\textcolor[rgb]{0.50,0.00,0.00}{{#1}}}
% \newcommand{\DecValTok}[1]{\textcolor[rgb]{0.00,0.00,1.00}{{#1}}}
% \newcommand{\BaseNTok}[1]{\textcolor[rgb]{0.00,0.00,1.00}{{#1}}}
% \newcommand{\FloatTok}[1]{\textcolor[rgb]{0.50,0.00,0.50}{{#1}}}
% \newcommand{\CharTok}[1]{\textcolor[rgb]{1.00,0.00,1.00}{{#1}}}
% \newcommand{\StringTok}[1]{\textcolor[rgb]{0.87,0.00,0.00}{{#1}}}
% \newcommand{\CommentTok}[1]{\textcolor[rgb]{0.50,0.50,0.50}{\textit{{#1}}}}
% \newcommand{\OtherTok}[1]{{#1}}
% \newcommand{\AlertTok}[1]{\textcolor[rgb]{0.00,1.00,0.00}{\textbf{{#1}}}}
% \newcommand{\FunctionTok}[1]{\textcolor[rgb]{0.00,0.00,0.50}{{#1}}}
% \newcommand{\RegionMarkerTok}[1]{{#1}}
% \newcommand{\ErrorTok}[1]{\textcolor[rgb]{1.00,0.00,0.00}{\textbf{{#1}}}}
% \newcommand{\NormalTok}[1]{{#1}}
% %end pandoc code hacks

% jodliterate colors
\usepackage{color}
\definecolor{shadecolor}{RGB}{248,248,248}
% j control structures 
\definecolor{keywcolor}{rgb}{0.13,0.29,0.53}
% j explicit arguments x y m n u v
\definecolor{datacolor}{rgb}{0.13,0.29,0.53}
% j numbers - all types see j.xml
\definecolor{decvcolor}{rgb}{0.00,0.00,0.81}
\definecolor{basencolor}{rgb}{0.00,0.00,0.81}
\definecolor{floatcolor}{rgb}{0.00,0.00,0.81}
% j local assignments
\definecolor{charcolor}{rgb}{0.31,0.60,0.02}
\definecolor{stringcolor}{rgb}{0.31,0.60,0.02}
\definecolor{commentcolor}{rgb}{0.56,0.35,0.01}
% primitive adverbs and conjunctions
%\definecolor{othercolor}{rgb}{0.56,0.35,0.01}   
\definecolor{othercolor}{RGB}{0,0,255}
% global assignments
\definecolor{alertcolor}{rgb}{0.94,0.16,0.16}
% primitive J verbs and noun names
\definecolor{funccolor}{rgb}{0.00,0.00,0.00}

% custom colors
\definecolor{CodeBackGround}{cmyk}{0.0,0.0,0,0.05}    % light gray
\definecolor{CodeComment}{rgb}{0,0.50,0.00}           % dark green {0,0.45,0.08}
\definecolor{TableStripes}{gray}{0.9}                 % odd/even background in tables

% Colors for the hyperref package
\definecolor{urlcolor}{rgb}{0,.145,.698}
\definecolor{linkcolor}{rgb}{.71,0.21,0.01}
\definecolor{citecolor}{rgb}{.12,.54,.11}

% % Exact colors from NB
\definecolor{incolor}{HTML}{303F9F}
\definecolor{outcolor}{HTML}{D84315}
\definecolor{cellborder}{HTML}{CFCFCF}
\definecolor{cellbackground}{HTML}{F7F7F7}

% % ANSI colors
\definecolor{ansi-black}{HTML}{3E424D}
\definecolor{ansi-black-intense}{HTML}{282C36}
\definecolor{ansi-red}{HTML}{E75C58}
\definecolor{ansi-red-intense}{HTML}{B22B31}
\definecolor{ansi-green}{HTML}{00A250}
\definecolor{ansi-green-intense}{HTML}{007427}
\definecolor{ansi-yellow}{HTML}{DDB62B}
\definecolor{ansi-yellow-intense}{HTML}{B27D12}
\definecolor{ansi-blue}{HTML}{208FFB}
\definecolor{ansi-blue-intense}{HTML}{0065CA}
\definecolor{ansi-magenta}{HTML}{D160C4}
\definecolor{ansi-magenta-intense}{HTML}{A03196}
\definecolor{ansi-cyan}{HTML}{60C6C8}
\definecolor{ansi-cyan-intense}{HTML}{258F8F}
\definecolor{ansi-white}{HTML}{C5C1B4}
\definecolor{ansi-white-intense}{HTML}{A1A6B2}
\definecolor{ansi-default-inverse-fg}{HTML}{FFFFFF}
\definecolor{ansi-default-inverse-bg}{HTML}{000000}
    

% \usepackage{framed}
% \newenvironment{Shaded}{}{}
% \newcommand{\KeywordTok}[1]{\textcolor{keywcolor}{\textbf{{#1}}}}
% \newcommand{\DataTypeTok}[1]{\textcolor{datacolor}{{#1}}}
% %\newcommand{\DecValTok}[1]{\textcolor{decvcolor}{{#1}}}
% \newcommand{\DecValTok}[1]{{#1}} 
% \newcommand{\BaseNTok}[1]{\textcolor{basencolor}{{#1}}}
% \newcommand{\FloatTok}[1]{\textcolor{floatcolor}{{#1}}}
% \newcommand{\CharTok}[1]{\textcolor{charcolor}{\textbf{{#1}}}}
% \newcommand{\StringTok}[1]{\textcolor{stringcolor}{{#1}}}
% \newcommand{\CommentTok}[1]{\textcolor{commentcolor}{\textit{{#1}}}}
% \newcommand{\OtherTok}[1]{\textcolor{othercolor}{{#1}}} 
% \newcommand{\AlertTok}[1]{\textcolor{alertcolor}{\textbf{{#1}}}}
% %\newcommand{\FunctionTok}[1]{\textcolor{funccolor}{{#1}}}
% \newcommand{\FunctionTok}[1]{{#1}}
% \newcommand{\RegionMarkerTok}[1]{{#1}}
% \newcommand{\ErrorTok}[1]{\textbf{{#1}}}
% \newcommand{\NormalTok}[1]{{#1}}

% The default LaTeX title has an obnoxious amount of whitespace. By default,
% titling removes some of it. It also provides customization options.
\usepackage{titling}

% headers and footers
\usepackage{fancyhdr}
%\pagestyle{fancy}
\pagestyle{plain}

\fancyhead{}
\fancyfoot{}

%\fancyhead[LE,RO]{\slshape \rightmark}
%\fancyhead[LO,RE]{\slshape \leftmark}
\fancyfoot[C]{\thepage}
%\headrulewidth 0.4pt
%\footrulewidth 0 pt

%\addtolength{\headheight}{\baselineskip}

%\lfoot{\emph{Analyze the Data not the Drivel}}
%\rfoot{\emph{\today}}

% subfigure handles figures that contain subfigures
%\usepackage{color,graphicx,subfigure,sidecap}
\usepackage{graphicx,sidecap}
\usepackage{subfigure}
\graphicspath{{./inclusions/}}

% floatflt provides for text wrapping around small figures and tables
\usepackage{floatflt}

% tweak caption formats 
\usepackage{caption} 
\usepackage{sidecap}
%\usepackage{subcaption} % not compatible with subfigure

\usepackage{rotating} % flip tables sideways

% complex footnotes
%\usepackage{bigfoot}

% weird logos \XeLaTeX
\usepackage{metalogo}

\newcommand{\HRule}{\rule{\linewidth}{0.5mm}}

\usepackage[breakable]{tcolorbox}

\usepackage{parskip} % Stop auto-indenting (to mimic markdown behaviour)
    
% Basic figure setup, for now with no caption control since it's done
% automatically by Pandoc (which extracts ![](path) syntax from Markdown).
\usepackage{graphicx}

%\DeclareCaptionFormat{nocaption}{}
%\captionsetup{format=nocaption,aboveskip=0pt,belowskip=0pt}

\usepackage[Export]{adjustbox} % Used to constrain images to a maximum size
\adjustboxset{max size={0.9\linewidth}{0.9\paperheight}}
\usepackage{float}

%\floatplacement{figure}{H} % forces figures to be placed at the correct location

\usepackage{xcolor} % Allow colors to be defined
\usepackage{enumerate} % Needed for markdown enumerations to work
\usepackage{geometry} % Used to adjust the document margins

%\usepackage{amsmath} % Equations
%\usepackage{amssymb} % Equations

\usepackage{textcomp} % defines textquotesingle

% Hack from http://tex.stackexchange.com/a/47451/13684:
\AtBeginDocument{%
	\def\PYZsq{\textquotesingle}% Upright quotes in Pygmentized code
}

\usepackage{upquote} % Upright quotes for verbatim code
\usepackage{eurosym} % defines \euro
\usepackage[mathletters]{ucs} % Extended unicode (utf-8) support

%\usepackage{fancyvrb} % verbatim replacement that allows latex

\usepackage{grffile} % extends the file name processing of package graphics 
					 % to support a larger range
					 
\makeatletter % fix for grffile with XeLaTeX
\def\Gread@@xetex#1{%
  \IfFileExists{"\Gin@base".bb}%
  {\Gread@eps{\Gin@base.bb}}%
  {\Gread@@xetex@aux#1}%
}
\makeatother

% The hyperref package gives us a pdf with properly built
% internal navigation ('pdf bookmarks' for the table of contents,
% internal cross-reference links, web links for URLs, etc.)
\usepackage{hyperref}
% The default LaTeX title has an obnoxious amount of whitespace. By default,
% titling removes some of it. It also provides customization options.
\usepackage{titling}
\usepackage{longtable} % longtable support required by pandoc >1.10
\usepackage{booktabs}  % table support for pandoc > 1.12.2
\usepackage[inline]{enumitem} % IRkernel/repr support (it uses the enumerate* environment)
\usepackage[normalem]{ulem} % ulem is needed to support strikethroughs (\sout)
							% normalem makes italics be italics, not underlines
\usepackage{mathrsfs}

% commands and environments needed by pandoc snippets
% extracted from the output of `pandoc -s`
\providecommand{\tightlist}{%
  \setlength{\itemsep}{0pt}\setlength{\parskip}{0pt}}
  
\DefineVerbatimEnvironment{Highlighting}{Verbatim}{commandchars=\\\{\}}
% Add ',fontsize=\small' for more characters per line
\newenvironment{Shaded}{}{}
\newcommand{\KeywordTok}[1]{\textcolor[rgb]{0.00,0.44,0.13}{\textbf{{#1}}}}
\newcommand{\DataTypeTok}[1]{\textcolor[rgb]{0.56,0.13,0.00}{{#1}}}
\newcommand{\DecValTok}[1]{\textcolor[rgb]{0.25,0.63,0.44}{{#1}}}
\newcommand{\BaseNTok}[1]{\textcolor[rgb]{0.25,0.63,0.44}{{#1}}}
\newcommand{\FloatTok}[1]{\textcolor[rgb]{0.25,0.63,0.44}{{#1}}}
\newcommand{\CharTok}[1]{\textcolor[rgb]{0.25,0.44,0.63}{{#1}}}
\newcommand{\StringTok}[1]{\textcolor[rgb]{0.25,0.44,0.63}{{#1}}}
\newcommand{\CommentTok}[1]{\textcolor[rgb]{0.38,0.63,0.69}{\textit{{#1}}}}
\newcommand{\OtherTok}[1]{\textcolor[rgb]{0.00,0.44,0.13}{{#1}}}
\newcommand{\AlertTok}[1]{\textcolor[rgb]{1.00,0.00,0.00}{\textbf{{#1}}}}
\newcommand{\FunctionTok}[1]{\textcolor[rgb]{0.02,0.16,0.49}{{#1}}}
\newcommand{\RegionMarkerTok}[1]{{#1}}
\newcommand{\ErrorTok}[1]{\textcolor[rgb]{1.00,0.00,0.00}{\textbf{{#1}}}}
\newcommand{\NormalTok}[1]{{#1}}

% Additional commands for more recent versions of Pandoc
\newcommand{\ConstantTok}[1]{\textcolor[rgb]{0.53,0.00,0.00}{{#1}}}
\newcommand{\SpecialCharTok}[1]{\textcolor[rgb]{0.25,0.44,0.63}{{#1}}}
\newcommand{\VerbatimStringTok}[1]{\textcolor[rgb]{0.25,0.44,0.63}{{#1}}}
\newcommand{\SpecialStringTok}[1]{\textcolor[rgb]{0.73,0.40,0.53}{{#1}}}
\newcommand{\ImportTok}[1]{{#1}}
\newcommand{\DocumentationTok}[1]{\textcolor[rgb]{0.73,0.13,0.13}{\textit{{#1}}}}
\newcommand{\AnnotationTok}[1]{\textcolor[rgb]{0.38,0.63,0.69}{\textbf{\textit{{#1}}}}}
\newcommand{\CommentVarTok}[1]{\textcolor[rgb]{0.38,0.63,0.69}{\textbf{\textit{{#1}}}}}
\newcommand{\VariableTok}[1]{\textcolor[rgb]{0.10,0.09,0.49}{{#1}}}
\newcommand{\ControlFlowTok}[1]{\textcolor[rgb]{0.00,0.44,0.13}{\textbf{{#1}}}}
\newcommand{\OperatorTok}[1]{\textcolor[rgb]{0.40,0.40,0.40}{{#1}}}
\newcommand{\BuiltInTok}[1]{{#1}}
\newcommand{\ExtensionTok}[1]{{#1}}
\newcommand{\PreprocessorTok}[1]{\textcolor[rgb]{0.74,0.48,0.00}{{#1}}}
\newcommand{\AttributeTok}[1]{\textcolor[rgb]{0.49,0.56,0.16}{{#1}}}
\newcommand{\InformationTok}[1]{\textcolor[rgb]{0.38,0.63,0.69}{\textbf{\textit{{#1}}}}}
\newcommand{\WarningTok}[1]{\textcolor[rgb]{0.38,0.63,0.69}{\textbf{\textit{{#1}}}}}

% Define a nice break command that doesn't care if a line doesn't already exist.
\def\br{\hspace*{\fill} \\* }
% Math Jax compatibility definitions
\def\gt{>}
\def\lt{<}
\let\Oldtex\TeX
\let\Oldlatex\LaTeX
\renewcommand{\TeX}{\textrm{\Oldtex}}
\renewcommand{\LaTeX}{\textrm{\Oldlatex}}
 
% Pygments definitions
\makeatletter
\def\PY@reset{\let\PY@it=\relax \let\PY@bf=\relax%
    \let\PY@ul=\relax \let\PY@tc=\relax%
    \let\PY@bc=\relax \let\PY@ff=\relax}
\def\PY@tok#1{\csname PY@tok@#1\endcsname}
\def\PY@toks#1+{\ifx\relax#1\empty\else%
    \PY@tok{#1}\expandafter\PY@toks\fi}
\def\PY@do#1{\PY@bc{\PY@tc{\PY@ul{%
    \PY@it{\PY@bf{\PY@ff{#1}}}}}}}
\def\PY#1#2{\PY@reset\PY@toks#1+\relax+\PY@do{#2}}

\expandafter\def\csname PY@tok@w\endcsname{\def\PY@tc##1{\textcolor[rgb]{0.73,0.73,0.73}{##1}}}
\expandafter\def\csname PY@tok@c\endcsname{\let\PY@it=\textit\def\PY@tc##1{\textcolor[rgb]{0.25,0.50,0.50}{##1}}}
\expandafter\def\csname PY@tok@cp\endcsname{\def\PY@tc##1{\textcolor[rgb]{0.74,0.48,0.00}{##1}}}
\expandafter\def\csname PY@tok@k\endcsname{\let\PY@bf=\textbf\def\PY@tc##1{\textcolor[rgb]{0.00,0.50,0.00}{##1}}}
\expandafter\def\csname PY@tok@kp\endcsname{\def\PY@tc##1{\textcolor[rgb]{0.00,0.50,0.00}{##1}}}
\expandafter\def\csname PY@tok@kt\endcsname{\def\PY@tc##1{\textcolor[rgb]{0.69,0.00,0.25}{##1}}}
\expandafter\def\csname PY@tok@o\endcsname{\def\PY@tc##1{\textcolor[rgb]{0.40,0.40,0.40}{##1}}}
\expandafter\def\csname PY@tok@ow\endcsname{\let\PY@bf=\textbf\def\PY@tc##1{\textcolor[rgb]{0.67,0.13,1.00}{##1}}}
\expandafter\def\csname PY@tok@nb\endcsname{\def\PY@tc##1{\textcolor[rgb]{0.00,0.50,0.00}{##1}}}
\expandafter\def\csname PY@tok@nf\endcsname{\def\PY@tc##1{\textcolor[rgb]{0.00,0.00,1.00}{##1}}}
\expandafter\def\csname PY@tok@nc\endcsname{\let\PY@bf=\textbf\def\PY@tc##1{\textcolor[rgb]{0.00,0.00,1.00}{##1}}}
\expandafter\def\csname PY@tok@nn\endcsname{\let\PY@bf=\textbf\def\PY@tc##1{\textcolor[rgb]{0.00,0.00,1.00}{##1}}}
\expandafter\def\csname PY@tok@ne\endcsname{\let\PY@bf=\textbf\def\PY@tc##1{\textcolor[rgb]{0.82,0.25,0.23}{##1}}}
\expandafter\def\csname PY@tok@nv\endcsname{\def\PY@tc##1{\textcolor[rgb]{0.10,0.09,0.49}{##1}}}
\expandafter\def\csname PY@tok@no\endcsname{\def\PY@tc##1{\textcolor[rgb]{0.53,0.00,0.00}{##1}}}
\expandafter\def\csname PY@tok@nl\endcsname{\def\PY@tc##1{\textcolor[rgb]{0.63,0.63,0.00}{##1}}}
\expandafter\def\csname PY@tok@ni\endcsname{\let\PY@bf=\textbf\def\PY@tc##1{\textcolor[rgb]{0.60,0.60,0.60}{##1}}}
\expandafter\def\csname PY@tok@na\endcsname{\def\PY@tc##1{\textcolor[rgb]{0.49,0.56,0.16}{##1}}}
\expandafter\def\csname PY@tok@nt\endcsname{\let\PY@bf=\textbf\def\PY@tc##1{\textcolor[rgb]{0.00,0.50,0.00}{##1}}}
\expandafter\def\csname PY@tok@nd\endcsname{\def\PY@tc##1{\textcolor[rgb]{0.67,0.13,1.00}{##1}}}
\expandafter\def\csname PY@tok@s\endcsname{\def\PY@tc##1{\textcolor[rgb]{0.73,0.13,0.13}{##1}}}
\expandafter\def\csname PY@tok@sd\endcsname{\let\PY@it=\textit\def\PY@tc##1{\textcolor[rgb]{0.73,0.13,0.13}{##1}}}
\expandafter\def\csname PY@tok@si\endcsname{\let\PY@bf=\textbf\def\PY@tc##1{\textcolor[rgb]{0.73,0.40,0.53}{##1}}}
\expandafter\def\csname PY@tok@se\endcsname{\let\PY@bf=\textbf\def\PY@tc##1{\textcolor[rgb]{0.73,0.40,0.13}{##1}}}
\expandafter\def\csname PY@tok@sr\endcsname{\def\PY@tc##1{\textcolor[rgb]{0.73,0.40,0.53}{##1}}}
\expandafter\def\csname PY@tok@ss\endcsname{\def\PY@tc##1{\textcolor[rgb]{0.10,0.09,0.49}{##1}}}
\expandafter\def\csname PY@tok@sx\endcsname{\def\PY@tc##1{\textcolor[rgb]{0.00,0.50,0.00}{##1}}}
\expandafter\def\csname PY@tok@m\endcsname{\def\PY@tc##1{\textcolor[rgb]{0.40,0.40,0.40}{##1}}}
\expandafter\def\csname PY@tok@gh\endcsname{\let\PY@bf=\textbf\def\PY@tc##1{\textcolor[rgb]{0.00,0.00,0.50}{##1}}}
\expandafter\def\csname PY@tok@gu\endcsname{\let\PY@bf=\textbf\def\PY@tc##1{\textcolor[rgb]{0.50,0.00,0.50}{##1}}}
\expandafter\def\csname PY@tok@gd\endcsname{\def\PY@tc##1{\textcolor[rgb]{0.63,0.00,0.00}{##1}}}
\expandafter\def\csname PY@tok@gi\endcsname{\def\PY@tc##1{\textcolor[rgb]{0.00,0.63,0.00}{##1}}}
\expandafter\def\csname PY@tok@gr\endcsname{\def\PY@tc##1{\textcolor[rgb]{1.00,0.00,0.00}{##1}}}
\expandafter\def\csname PY@tok@ge\endcsname{\let\PY@it=\textit}
\expandafter\def\csname PY@tok@gs\endcsname{\let\PY@bf=\textbf}
\expandafter\def\csname PY@tok@gp\endcsname{\let\PY@bf=\textbf\def\PY@tc##1{\textcolor[rgb]{0.00,0.00,0.50}{##1}}}
\expandafter\def\csname PY@tok@go\endcsname{\def\PY@tc##1{\textcolor[rgb]{0.53,0.53,0.53}{##1}}}
\expandafter\def\csname PY@tok@gt\endcsname{\def\PY@tc##1{\textcolor[rgb]{0.00,0.27,0.87}{##1}}}
\expandafter\def\csname PY@tok@err\endcsname{\def\PY@bc##1{\setlength{\fboxsep}{0pt}\fcolorbox[rgb]{1.00,0.00,0.00}{1,1,1}{\strut ##1}}}
\expandafter\def\csname PY@tok@kc\endcsname{\let\PY@bf=\textbf\def\PY@tc##1{\textcolor[rgb]{0.00,0.50,0.00}{##1}}}
\expandafter\def\csname PY@tok@kd\endcsname{\let\PY@bf=\textbf\def\PY@tc##1{\textcolor[rgb]{0.00,0.50,0.00}{##1}}}
\expandafter\def\csname PY@tok@kn\endcsname{\let\PY@bf=\textbf\def\PY@tc##1{\textcolor[rgb]{0.00,0.50,0.00}{##1}}}
\expandafter\def\csname PY@tok@kr\endcsname{\let\PY@bf=\textbf\def\PY@tc##1{\textcolor[rgb]{0.00,0.50,0.00}{##1}}}
\expandafter\def\csname PY@tok@bp\endcsname{\def\PY@tc##1{\textcolor[rgb]{0.00,0.50,0.00}{##1}}}
\expandafter\def\csname PY@tok@fm\endcsname{\def\PY@tc##1{\textcolor[rgb]{0.00,0.00,1.00}{##1}}}
\expandafter\def\csname PY@tok@vc\endcsname{\def\PY@tc##1{\textcolor[rgb]{0.10,0.09,0.49}{##1}}}
\expandafter\def\csname PY@tok@vg\endcsname{\def\PY@tc##1{\textcolor[rgb]{0.10,0.09,0.49}{##1}}}
\expandafter\def\csname PY@tok@vi\endcsname{\def\PY@tc##1{\textcolor[rgb]{0.10,0.09,0.49}{##1}}}
\expandafter\def\csname PY@tok@vm\endcsname{\def\PY@tc##1{\textcolor[rgb]{0.10,0.09,0.49}{##1}}}
\expandafter\def\csname PY@tok@sa\endcsname{\def\PY@tc##1{\textcolor[rgb]{0.73,0.13,0.13}{##1}}}
\expandafter\def\csname PY@tok@sb\endcsname{\def\PY@tc##1{\textcolor[rgb]{0.73,0.13,0.13}{##1}}}
\expandafter\def\csname PY@tok@sc\endcsname{\def\PY@tc##1{\textcolor[rgb]{0.73,0.13,0.13}{##1}}}
\expandafter\def\csname PY@tok@dl\endcsname{\def\PY@tc##1{\textcolor[rgb]{0.73,0.13,0.13}{##1}}}
\expandafter\def\csname PY@tok@s2\endcsname{\def\PY@tc##1{\textcolor[rgb]{0.73,0.13,0.13}{##1}}}
\expandafter\def\csname PY@tok@sh\endcsname{\def\PY@tc##1{\textcolor[rgb]{0.73,0.13,0.13}{##1}}}
\expandafter\def\csname PY@tok@s1\endcsname{\def\PY@tc##1{\textcolor[rgb]{0.73,0.13,0.13}{##1}}}
\expandafter\def\csname PY@tok@mb\endcsname{\def\PY@tc##1{\textcolor[rgb]{0.40,0.40,0.40}{##1}}}
\expandafter\def\csname PY@tok@mf\endcsname{\def\PY@tc##1{\textcolor[rgb]{0.40,0.40,0.40}{##1}}}
\expandafter\def\csname PY@tok@mh\endcsname{\def\PY@tc##1{\textcolor[rgb]{0.40,0.40,0.40}{##1}}}
\expandafter\def\csname PY@tok@mi\endcsname{\def\PY@tc##1{\textcolor[rgb]{0.40,0.40,0.40}{##1}}}
\expandafter\def\csname PY@tok@il\endcsname{\def\PY@tc##1{\textcolor[rgb]{0.40,0.40,0.40}{##1}}}
\expandafter\def\csname PY@tok@mo\endcsname{\def\PY@tc##1{\textcolor[rgb]{0.40,0.40,0.40}{##1}}}
\expandafter\def\csname PY@tok@ch\endcsname{\let\PY@it=\textit\def\PY@tc##1{\textcolor[rgb]{0.25,0.50,0.50}{##1}}}
\expandafter\def\csname PY@tok@cm\endcsname{\let\PY@it=\textit\def\PY@tc##1{\textcolor[rgb]{0.25,0.50,0.50}{##1}}}
\expandafter\def\csname PY@tok@cpf\endcsname{\let\PY@it=\textit\def\PY@tc##1{\textcolor[rgb]{0.25,0.50,0.50}{##1}}}
\expandafter\def\csname PY@tok@c1\endcsname{\let\PY@it=\textit\def\PY@tc##1{\textcolor[rgb]{0.25,0.50,0.50}{##1}}}
\expandafter\def\csname PY@tok@cs\endcsname{\let\PY@it=\textit\def\PY@tc##1{\textcolor[rgb]{0.25,0.50,0.50}{##1}}}

\def\PYZbs{\char`\\}
\def\PYZus{\char`\_}
\def\PYZob{\char`\{}
\def\PYZcb{\char`\}}
\def\PYZca{\char`\^}
\def\PYZam{\char`\&}
\def\PYZlt{\char`\<}
\def\PYZgt{\char`\>}
\def\PYZsh{\char`\#}
\def\PYZpc{\char`\%}
\def\PYZdl{\char`\$}
\def\PYZhy{\char`\-}
\def\PYZsq{\char`\'}
\def\PYZdq{\char`\"}
\def\PYZti{\char`\~}
% for compatibility with earlier versions
\def\PYZat{@}
\def\PYZlb{[}
\def\PYZrb{]}
\makeatother

% For linebreaks inside Verbatim environment from package fancyvrb. 
\makeatletter
	\newbox\Wrappedcontinuationbox 
	\newbox\Wrappedvisiblespacebox 
	\newcommand*\Wrappedvisiblespace {\textcolor{red}{\textvisiblespace}} 
	\newcommand*\Wrappedcontinuationsymbol {\textcolor{red}{\llap{\tiny$\m@th\hookrightarrow$}}} 
	\newcommand*\Wrappedcontinuationindent {3ex } 
	\newcommand*\Wrappedafterbreak {\kern\Wrappedcontinuationindent\copy\Wrappedcontinuationbox} 
	% Take advantage of the already applied Pygments mark-up to insert 
	% potential linebreaks for TeX processing. 
	%        {, <, #, %, $, ' and ": go to next line. 
	%        _, }, ^, &, >, - and ~: stay at end of broken line. 
	% Use of \textquotesingle for straight quote. 
	\newcommand*\Wrappedbreaksatspecials {% 
		\def\PYGZus{\discretionary{\char`\_}{\Wrappedafterbreak}{\char`\_}}% 
		\def\PYGZob{\discretionary{}{\Wrappedafterbreak\char`\{}{\char`\{}}% 
		\def\PYGZcb{\discretionary{\char`\}}{\Wrappedafterbreak}{\char`\}}}% 
		\def\PYGZca{\discretionary{\char`\^}{\Wrappedafterbreak}{\char`\^}}% 
		\def\PYGZam{\discretionary{\char`\&}{\Wrappedafterbreak}{\char`\&}}% 
		\def\PYGZlt{\discretionary{}{\Wrappedafterbreak\char`\<}{\char`\<}}% 
		\def\PYGZgt{\discretionary{\char`\>}{\Wrappedafterbreak}{\char`\>}}% 
		\def\PYGZsh{\discretionary{}{\Wrappedafterbreak\char`\#}{\char`\#}}% 
		\def\PYGZpc{\discretionary{}{\Wrappedafterbreak\char`\%}{\char`\%}}% 
		\def\PYGZdl{\discretionary{}{\Wrappedafterbreak\char`\$}{\char`\$}}% 
		\def\PYGZhy{\discretionary{\char`\-}{\Wrappedafterbreak}{\char`\-}}% 
		\def\PYGZsq{\discretionary{}{\Wrappedafterbreak\textquotesingle}{\textquotesingle}}% 
		\def\PYGZdq{\discretionary{}{\Wrappedafterbreak\char`\"}{\char`\"}}% 
		\def\PYGZti{\discretionary{\char`\~}{\Wrappedafterbreak}{\char`\~}}% 
	} 
	% Some characters . , ; ? ! / are not pygmentized. 
	% This macro makes them "active" and they will insert potential linebreaks 
	\newcommand*\Wrappedbreaksatpunct {% 
		\lccode`\~`\.\lowercase{\def~}{\discretionary{\hbox{\char`\.}}{\Wrappedafterbreak}{\hbox{\char`\.}}}% 
		\lccode`\~`\,\lowercase{\def~}{\discretionary{\hbox{\char`\,}}{\Wrappedafterbreak}{\hbox{\char`\,}}}% 
		\lccode`\~`\;\lowercase{\def~}{\discretionary{\hbox{\char`\;}}{\Wrappedafterbreak}{\hbox{\char`\;}}}% 
		\lccode`\~`\:\lowercase{\def~}{\discretionary{\hbox{\char`\:}}{\Wrappedafterbreak}{\hbox{\char`\:}}}% 
		\lccode`\~`\?\lowercase{\def~}{\discretionary{\hbox{\char`\?}}{\Wrappedafterbreak}{\hbox{\char`\?}}}% 
		\lccode`\~`\!\lowercase{\def~}{\discretionary{\hbox{\char`\!}}{\Wrappedafterbreak}{\hbox{\char`\!}}}% 
		\lccode`\~`\/\lowercase{\def~}{\discretionary{\hbox{\char`\/}}{\Wrappedafterbreak}{\hbox{\char`\/}}}% 
		\catcode`\.\active
		\catcode`\,\active 
		\catcode`\;\active
		\catcode`\:\active
		\catcode`\?\active
		\catcode`\!\active
		\catcode`\/\active 
		\lccode`\~`\~ 	
	}
\makeatother

\let\OriginalVerbatim=\Verbatim
\makeatletter
\renewcommand{\Verbatim}[1][1]{%
	%\parskip\z@skip
	\sbox\Wrappedcontinuationbox {\Wrappedcontinuationsymbol}%
	\sbox\Wrappedvisiblespacebox {\FV@SetupFont\Wrappedvisiblespace}%
	\def\FancyVerbFormatLine ##1{\hsize\linewidth
		\vtop{\raggedright\hyphenpenalty\z@\exhyphenpenalty\z@
			\doublehyphendemerits\z@\finalhyphendemerits\z@
			\strut ##1\strut}%
	}%
	% If the linebreak is at a space, the latter will be displayed as visible
	% space at end of first line, and a continuation symbol starts next line.
	% Stretch/shrink are however usually zero for typewriter font.
	\def\FV@Space {%
		\nobreak\hskip\z@ plus\fontdimen3\font minus\fontdimen4\font
		\discretionary{\copy\Wrappedvisiblespacebox}{\Wrappedafterbreak}
		{\kern\fontdimen2\font}%
	}%
	
	% Allow breaks at special characters using \PYG... macros.
	\Wrappedbreaksatspecials
	% Breaks at punctuation characters . , ; ? ! and / need catcode=\active 	
	\OriginalVerbatim[#1,codes*=\Wrappedbreaksatpunct]%
}
\makeatother


% prompt
\makeatletter
\newcommand{\boxspacing}{\kern\kvtcb@left@rule\kern\kvtcb@boxsep}
\makeatother
\newcommand{\prompt}[4]{
	\ttfamily\llap{{\color{#2}[#3]:\hspace{3pt}#4}}\vspace{-\baselineskip}
}
    

% Prevent overflowing lines due to hard-to-break entities
\sloppy 

% Setup hyperref package
\hypersetup{
  breaklinks=true,  % so long urls are correctly broken across lines
  colorlinks=true,
  urlcolor=urlcolor,
  linkcolor=linkcolor,
  citecolor=citecolor,
  pdfauthor={John D. Baker},
  pdftitle={Analyze the Data not the Drivel},
  pdfsubject={Blog},
  pdfcreator={MikTeX+LaTeXe},
  pdfkeywords={blog,wordpress},
  }
  
% Slightly bigger margins than the latex defaults
% \geometry{verbose,tmargin=1in,bmargin=1in,lmargin=1in,rmargin=1in}  

%\usepackage{wrapfig}

% source code listings
\usepackage{listings}

\lstdefinelanguage{bat}
{morekeywords={echo,title,pushd,popd,setlocal,endlocal,off,if,not,exist,set,goto,pause},
sensitive=True,
morecomment=[l]{rem}
}

\lstdefinelanguage{jdoc}
{
morekeywords={},
otherkeywords={assert.,break.,continue.,for.,do.,if.,else.,elseif.,return.,select.,end.
,while.,whilst.,throw.,catch.,catchd.,catcht.,try.,case.,fcase.},
sensitive=True,
morecomment=[l]{NB.},
morestring=[b]',
morestring=[d]',
}

% latex size ordering - can never remember it
% \tiny
% \scriptsize
% \footnotesize
% \small
% \normalsize
% \large
% \Large
% \LARGE
% \huge
% \Huge
 
% listings package settings  
\lstset{%
  language=jdoc,                                % j document settings
  basicstyle=\ttfamily\footnotesize,            
  keywordstyle=\bfseries\color{keywcolor}\footnotesize,
  identifierstyle=\color{black},
  commentstyle=\slshape\color{CodeComment},     % colored slanted comments
  stringstyle=\color{red}\ttfamily,
  showstringspaces=false,                       
  %backgroundcolor=\color{CodeBackGround},       
  frame=single,                                
  framesep=1pt,                                 
  framerule=0.8pt,                             
  rulecolor=\color{CodeBackGround},   
  showspaces=false,
  %columns=fullflexible,
  %numbers=left,
  %numberstyle=\footnotesize,
  %numbersep=9pt,
  tabsize=2,
  showtabs=false,
  captionpos=b
  breaklines=true,                              
  breakindent=5pt                              
}

\lstdefinelanguage{JavaScript}{
  keywords={typeof, new, true, false, catch, function, return, null, catch, switch, var, if, in, while, do, else, case, break},
  ndkeywords={class, export, boolean, throw, implements, import, this},
  ndkeywordstyle=\color{darkgray}\bfseries,
  sensitive=false,
  comment=[l]{//},
  morecomment=[s]{/*}{*/},
  morestring=[b]',
  morestring=[b]"
}

% C# settings
\lstdefinestyle{sharpc}{
language=[Sharp]C,
basicstyle=\ttfamily\scriptsize, 
keywordstyle=\bfseries\color{keywcolor}\scriptsize,
framerule=0pt
}

% for source code listing longer than two use smaller font
\lstdefinestyle{smallersource}{
basicstyle=\ttfamily\scriptsize, 
keywordstyle=\bfseries\color{keywcolor}\scriptsize,
framerule=0pt
}

\lstdefinestyle{resetdefaults}{
language=jdoc,
basicstyle=\ttfamily\footnotesize,  
keywordstyle=\bfseries\color{keywcolor}\footnotesize,                                                               
framerule=0.8pt 
}

% APL UTF8 code points listed for lstlisting processing
\makeatletter
\lst@InputCatcodes
\def\lst@DefEC{%
 \lst@CCECUse \lst@ProcessLetter
  ^^80^^81^^82^^83^^84^^85^^86^^87^^88^^89^^8a^^8b^^8c^^8d^^8e^^8f%
  ^^90^^91^^92^^93^^94^^95^^96^^97^^98^^99^^9a^^9b^^9c^^9d^^9e^^9f%
  ^^a0^^a1^^a2^^a3^^a4^^a5^^a6^^a7^^a8^^a9^^aa^^ab^^ac^^ad^^ae^^af%
  ^^b0^^b1^^b2^^b3^^b4^^b5^^b6^^b7^^b8^^b9^^ba^^bb^^bc^^bd^^be^^bf%
  ^^c0^^c1^^c2^^c3^^c4^^c5^^c6^^c7^^c8^^c9^^ca^^cb^^cc^^cd^^ce^^cf%
  ^^d0^^d1^^d2^^d3^^d4^^d5^^d6^^d7^^d8^^d9^^da^^db^^dc^^dd^^de^^df%
  ^^e0^^e1^^e2^^e3^^e4^^e5^^e6^^e7^^e8^^e9^^ea^^eb^^ec^^ed^^ee^^ef%
  ^^f0^^f1^^f2^^f3^^f4^^f5^^f6^^f7^^f8^^f9^^fa^^fb^^fc^^fd^^fe^^ff%
  ^^^^20ac^^^^0153^^^^0152%
  ^^^^20a7^^^^2190^^^^2191^^^^2192^^^^2193^^^^2206^^^^2207^^^^220a%
  ^^^^2218^^^^2228^^^^2229^^^^222a^^^^2235^^^^223c^^^^2260^^^^2261%
  ^^^^2262^^^^2264^^^^2265^^^^2282^^^^2283^^^^2296^^^^22a2^^^^22a3%
  ^^^^22a4^^^^22a5^^^^22c4^^^^2308^^^^230a^^^^2336^^^^2337^^^^2339%
  ^^^^233b^^^^233d^^^^233f^^^^2340^^^^2342^^^^2347^^^^2348^^^^2349%
  ^^^^234b^^^^234e^^^^2350^^^^2352^^^^2355^^^^2357^^^^2359^^^^235d%
  ^^^^235e^^^^235f^^^^2361^^^^2362^^^^2363^^^^2364^^^^2365^^^^2368%
  ^^^^236a^^^^236b^^^^236c^^^^2371^^^^2372^^^^2373^^^^2374^^^^2375%
  ^^^^2377^^^^2378^^^^237a^^^^2395^^^^25af^^^^25ca^^^^25cb%  
  ^^00}
\lst@RestoreCatcodes
\makeatother

% custom lengths used within minipages
\newcommand{\minindent}{17pt}

\makeindex

\begin{document}

\subsection*{\href{http://analyzethedatanotthedrivel.org/2025/03/16/gonggone-gone-parts-9-10/}{Gonggone Gone --- Parts 9 \& 10}}
\addcontentsline{toc}{subsection}{Gonggone Gone --- Parts 9 \& 10}


\noindent\emph{Posted: 16 Mar 2025 19:45:03}
\vspace{6pt}

%\subsection{breaking chats}\label{breaking-chats}
\begin{center}\large\textbf{-- \emph{breaking chats} --}\normalsize\end{center}

On the first anniversary of runaway, as Earth crossed the gulf between
the orbits of Jupiter and Saturn, Alex and Doug took a day off and gave
themselves sponge baths in the airlock. It was one of their paramount
luxuries. Every few weeks, they'd fire up the airlock stove to the max
and heat bath water on it. When the water reached the verge of boiling,
they'd strip down, stand in a plastic tub, and rub themselves with hot,
wet, soapy rags. Life in the mine was a dirty business. Their tub wash
water was always dark from coal and cinder dust when they finished. To
conserve coal and water, they sponged together. It helped to have
another scrub your back in the tiny airlock.

As Alex rubbed the black grime off Doug's back, Doug always said, ``Just
for the record, we're not gay.''

Alex always responded with a quip like, ``Warm showers signaled peak gay
civilization.''

It had only been fifty-two weeks since they moved into the mine, but it
seemed like they had always been there. Reminiscing about their old,
warm, comfortable pre-runaway life felt like discussing Bronze Age
history. It happened, but so long ago, it no longer mattered. The daily
grind wore them down. Their food consumption had increased as they
worked longer and longer at the coal seam, extracting the fuel that kept
them alive. Like all heat engines, bodies require more fuel to do more
work. Alex's spreadsheet doom date kept moving up.

Despite their precarious circumstances, they were lucky. For most of
human history, people have lived two harvests from starvation. The
modern world thought it had broken this ancient cycle. Nations
stockpiled grains, rice, and other staples. Billions of chickens, cows,
and goats roamed well-tended pasture lands. If famine broke out in one
country, others would usually provide aid. Because
20\textsuperscript{th} and 21st-century famines were largely
self-inflicted \emph{regional} catastrophes, people lost sight of how
\emph{spare} their food margins were. Runaway Earth reminded everyone.
Never had harvests failed everywhere. Nor had it ever gotten so cold
livestock couldn't be grazed anywhere. Even ocean fishing failed. As the
ice packs expanded, fish stocks collapsed. Most species couldn't adapt
to changing ocean currents and lower water temperatures. With little
coming in and a lot going out, global food stores soon ran out.

People might have lasted longer with less food, but the cold crushed
their ability to move and adjust. Temperatures routinely dropped below
-100 degrees Celsius outside the mine: too cold for even the hardiest of
Arctic and Antarctic animals and far beyond what poorly equipped people
could handle. It was still warmer in the tropics, if you consider -60C
warm, but not enough to help. Without the cold, the landscape would be
swarming with refugees, but refugees wouldn't make it far without the
best cold-weather survival gear. For the last year, Alex lived in
perpetual fear of others finding them and forcing them into the cold.
Doug shared his concerns but constantly reminded Alex they hadn't seen a
single goddamn person despite watching hundreds of hours of 24x7 trail
camera video.

Fewer radio stations than expected observed the one-year runaway
anniversary. Propaganda FM said nothing. Some surviving national
short-wave services briefly mentioned it, but nobody devoted significant
broadcast time to the anniversary except for a few hams. Alex and Doug
didn't ``celebrate'' either. After drying themselves and eating a hot
dinner in the airlock, they squirmed into two \emph{cleanish} (their
limited hand tub laundry never got anything clean) layers of base layer
underwear and hurried back to the inner box where they crawled into
their warm sleeping bags. Getting extra sleep was celebration enough.

The following morning, after his outdoor astrograph observing session
(Alex focused on Saturn), Doug didn't join him in the airlock for lunch.
Doug started skipping meals. Their food supplies were dropping. Things
weren't critical yet; Alex's spreadsheets gave them about twenty more
months. But eventually, they'd starve like everyone else. Alex suspected
Doug didn't eat to prolong things. It worried him.

Calling on the walkie-talkie, Alex told Doug, ``I'm finishing up here.
The hot wash water thermos is ready; I'll grab some clean rags after
stoking the stove. I'll be at the box in a few minutes.''

They had given up tamping down their stoves to suppress exhaust plumes.
Severe cold forced them to pump as much warm air into the mine shaft as
possible. Doug suggested leaving the back airlock door open with the
stove burning while they mined coal, but Alex talked him out of it.

``Remember, we're in a coal mine. Open flames aren't a good idea.''

They compromised by making a second lightweight \emph{uninsulated}
airlock back door by cutting the last of their fence planks and wrapping
them in aluminum foil. Sealing the second foil-covered door with Velcro
and stapled duct tape; it neatly isolated the stove from the mine shaft.
When the stove burned full blast, they could feel the heat radiating
from the door. By tilting some of their small vent fans to blow over the
aluminum foil, they directed as much warm air as they could deeper into
the mine.

Doug crawled out of the inner box when Alex reached it. He carried a
recharged vacuum cleaner battery and looked like a giant Santa Claus in
his long red underwear. Before getting up, he put on one of the bicycle
helmets, which hung outside the inner box. Doug had repeatedly hit his
head in the mine shaft. Maintaining a bent, stooped posture is
exhausting. Doug could only completely stand up outside the mine. He
said nothing as he put on the helmet and squeezed around the box to the
back.

Alex quickly removed his outerwear and hung his parka and ski pants on a
clothesline they had rigged just below the inner box stove exhaust pipe.
On the clothesline, their outerwear draped over the Styrofoam-lined
generator box. Hanging outerwear helped keep the generator batteries
warm. Recently, they'd been having problems starting the generators, and
they had to run a jumper cable into the inner box powerpacks to jolt the
reticent machines. Like everything else, the generators assumed
\emph{normal} operating temperatures. And nothing is normal.

Out of his outerwear and exposed to the freezing shaft air, Alex
squeezed around the box carrying the wash water thermos. While Alex
moved around the box, Doug pulled on his mining parka. All the clothes
they used to work the seam hung \emph{behind} the box. They
\emph{always} changed out of their mining garb before squeezing around
the box. It helped keep coal and rock dust down in the inner and airlock
boxes. Alex hurried into his mining parka, ski pants, and insulated work
boots. Before dragging their tools to the seam on a little sled, they
put on their cut overalls. The overalls covered their cold-weather gear.
They had to cut the legs and waists to fit over their parkas. Their
overalls were their dirtiest garments. After working the coal seam, they
swept, vacuumed, and hung their overalls. And then washed their hands
and faces with warm, soapy water from the thermos. Keeping clean seemed
like a lost cause, but they did their best.

At the coal seam, Alex and Doug put on COVID and scuba masks. When
apocalypse shopping, Doug looked for transparent ski masks but couldn't
find any. Scuba masks were not tinted and kept coal dust out of their
eyes in the LED lantern-lit shaft. They protected their heads with
reworked bike helmets. To fit the helmets over their parka hoods, they
had to gouge out most of the foam lining.

Making constant adjustments made up a big part of their lives, or as
Alex often repeated, ``As long as you're alive, you have options.''

While working the seam, one of them sledgehammered pikes and chisels
into the face to break out coal or surrounding rock while the other,
sitting a few meters back, would clear away tailings and ``break'' coal.
Coal seldom comes out of the ground, ready to burn. It's often mixed
with inflammable detritus. How much depends on the quality of the seam.
In this mine, about a third of the material extracted from the seam
could not be burned. It had to be painstakingly broken out. In the
19\textsuperscript{th} century, coal mines often employed young boys,
called ``breakers,'' to do this tedious, exhausting job. Alex and Doug
did their own breaking. They'd dump newly mined chunks of coal on the
metal pan they found when they first opened the mine and hammer them
into smaller chunks. In the dim LED lantern light, it wasn't always easy
to distinguish coal from rock. Picking out the stones was indeed
tedious. Breaking was the dirtiest of their jobs. Emptying shit pails
paled in comparison.

This morning, Doug hesitated as he picked up the freezing steel pike
below the seam. In the shadow-making LED lantern light, the pit they had
laboriously dug back into the black coal seam looked like the mouth of a
giant grouper fish opening wide. The jagged, light-colored stone
surrounding the coal layer even made convincing teeth. To the shaft
grouper, Doug and Alex played the role of cleaner crustaceans who picked
parasites from the gaping mouths of fish. Both organisms benefit. The
crustaceans eat, and the fish shed parasites. The metaphor didn't quite
work for the mine. The coal kept them going, but what did the mine get?
Resentment?

As he reached for a mallet to drive the pike into the seam, Doug said,
``We're not going to last much longer.''

Alex almost asked \emph{where this came from.} But he didn't because he
knew where it came from---the truth. They had always known this only
delayed the inevitable: pushing it back another day, another week,
another month.

Before Doug could hammer the pike, Alex said, ``From now on, let's cut
all the awkward pretense. There's nobody left to impress or embarrass.
You know there is nothing you can say that will stop me from loving you.
So, say what you want because you're right; we won't last much longer.''

``Ok, I miss mom. I ask myself all the time if she's still alive. Do
you?''

``I doubt she's still alive, and no, I stopped loving your mother years
ago, long before she left. I don't know who gave up first. Alice went
cold right after your birth. She was always self-centered and
uncharitable, nicer to total strangers than her own damn family. She
harbored grudges and blamed me for everything. Honestly, I don't miss
her. I never think of her. But I know that if she were here, we'd both
be more miserable, admit it.''

Perhaps not expecting such a candid response, Doug paused before saying,
``You're right about the miserable bit. I loved mom. I still love Mom,
but sometimes she was \ldots''

``A bitch.''

``Not exactly. I'd say distant. I can't remember her ever taking an
interest in anything I did. Normal mothers take their kids to scouts;
they show up at teacher-parent conferences. Mom did some of this, but
she couldn't keep it up. She'd always bail when things got boring--- for
her. It felt like she didn't want me.''

``Oh, she wanted you; you're an invitro baby. She endured two or three
rounds to get pregnant, but you're right. She didn't support others.
\emph{She couldn't even be indifferent!} If you liked something she
didn't, she'd resent every goddamn second you spent on it. You were
always wasting your time. \emph{You should be making money for her!} It
pissed me off. She never, in twenty goddamn years, came outside and
looked through telescopes with me, but I was supposed to cater to her
hobbies. Remember line dancing?''

They both laughed.

``Yeah, she dragged both of us into that.''

``I was an awful dance partner; she kept pushing me for missing beats.
She was furious that I couldn't get it.''

``Ellen wasn't like that.''

``No, not at all. Ellen was genuinely \emph{kind}; you really lucked out
\ldots{} with her. Sorry.''

``Don't be, the entire fucking time I was with her, I was \ldots{} just
\emph{grateful}. I couldn't believe she was with me.''

``Well, I liked her, I really liked her, and honestly, it surprised me
too when she picked you.''

``Since we're dropping the BS, she was my first and only girlfriend. I
was a twenty-seven-year-old virgin until Ellen---a total loser.''

``I know young men get more shit than we did; I should have told you to
ignore the bitches. It's a key male coping strategy.''

``Dad, guys have always taken shit from women. It's part of their mating
rituals. We're just the first generation to grow up with dating apps in
the greedy paws of self-absorbed feminists. Girls do all the picking,
and unless you're a six-foot, six-figure, six-pack bro, you'll be
playing a lot of X-Box: a lot do ---or did.''

``Well, you met the six-foot criterion.''

They laughed again.

Waving his mallet like a gym weight, Doug added, ``After a year of this,
I've got the six-pack too.''

``We need to get busy. Let's keep doing this. Mining is boring.''

%\subsection{the queen}\label{the-queen}
\begin{center}\large\textbf{-- \emph{the queen} --}\normalsize\end{center}

Jupiter's approach had been impressive, but Saturn's exceeded Alex's
wildest expectations. From the 17th-century days of Christiaan Huygens,
Saturn has securely reigned as the solar system's beauty queen. Alex
dismissed people making claims for Earth, \emph{the piddly pale blue
dot}, as deranged child molesting woke commies with the aesthetic sense
of garden slugs. Anyone looking at magnificent Voyager and Cassini
images of Saturn with its rapturous rings could see \emph{it was the
maximum babe in a solar system of fat tattooed trannies.}

Alex would have spent more time outside the mine watching Saturn rise
and swell in the east, but with the Sun now reduced to less than a fifth
of its old size and delivering less than four percent of its old energy,
the already freezing surface cooled even faster. Earth's heat flow
resembled filling a big bathtub with a constantly expanding open drain.
It got so cold simple tasks --- like dumping shit outside --- became
deadly.

The garage was now so cold, about -80C, they worried about getting ass
frostbite when taking craps. They rigged a new ``bathroom'' near the
inner box. They cut tarps and hung them from shaft shoring to trap heat
from the inner box's stove exhaust vent. Inside the hanging tarps, they
set up Doug's toilet box and kept his neatly cut toilet seat inside the
inner box to keep it warm. It was warmer in the new ``bathroom'' than in
the garage. But, sitting on the toilet box remained unpleasant, and it
became more so with use because they didn't dump the box pail daily. You
learn all sorts of things when the world ends, like even frozen shit
stinks.

Their water supply concerned them. Dragging the sled down to the
catchment pond --- glacier --- became a risky undertaking. They could
stay outside about half an hour before the -120C weather forced a
retreat. This wasn't enough time to mine ice, which had solidified to
rock strength in the intense cold. It was hard to sledgehammer off
chunks and being so cold, it required more time in their melting pans,
which needed more coal. They started scooping snow instead of ice. It
was easier to melt, but being less dense, it produced less water and
needed more trips outside. To compensate, they reduced their water
consumption to an absolute minimum. One luxury they dearly missed was
their infrequent sponge baths. From now on, they only wiped their faces
and hands, and maybe, if they couldn't stand it anymore, their itchy
butts. They couldn't win; the cold was stalking them.

To conserve heat, they stopped opening the mine every day. This brought
new worries, such as the buildup of carbon monoxide. Alex obsessively
checked the detectors and alarms in the shaft and boxes. On top of all
these worries, their only relief, listening to the radio, was going
away. Seventy weeks after runaway, station after station went off the
air. Satellite propaganda FM stopped without warning. Imagine missing
mainstream media morons. A few short-wave stations and a handful of hams
kept broadcasting, but they were hard to find even with the radios'
automatic band scanning feature.

Frustrated with the radio, they tried reading some of their books. Doug
looked through his Manga stack several times but wasn't interested. They
started tearing out Manga pages to help start stove fires. Alex found it
impossible to concentrate or take the trivial problems of Stephen
Dedalus or Ishmael seriously. So, your wife is masturbating to other
men, and you, whiney little pussy that you are, go on a walking tour of
your hometown only to cast literary allusions like dogs peeing in the
park. You're no Ulysses you putz. He would have drowned himself before
enduring such debasements. And call you Ishmael; you've got a mean white
whale problem. Must be nice. Try living on a fucking planet that's done
with its Sun. Even Borges' clever circumlocutions didn't land anymore.
Years ago, Alex enjoyed the \emph{Library of Babel} but could no longer
muster sympathy for the poor inhabitants of the library. Try turning
down the thermostat to dry ice cold in the frigging library, then get
back to me about how tough life is.

Reading was out, but writing, image processing, and computing still
commanded Alex's attention. Every day, he meticulously updated his diary
and logbooks and transcribed daily data to a series of spreadsheet
models. His models tracked the relentless decline of outer and inner
temperatures, food and water consumption, coolers of coal burned, time
spent outside, time spent mining, propane levels, and powerpack levels.
All his calculations converged on a simple conclusion: they could not
hold out beyond three years. Imminent death makes us choose wisely or
foolishly or not at all. What would be a wise choice here? Alex told
himself over and over this was his last observing session, so he
observed.

In one of their now candid mining chats, Alex asked Doug why he kept
going. ``To spend time with you, Dad. It's why I don't walk out into the
snow. Mind you, your bean farts are a strong counterargument.''

Observing sessions split into night and day. In daylight hours, Alex
quickly screwed on the astrograph's solar filter and snapped a few
images of the Sun's disk. The shrinking Sun was an intense dot in the
sky. Far less light reached the Earth, but the landscape didn't look
much different to their eyes. Daylight had a subdued solar eclipse
quality: dimmer but not to the degree expected. Human vision responds to
changing light levels logarithmically: a tenfold decrease in
illumination seems about half as bright.

Runaway put an end to seasons. Spinning like a giant top, the Earth
maintained its orientation in space. It had always done so, but when
circling the Sun, this changed the view of the stellar background. The
stellar background no longer changed. The Sun slowly moved to a fixed
point in the sky. Each day, it would rise to the same altitude, in
almost precisely the same spot as the previous day. Maybe this would
change in thousands of years with polar precession, but for now days
were unchanging. So were nights. The seasonal march of constellations
ceased. The same stars rose at the same time and reached the same
altitude night after night. The biggest change, total daylight hours
remained the same, but the \emph{clock} time of sunrise and sunset kept
moving forward. At great distances, the Sun would rise and set about six
hours earlier than it had before. In the first months of runaway,
daylight hours shifted rapidly; now, they crept forward so slowly they
stopped adjusting their wake-up time.

Alex's solar disk images still showed sunspots. Quick calculations
indicated they would die before they could no longer resolve big
sunspots.

Nights went to the queen. In two months, Earth would make its closest
approach to Saturn. It would zoom over the northern pole within the
orbits of some Saturnian moons. Saturn has over a hundred moons. Alex
didn't have reliable orbits for all of them, so he couldn't say if we'd
slam into one. It didn't matter. Smashing into a large moon would be an
act of mercy, but it was unlikely. Douglas Adams said it best, ``Space
is big. You just won't believe how vastly, hugely, mind-bogglingly big
it is.'' You could fit millions of Earths within the orbits of Saturn's
moons. We'll still die, but probably not by moon impact. Saturn was
already closer than Mars used to be. Mars is a pathetic little thing,
barely qualifying as an official planet, but Saturn, now we're talking.
Alex's astrograph images were magnificent and improving.

Every third or fourth day, they opened the mine, braved the intense
cold, and snapped a few more astrograph images of Saturn. Because of the
dangerous cold, Doug stayed in the heated airlock, where he could
quickly help his dad if something came up. Whenever they went outside,
they kept up a running chat on their walkie-talkies. Any long pause in
the chatter probably meant trouble.

Alex had to work fast. The little telescope cooled rapidly when he
pulled it out of the scope box and unwrapped the heated electric blanket
around it. To keep a little warmer, Alex wrapped the electric blanket
over his outerwear. He had precisely choreographed his observing
routine. First, he quickly cleared snow off the patch of ground outside
the mine. Clearing snow got easier; precipitation had almost stopped at
mid-latitudes as the air dried. Any snow on Alex's observing ground
drifted in, but drifting also decreased as the solar energy driving
winds decreased. Some evenings, only a few inches of light ice crystal
snow needed pushing aside.

Next, he located and swept out tripod holes chiseled in the frozen
ground. It sped things up to know precisely where to place the
telescope. Extending the tripod legs, he set them in the guide holes. He
didn't bother with polar alignment, taking long-tracked guided
exposures, or CCD frame stacking. Advanced imaging techniques were
unnecessary for Saturn, a bright object he could see on the webcams. He
only needed to center it and start shooting sub-second frames. Basic
no-brainer imaging sufficed; he would have frozen solid shooting
night-long guided exposures.

Most nights, Alex finished imaging in twenty minutes. Fleeing into the
mine, he lowered and sealed the outer door, rewrapped the astrograph in
its electric blanket, and waited about twenty minutes for the scope to
warm up enough to remove the CCD memory card. After getting the card, he
put the scope back in its Styrofoam-lined storage box and scurried into
the airlock. Later, at the end of the day, when they were back in the
inner box and snuggled up in their sleeping bags, Alex copied the images
from the card to his laptop.

More and more details became visible. In a few more days, Saturn would
look to the naked eye like it used to in his Dobsonian telescope, a
pale-yellow dot encircled by rings and buzzed by tiny moons. Astrograph
images resolved many of the ring gaps and numerous moons. Earth would
come within three million kilometers of Saturn at its closest approach,
closer than Iapetus\textbf{.} Three million kilometers is almost eight
times the old Earth-Moon distance\emph{,} but Saturn is over thirty
times the diameter of the Moon. Even at three million kilometers, the
planet's disk would be four times the size of the old Moon and the rings
eight times.

In the final week of approach, Alex and Doug went outside every night.
They set up the astrograph and shot images of the expanding ringed
giant. After imaging, they peeled back their ski masks and scanned the
skies around Saturn with binoculars. From Earth's angle of approach,
Saturn's north pole hexagonal storm was visible in binoculars. They
could only take quick peeks through their binoculars. The cold burned
the skin of their exposed faces, and they had to avoid pressing the
binocular eyepieces to their eyes. Flesh would freeze on the eyepieces.

On fly-by day, they eagerly opened the mine. Saturn hung above the
eastern horizon, and its rings spanned their mittened hands. Earth's
viewpoint above Saturn yielded a partial crescent and long disk shadows
on the rings. Saturn's rings didn't fit in their binoculars; the
five-degree field of view wasn't wide enough. Before taking astrograph
images, they looked for the polar hexagon storm; it stood out as a
darker shade of pale yellow at the planet's pole. In binoculars, they
saw rich swirling cloud details, and Alex's best astrograph images
rivaled those of Voyager and Cassini.

Passing the queen felt sweet and sad. There would be no more planet
fly-bys, no more spectacles. Earth would plunge deeper and deeper into
cold, dark space.


%\end{document}
 
