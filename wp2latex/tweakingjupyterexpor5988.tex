%% uncomment to list all files in log
%\listfiles

\documentclass[12pt]{report}

\usepackage{fontspec}

%\setmainfont[Scale=MatchLowercase]{Lucida Bright}
%\setmonofont{FreeMono}
%\setmonofont{Source Code Pro}
\setmonofont[Scale=MatchLowercase]{Ubuntu Mono}

\usepackage[headings]{fullpage}

% national use characters 
%\usepackage{inputenc}

% ams mathematical symbols
\usepackage{amsmath,amssymb}

% added to support pandoc highlighting
\usepackage{microtype}

\usepackage{makeidx}

% add index and bibliographies to table of contents
\usepackage[nottoc]{tocbibind}

% postscript courier and times in place of cm fonts
%\usepackage{courier}
%\usepackage{times}

% extended coloring
\usepackage{color}
\usepackage[table,dvipsnames]{xcolor}
\usepackage{colortbl}

% advanced date formating
\usepackage{datetime}

%support pandoc code highlighting
\usepackage{fancyvrb}
\DefineShortVerb[commandchars=\\\{\}]{\|}
\DefineVerbatimEnvironment{Highlighting}{Verbatim}{commandchars=\\\{\}}
% Add ',fontsize=\small' for more characters per line

%tango style colors
% \usepackage{framed}
% \definecolor{shadecolor}{RGB}{255,255,255}
% \newenvironment{Shaded}{\begin{snugshade}}{\end{snugshade}}
% \newcommand{\KeywordTok}[1]{\textcolor[rgb]{0.13,0.29,0.53}{\textbf{{#1}}}}
% \newcommand{\DataTypeTok}[1]{\textcolor[rgb]{0.13,0.29,0.53}{{#1}}}
% \newcommand{\DecValTok}[1]{\textcolor[rgb]{0.00,0.00,0.81}{{#1}}}
% \newcommand{\BaseNTok}[1]{\textcolor[rgb]{0.00,0.00,0.81}{{#1}}}
% \newcommand{\FloatTok}[1]{\textcolor[rgb]{0.00,0.00,0.81}{{#1}}}
% \newcommand{\CharTok}[1]{\textcolor[rgb]{0.31,0.60,0.02}{{#1}}}
% \newcommand{\StringTok}[1]{\textcolor[rgb]{0.31,0.60,0.02}{{#1}}}
% \newcommand{\CommentTok}[1]{\textcolor[rgb]{0.56,0.35,0.01}{\textit{{#1}}}}
% \newcommand{\OtherTok}[1]{\textcolor[rgb]{0.56,0.35,0.01}{{#1}}}
% \newcommand{\AlertTok}[1]{\textcolor[rgb]{0.94,0.16,0.16}{{#1}}}
% \newcommand{\FunctionTok}[1]{\textcolor[rgb]{0.00,0.00,0.00}{{#1}}}
% \newcommand{\RegionMarkerTok}[1]{{#1}}
% \newcommand{\ErrorTok}[1]{\textbf{{#1}}}
% \newcommand{\NormalTok}[1]{{#1}}

%espresso style colors
% \usepackage{framed}
% \definecolor{shadecolor}{RGB}{42,33,28}
% \newenvironment{Shaded}{\begin{snugshade}}{\end{snugshade}}
% \newcommand{\KeywordTok}[1]{\textcolor[rgb]{0.26,0.66,0.93}{\textbf{{#1}}}}
% \newcommand{\DataTypeTok}[1]{\textcolor[rgb]{0.74,0.68,0.62}{\underline{{#1}}}}
% \newcommand{\DecValTok}[1]{\textcolor[rgb]{0.27,0.67,0.26}{{#1}}}
% \newcommand{\BaseNTok}[1]{\textcolor[rgb]{0.27,0.67,0.26}{{#1}}}
% \newcommand{\FloatTok}[1]{\textcolor[rgb]{0.27,0.67,0.26}{{#1}}}
% \newcommand{\CharTok}[1]{\textcolor[rgb]{0.02,0.61,0.04}{{#1}}}
% \newcommand{\StringTok}[1]{\textcolor[rgb]{0.02,0.61,0.04}{{#1}}}
% \newcommand{\CommentTok}[1]{\textcolor[rgb]{0.00,0.40,1.00}{\textit{{#1}}}}
% \newcommand{\OtherTok}[1]{\textcolor[rgb]{0.74,0.68,0.62}{{#1}}}
% \newcommand{\AlertTok}[1]{\textcolor[rgb]{1.00,1.00,0.00}{{#1}}}
% \newcommand{\FunctionTok}[1]{\textcolor[rgb]{1.00,0.58,0.35}{\textbf{{#1}}}}
% \newcommand{\RegionMarkerTok}[1]{\textcolor[rgb]{0.74,0.68,0.62}{{#1}}}
% \newcommand{\ErrorTok}[1]{\textcolor[rgb]{0.74,0.68,0.62}{\textbf{{#1}}}}
% \newcommand{\NormalTok}[1]{\textcolor[rgb]{0.74,0.68,0.62}{{#1}}}

%kete style colors
% \newenvironment{Shaded}{}{}
% \newcommand{\KeywordTok}[1]{\textbf{{#1}}}
% \newcommand{\DataTypeTok}[1]{\textcolor[rgb]{0.50,0.00,0.00}{{#1}}}
% \newcommand{\DecValTok}[1]{\textcolor[rgb]{0.00,0.00,1.00}{{#1}}}
% \newcommand{\BaseNTok}[1]{\textcolor[rgb]{0.00,0.00,1.00}{{#1}}}
% \newcommand{\FloatTok}[1]{\textcolor[rgb]{0.50,0.00,0.50}{{#1}}}
% \newcommand{\CharTok}[1]{\textcolor[rgb]{1.00,0.00,1.00}{{#1}}}
% \newcommand{\StringTok}[1]{\textcolor[rgb]{0.87,0.00,0.00}{{#1}}}
% \newcommand{\CommentTok}[1]{\textcolor[rgb]{0.50,0.50,0.50}{\textit{{#1}}}}
% \newcommand{\OtherTok}[1]{{#1}}
% \newcommand{\AlertTok}[1]{\textcolor[rgb]{0.00,1.00,0.00}{\textbf{{#1}}}}
% \newcommand{\FunctionTok}[1]{\textcolor[rgb]{0.00,0.00,0.50}{{#1}}}
% \newcommand{\RegionMarkerTok}[1]{{#1}}
% \newcommand{\ErrorTok}[1]{\textcolor[rgb]{1.00,0.00,0.00}{\textbf{{#1}}}}
% \newcommand{\NormalTok}[1]{{#1}}
%end pandoc code hacks

% jodliterate colors
\usepackage{color}
\definecolor{shadecolor}{RGB}{248,248,248}
% j control structures 
\definecolor{keywcolor}{rgb}{0.13,0.29,0.53}
% j explicit arguments x y m n u v
\definecolor{datacolor}{rgb}{0.13,0.29,0.53}
% j numbers - all types see j.xml
\definecolor{decvcolor}{rgb}{0.00,0.00,0.81}
\definecolor{basencolor}{rgb}{0.00,0.00,0.81}
\definecolor{floatcolor}{rgb}{0.00,0.00,0.81}
% j local assignments
\definecolor{charcolor}{rgb}{0.31,0.60,0.02}
\definecolor{stringcolor}{rgb}{0.31,0.60,0.02}
\definecolor{commentcolor}{rgb}{0.56,0.35,0.01}
% primitive adverbs and conjunctions
%\definecolor{othercolor}{rgb}{0.56,0.35,0.01}   
\definecolor{othercolor}{RGB}{0,0,255}
% global assignments
\definecolor{alertcolor}{rgb}{0.94,0.16,0.16}
% primitive J verbs and noun names
\definecolor{funccolor}{rgb}{0.00,0.00,0.00}    

\usepackage{framed}
\newenvironment{Shaded}{}{}
\newcommand{\KeywordTok}[1]{\textcolor{keywcolor}{\textbf{{#1}}}}
\newcommand{\DataTypeTok}[1]{\textcolor{datacolor}{{#1}}}
%\newcommand{\DecValTok}[1]{\textcolor{decvcolor}{{#1}}}
\newcommand{\DecValTok}[1]{{#1}} 
\newcommand{\BaseNTok}[1]{\textcolor{basencolor}{{#1}}}
\newcommand{\FloatTok}[1]{\textcolor{floatcolor}{{#1}}}
\newcommand{\CharTok}[1]{\textcolor{charcolor}{\textbf{{#1}}}}
\newcommand{\StringTok}[1]{\textcolor{stringcolor}{{#1}}}
\newcommand{\CommentTok}[1]{\textcolor{commentcolor}{\textit{{#1}}}}
\newcommand{\OtherTok}[1]{\textcolor{othercolor}{{#1}}} 
\newcommand{\AlertTok}[1]{\textcolor{alertcolor}{\textbf{{#1}}}}
%\newcommand{\FunctionTok}[1]{\textcolor{funccolor}{{#1}}}
\newcommand{\FunctionTok}[1]{{#1}}
\newcommand{\RegionMarkerTok}[1]{{#1}}
\newcommand{\ErrorTok}[1]{\textbf{{#1}}}
\newcommand{\NormalTok}[1]{{#1}}

% headers and footers
\usepackage{fancyhdr}
\pagestyle{fancy}

\fancyhead{}
\fancyfoot{}

%\fancyhead[LE,RO]{\slshape \rightmark}
%\fancyhead[LO,RE]{\slshape \leftmark}
\fancyfoot[C]{\thepage}
%\headrulewidth 0.4pt
%\footrulewidth 0 pt

%\addtolength{\headheight}{\baselineskip}

%\lfoot{\emph{Analyze the Data not the Drivel}}
%\rfoot{\emph{\today}}

% subfigure handles figures that contain subfigures
%\usepackage{color,graphicx,subfigure,sidecap}
\usepackage{graphicx,sidecap}
\usepackage{subfigure}
\graphicspath{{./inclusions/}}

% floatflt provides for text wrapping around small figures and tables
\usepackage{floatflt}

% tweak caption formats 
\usepackage{caption} 
\usepackage{sidecap}
%\usepackage{subcaption} % not compatible with subfigure

\usepackage{rotating} % flip tables sideways

% complex footnotes
%\usepackage{bigfoot}

% weird logos \XeLaTeX
\usepackage{metalogo}

% source code listings
\usepackage{listings}

% long tables
% \usepackage{longtable}

\newcommand{\HRule}{\rule{\linewidth}{0.5mm}}

% map LaTeX cross references into PDF cross references
\usepackage[
            %dvips,
            colorlinks,
            linkcolor=blue,
            citecolor=blue,
            urlcolor=blue,   % magenta, cyan default        
            pdfauthor={John D. Baker},
            pdftitle={Analyze the Data not the Drivel},
            pdfsubject={Blog},
            pdfcreator={MikTeX+LaTeXe with hyperref package},
            pdfkeywords={blog,wordpress},
            ]{hyperref}
           
% custom colors
\definecolor{CodeBackGround}{cmyk}{0.0,0.0,0,0.05}    % light gray
\definecolor{CodeComment}{rgb}{0,0.50,0.00}           % dark green {0,0.45,0.08}
\definecolor{TableStripes}{gray}{0.9}                 % odd/even background in tables

\lstdefinelanguage{bat}
{morekeywords={echo,title,pushd,popd,setlocal,endlocal,off,if,not,exist,set,goto,pause},
sensitive=True,
morecomment=[l]{rem}
}

\lstdefinelanguage{jdoc}
{
morekeywords={},
otherkeywords={assert.,break.,continue.,for.,do.,if.,else.,elseif.,return.,select.,end.
,while.,whilst.,throw.,catch.,catchd.,catcht.,try.,case.,fcase.},
sensitive=True,
morecomment=[l]{NB.},
morestring=[b]',
morestring=[d]',
}

% latex size ordering - can never remember it
% \tiny
% \scriptsize
% \footnotesize
% \small
% \normalsize
% \large
% \Large
% \LARGE
% \huge
% \Huge
 
% listings package settings  
\lstset{%
  language=jdoc,                                % j document settings
  basicstyle=\ttfamily\footnotesize,            
  keywordstyle=\bfseries\color{keywcolor}\footnotesize,
  identifierstyle=\color{black},
  commentstyle=\slshape\color{CodeComment},     % colored slanted comments
  stringstyle=\color{red}\ttfamily,
  showstringspaces=false,                       
  %backgroundcolor=\color{CodeBackGround},       
  frame=single,                                
  framesep=1pt,                                 
  framerule=0.8pt,                             
  rulecolor=\color{CodeBackGround},   
  showspaces=false,
  %columns=fullflexible,
  %numbers=left,
  %numberstyle=\footnotesize,
  %numbersep=9pt,
  tabsize=2,
  showtabs=false,
  captionpos=b
  breaklines=true,                              
  breakindent=5pt                              
}

\lstdefinelanguage{JavaScript}{
  keywords={typeof, new, true, false, catch, function, return, null, catch, switch, var, if, in, while, do, else, case, break},
  ndkeywords={class, export, boolean, throw, implements, import, this},
  ndkeywordstyle=\color{darkgray}\bfseries,
  sensitive=false,
  comment=[l]{//},
  morecomment=[s]{/*}{*/},
  morestring=[b]',
  morestring=[b]"
}

% C# settings
\lstdefinestyle{sharpc}{
language=[Sharp]C,
basicstyle=\ttfamily\scriptsize, 
keywordstyle=\bfseries\color{keywcolor}\scriptsize,
framerule=0pt
}

% for source code listing longer than two use smaller font
\lstdefinestyle{smallersource}{
basicstyle=\ttfamily\scriptsize, 
keywordstyle=\bfseries\color{keywcolor}\scriptsize,
framerule=0pt
}

\lstdefinestyle{resetdefaults}{
language=jdoc,
basicstyle=\ttfamily\footnotesize,  
keywordstyle=\bfseries\color{keywcolor}\footnotesize,                                                               
framerule=0.8pt 
}

% APL UTF8 code points listed for lstlisting processing
\makeatletter
\lst@InputCatcodes
\def\lst@DefEC{%
 \lst@CCECUse \lst@ProcessLetter
  ^^80^^81^^82^^83^^84^^85^^86^^87^^88^^89^^8a^^8b^^8c^^8d^^8e^^8f%
  ^^90^^91^^92^^93^^94^^95^^96^^97^^98^^99^^9a^^9b^^9c^^9d^^9e^^9f%
  ^^a0^^a1^^a2^^a3^^a4^^a5^^a6^^a7^^a8^^a9^^aa^^ab^^ac^^ad^^ae^^af%
  ^^b0^^b1^^b2^^b3^^b4^^b5^^b6^^b7^^b8^^b9^^ba^^bb^^bc^^bd^^be^^bf%
  ^^c0^^c1^^c2^^c3^^c4^^c5^^c6^^c7^^c8^^c9^^ca^^cb^^cc^^cd^^ce^^cf%
  ^^d0^^d1^^d2^^d3^^d4^^d5^^d6^^d7^^d8^^d9^^da^^db^^dc^^dd^^de^^df%
  ^^e0^^e1^^e2^^e3^^e4^^e5^^e6^^e7^^e8^^e9^^ea^^eb^^ec^^ed^^ee^^ef%
  ^^f0^^f1^^f2^^f3^^f4^^f5^^f6^^f7^^f8^^f9^^fa^^fb^^fc^^fd^^fe^^ff%
  ^^^^20ac^^^^0153^^^^0152%
  ^^^^20a7^^^^2190^^^^2191^^^^2192^^^^2193^^^^2206^^^^2207^^^^220a%
  ^^^^2218^^^^2228^^^^2229^^^^222a^^^^2235^^^^223c^^^^2260^^^^2261%
  ^^^^2262^^^^2264^^^^2265^^^^2282^^^^2283^^^^2296^^^^22a2^^^^22a3%
  ^^^^22a4^^^^22a5^^^^22c4^^^^2308^^^^230a^^^^2336^^^^2337^^^^2339%
  ^^^^233b^^^^233d^^^^233f^^^^2340^^^^2342^^^^2347^^^^2348^^^^2349%
  ^^^^234b^^^^234e^^^^2350^^^^2352^^^^2355^^^^2357^^^^2359^^^^235d%
  ^^^^235e^^^^235f^^^^2361^^^^2362^^^^2363^^^^2364^^^^2365^^^^2368%
  ^^^^236a^^^^236b^^^^236c^^^^2371^^^^2372^^^^2373^^^^2374^^^^2375%
  ^^^^2377^^^^2378^^^^237a^^^^2395^^^^25af^^^^25ca^^^^25cb%  
  ^^00}
\lst@RestoreCatcodes
\makeatother

% custom lengths used within minipages
\newcommand{\minindent}{17pt}


\makeindex

\begin{document}

\subsection*{\href{http://analyzethedatanotthedrivel.org/2020/07/31/tweaking-jupyter-export-html-with-templates/}{Tweaking Jupyter export HTML with Templates}}
\addcontentsline{toc}{subsection}{Tweaking Jupyter export HTML with Templates}


\noindent\emph{Posted: 31 Jul 2020 22:34:35}
\vspace{6pt}

In a
\href{https://analyzethedatanotthedrivel.org/2020/06/01/better-blogging-with-jupyter-notebooks-on-wordpress-com/}{previous
post}, I outlined some handy hacks for converting Jupyter notebooks to
WordPress.com oriented HTML. This addendum describes the use of
\href{https://www.datacamp.com/community/tutorials/jinja2-custom-export-templates-jupyter}{Jupyter
templates} and CSS edits to fine-tune exported HTML.

Jupyter exports notebooks in a variety of formats. I regularly export
notebooks as Markdown, HTML, and $\LaTeX$. When blogging I mainly
use HTML and Markdown. Standard HTML output labels code blocks with
numbered prompts and indents text. The following
\href{https://www.rust-lang.org/}{Rust}
\href{https://en.wikipedia.org/wiki/\%22Hello,_World!\%22_program}{hello
world} program illustrates this layout.

In~
%{[}2{]}
:

\begin{verbatim}
// This is the main function
fn main() {
    // Statements here are executed when the compiled binary is called

    // Print text to the console
    println!("Hello World!");
}

main()
\end{verbatim}

\begin{verbatim}
Hello World!
\end{verbatim}

Out
%{[}2{]}
:

\begin{verbatim}
()
\end{verbatim}

For a few simple examples numbered indented blocks are fine but for
longer posts, the redundant numbering and space-wasting indents grow
tiresome. I used standard HTML output for
\href{https://analyzethedatanotthedrivel.org/2020/05/25/using-jodliterate/}{Using
jodliterate} and around prompt \texttt{In
%{[}7{]}
:} I was irritated with
Jupyter's defaults.

Here's something you should know about me.

\textbf{\emph{I don't like being irritated!}}

In addition to indented numbered prompts the standard format suffers
from other
\href{https://www.dictionary.com/browse/boo-boo}{\emph{boo-boos}} when
run through \href{https://github.com/bennylp}{Benny's} Jupyter to
WordPress.com
\href{https://calvinandhobbes.fandom.com/wiki/Transmogrifier}{transmogrifier}.
Markdown source code like \texttt{source} code \texttt{McCodeyFace} is
set off in ugly blocks and any use of $\LaTeX$ generates hideous
inline images with nontransparent backgrounds like:

$\varphi (u) = \int_{\Omega} \left[
\dfrac{\|\nabla u\|^2}{2} +
\lambda\dfrac{u^2}{2} -
\dfrac{(u^+)^p}{p} \right] d\mu $

You can hack the background color of WordPress.com $\LaTeX$ by
bolting in a HEX color code suffix. This blog's background color is
\emph{currently} set to \texttt{cfcdcd} so
\texttt{\textbackslash{}sqrt
%{[}n{]}
\{1+x+x\^{}2+x\^{}3+\textbackslash{}dots+x\^{}n\}\&bg=cfcdcd}
yields:

$\sqrt[n]{1+x+x^2+x^3+\dots+x^n}$

As far as I know, there is no global WordPress.com $\LaTeX$
background color setting. You are forced to hard code the suffix into
every expression and if you ever change your theme's background color
you have to do it all over again.

\textbf{\emph{Remember that bit about how I don't like being
irritated?}}

Hard coding is euphemistically called a
\href{https://www.codegrip.tech/productivity/everything-you-need-to-know-about-code-smells/}{"code
smell."} In this case, it's a foul stench. This isn't Jupyter's fault.
Jupyter sensibly uses \href{https://www.mathjax.org/}{MathJaX} to render
mathematics with proper scaled outline fonts, but MathJax is an external
JavaScript library and WordPress.com does not allow cheap peons like
\emph{moi} to drag in alien JavaScript libraries.

\begin{quote}
Side note to WordPress.com. Make an exception to your no JavaScript rule
for MathJax. MathJax is a well established standard library and you're
being pigheaded by not allowing it.
\end{quote}

Some of these problems can be fixed with CSS tweaks, Jupyter templates,
and \texttt{nb2wp} edits.

\hypertarget{tweaking-jupyter-export-formats}{%
\paragraph{\texorpdfstring{Tweaking Jupyter export
formats\protect\hyperlink{Tweaking-Jupyter-export-formats}{}}{Tweaking Jupyter export formats}}\label{tweaking-jupyter-export-formats}}

You can adjust Jupyter export formats with templates. Templates modify
how \href{https://nbconvert.readthedocs.io/en/latest/}{nbconvert}
transforms notebook JSON \texttt{.ipynb} files into HTML, Markdown, and
other formats. Here's a good basic
\href{https://nbconvert.readthedocs.io/en/latest/customizing.html}{introduction
to Jupyter templates}.

Jupyter templates are versatile and powerful but with "\emph{great
formatting power comes extreme annoyance.}" I don't want to learn yet
another programming language to tweak my fricking blog! Fortunately,
it's not necessary to get deep in the
\href{https://jinja.palletsprojects.com/en/2.11.x/}{jinja} weeds to
remove a few block labels. The following template \texttt{tpl} file
meets my meager needs.

\hypertarget{a-brain-dead-jupyter-template-blankfull.tpl}{%
\paragraph{\texorpdfstring{A brain dead Jupyter template:
blankfull.tpl\href{-blankfull.tpl}{}}{A brain dead Jupyter template: blankfull.tpl}}\label{a-brain-dead-jupyter-template-blankfull.tpl}}

\begin{verbatim}


## change the appearance of execution count

&nbsp;&nbsp;&nbsp;&nbsp;






&nbsp;&nbsp;&nbsp;&nbsp;




\end{verbatim}

Save the preceding as UTF-8 text with extension \texttt{tpl} in
Jupyter's \texttt{nbconvert} share directory. The location of this
directory will vary for different systems. On this Windows 10 machine it
can be found here:

\begin{verbatim}
C:\Anaconda\share\jupyter\nbconvert\templates\html
\end{verbatim}

Refer to \texttt{nbconvert} documentation to determine where to save
\texttt{tpl} files on your system.

Stashing a custom template in the proper location removes block prompts
and indentation but it doesn't fix CSS issues. The
\href{https://github.com/bennylp/nb2wp/blob/master/style.css}{CSS style
file} distributed with \texttt{nb2wp} needs a few lines changed to
prevent \texttt{MarkdownText} from getting a white background. The
values after the line numbers on the left are the settings you need to
give \texttt{CodeyMcCodeFace} text \texttt{FRAGMENTS} transparent
backgrounds.

\begin{verbatim}
1077:   background-color: transparent;  background-color: #fff;
 1529:   background-color: transparent;  background-color: #f9f2f4;
10126:   background-color: transparent;  background-color: #fff;
\end{verbatim}

Finally, to control annoying $\LaTeX$ color backgrounds add another
argument \texttt{bg\_color\_hex=\textquotesingle{}\textquotesingle{}} to
the the Python function \texttt{nb2wp}.

\hypertarget{running-nb2wp-with-adjustments}{%
\paragraph{\texorpdfstring{Running nb2wp with
adjustments\protect\hyperlink{Running-nb2wp-with-adjustments}{}}{Running nb2wp with adjustments}}\label{running-nb2wp-with-adjustments}}

~~~~

\begin{verbatim}
import os
import sys

# append nb2* script directory to system path
sys.path.append(r'C:\temp\nb2wp')

import nb2wput as nbu
\end{verbatim}

~~~~

\begin{verbatim}
# set notebook files - the source for this blog post
nb_file_0 = r'C:\temp\nb2wp\MoreBetterBloggingWithJupyter0.ipynb'
nb_file_1 = r'C:\temp\nb2wp\MoreBetterBloggingWithJupyter1.ipynb'
\end{verbatim}

~~~~

\begin{verbatim}
# convert notebook with (nb2wp) using
# defaults for unspecified arguments
nbu.nb2wp(nb_file_0, out_dir=r'C:\temp\nb2wp\tmp',
    css_files=[r'C:\temp\nb2wp\style.css'], save_css=True, 
    remove_attrs=True, template='full', latex="wp",
    img_dir=r'C:\temp\nb2wp\tmp\img', footer=False)
\end{verbatim}

~~~~

\begin{verbatim}
Using template: full
Using CSS files ['C:\\temp\\nb2wp\\style.css']
Saving CSS to C:\temp\nb2wp\tmp\style.css
C:\temp\nb2wp\tmp\MoreBetterBloggingWithJupyter0.html: 23132 bytes written in 5.863s
\end{verbatim}

~~~~

\begin{verbatim}
# convert notebook using tweaked CSS and custom template
nbu.nb2wp(nb_file_1, out_dir=r'C:\temp\nb2wp\tmp',
    css_files=[r'C:\temp\nb2wp\my_nb2wp_style.css'], save_css=True, 
    remove_attrs=True, template='blankfull', latex="wp", bg_color_hex='cfcdcd',
    img_dir=r'C:\temp\nb2wp\tmp\img', footer=False)
\end{verbatim}

~~~~

\begin{verbatim}
Using template: blankfull
Using CSS files ['C:\\temp\\nb2wp\\my_nb2wp_style.css']
Saving CSS to C:\temp\nb2wp\tmp\style.css
C:\temp\nb2wp\tmp\MoreBetterBloggingWithJupyter1.html: 53009 bytes written in 10.377s
\end{verbatim}



%\end{document}
 
% standard floating figure
% \captionsetup[figure]{labelformat=empty}
% \begin{figure}[htbp]
% \centering
% \href{}{\includegraphics[width=0.50\textwidth]{}}
% \caption{}
% \label{fig:????x0}
% \end{figure}
 
% captions beside figure
% \captionsetup[figure]{labelformat=empty}
% \begin{SCfigure}
% \centering
% \href{}{\includegraphics[width=0.40\textwidth]{}}
% \caption{}
% \label{fig:????x0}
% \end{SCfigure}
 
% wrapped figure - outer size > inner size
% \captionsetup[floatingfigure]{labelformat=empty}
% \begin{floatingfigure}[l]{0.23\textwidth}
% \centering
% \href{}{\includegraphics[width=0.22\textwidth]{}}
% \caption{}
% \label{fig:????x0}
% \end{floatingfigure}
