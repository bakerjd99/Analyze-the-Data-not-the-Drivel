%% bm.pdf preamble - material merged from previous preamble and current pandoc preamable output
% NOTE: float placement required changes to the source files referenced by bm.tex
% May 28, 2020
%
% Use lualatex to compile - test with MiKTeX 2.9

% uncomment to list all files in log
%\listfiles

\documentclass[12pt]{report}


\usepackage{fontspec}

%\setmainfont[Scale=MatchLowercase]{Lucida Bright}
%\setmonofont{FreeMono}
%\setmonofont{Source Code Pro}
\setmonofont[Scale=MatchLowercase]{Ubuntu Mono}

\usepackage[headings]{fullpage}

% national use characters 
%\usepackage{inputenc}

% ams mathematical symbols
\usepackage{amsmath,amssymb}

% added to support pandoc highlighting
\usepackage{microtype}

\usepackage{makeidx}

% add index and bibliographies to table of contents
\usepackage[nottoc]{tocbibind}

% postscript courier and times in place of cm fonts
%\usepackage{courier}
%\usepackage{times}

% extended coloring
\usepackage{color}
\usepackage[table,dvipsnames]{xcolor}
\usepackage{colortbl}

% advanced date formating
\usepackage{datetime}

%support pandoc code highlighting
\usepackage{fancyvrb}

% \DefineShortVerb[commandchars=\\\{\}]{\|}
% \DefineVerbatimEnvironment{Highlighting}{Verbatim}{commandchars=\\\{\}}
% % Add ',fontsize=\small' for more characters per line

% tango style colors
% \usepackage{framed}
% \definecolor{shadecolor}{RGB}{255,255,255}
% \newenvironment{Shaded}{\begin{snugshade}}{\end{snugshade}}
% \newcommand{\KeywordTok}[1]{\textcolor[rgb]{0.13,0.29,0.53}{\textbf{{#1}}}}
% \newcommand{\DataTypeTok}[1]{\textcolor[rgb]{0.13,0.29,0.53}{{#1}}}
% \newcommand{\DecValTok}[1]{\textcolor[rgb]{0.00,0.00,0.81}{{#1}}}
% \newcommand{\BaseNTok}[1]{\textcolor[rgb]{0.00,0.00,0.81}{{#1}}}
% \newcommand{\FloatTok}[1]{\textcolor[rgb]{0.00,0.00,0.81}{{#1}}}
% \newcommand{\CharTok}[1]{\textcolor[rgb]{0.31,0.60,0.02}{{#1}}}
% \newcommand{\StringTok}[1]{\textcolor[rgb]{0.31,0.60,0.02}{{#1}}}
% \newcommand{\CommentTok}[1]{\textcolor[rgb]{0.56,0.35,0.01}{\textit{{#1}}}}
% \newcommand{\OtherTok}[1]{\textcolor[rgb]{0.56,0.35,0.01}{{#1}}}
% \newcommand{\AlertTok}[1]{\textcolor[rgb]{0.94,0.16,0.16}{{#1}}}
% \newcommand{\FunctionTok}[1]{\textcolor[rgb]{0.00,0.00,0.00}{{#1}}}
% \newcommand{\RegionMarkerTok}[1]{{#1}}
% \newcommand{\ErrorTok}[1]{\textbf{{#1}}}
% \newcommand{\NormalTok}[1]{{#1}}

% %espresso style colors
% \usepackage{framed}
% \definecolor{shadecolor}{RGB}{42,33,28}
% \newenvironment{Shaded}{\begin{snugshade}}{\end{snugshade}}
% \newcommand{\KeywordTok}[1]{\textcolor[rgb]{0.26,0.66,0.93}{\textbf{{#1}}}}
% \newcommand{\DataTypeTok}[1]{\textcolor[rgb]{0.74,0.68,0.62}{\underline{{#1}}}}
% \newcommand{\DecValTok}[1]{\textcolor[rgb]{0.27,0.67,0.26}{{#1}}}
% \newcommand{\BaseNTok}[1]{\textcolor[rgb]{0.27,0.67,0.26}{{#1}}}
% \newcommand{\FloatTok}[1]{\textcolor[rgb]{0.27,0.67,0.26}{{#1}}}
% \newcommand{\CharTok}[1]{\textcolor[rgb]{0.02,0.61,0.04}{{#1}}}
% \newcommand{\StringTok}[1]{\textcolor[rgb]{0.02,0.61,0.04}{{#1}}}
% \newcommand{\CommentTok}[1]{\textcolor[rgb]{0.00,0.40,1.00}{\textit{{#1}}}}
% \newcommand{\OtherTok}[1]{\textcolor[rgb]{0.74,0.68,0.62}{{#1}}}
% \newcommand{\AlertTok}[1]{\textcolor[rgb]{1.00,1.00,0.00}{{#1}}}
% \newcommand{\FunctionTok}[1]{\textcolor[rgb]{1.00,0.58,0.35}{\textbf{{#1}}}}
% \newcommand{\RegionMarkerTok}[1]{\textcolor[rgb]{0.74,0.68,0.62}{{#1}}}
% \newcommand{\ErrorTok}[1]{\textcolor[rgb]{0.74,0.68,0.62}{\textbf{{#1}}}}
% \newcommand{\NormalTok}[1]{\textcolor[rgb]{0.74,0.68,0.62}{{#1}}}

% %kete style colors
% \newenvironment{Shaded}{}{}
% \newcommand{\KeywordTok}[1]{\textbf{{#1}}}
% \newcommand{\DataTypeTok}[1]{\textcolor[rgb]{0.50,0.00,0.00}{{#1}}}
% \newcommand{\DecValTok}[1]{\textcolor[rgb]{0.00,0.00,1.00}{{#1}}}
% \newcommand{\BaseNTok}[1]{\textcolor[rgb]{0.00,0.00,1.00}{{#1}}}
% \newcommand{\FloatTok}[1]{\textcolor[rgb]{0.50,0.00,0.50}{{#1}}}
% \newcommand{\CharTok}[1]{\textcolor[rgb]{1.00,0.00,1.00}{{#1}}}
% \newcommand{\StringTok}[1]{\textcolor[rgb]{0.87,0.00,0.00}{{#1}}}
% \newcommand{\CommentTok}[1]{\textcolor[rgb]{0.50,0.50,0.50}{\textit{{#1}}}}
% \newcommand{\OtherTok}[1]{{#1}}
% \newcommand{\AlertTok}[1]{\textcolor[rgb]{0.00,1.00,0.00}{\textbf{{#1}}}}
% \newcommand{\FunctionTok}[1]{\textcolor[rgb]{0.00,0.00,0.50}{{#1}}}
% \newcommand{\RegionMarkerTok}[1]{{#1}}
% \newcommand{\ErrorTok}[1]{\textcolor[rgb]{1.00,0.00,0.00}{\textbf{{#1}}}}
% \newcommand{\NormalTok}[1]{{#1}}
% %end pandoc code hacks

% jodliterate colors
\usepackage{color}
\definecolor{shadecolor}{RGB}{248,248,248}
% j control structures 
\definecolor{keywcolor}{rgb}{0.13,0.29,0.53}
% j explicit arguments x y m n u v
\definecolor{datacolor}{rgb}{0.13,0.29,0.53}
% j numbers - all types see j.xml
\definecolor{decvcolor}{rgb}{0.00,0.00,0.81}
\definecolor{basencolor}{rgb}{0.00,0.00,0.81}
\definecolor{floatcolor}{rgb}{0.00,0.00,0.81}
% j local assignments
\definecolor{charcolor}{rgb}{0.31,0.60,0.02}
\definecolor{stringcolor}{rgb}{0.31,0.60,0.02}
\definecolor{commentcolor}{rgb}{0.56,0.35,0.01}
% primitive adverbs and conjunctions
%\definecolor{othercolor}{rgb}{0.56,0.35,0.01}   
\definecolor{othercolor}{RGB}{0,0,255}
% global assignments
\definecolor{alertcolor}{rgb}{0.94,0.16,0.16}
% primitive J verbs and noun names
\definecolor{funccolor}{rgb}{0.00,0.00,0.00}

% custom colors
\definecolor{CodeBackGround}{cmyk}{0.0,0.0,0,0.05}    % light gray
\definecolor{CodeComment}{rgb}{0,0.50,0.00}           % dark green {0,0.45,0.08}
\definecolor{TableStripes}{gray}{0.9}                 % odd/even background in tables

% Colors for the hyperref package
\definecolor{urlcolor}{rgb}{0,.145,.698}
\definecolor{linkcolor}{rgb}{.71,0.21,0.01}
\definecolor{citecolor}{rgb}{.12,.54,.11}

% % Exact colors from NB
\definecolor{incolor}{HTML}{303F9F}
\definecolor{outcolor}{HTML}{D84315}
\definecolor{cellborder}{HTML}{CFCFCF}
\definecolor{cellbackground}{HTML}{F7F7F7}

% % ANSI colors
\definecolor{ansi-black}{HTML}{3E424D}
\definecolor{ansi-black-intense}{HTML}{282C36}
\definecolor{ansi-red}{HTML}{E75C58}
\definecolor{ansi-red-intense}{HTML}{B22B31}
\definecolor{ansi-green}{HTML}{00A250}
\definecolor{ansi-green-intense}{HTML}{007427}
\definecolor{ansi-yellow}{HTML}{DDB62B}
\definecolor{ansi-yellow-intense}{HTML}{B27D12}
\definecolor{ansi-blue}{HTML}{208FFB}
\definecolor{ansi-blue-intense}{HTML}{0065CA}
\definecolor{ansi-magenta}{HTML}{D160C4}
\definecolor{ansi-magenta-intense}{HTML}{A03196}
\definecolor{ansi-cyan}{HTML}{60C6C8}
\definecolor{ansi-cyan-intense}{HTML}{258F8F}
\definecolor{ansi-white}{HTML}{C5C1B4}
\definecolor{ansi-white-intense}{HTML}{A1A6B2}
\definecolor{ansi-default-inverse-fg}{HTML}{FFFFFF}
\definecolor{ansi-default-inverse-bg}{HTML}{000000}
    

% \usepackage{framed}
% \newenvironment{Shaded}{}{}
% \newcommand{\KeywordTok}[1]{\textcolor{keywcolor}{\textbf{{#1}}}}
% \newcommand{\DataTypeTok}[1]{\textcolor{datacolor}{{#1}}}
% %\newcommand{\DecValTok}[1]{\textcolor{decvcolor}{{#1}}}
% \newcommand{\DecValTok}[1]{{#1}} 
% \newcommand{\BaseNTok}[1]{\textcolor{basencolor}{{#1}}}
% \newcommand{\FloatTok}[1]{\textcolor{floatcolor}{{#1}}}
% \newcommand{\CharTok}[1]{\textcolor{charcolor}{\textbf{{#1}}}}
% \newcommand{\StringTok}[1]{\textcolor{stringcolor}{{#1}}}
% \newcommand{\CommentTok}[1]{\textcolor{commentcolor}{\textit{{#1}}}}
% \newcommand{\OtherTok}[1]{\textcolor{othercolor}{{#1}}} 
% \newcommand{\AlertTok}[1]{\textcolor{alertcolor}{\textbf{{#1}}}}
% %\newcommand{\FunctionTok}[1]{\textcolor{funccolor}{{#1}}}
% \newcommand{\FunctionTok}[1]{{#1}}
% \newcommand{\RegionMarkerTok}[1]{{#1}}
% \newcommand{\ErrorTok}[1]{\textbf{{#1}}}
% \newcommand{\NormalTok}[1]{{#1}}

% The default LaTeX title has an obnoxious amount of whitespace. By default,
% titling removes some of it. It also provides customization options.
\usepackage{titling}

% headers and footers
\usepackage{fancyhdr}
%\pagestyle{fancy}
\pagestyle{plain}

\fancyhead{}
\fancyfoot{}

%\fancyhead[LE,RO]{\slshape \rightmark}
%\fancyhead[LO,RE]{\slshape \leftmark}
\fancyfoot[C]{\thepage}
%\headrulewidth 0.4pt
%\footrulewidth 0 pt

%\addtolength{\headheight}{\baselineskip}

%\lfoot{\emph{Analyze the Data not the Drivel}}
%\rfoot{\emph{\today}}

% subfigure handles figures that contain subfigures
%\usepackage{color,graphicx,subfigure,sidecap}
\usepackage{graphicx,sidecap}
\usepackage{subfigure}
\graphicspath{{./inclusions/}}

% floatflt provides for text wrapping around small figures and tables
\usepackage{floatflt}

% tweak caption formats 
\usepackage{caption} 
\usepackage{sidecap}
%\usepackage{subcaption} % not compatible with subfigure

\usepackage{rotating} % flip tables sideways

% complex footnotes
%\usepackage{bigfoot}

% weird logos \XeLaTeX
\usepackage{metalogo}

\newcommand{\HRule}{\rule{\linewidth}{0.5mm}}

\usepackage[breakable]{tcolorbox}

\usepackage{parskip} % Stop auto-indenting (to mimic markdown behaviour)
    
% Basic figure setup, for now with no caption control since it's done
% automatically by Pandoc (which extracts ![](path) syntax from Markdown).
\usepackage{graphicx}

%\DeclareCaptionFormat{nocaption}{}
%\captionsetup{format=nocaption,aboveskip=0pt,belowskip=0pt}

\usepackage[Export]{adjustbox} % Used to constrain images to a maximum size
\adjustboxset{max size={0.9\linewidth}{0.9\paperheight}}
\usepackage{float}

%\floatplacement{figure}{H} % forces figures to be placed at the correct location

\usepackage{xcolor} % Allow colors to be defined
\usepackage{enumerate} % Needed for markdown enumerations to work
\usepackage{geometry} % Used to adjust the document margins

%\usepackage{amsmath} % Equations
%\usepackage{amssymb} % Equations

\usepackage{textcomp} % defines textquotesingle

% Hack from http://tex.stackexchange.com/a/47451/13684:
\AtBeginDocument{%
	\def\PYZsq{\textquotesingle}% Upright quotes in Pygmentized code
}

\usepackage{upquote} % Upright quotes for verbatim code
\usepackage{eurosym} % defines \euro
\usepackage[mathletters]{ucs} % Extended unicode (utf-8) support

%\usepackage{fancyvrb} % verbatim replacement that allows latex

\usepackage{grffile} % extends the file name processing of package graphics 
					 % to support a larger range
					 
\makeatletter % fix for grffile with XeLaTeX
\def\Gread@@xetex#1{%
  \IfFileExists{"\Gin@base".bb}%
  {\Gread@eps{\Gin@base.bb}}%
  {\Gread@@xetex@aux#1}%
}
\makeatother

% The hyperref package gives us a pdf with properly built
% internal navigation ('pdf bookmarks' for the table of contents,
% internal cross-reference links, web links for URLs, etc.)
\usepackage{hyperref}
% The default LaTeX title has an obnoxious amount of whitespace. By default,
% titling removes some of it. It also provides customization options.
\usepackage{titling}
\usepackage{longtable} % longtable support required by pandoc >1.10
\usepackage{booktabs}  % table support for pandoc > 1.12.2
\usepackage[inline]{enumitem} % IRkernel/repr support (it uses the enumerate* environment)
\usepackage[normalem]{ulem} % ulem is needed to support strikethroughs (\sout)
							% normalem makes italics be italics, not underlines
\usepackage{mathrsfs}

% commands and environments needed by pandoc snippets
% extracted from the output of `pandoc -s`
\providecommand{\tightlist}{%
  \setlength{\itemsep}{0pt}\setlength{\parskip}{0pt}}
  
\DefineVerbatimEnvironment{Highlighting}{Verbatim}{commandchars=\\\{\}}
% Add ',fontsize=\small' for more characters per line
\newenvironment{Shaded}{}{}
\newcommand{\KeywordTok}[1]{\textcolor[rgb]{0.00,0.44,0.13}{\textbf{{#1}}}}
\newcommand{\DataTypeTok}[1]{\textcolor[rgb]{0.56,0.13,0.00}{{#1}}}
\newcommand{\DecValTok}[1]{\textcolor[rgb]{0.25,0.63,0.44}{{#1}}}
\newcommand{\BaseNTok}[1]{\textcolor[rgb]{0.25,0.63,0.44}{{#1}}}
\newcommand{\FloatTok}[1]{\textcolor[rgb]{0.25,0.63,0.44}{{#1}}}
\newcommand{\CharTok}[1]{\textcolor[rgb]{0.25,0.44,0.63}{{#1}}}
\newcommand{\StringTok}[1]{\textcolor[rgb]{0.25,0.44,0.63}{{#1}}}
\newcommand{\CommentTok}[1]{\textcolor[rgb]{0.38,0.63,0.69}{\textit{{#1}}}}
\newcommand{\OtherTok}[1]{\textcolor[rgb]{0.00,0.44,0.13}{{#1}}}
\newcommand{\AlertTok}[1]{\textcolor[rgb]{1.00,0.00,0.00}{\textbf{{#1}}}}
\newcommand{\FunctionTok}[1]{\textcolor[rgb]{0.02,0.16,0.49}{{#1}}}
\newcommand{\RegionMarkerTok}[1]{{#1}}
\newcommand{\ErrorTok}[1]{\textcolor[rgb]{1.00,0.00,0.00}{\textbf{{#1}}}}
\newcommand{\NormalTok}[1]{{#1}}

% Additional commands for more recent versions of Pandoc
\newcommand{\ConstantTok}[1]{\textcolor[rgb]{0.53,0.00,0.00}{{#1}}}
\newcommand{\SpecialCharTok}[1]{\textcolor[rgb]{0.25,0.44,0.63}{{#1}}}
\newcommand{\VerbatimStringTok}[1]{\textcolor[rgb]{0.25,0.44,0.63}{{#1}}}
\newcommand{\SpecialStringTok}[1]{\textcolor[rgb]{0.73,0.40,0.53}{{#1}}}
\newcommand{\ImportTok}[1]{{#1}}
\newcommand{\DocumentationTok}[1]{\textcolor[rgb]{0.73,0.13,0.13}{\textit{{#1}}}}
\newcommand{\AnnotationTok}[1]{\textcolor[rgb]{0.38,0.63,0.69}{\textbf{\textit{{#1}}}}}
\newcommand{\CommentVarTok}[1]{\textcolor[rgb]{0.38,0.63,0.69}{\textbf{\textit{{#1}}}}}
\newcommand{\VariableTok}[1]{\textcolor[rgb]{0.10,0.09,0.49}{{#1}}}
\newcommand{\ControlFlowTok}[1]{\textcolor[rgb]{0.00,0.44,0.13}{\textbf{{#1}}}}
\newcommand{\OperatorTok}[1]{\textcolor[rgb]{0.40,0.40,0.40}{{#1}}}
\newcommand{\BuiltInTok}[1]{{#1}}
\newcommand{\ExtensionTok}[1]{{#1}}
\newcommand{\PreprocessorTok}[1]{\textcolor[rgb]{0.74,0.48,0.00}{{#1}}}
\newcommand{\AttributeTok}[1]{\textcolor[rgb]{0.49,0.56,0.16}{{#1}}}
\newcommand{\InformationTok}[1]{\textcolor[rgb]{0.38,0.63,0.69}{\textbf{\textit{{#1}}}}}
\newcommand{\WarningTok}[1]{\textcolor[rgb]{0.38,0.63,0.69}{\textbf{\textit{{#1}}}}}

% Define a nice break command that doesn't care if a line doesn't already exist.
\def\br{\hspace*{\fill} \\* }
% Math Jax compatibility definitions
\def\gt{>}
\def\lt{<}
\let\Oldtex\TeX
\let\Oldlatex\LaTeX
\renewcommand{\TeX}{\textrm{\Oldtex}}
\renewcommand{\LaTeX}{\textrm{\Oldlatex}}
 
% Pygments definitions
\makeatletter
\def\PY@reset{\let\PY@it=\relax \let\PY@bf=\relax%
    \let\PY@ul=\relax \let\PY@tc=\relax%
    \let\PY@bc=\relax \let\PY@ff=\relax}
\def\PY@tok#1{\csname PY@tok@#1\endcsname}
\def\PY@toks#1+{\ifx\relax#1\empty\else%
    \PY@tok{#1}\expandafter\PY@toks\fi}
\def\PY@do#1{\PY@bc{\PY@tc{\PY@ul{%
    \PY@it{\PY@bf{\PY@ff{#1}}}}}}}
\def\PY#1#2{\PY@reset\PY@toks#1+\relax+\PY@do{#2}}

\expandafter\def\csname PY@tok@w\endcsname{\def\PY@tc##1{\textcolor[rgb]{0.73,0.73,0.73}{##1}}}
\expandafter\def\csname PY@tok@c\endcsname{\let\PY@it=\textit\def\PY@tc##1{\textcolor[rgb]{0.25,0.50,0.50}{##1}}}
\expandafter\def\csname PY@tok@cp\endcsname{\def\PY@tc##1{\textcolor[rgb]{0.74,0.48,0.00}{##1}}}
\expandafter\def\csname PY@tok@k\endcsname{\let\PY@bf=\textbf\def\PY@tc##1{\textcolor[rgb]{0.00,0.50,0.00}{##1}}}
\expandafter\def\csname PY@tok@kp\endcsname{\def\PY@tc##1{\textcolor[rgb]{0.00,0.50,0.00}{##1}}}
\expandafter\def\csname PY@tok@kt\endcsname{\def\PY@tc##1{\textcolor[rgb]{0.69,0.00,0.25}{##1}}}
\expandafter\def\csname PY@tok@o\endcsname{\def\PY@tc##1{\textcolor[rgb]{0.40,0.40,0.40}{##1}}}
\expandafter\def\csname PY@tok@ow\endcsname{\let\PY@bf=\textbf\def\PY@tc##1{\textcolor[rgb]{0.67,0.13,1.00}{##1}}}
\expandafter\def\csname PY@tok@nb\endcsname{\def\PY@tc##1{\textcolor[rgb]{0.00,0.50,0.00}{##1}}}
\expandafter\def\csname PY@tok@nf\endcsname{\def\PY@tc##1{\textcolor[rgb]{0.00,0.00,1.00}{##1}}}
\expandafter\def\csname PY@tok@nc\endcsname{\let\PY@bf=\textbf\def\PY@tc##1{\textcolor[rgb]{0.00,0.00,1.00}{##1}}}
\expandafter\def\csname PY@tok@nn\endcsname{\let\PY@bf=\textbf\def\PY@tc##1{\textcolor[rgb]{0.00,0.00,1.00}{##1}}}
\expandafter\def\csname PY@tok@ne\endcsname{\let\PY@bf=\textbf\def\PY@tc##1{\textcolor[rgb]{0.82,0.25,0.23}{##1}}}
\expandafter\def\csname PY@tok@nv\endcsname{\def\PY@tc##1{\textcolor[rgb]{0.10,0.09,0.49}{##1}}}
\expandafter\def\csname PY@tok@no\endcsname{\def\PY@tc##1{\textcolor[rgb]{0.53,0.00,0.00}{##1}}}
\expandafter\def\csname PY@tok@nl\endcsname{\def\PY@tc##1{\textcolor[rgb]{0.63,0.63,0.00}{##1}}}
\expandafter\def\csname PY@tok@ni\endcsname{\let\PY@bf=\textbf\def\PY@tc##1{\textcolor[rgb]{0.60,0.60,0.60}{##1}}}
\expandafter\def\csname PY@tok@na\endcsname{\def\PY@tc##1{\textcolor[rgb]{0.49,0.56,0.16}{##1}}}
\expandafter\def\csname PY@tok@nt\endcsname{\let\PY@bf=\textbf\def\PY@tc##1{\textcolor[rgb]{0.00,0.50,0.00}{##1}}}
\expandafter\def\csname PY@tok@nd\endcsname{\def\PY@tc##1{\textcolor[rgb]{0.67,0.13,1.00}{##1}}}
\expandafter\def\csname PY@tok@s\endcsname{\def\PY@tc##1{\textcolor[rgb]{0.73,0.13,0.13}{##1}}}
\expandafter\def\csname PY@tok@sd\endcsname{\let\PY@it=\textit\def\PY@tc##1{\textcolor[rgb]{0.73,0.13,0.13}{##1}}}
\expandafter\def\csname PY@tok@si\endcsname{\let\PY@bf=\textbf\def\PY@tc##1{\textcolor[rgb]{0.73,0.40,0.53}{##1}}}
\expandafter\def\csname PY@tok@se\endcsname{\let\PY@bf=\textbf\def\PY@tc##1{\textcolor[rgb]{0.73,0.40,0.13}{##1}}}
\expandafter\def\csname PY@tok@sr\endcsname{\def\PY@tc##1{\textcolor[rgb]{0.73,0.40,0.53}{##1}}}
\expandafter\def\csname PY@tok@ss\endcsname{\def\PY@tc##1{\textcolor[rgb]{0.10,0.09,0.49}{##1}}}
\expandafter\def\csname PY@tok@sx\endcsname{\def\PY@tc##1{\textcolor[rgb]{0.00,0.50,0.00}{##1}}}
\expandafter\def\csname PY@tok@m\endcsname{\def\PY@tc##1{\textcolor[rgb]{0.40,0.40,0.40}{##1}}}
\expandafter\def\csname PY@tok@gh\endcsname{\let\PY@bf=\textbf\def\PY@tc##1{\textcolor[rgb]{0.00,0.00,0.50}{##1}}}
\expandafter\def\csname PY@tok@gu\endcsname{\let\PY@bf=\textbf\def\PY@tc##1{\textcolor[rgb]{0.50,0.00,0.50}{##1}}}
\expandafter\def\csname PY@tok@gd\endcsname{\def\PY@tc##1{\textcolor[rgb]{0.63,0.00,0.00}{##1}}}
\expandafter\def\csname PY@tok@gi\endcsname{\def\PY@tc##1{\textcolor[rgb]{0.00,0.63,0.00}{##1}}}
\expandafter\def\csname PY@tok@gr\endcsname{\def\PY@tc##1{\textcolor[rgb]{1.00,0.00,0.00}{##1}}}
\expandafter\def\csname PY@tok@ge\endcsname{\let\PY@it=\textit}
\expandafter\def\csname PY@tok@gs\endcsname{\let\PY@bf=\textbf}
\expandafter\def\csname PY@tok@gp\endcsname{\let\PY@bf=\textbf\def\PY@tc##1{\textcolor[rgb]{0.00,0.00,0.50}{##1}}}
\expandafter\def\csname PY@tok@go\endcsname{\def\PY@tc##1{\textcolor[rgb]{0.53,0.53,0.53}{##1}}}
\expandafter\def\csname PY@tok@gt\endcsname{\def\PY@tc##1{\textcolor[rgb]{0.00,0.27,0.87}{##1}}}
\expandafter\def\csname PY@tok@err\endcsname{\def\PY@bc##1{\setlength{\fboxsep}{0pt}\fcolorbox[rgb]{1.00,0.00,0.00}{1,1,1}{\strut ##1}}}
\expandafter\def\csname PY@tok@kc\endcsname{\let\PY@bf=\textbf\def\PY@tc##1{\textcolor[rgb]{0.00,0.50,0.00}{##1}}}
\expandafter\def\csname PY@tok@kd\endcsname{\let\PY@bf=\textbf\def\PY@tc##1{\textcolor[rgb]{0.00,0.50,0.00}{##1}}}
\expandafter\def\csname PY@tok@kn\endcsname{\let\PY@bf=\textbf\def\PY@tc##1{\textcolor[rgb]{0.00,0.50,0.00}{##1}}}
\expandafter\def\csname PY@tok@kr\endcsname{\let\PY@bf=\textbf\def\PY@tc##1{\textcolor[rgb]{0.00,0.50,0.00}{##1}}}
\expandafter\def\csname PY@tok@bp\endcsname{\def\PY@tc##1{\textcolor[rgb]{0.00,0.50,0.00}{##1}}}
\expandafter\def\csname PY@tok@fm\endcsname{\def\PY@tc##1{\textcolor[rgb]{0.00,0.00,1.00}{##1}}}
\expandafter\def\csname PY@tok@vc\endcsname{\def\PY@tc##1{\textcolor[rgb]{0.10,0.09,0.49}{##1}}}
\expandafter\def\csname PY@tok@vg\endcsname{\def\PY@tc##1{\textcolor[rgb]{0.10,0.09,0.49}{##1}}}
\expandafter\def\csname PY@tok@vi\endcsname{\def\PY@tc##1{\textcolor[rgb]{0.10,0.09,0.49}{##1}}}
\expandafter\def\csname PY@tok@vm\endcsname{\def\PY@tc##1{\textcolor[rgb]{0.10,0.09,0.49}{##1}}}
\expandafter\def\csname PY@tok@sa\endcsname{\def\PY@tc##1{\textcolor[rgb]{0.73,0.13,0.13}{##1}}}
\expandafter\def\csname PY@tok@sb\endcsname{\def\PY@tc##1{\textcolor[rgb]{0.73,0.13,0.13}{##1}}}
\expandafter\def\csname PY@tok@sc\endcsname{\def\PY@tc##1{\textcolor[rgb]{0.73,0.13,0.13}{##1}}}
\expandafter\def\csname PY@tok@dl\endcsname{\def\PY@tc##1{\textcolor[rgb]{0.73,0.13,0.13}{##1}}}
\expandafter\def\csname PY@tok@s2\endcsname{\def\PY@tc##1{\textcolor[rgb]{0.73,0.13,0.13}{##1}}}
\expandafter\def\csname PY@tok@sh\endcsname{\def\PY@tc##1{\textcolor[rgb]{0.73,0.13,0.13}{##1}}}
\expandafter\def\csname PY@tok@s1\endcsname{\def\PY@tc##1{\textcolor[rgb]{0.73,0.13,0.13}{##1}}}
\expandafter\def\csname PY@tok@mb\endcsname{\def\PY@tc##1{\textcolor[rgb]{0.40,0.40,0.40}{##1}}}
\expandafter\def\csname PY@tok@mf\endcsname{\def\PY@tc##1{\textcolor[rgb]{0.40,0.40,0.40}{##1}}}
\expandafter\def\csname PY@tok@mh\endcsname{\def\PY@tc##1{\textcolor[rgb]{0.40,0.40,0.40}{##1}}}
\expandafter\def\csname PY@tok@mi\endcsname{\def\PY@tc##1{\textcolor[rgb]{0.40,0.40,0.40}{##1}}}
\expandafter\def\csname PY@tok@il\endcsname{\def\PY@tc##1{\textcolor[rgb]{0.40,0.40,0.40}{##1}}}
\expandafter\def\csname PY@tok@mo\endcsname{\def\PY@tc##1{\textcolor[rgb]{0.40,0.40,0.40}{##1}}}
\expandafter\def\csname PY@tok@ch\endcsname{\let\PY@it=\textit\def\PY@tc##1{\textcolor[rgb]{0.25,0.50,0.50}{##1}}}
\expandafter\def\csname PY@tok@cm\endcsname{\let\PY@it=\textit\def\PY@tc##1{\textcolor[rgb]{0.25,0.50,0.50}{##1}}}
\expandafter\def\csname PY@tok@cpf\endcsname{\let\PY@it=\textit\def\PY@tc##1{\textcolor[rgb]{0.25,0.50,0.50}{##1}}}
\expandafter\def\csname PY@tok@c1\endcsname{\let\PY@it=\textit\def\PY@tc##1{\textcolor[rgb]{0.25,0.50,0.50}{##1}}}
\expandafter\def\csname PY@tok@cs\endcsname{\let\PY@it=\textit\def\PY@tc##1{\textcolor[rgb]{0.25,0.50,0.50}{##1}}}

\def\PYZbs{\char`\\}
\def\PYZus{\char`\_}
\def\PYZob{\char`\{}
\def\PYZcb{\char`\}}
\def\PYZca{\char`\^}
\def\PYZam{\char`\&}
\def\PYZlt{\char`\<}
\def\PYZgt{\char`\>}
\def\PYZsh{\char`\#}
\def\PYZpc{\char`\%}
\def\PYZdl{\char`\$}
\def\PYZhy{\char`\-}
\def\PYZsq{\char`\'}
\def\PYZdq{\char`\"}
\def\PYZti{\char`\~}
% for compatibility with earlier versions
\def\PYZat{@}
\def\PYZlb{[}
\def\PYZrb{]}
\makeatother

% For linebreaks inside Verbatim environment from package fancyvrb. 
\makeatletter
	\newbox\Wrappedcontinuationbox 
	\newbox\Wrappedvisiblespacebox 
	\newcommand*\Wrappedvisiblespace {\textcolor{red}{\textvisiblespace}} 
	\newcommand*\Wrappedcontinuationsymbol {\textcolor{red}{\llap{\tiny$\m@th\hookrightarrow$}}} 
	\newcommand*\Wrappedcontinuationindent {3ex } 
	\newcommand*\Wrappedafterbreak {\kern\Wrappedcontinuationindent\copy\Wrappedcontinuationbox} 
	% Take advantage of the already applied Pygments mark-up to insert 
	% potential linebreaks for TeX processing. 
	%        {, <, #, %, $, ' and ": go to next line. 
	%        _, }, ^, &, >, - and ~: stay at end of broken line. 
	% Use of \textquotesingle for straight quote. 
	\newcommand*\Wrappedbreaksatspecials {% 
		\def\PYGZus{\discretionary{\char`\_}{\Wrappedafterbreak}{\char`\_}}% 
		\def\PYGZob{\discretionary{}{\Wrappedafterbreak\char`\{}{\char`\{}}% 
		\def\PYGZcb{\discretionary{\char`\}}{\Wrappedafterbreak}{\char`\}}}% 
		\def\PYGZca{\discretionary{\char`\^}{\Wrappedafterbreak}{\char`\^}}% 
		\def\PYGZam{\discretionary{\char`\&}{\Wrappedafterbreak}{\char`\&}}% 
		\def\PYGZlt{\discretionary{}{\Wrappedafterbreak\char`\<}{\char`\<}}% 
		\def\PYGZgt{\discretionary{\char`\>}{\Wrappedafterbreak}{\char`\>}}% 
		\def\PYGZsh{\discretionary{}{\Wrappedafterbreak\char`\#}{\char`\#}}% 
		\def\PYGZpc{\discretionary{}{\Wrappedafterbreak\char`\%}{\char`\%}}% 
		\def\PYGZdl{\discretionary{}{\Wrappedafterbreak\char`\$}{\char`\$}}% 
		\def\PYGZhy{\discretionary{\char`\-}{\Wrappedafterbreak}{\char`\-}}% 
		\def\PYGZsq{\discretionary{}{\Wrappedafterbreak\textquotesingle}{\textquotesingle}}% 
		\def\PYGZdq{\discretionary{}{\Wrappedafterbreak\char`\"}{\char`\"}}% 
		\def\PYGZti{\discretionary{\char`\~}{\Wrappedafterbreak}{\char`\~}}% 
	} 
	% Some characters . , ; ? ! / are not pygmentized. 
	% This macro makes them "active" and they will insert potential linebreaks 
	\newcommand*\Wrappedbreaksatpunct {% 
		\lccode`\~`\.\lowercase{\def~}{\discretionary{\hbox{\char`\.}}{\Wrappedafterbreak}{\hbox{\char`\.}}}% 
		\lccode`\~`\,\lowercase{\def~}{\discretionary{\hbox{\char`\,}}{\Wrappedafterbreak}{\hbox{\char`\,}}}% 
		\lccode`\~`\;\lowercase{\def~}{\discretionary{\hbox{\char`\;}}{\Wrappedafterbreak}{\hbox{\char`\;}}}% 
		\lccode`\~`\:\lowercase{\def~}{\discretionary{\hbox{\char`\:}}{\Wrappedafterbreak}{\hbox{\char`\:}}}% 
		\lccode`\~`\?\lowercase{\def~}{\discretionary{\hbox{\char`\?}}{\Wrappedafterbreak}{\hbox{\char`\?}}}% 
		\lccode`\~`\!\lowercase{\def~}{\discretionary{\hbox{\char`\!}}{\Wrappedafterbreak}{\hbox{\char`\!}}}% 
		\lccode`\~`\/\lowercase{\def~}{\discretionary{\hbox{\char`\/}}{\Wrappedafterbreak}{\hbox{\char`\/}}}% 
		\catcode`\.\active
		\catcode`\,\active 
		\catcode`\;\active
		\catcode`\:\active
		\catcode`\?\active
		\catcode`\!\active
		\catcode`\/\active 
		\lccode`\~`\~ 	
	}
\makeatother

\let\OriginalVerbatim=\Verbatim
\makeatletter
\renewcommand{\Verbatim}[1][1]{%
	%\parskip\z@skip
	\sbox\Wrappedcontinuationbox {\Wrappedcontinuationsymbol}%
	\sbox\Wrappedvisiblespacebox {\FV@SetupFont\Wrappedvisiblespace}%
	\def\FancyVerbFormatLine ##1{\hsize\linewidth
		\vtop{\raggedright\hyphenpenalty\z@\exhyphenpenalty\z@
			\doublehyphendemerits\z@\finalhyphendemerits\z@
			\strut ##1\strut}%
	}%
	% If the linebreak is at a space, the latter will be displayed as visible
	% space at end of first line, and a continuation symbol starts next line.
	% Stretch/shrink are however usually zero for typewriter font.
	\def\FV@Space {%
		\nobreak\hskip\z@ plus\fontdimen3\font minus\fontdimen4\font
		\discretionary{\copy\Wrappedvisiblespacebox}{\Wrappedafterbreak}
		{\kern\fontdimen2\font}%
	}%
	
	% Allow breaks at special characters using \PYG... macros.
	\Wrappedbreaksatspecials
	% Breaks at punctuation characters . , ; ? ! and / need catcode=\active 	
	\OriginalVerbatim[#1,codes*=\Wrappedbreaksatpunct]%
}
\makeatother


% prompt
\makeatletter
\newcommand{\boxspacing}{\kern\kvtcb@left@rule\kern\kvtcb@boxsep}
\makeatother
\newcommand{\prompt}[4]{
	\ttfamily\llap{{\color{#2}[#3]:\hspace{3pt}#4}}\vspace{-\baselineskip}
}
    

% Prevent overflowing lines due to hard-to-break entities
\sloppy 

% Setup hyperref package
\hypersetup{
  breaklinks=true,  % so long urls are correctly broken across lines
  colorlinks=true,
  urlcolor=urlcolor,
  linkcolor=linkcolor,
  citecolor=citecolor,
  pdfauthor={John D. Baker},
  pdftitle={Analyze the Data not the Drivel},
  pdfsubject={Blog},
  pdfcreator={MikTeX+LaTeXe},
  pdfkeywords={blog,wordpress},
  }
  
% Slightly bigger margins than the latex defaults
% \geometry{verbose,tmargin=1in,bmargin=1in,lmargin=1in,rmargin=1in}  

%\usepackage{wrapfig}

% source code listings
\usepackage{listings}

\lstdefinelanguage{bat}
{morekeywords={echo,title,pushd,popd,setlocal,endlocal,off,if,not,exist,set,goto,pause},
sensitive=True,
morecomment=[l]{rem}
}

\lstdefinelanguage{jdoc}
{
morekeywords={},
otherkeywords={assert.,break.,continue.,for.,do.,if.,else.,elseif.,return.,select.,end.
,while.,whilst.,throw.,catch.,catchd.,catcht.,try.,case.,fcase.},
sensitive=True,
morecomment=[l]{NB.},
morestring=[b]',
morestring=[d]',
}

% latex size ordering - can never remember it
% \tiny
% \scriptsize
% \footnotesize
% \small
% \normalsize
% \large
% \Large
% \LARGE
% \huge
% \Huge
 
% listings package settings  
\lstset{%
  language=jdoc,                                % j document settings
  basicstyle=\ttfamily\footnotesize,            
  keywordstyle=\bfseries\color{keywcolor}\footnotesize,
  identifierstyle=\color{black},
  commentstyle=\slshape\color{CodeComment},     % colored slanted comments
  stringstyle=\color{red}\ttfamily,
  showstringspaces=false,                       
  %backgroundcolor=\color{CodeBackGround},       
  frame=single,                                
  framesep=1pt,                                 
  framerule=0.8pt,                             
  rulecolor=\color{CodeBackGround},   
  showspaces=false,
  %columns=fullflexible,
  %numbers=left,
  %numberstyle=\footnotesize,
  %numbersep=9pt,
  tabsize=2,
  showtabs=false,
  captionpos=b
  breaklines=true,                              
  breakindent=5pt                              
}

\lstdefinelanguage{JavaScript}{
  keywords={typeof, new, true, false, catch, function, return, null, catch, switch, var, if, in, while, do, else, case, break},
  ndkeywords={class, export, boolean, throw, implements, import, this},
  ndkeywordstyle=\color{darkgray}\bfseries,
  sensitive=false,
  comment=[l]{//},
  morecomment=[s]{/*}{*/},
  morestring=[b]',
  morestring=[b]"
}

% C# settings
\lstdefinestyle{sharpc}{
language=[Sharp]C,
basicstyle=\ttfamily\scriptsize, 
keywordstyle=\bfseries\color{keywcolor}\scriptsize,
framerule=0pt
}

% for source code listing longer than two use smaller font
\lstdefinestyle{smallersource}{
basicstyle=\ttfamily\scriptsize, 
keywordstyle=\bfseries\color{keywcolor}\scriptsize,
framerule=0pt
}

\lstdefinestyle{resetdefaults}{
language=jdoc,
basicstyle=\ttfamily\footnotesize,  
keywordstyle=\bfseries\color{keywcolor}\footnotesize,                                                               
framerule=0.8pt 
}

% APL UTF8 code points listed for lstlisting processing
\makeatletter
\lst@InputCatcodes
\def\lst@DefEC{%
 \lst@CCECUse \lst@ProcessLetter
  ^^80^^81^^82^^83^^84^^85^^86^^87^^88^^89^^8a^^8b^^8c^^8d^^8e^^8f%
  ^^90^^91^^92^^93^^94^^95^^96^^97^^98^^99^^9a^^9b^^9c^^9d^^9e^^9f%
  ^^a0^^a1^^a2^^a3^^a4^^a5^^a6^^a7^^a8^^a9^^aa^^ab^^ac^^ad^^ae^^af%
  ^^b0^^b1^^b2^^b3^^b4^^b5^^b6^^b7^^b8^^b9^^ba^^bb^^bc^^bd^^be^^bf%
  ^^c0^^c1^^c2^^c3^^c4^^c5^^c6^^c7^^c8^^c9^^ca^^cb^^cc^^cd^^ce^^cf%
  ^^d0^^d1^^d2^^d3^^d4^^d5^^d6^^d7^^d8^^d9^^da^^db^^dc^^dd^^de^^df%
  ^^e0^^e1^^e2^^e3^^e4^^e5^^e6^^e7^^e8^^e9^^ea^^eb^^ec^^ed^^ee^^ef%
  ^^f0^^f1^^f2^^f3^^f4^^f5^^f6^^f7^^f8^^f9^^fa^^fb^^fc^^fd^^fe^^ff%
  ^^^^20ac^^^^0153^^^^0152%
  ^^^^20a7^^^^2190^^^^2191^^^^2192^^^^2193^^^^2206^^^^2207^^^^220a%
  ^^^^2218^^^^2228^^^^2229^^^^222a^^^^2235^^^^223c^^^^2260^^^^2261%
  ^^^^2262^^^^2264^^^^2265^^^^2282^^^^2283^^^^2296^^^^22a2^^^^22a3%
  ^^^^22a4^^^^22a5^^^^22c4^^^^2308^^^^230a^^^^2336^^^^2337^^^^2339%
  ^^^^233b^^^^233d^^^^233f^^^^2340^^^^2342^^^^2347^^^^2348^^^^2349%
  ^^^^234b^^^^234e^^^^2350^^^^2352^^^^2355^^^^2357^^^^2359^^^^235d%
  ^^^^235e^^^^235f^^^^2361^^^^2362^^^^2363^^^^2364^^^^2365^^^^2368%
  ^^^^236a^^^^236b^^^^236c^^^^2371^^^^2372^^^^2373^^^^2374^^^^2375%
  ^^^^2377^^^^2378^^^^237a^^^^2395^^^^25af^^^^25ca^^^^25cb%  
  ^^00}
\lst@RestoreCatcodes
\makeatother

% custom lengths used within minipages
\newcommand{\minindent}{17pt}

\makeindex

\begin{document}

\subsection*{\href{http://analyzethedatanotthedrivel.org/2020/07/31/tweaking-jupyter-export-html-with-templates/}{Tweaking Jupyter export HTML with Templates}}
\addcontentsline{toc}{subsection}{Tweaking Jupyter export HTML with Templates}


\noindent\emph{Posted: 31 Jul 2020 22:34:35}
\vspace{6pt}
    
    
    In a
\href{https://analyzethedatanotthedrivel.org/2020/06/01/better-blogging-with-jupyter-notebooks-on-wordpress-com/}{previous
post}, I outlined some handy hacks for converting Jupyter notebooks to
WordPress.com oriented HTML. This addendum describes the use of
\href{https://www.datacamp.com/community/tutorials/jinja2-custom-export-templates-jupyter}{Jupyter
templates} and CSS edits to fine-tune exported HTML.

Jupyter exports notebooks in a variety of formats. I regularly export
notebooks as Markdown, HTML, and \(\LaTeX\). When blogging I mainly use
HTML and Markdown. Standard HTML output labels code blocks with numbered
prompts and indents text. The following
\href{https://www.rust-lang.org/}{Rust}
\href{https://en.wikipedia.org/wiki/\%22Hello,_World!\%22_program}{hello
world} program illustrates this layout.

    \begin{tcolorbox}[breakable, size=fbox, boxrule=1pt, pad at break*=1mm,colback=cellbackground, colframe=cellborder]
\prompt{In}{incolor}{2}{\boxspacing}
\begin{Verbatim}[commandchars=\\\{\}]
\PY{c+c1}{// This is the main function}
\PY{k}{fn} \PY{n+nf}{main}\PY{p}{(}\PY{p}{)}\PY{+w}{ }\PY{p}{\PYZob{}}
\PY{+w}{    }\PY{c+c1}{// Statements here are executed when the compiled binary is called}

\PY{+w}{    }\PY{c+c1}{// Print text to the console}
\PY{+w}{    }\PY{n}{println}\PY{o}{!}\PY{p}{(}\PY{l+s}{\PYZdq{}}\PY{l+s}{Hello World!}\PY{l+s}{\PYZdq{}}\PY{p}{)}\PY{p}{;}
\PY{p}{\PYZcb{}}

\PY{n}{main}\PY{p}{(}\PY{p}{)}
\end{Verbatim}
\end{tcolorbox}

    \begin{Verbatim}[commandchars=\\\{\}]
Hello World!
    \end{Verbatim}

            \begin{tcolorbox}[breakable, size=fbox, boxrule=.5pt, pad at break*=1mm, opacityfill=0]
\prompt{Out}{outcolor}{2}{\boxspacing}
\begin{Verbatim}[commandchars=\\\{\}]
()
\end{Verbatim}
\end{tcolorbox}
        
    For a few simple examples numbered indented blocks are fine but for
longer posts, the redundant numbering and space-wasting indents grow
tiresome. I used standard HTML output for
\href{https://analyzethedatanotthedrivel.org/2020/05/25/using-jodliterate/}{Using
jodliterate} and around prompt \texttt{In{[}7{]}:} I was irritated with
Jupyter's defaults.

Here's something you should know about me.

\textbf{\emph{I don't like being irritated!}}

In addition to indented numbered prompts the standard format suffers
from other
\href{https://www.dictionary.com/browse/boo-boo}{\emph{boo-boos}} when
run through \href{https://github.com/bennylp}{Benny's} Jupyter to
WordPress.com
\href{https://calvinandhobbes.fandom.com/wiki/Transmogrifier}{transmogrifier}.
Markdown source code like \texttt{source} code \texttt{McCodeyFace} is
set off in ugly blocks and any use of \(\LaTeX\) generates hideous
inline images with nontransparent backgrounds like:

%\begin{tcolorbox}[breakable, size=fbox, boxrule=1pt, pad at break*=1mm,colback=cellbackground, colframe=cellborder]
\begin{tcolorbox}[breakable, width=9cm, center, boxrule=1pt, pad at break*=1mm,colback=cellbackground, colframe=cellborder]
\[\varphi (u) = \int\_\{\Omega\} \left[
\dfrac{\|\nabla u\|^2}{2} +
\lambda\dfrac{u^2}{2} -
\dfrac{(u^+)^p}{p} \right] d\mu \]
\end{tcolorbox}

You can hack the background color of WordPress.com \(\LaTeX\) by bolting
in a HEX color code suffix. This blog's background color is
\emph{currently} set to \texttt{cfcdcd} so
\texttt{\textbackslash{}sqrt{[}n{]}\{1+x+x\^{}2+x\^{}3+\textbackslash{}dots+x\^{}n\}\&bg=cfcdcd}
yields:

\[ \sqrt[n]{1+x+x^2+x^3+\dots+x^n} \]

As far as I know, there is no global WordPress.com \(\LaTeX\) background
color setting. You are forced to hard code the suffix into every
expression and if you ever change your theme's background color you have
to do it all over again.

\textbf{\emph{Remember that bit about how I don't like being
irritated?}}

Hard coding is euphemistically called a
\href{https://www.codegrip.tech/productivity/everything-you-need-to-know-about-code-smells/}{``code
smell.''} In this case, it's a foul stench. This isn't Jupyter's fault.
Jupyter sensibly uses \href{https://www.mathjax.org/}{MathJaX} to render
mathematics with proper scaled outline fonts, but MathJax is an external
JavaScript library and WordPress.com does not allow cheap peons like
\emph{moi} to drag in alien JavaScript libraries.

\begin{quote}
Side note to WordPress.com. Make an exception to your no JavaScript rule
for MathJax. MathJax is a well established standard library and you're
being pigheaded by not allowing it.
\end{quote}

Some of these problems can be fixed with CSS tweaks, Jupyter templates,
and \texttt{nb2wp} edits.

    
    \hypertarget{tweaking-jupyter-export-formats}{%
\paragraph{Tweaking Jupyter export
formats}\label{tweaking-jupyter-export-formats}}

You can adjust Jupyter export formats with templates. Templates modify
how \href{https://nbconvert.readthedocs.io/en/latest/}{nbconvert}
transforms notebook JSON \texttt{.ipynb} files into HTML, Markdown, and
other formats. Here's a good basic
\href{https://nbconvert.readthedocs.io/en/latest/customizing.html}{introduction
to Jupyter templates}.

Jupyter templates are versatile and powerful but with ``\emph{great
formatting power comes extreme annoyance.}'' I don't want to learn yet
another programming language to tweak my fricking blog! Fortunately,
it's not necessary to get deep in the
\href{https://jinja.palletsprojects.com/en/2.11.x/}{jinja} weeds to
remove a few block labels. The following template \texttt{tpl} file
meets my meager needs.

\hypertarget{a-brain-dead-jupyter-template-blankfull.tpl}\NormalTok{ extends }\StringTok{'full.tpl'}\OperatorTok\NormalTok{ block in_prompt }\OperatorTok\NormalTok{ endblock in_prompt }\OperatorTok\NormalTok{ block prompt }\OperatorTok\NormalTok{ endblock prompt }\OperatorTok\NormalTok{ block output_area_prompt }\OperatorTok\NormalTok{ endblock output_area_prompt }\OperatorTok\NormalTok{ block output_area }\OperatorTok\NormalTok{ endblock output_area }\OperatorTok{%}\NormalTok{\}}
\end{Highlighting}
\end{Shaded}

Save the preceding as UTF-8 text with extension \texttt{tpl} in
Jupyter's \texttt{nbconvert} share directory. The location of this
directory will vary for different systems. On this Windows 10 machine it
can be found here:

\begin{verbatim}
C:\Anaconda\share\jupyter\nbconvert\templates\html
\end{verbatim}

Refer to \texttt{nbconvert} documentation to determine where to save
\texttt{tpl} files on your system.

Stashing a custom template in the proper location removes block prompts
and indentation but it doesn't fix CSS issues. The
\href{https://github.com/bennylp/nb2wp/blob/master/style.css}{CSS style
file} distributed with \texttt{nb2wp} needs a few lines changed to
prevent \texttt{MarkdownText} from getting a white background. The
values after the line numbers on the left are the settings you need to
give \texttt{CodeyMcCodeFace} text \texttt{FRAGMENTS} transparent
backgrounds.

\begin{Shaded}
\begin{Highlighting}[]
\NormalTok{ 1077}\InformationTok{:}\NormalTok{   background-color}\InformationTok{:}\NormalTok{ transparent;  background-color}\InformationTok{:} \PreprocessorTok{#fff}\NormalTok{;}
\NormalTok{ 1529}\InformationTok{:}\NormalTok{   background-color}\InformationTok{:}\NormalTok{ transparent;  background-color}\InformationTok{:} \PreprocessorTok{#f9f2f4}\NormalTok{;}
\NormalTok{10126}\InformationTok{:}\NormalTok{   background-color}\InformationTok{:}\NormalTok{ transparent;  background-color}\InformationTok{:} \PreprocessorTok{#fff}\NormalTok{;}
\end{Highlighting}
\end{Shaded}

Finally, to control annoying \(\LaTeX\) color backgrounds add another
argument \texttt{bg\_color\_hex=\textquotesingle{}\textquotesingle{}} to
the Python function \texttt{nb2wp}.

    \hypertarget{running-nb2wp-with-adjustments}{%
\paragraph{Running nb2wp with
adjustments}\label{running-nb2wp-with-adjustments}}

    \begin{tcolorbox}[breakable, size=fbox, boxrule=1pt, pad at break*=1mm,colback=cellbackground, colframe=cellborder]
%\prompt{In}{incolor}{1}{\boxspacing}
\begin{Verbatim}[commandchars=\\\{\}]
\PY{k+kn}{import} \PY{n+nn}{os}
\PY{k+kn}{import} \PY{n+nn}{sys}

\PY{c+c1}{\PYZsh{} append nb2* script directory to system path}
\PY{n}{sys}\PY{o}{.}\PY{n}{path}\PY{o}{.}\PY{n}{append}\PY{p}{(}\PY{l+s+sa}{r}\PY{l+s+s1}{\PYZsq{}}\PY{l+s+s1}{C:}\PY{l+s+s1}{\PYZbs{}}\PY{l+s+s1}{temp}\PY{l+s+s1}{\PYZbs{}}\PY{l+s+s1}{nb2wp}\PY{l+s+s1}{\PYZsq{}}\PY{p}{)}

\PY{k+kn}{import} \PY{n+nn}{nb2wput} \PY{k}{as} \PY{n+nn}{nbu}
\end{Verbatim}
\end{tcolorbox}

    \begin{tcolorbox}[breakable, size=fbox, boxrule=1pt, pad at break*=1mm,colback=cellbackground, colframe=cellborder]
%\prompt{In}{incolor}{2}{\boxspacing}
\begin{Verbatim}[commandchars=\\\{\}]
\PY{c+c1}{\PYZsh{} set notebook files \PYZhy{} the source for this blog post}
\PY{n}{nb\PYZus{}file\PYZus{}0} \PY{o}{=} \PY{l+s+sa}{r}\PY{l+s+s1}{\PYZsq{}}\PY{l+s+s1}{C:}\PY{l+s+s1}{\PYZbs{}}\PY{l+s+s1}{temp}\PY{l+s+s1}{\PYZbs{}}\PY{l+s+s1}{nb2wp}\PY{l+s+s1}{\PYZbs{}}\PY{l+s+s1}{TweakingJupyterExportTemplates0.ipynb}\PY{l+s+s1}{\PYZsq{}}
\PY{n}{nb\PYZus{}file\PYZus{}1} \PY{o}{=} \PY{l+s+sa}{r}\PY{l+s+s1}{\PYZsq{}}\PY{l+s+s1}{C:}\PY{l+s+s1}{\PYZbs{}}\PY{l+s+s1}{temp}\PY{l+s+s1}{\PYZbs{}}\PY{l+s+s1}{nb2wp}\PY{l+s+s1}{\PYZbs{}}\PY{l+s+s1}{TweakingJupyterExportTemplates1.ipynb}\PY{l+s+s1}{\PYZsq{}}
\end{Verbatim}
\end{tcolorbox}

    \begin{tcolorbox}[breakable, size=fbox, boxrule=1pt, pad at break*=1mm,colback=cellbackground, colframe=cellborder]
%\prompt{In}{incolor}{3}{\boxspacing}
\begin{Verbatim}[commandchars=\\\{\}]
\PY{c+c1}{\PYZsh{} convert notebook with (nb2wp) using}
\PY{c+c1}{\PYZsh{} defaults for unspecified arguments}
\PY{n}{nbu}\PY{o}{.}\PY{n}{nb2wp}\PY{p}{(}\PY{n}{nb\PYZus{}file\PYZus{}0}\PY{p}{,} \PY{n}{out\PYZus{}dir}\PY{o}{=}\PY{l+s+sa}{r}\PY{l+s+s1}{\PYZsq{}}\PY{l+s+s1}{C:}\PY{l+s+s1}{\PYZbs{}}\PY{l+s+s1}{temp}\PY{l+s+s1}{\PYZbs{}}\PY{l+s+s1}{nb2wp}\PY{l+s+s1}{\PYZbs{}}\PY{l+s+s1}{tmp}\PY{l+s+s1}{\PYZsq{}}\PY{p}{,}
    \PY{n}{css\PYZus{}files}\PY{o}{=}\PY{p}{[}\PY{l+s+sa}{r}\PY{l+s+s1}{\PYZsq{}}\PY{l+s+s1}{C:}\PY{l+s+s1}{\PYZbs{}}\PY{l+s+s1}{temp}\PY{l+s+s1}{\PYZbs{}}\PY{l+s+s1}{nb2wp}\PY{l+s+s1}{\PYZbs{}}\PY{l+s+s1}{style.css}\PY{l+s+s1}{\PYZsq{}}\PY{p}{]}\PY{p}{,} \PY{n}{save\PYZus{}css}\PY{o}{=}\PY{k+kc}{True}\PY{p}{,} 
    \PY{n}{remove\PYZus{}attrs}\PY{o}{=}\PY{k+kc}{True}\PY{p}{,} \PY{n}{template}\PY{o}{=}\PY{l+s+s1}{\PYZsq{}}\PY{l+s+s1}{full}\PY{l+s+s1}{\PYZsq{}}\PY{p}{,} \PY{n}{latex}\PY{o}{=}\PY{l+s+s2}{\PYZdq{}}\PY{l+s+s2}{wp}\PY{l+s+s2}{\PYZdq{}}\PY{p}{,}
    \PY{n}{img\PYZus{}dir}\PY{o}{=}\PY{l+s+sa}{r}\PY{l+s+s1}{\PYZsq{}}\PY{l+s+s1}{C:}\PY{l+s+s1}{\PYZbs{}}\PY{l+s+s1}{temp}\PY{l+s+s1}{\PYZbs{}}\PY{l+s+s1}{nb2wp}\PY{l+s+s1}{\PYZbs{}}\PY{l+s+s1}{tmp}\PY{l+s+s1}{\PYZbs{}}\PY{l+s+s1}{img}\PY{l+s+s1}{\PYZsq{}}\PY{p}{,} \PY{n}{footer}\PY{o}{=}\PY{k+kc}{False}\PY{p}{)}
\end{Verbatim}
\end{tcolorbox}

    \begin{Verbatim}[commandchars=\\\{\}]
Using template: full
Using CSS files ['C:\textbackslash{}\textbackslash{}temp\textbackslash{}\textbackslash{}nb2wp\textbackslash{}\textbackslash{}style.css']
Saving CSS to C:\textbackslash{}temp\textbackslash{}nb2wp\textbackslash{}tmp\textbackslash{}style.css
C:\textbackslash{}temp\textbackslash{}nb2wp\textbackslash{}tmp\textbackslash{}TweakingJupyterExportTemplates0.html: 23132 bytes written in
7.650s
    \end{Verbatim}

    \begin{tcolorbox}[breakable, size=fbox, boxrule=1pt, pad at break*=1mm,colback=cellbackground, colframe=cellborder]
%\prompt{In}{incolor}{4}{\boxspacing}
\begin{Verbatim}[commandchars=\\\{\}]
\PY{c+c1}{\PYZsh{} convert notebook using tweaked CSS and custom template}
\PY{n}{nbu}\PY{o}{.}\PY{n}{nb2wp}\PY{p}{(}\PY{n}{nb\PYZus{}file\PYZus{}1}\PY{p}{,} \PY{n}{out\PYZus{}dir}\PY{o}{=}\PY{l+s+sa}{r}\PY{l+s+s1}{\PYZsq{}}\PY{l+s+s1}{C:}\PY{l+s+s1}{\PYZbs{}}\PY{l+s+s1}{temp}\PY{l+s+s1}{\PYZbs{}}\PY{l+s+s1}{nb2wp}\PY{l+s+s1}{\PYZbs{}}\PY{l+s+s1}{tmp}\PY{l+s+s1}{\PYZsq{}}\PY{p}{,}
    \PY{n}{css\PYZus{}files}\PY{o}{=}\PY{p}{[}\PY{l+s+sa}{r}\PY{l+s+s1}{\PYZsq{}}\PY{l+s+s1}{C:}\PY{l+s+s1}{\PYZbs{}}\PY{l+s+s1}{temp}\PY{l+s+s1}{\PYZbs{}}\PY{l+s+s1}{nb2wp}\PY{l+s+s1}{\PYZbs{}}\PY{l+s+s1}{my\PYZus{}nb2wp\PYZus{}style.css}\PY{l+s+s1}{\PYZsq{}}\PY{p}{]}\PY{p}{,} \PY{n}{save\PYZus{}css}\PY{o}{=}\PY{k+kc}{True}\PY{p}{,} 
    \PY{n}{remove\PYZus{}attrs}\PY{o}{=}\PY{k+kc}{True}\PY{p}{,} \PY{n}{template}\PY{o}{=}\PY{l+s+s1}{\PYZsq{}}\PY{l+s+s1}{blankfull}\PY{l+s+s1}{\PYZsq{}}\PY{p}{,} \PY{n}{latex}\PY{o}{=}\PY{l+s+s2}{\PYZdq{}}\PY{l+s+s2}{wp}\PY{l+s+s2}{\PYZdq{}}\PY{p}{,} \PY{n}{bg\PYZus{}color\PYZus{}hex}\PY{o}{=}\PY{l+s+s1}{\PYZsq{}}\PY{l+s+s1}{cfcdcd}\PY{l+s+s1}{\PYZsq{}}\PY{p}{,}
    \PY{n}{img\PYZus{}dir}\PY{o}{=}\PY{l+s+sa}{r}\PY{l+s+s1}{\PYZsq{}}\PY{l+s+s1}{C:}\PY{l+s+s1}{\PYZbs{}}\PY{l+s+s1}{temp}\PY{l+s+s1}{\PYZbs{}}\PY{l+s+s1}{nb2wp}\PY{l+s+s1}{\PYZbs{}}\PY{l+s+s1}{tmp}\PY{l+s+s1}{\PYZbs{}}\PY{l+s+s1}{img}\PY{l+s+s1}{\PYZsq{}}\PY{p}{,} \PY{n}{footer}\PY{o}{=}\PY{k+kc}{False}\PY{p}{)}
\end{Verbatim}
\end{tcolorbox}

    \begin{Verbatim}[commandchars=\\\{\}]
Using template: blankfull
Using CSS files ['C:\textbackslash{}\textbackslash{}temp\textbackslash{}\textbackslash{}nb2wp\textbackslash{}\textbackslash{}my\_nb2wp\_style.css']
Saving CSS to C:\textbackslash{}temp\textbackslash{}nb2wp\textbackslash{}tmp\textbackslash{}style.css
C:\textbackslash{}temp\textbackslash{}nb2wp\textbackslash{}tmp\textbackslash{}TweakingJupyterExportTemplates1.html: 57127 bytes written in
12.055s
    \end{Verbatim}



    % Add a bibliography block to the postdoc
    
    
    
%\end{document}
