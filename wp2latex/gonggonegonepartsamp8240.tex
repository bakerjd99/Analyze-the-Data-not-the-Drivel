%% bm.pdf preamble - material merged from previous preamble and current pandoc preamable output
% NOTE: float placement required changes to the source files referenced by bm.tex
% May 28, 2020
%
% Use lualatex to compile - test with MiKTeX 2.9

% uncomment to list all files in log
%\listfiles

\documentclass[12pt]{report}


\usepackage{fontspec}

%\setmainfont[Scale=MatchLowercase]{Lucida Bright}
%\setmonofont{FreeMono}
%\setmonofont{Source Code Pro}
\setmonofont[Scale=MatchLowercase]{Ubuntu Mono}

% short snippets of asian languages
\newfontfamily\myAsian{Noto Serif TC Medium}

\usepackage[headings]{fullpage}

% national use characters 
%\usepackage{inputenc}

% ams mathematical symbols
\usepackage{amsmath,amssymb}

% added to support pandoc highlighting
\usepackage{microtype}

\usepackage{makeidx}

% add index and bibliographies to table of contents
\usepackage[nottoc]{tocbibind}

% postscript courier and times in place of cm fonts
%\usepackage{courier}
%\usepackage{times}

% extended coloring
\usepackage{color}
\usepackage[table,dvipsnames]{xcolor}
\usepackage{colortbl}

% advanced date formating
\usepackage{datetime}

%support pandoc code highlighting
\usepackage{fancyvrb}

% \DefineShortVerb[commandchars=\\\{\}]{\|}
% \DefineVerbatimEnvironment{Highlighting}{Verbatim}{commandchars=\\\{\}}
% % Add ',fontsize=\small' for more characters per line

% tango style colors
% \usepackage{framed}
% \definecolor{shadecolor}{RGB}{255,255,255}
% \newenvironment{Shaded}{\begin{snugshade}}{\end{snugshade}}
% \newcommand{\KeywordTok}[1]{\textcolor[rgb]{0.13,0.29,0.53}{\textbf{{#1}}}}
% \newcommand{\DataTypeTok}[1]{\textcolor[rgb]{0.13,0.29,0.53}{{#1}}}
% \newcommand{\DecValTok}[1]{\textcolor[rgb]{0.00,0.00,0.81}{{#1}}}
% \newcommand{\BaseNTok}[1]{\textcolor[rgb]{0.00,0.00,0.81}{{#1}}}
% \newcommand{\FloatTok}[1]{\textcolor[rgb]{0.00,0.00,0.81}{{#1}}}
% \newcommand{\CharTok}[1]{\textcolor[rgb]{0.31,0.60,0.02}{{#1}}}
% \newcommand{\StringTok}[1]{\textcolor[rgb]{0.31,0.60,0.02}{{#1}}}
% \newcommand{\CommentTok}[1]{\textcolor[rgb]{0.56,0.35,0.01}{\textit{{#1}}}}
% \newcommand{\OtherTok}[1]{\textcolor[rgb]{0.56,0.35,0.01}{{#1}}}
% \newcommand{\AlertTok}[1]{\textcolor[rgb]{0.94,0.16,0.16}{{#1}}}
% \newcommand{\FunctionTok}[1]{\textcolor[rgb]{0.00,0.00,0.00}{{#1}}}
% \newcommand{\RegionMarkerTok}[1]{{#1}}
% \newcommand{\ErrorTok}[1]{\textbf{{#1}}}
% \newcommand{\NormalTok}[1]{{#1}}

% %espresso style colors
% \usepackage{framed}
% \definecolor{shadecolor}{RGB}{42,33,28}
% \newenvironment{Shaded}{\begin{snugshade}}{\end{snugshade}}
% \newcommand{\KeywordTok}[1]{\textcolor[rgb]{0.26,0.66,0.93}{\textbf{{#1}}}}
% \newcommand{\DataTypeTok}[1]{\textcolor[rgb]{0.74,0.68,0.62}{\underline{{#1}}}}
% \newcommand{\DecValTok}[1]{\textcolor[rgb]{0.27,0.67,0.26}{{#1}}}
% \newcommand{\BaseNTok}[1]{\textcolor[rgb]{0.27,0.67,0.26}{{#1}}}
% \newcommand{\FloatTok}[1]{\textcolor[rgb]{0.27,0.67,0.26}{{#1}}}
% \newcommand{\CharTok}[1]{\textcolor[rgb]{0.02,0.61,0.04}{{#1}}}
% \newcommand{\StringTok}[1]{\textcolor[rgb]{0.02,0.61,0.04}{{#1}}}
% \newcommand{\CommentTok}[1]{\textcolor[rgb]{0.00,0.40,1.00}{\textit{{#1}}}}
% \newcommand{\OtherTok}[1]{\textcolor[rgb]{0.74,0.68,0.62}{{#1}}}
% \newcommand{\AlertTok}[1]{\textcolor[rgb]{1.00,1.00,0.00}{{#1}}}
% \newcommand{\FunctionTok}[1]{\textcolor[rgb]{1.00,0.58,0.35}{\textbf{{#1}}}}
% \newcommand{\RegionMarkerTok}[1]{\textcolor[rgb]{0.74,0.68,0.62}{{#1}}}
% \newcommand{\ErrorTok}[1]{\textcolor[rgb]{0.74,0.68,0.62}{\textbf{{#1}}}}
% \newcommand{\NormalTok}[1]{\textcolor[rgb]{0.74,0.68,0.62}{{#1}}}

% %kete style colors
% \newenvironment{Shaded}{}{}
% \newcommand{\KeywordTok}[1]{\textbf{{#1}}}
% \newcommand{\DataTypeTok}[1]{\textcolor[rgb]{0.50,0.00,0.00}{{#1}}}
% \newcommand{\DecValTok}[1]{\textcolor[rgb]{0.00,0.00,1.00}{{#1}}}
% \newcommand{\BaseNTok}[1]{\textcolor[rgb]{0.00,0.00,1.00}{{#1}}}
% \newcommand{\FloatTok}[1]{\textcolor[rgb]{0.50,0.00,0.50}{{#1}}}
% \newcommand{\CharTok}[1]{\textcolor[rgb]{1.00,0.00,1.00}{{#1}}}
% \newcommand{\StringTok}[1]{\textcolor[rgb]{0.87,0.00,0.00}{{#1}}}
% \newcommand{\CommentTok}[1]{\textcolor[rgb]{0.50,0.50,0.50}{\textit{{#1}}}}
% \newcommand{\OtherTok}[1]{{#1}}
% \newcommand{\AlertTok}[1]{\textcolor[rgb]{0.00,1.00,0.00}{\textbf{{#1}}}}
% \newcommand{\FunctionTok}[1]{\textcolor[rgb]{0.00,0.00,0.50}{{#1}}}
% \newcommand{\RegionMarkerTok}[1]{{#1}}
% \newcommand{\ErrorTok}[1]{\textcolor[rgb]{1.00,0.00,0.00}{\textbf{{#1}}}}
% \newcommand{\NormalTok}[1]{{#1}}
% %end pandoc code hacks

% jodliterate colors
\usepackage{color}
\definecolor{shadecolor}{RGB}{248,248,248}
% j control structures 
\definecolor{keywcolor}{rgb}{0.13,0.29,0.53}
% j explicit arguments x y m n u v
\definecolor{datacolor}{rgb}{0.13,0.29,0.53}
% j numbers - all types see j.xml
\definecolor{decvcolor}{rgb}{0.00,0.00,0.81}
\definecolor{basencolor}{rgb}{0.00,0.00,0.81}
\definecolor{floatcolor}{rgb}{0.00,0.00,0.81}
% j local assignments
\definecolor{charcolor}{rgb}{0.31,0.60,0.02}
\definecolor{stringcolor}{rgb}{0.31,0.60,0.02}
\definecolor{commentcolor}{rgb}{0.56,0.35,0.01}
% primitive adverbs and conjunctions
%\definecolor{othercolor}{rgb}{0.56,0.35,0.01}   
\definecolor{othercolor}{RGB}{0,0,255}
% global assignments
\definecolor{alertcolor}{rgb}{0.94,0.16,0.16}
% primitive J verbs and noun names
\definecolor{funccolor}{rgb}{0.00,0.00,0.00}

% custom colors
\definecolor{CodeBackGround}{cmyk}{0.0,0.0,0,0.05}    % light gray
\definecolor{CodeComment}{rgb}{0,0.50,0.00}           % dark green {0,0.45,0.08}
\definecolor{TableStripes}{gray}{0.9}                 % odd/even background in tables

% Colors for the hyperref package
\definecolor{urlcolor}{rgb}{0,.145,.698}
\definecolor{linkcolor}{rgb}{.71,0.21,0.01}
\definecolor{citecolor}{rgb}{.12,.54,.11}

% % Exact colors from NB
\definecolor{incolor}{HTML}{303F9F}
\definecolor{outcolor}{HTML}{D84315}
\definecolor{cellborder}{HTML}{CFCFCF}
\definecolor{cellbackground}{HTML}{F7F7F7}

% % ANSI colors
\definecolor{ansi-black}{HTML}{3E424D}
\definecolor{ansi-black-intense}{HTML}{282C36}
\definecolor{ansi-red}{HTML}{E75C58}
\definecolor{ansi-red-intense}{HTML}{B22B31}
\definecolor{ansi-green}{HTML}{00A250}
\definecolor{ansi-green-intense}{HTML}{007427}
\definecolor{ansi-yellow}{HTML}{DDB62B}
\definecolor{ansi-yellow-intense}{HTML}{B27D12}
\definecolor{ansi-blue}{HTML}{208FFB}
\definecolor{ansi-blue-intense}{HTML}{0065CA}
\definecolor{ansi-magenta}{HTML}{D160C4}
\definecolor{ansi-magenta-intense}{HTML}{A03196}
\definecolor{ansi-cyan}{HTML}{60C6C8}
\definecolor{ansi-cyan-intense}{HTML}{258F8F}
\definecolor{ansi-white}{HTML}{C5C1B4}
\definecolor{ansi-white-intense}{HTML}{A1A6B2}
\definecolor{ansi-default-inverse-fg}{HTML}{FFFFFF}
\definecolor{ansi-default-inverse-bg}{HTML}{000000}
    

% \usepackage{framed}
% \newenvironment{Shaded}{}{}
% \newcommand{\KeywordTok}[1]{\textcolor{keywcolor}{\textbf{{#1}}}}
% \newcommand{\DataTypeTok}[1]{\textcolor{datacolor}{{#1}}}
% %\newcommand{\DecValTok}[1]{\textcolor{decvcolor}{{#1}}}
% \newcommand{\DecValTok}[1]{{#1}} 
% \newcommand{\BaseNTok}[1]{\textcolor{basencolor}{{#1}}}
% \newcommand{\FloatTok}[1]{\textcolor{floatcolor}{{#1}}}
% \newcommand{\CharTok}[1]{\textcolor{charcolor}{\textbf{{#1}}}}
% \newcommand{\StringTok}[1]{\textcolor{stringcolor}{{#1}}}
% \newcommand{\CommentTok}[1]{\textcolor{commentcolor}{\textit{{#1}}}}
% \newcommand{\OtherTok}[1]{\textcolor{othercolor}{{#1}}} 
% \newcommand{\AlertTok}[1]{\textcolor{alertcolor}{\textbf{{#1}}}}
% %\newcommand{\FunctionTok}[1]{\textcolor{funccolor}{{#1}}}
% \newcommand{\FunctionTok}[1]{{#1}}
% \newcommand{\RegionMarkerTok}[1]{{#1}}
% \newcommand{\ErrorTok}[1]{\textbf{{#1}}}
% \newcommand{\NormalTok}[1]{{#1}}

% The default LaTeX title has an obnoxious amount of whitespace. By default,
% titling removes some of it. It also provides customization options.
\usepackage{titling}

% headers and footers
\usepackage{fancyhdr}
%\pagestyle{fancy}
\pagestyle{plain}

\fancyhead{}
\fancyfoot{}

%\fancyhead[LE,RO]{\slshape \rightmark}
%\fancyhead[LO,RE]{\slshape \leftmark}
\fancyfoot[C]{\thepage}
%\headrulewidth 0.4pt
%\footrulewidth 0 pt

%\addtolength{\headheight}{\baselineskip}

%\lfoot{\emph{Analyze the Data not the Drivel}}
%\rfoot{\emph{\today}}

% subfigure handles figures that contain subfigures
%\usepackage{color,graphicx,subfigure,sidecap}
\usepackage{graphicx,sidecap}
\usepackage{subfigure}
\graphicspath{{./inclusions/}}

% floatflt provides for text wrapping around small figures and tables
\usepackage{floatflt}

% tweak caption formats 
\usepackage{caption} 
\usepackage{sidecap}
%\usepackage{subcaption} % not compatible with subfigure

\usepackage{rotating} % flip tables sideways

% complex footnotes
%\usepackage{bigfoot}

% weird logos \XeLaTeX
\usepackage{metalogo}

\newcommand{\HRule}{\rule{\linewidth}{0.5mm}}

\usepackage[breakable]{tcolorbox}

\usepackage{parskip} % Stop auto-indenting (to mimic markdown behaviour)
    
% Basic figure setup, for now with no caption control since it's done
% automatically by Pandoc (which extracts ![](path) syntax from Markdown).
\usepackage{graphicx}

%\DeclareCaptionFormat{nocaption}{}
%\captionsetup{format=nocaption,aboveskip=0pt,belowskip=0pt}

\usepackage[Export]{adjustbox} % Used to constrain images to a maximum size
\adjustboxset{max size={0.9\linewidth}{0.9\paperheight}}
\usepackage{float}

%\floatplacement{figure}{H} % forces figures to be placed at the correct location

\usepackage{xcolor} % Allow colors to be defined
\usepackage{enumerate} % Needed for markdown enumerations to work
\usepackage{geometry} % Used to adjust the document margins

%\usepackage{amsmath} % Equations
%\usepackage{amssymb} % Equations

\usepackage{textcomp} % defines textquotesingle

% Hack from http://tex.stackexchange.com/a/47451/13684:
\AtBeginDocument{%
	\def\PYZsq{\textquotesingle}% Upright quotes in Pygmentized code
}

\usepackage{upquote} % Upright quotes for verbatim code
\usepackage{eurosym} % defines \euro
\usepackage[mathletters]{ucs} % Extended unicode (utf-8) support

%\usepackage{fancyvrb} % verbatim replacement that allows latex

\usepackage{grffile} % extends the file name processing of package graphics 
					 % to support a larger range
					 
\makeatletter % fix for grffile with XeLaTeX
\def\Gread@@xetex#1{%
  \IfFileExists{"\Gin@base".bb}%
  {\Gread@eps{\Gin@base.bb}}%
  {\Gread@@xetex@aux#1}%
}
\makeatother

% The hyperref package gives us a pdf with properly built
% internal navigation ('pdf bookmarks' for the table of contents,
% internal cross-reference links, web links for URLs, etc.)
\usepackage{hyperref}
% The default LaTeX title has an obnoxious amount of whitespace. By default,
% titling removes some of it. It also provides customization options.
\usepackage{titling}
\usepackage{longtable} % longtable support required by pandoc >1.10
\usepackage{booktabs}  % table support for pandoc > 1.12.2
\usepackage[inline]{enumitem} % IRkernel/repr support (it uses the enumerate* environment)
\usepackage[normalem]{ulem} % ulem is needed to support strikethroughs (\sout)
							% normalem makes italics be italics, not underlines
\usepackage{mathrsfs}

% commands and environments needed by pandoc snippets
% extracted from the output of `pandoc -s`
\providecommand{\tightlist}{%
  \setlength{\itemsep}{0pt}\setlength{\parskip}{0pt}}
  
\DefineVerbatimEnvironment{Highlighting}{Verbatim}{commandchars=\\\{\}}
% Add ',fontsize=\small' for more characters per line
\newenvironment{Shaded}{}{}
\newcommand{\KeywordTok}[1]{\textcolor[rgb]{0.00,0.44,0.13}{\textbf{{#1}}}}
\newcommand{\DataTypeTok}[1]{\textcolor[rgb]{0.56,0.13,0.00}{{#1}}}
\newcommand{\DecValTok}[1]{\textcolor[rgb]{0.25,0.63,0.44}{{#1}}}
\newcommand{\BaseNTok}[1]{\textcolor[rgb]{0.25,0.63,0.44}{{#1}}}
\newcommand{\FloatTok}[1]{\textcolor[rgb]{0.25,0.63,0.44}{{#1}}}
\newcommand{\CharTok}[1]{\textcolor[rgb]{0.25,0.44,0.63}{{#1}}}
\newcommand{\StringTok}[1]{\textcolor[rgb]{0.25,0.44,0.63}{{#1}}}
\newcommand{\CommentTok}[1]{\textcolor[rgb]{0.38,0.63,0.69}{\textit{{#1}}}}
\newcommand{\OtherTok}[1]{\textcolor[rgb]{0.00,0.44,0.13}{{#1}}}
\newcommand{\AlertTok}[1]{\textcolor[rgb]{1.00,0.00,0.00}{\textbf{{#1}}}}
\newcommand{\FunctionTok}[1]{\textcolor[rgb]{0.02,0.16,0.49}{{#1}}}
\newcommand{\RegionMarkerTok}[1]{{#1}}
\newcommand{\ErrorTok}[1]{\textcolor[rgb]{1.00,0.00,0.00}{\textbf{{#1}}}}
\newcommand{\NormalTok}[1]{{#1}}

% Additional commands for more recent versions of Pandoc
\newcommand{\ConstantTok}[1]{\textcolor[rgb]{0.53,0.00,0.00}{{#1}}}
\newcommand{\SpecialCharTok}[1]{\textcolor[rgb]{0.25,0.44,0.63}{{#1}}}
\newcommand{\VerbatimStringTok}[1]{\textcolor[rgb]{0.25,0.44,0.63}{{#1}}}
\newcommand{\SpecialStringTok}[1]{\textcolor[rgb]{0.73,0.40,0.53}{{#1}}}
\newcommand{\ImportTok}[1]{{#1}}
\newcommand{\DocumentationTok}[1]{\textcolor[rgb]{0.73,0.13,0.13}{\textit{{#1}}}}
\newcommand{\AnnotationTok}[1]{\textcolor[rgb]{0.38,0.63,0.69}{\textbf{\textit{{#1}}}}}
\newcommand{\CommentVarTok}[1]{\textcolor[rgb]{0.38,0.63,0.69}{\textbf{\textit{{#1}}}}}
\newcommand{\VariableTok}[1]{\textcolor[rgb]{0.10,0.09,0.49}{{#1}}}
\newcommand{\ControlFlowTok}[1]{\textcolor[rgb]{0.00,0.44,0.13}{\textbf{{#1}}}}
\newcommand{\OperatorTok}[1]{\textcolor[rgb]{0.40,0.40,0.40}{{#1}}}
\newcommand{\BuiltInTok}[1]{{#1}}
\newcommand{\ExtensionTok}[1]{{#1}}
\newcommand{\PreprocessorTok}[1]{\textcolor[rgb]{0.74,0.48,0.00}{{#1}}}
\newcommand{\AttributeTok}[1]{\textcolor[rgb]{0.49,0.56,0.16}{{#1}}}
\newcommand{\InformationTok}[1]{\textcolor[rgb]{0.38,0.63,0.69}{\textbf{\textit{{#1}}}}}
\newcommand{\WarningTok}[1]{\textcolor[rgb]{0.38,0.63,0.69}{\textbf{\textit{{#1}}}}}

% Define a nice break command that doesn't care if a line doesn't already exist.
\def\br{\hspace*{\fill} \\* }
% Math Jax compatibility definitions
\def\gt{>}
\def\lt{<}
\let\Oldtex\TeX
\let\Oldlatex\LaTeX
\renewcommand{\TeX}{\textrm{\Oldtex}}
\renewcommand{\LaTeX}{\textrm{\Oldlatex}}
 
% Pygments definitions
\makeatletter
\def\PY@reset{\let\PY@it=\relax \let\PY@bf=\relax%
    \let\PY@ul=\relax \let\PY@tc=\relax%
    \let\PY@bc=\relax \let\PY@ff=\relax}
\def\PY@tok#1{\csname PY@tok@#1\endcsname}
\def\PY@toks#1+{\ifx\relax#1\empty\else%
    \PY@tok{#1}\expandafter\PY@toks\fi}
\def\PY@do#1{\PY@bc{\PY@tc{\PY@ul{%
    \PY@it{\PY@bf{\PY@ff{#1}}}}}}}
\def\PY#1#2{\PY@reset\PY@toks#1+\relax+\PY@do{#2}}

\expandafter\def\csname PY@tok@w\endcsname{\def\PY@tc##1{\textcolor[rgb]{0.73,0.73,0.73}{##1}}}
\expandafter\def\csname PY@tok@c\endcsname{\let\PY@it=\textit\def\PY@tc##1{\textcolor[rgb]{0.25,0.50,0.50}{##1}}}
\expandafter\def\csname PY@tok@cp\endcsname{\def\PY@tc##1{\textcolor[rgb]{0.74,0.48,0.00}{##1}}}
\expandafter\def\csname PY@tok@k\endcsname{\let\PY@bf=\textbf\def\PY@tc##1{\textcolor[rgb]{0.00,0.50,0.00}{##1}}}
\expandafter\def\csname PY@tok@kp\endcsname{\def\PY@tc##1{\textcolor[rgb]{0.00,0.50,0.00}{##1}}}
\expandafter\def\csname PY@tok@kt\endcsname{\def\PY@tc##1{\textcolor[rgb]{0.69,0.00,0.25}{##1}}}
\expandafter\def\csname PY@tok@o\endcsname{\def\PY@tc##1{\textcolor[rgb]{0.40,0.40,0.40}{##1}}}
\expandafter\def\csname PY@tok@ow\endcsname{\let\PY@bf=\textbf\def\PY@tc##1{\textcolor[rgb]{0.67,0.13,1.00}{##1}}}
\expandafter\def\csname PY@tok@nb\endcsname{\def\PY@tc##1{\textcolor[rgb]{0.00,0.50,0.00}{##1}}}
\expandafter\def\csname PY@tok@nf\endcsname{\def\PY@tc##1{\textcolor[rgb]{0.00,0.00,1.00}{##1}}}
\expandafter\def\csname PY@tok@nc\endcsname{\let\PY@bf=\textbf\def\PY@tc##1{\textcolor[rgb]{0.00,0.00,1.00}{##1}}}
\expandafter\def\csname PY@tok@nn\endcsname{\let\PY@bf=\textbf\def\PY@tc##1{\textcolor[rgb]{0.00,0.00,1.00}{##1}}}
\expandafter\def\csname PY@tok@ne\endcsname{\let\PY@bf=\textbf\def\PY@tc##1{\textcolor[rgb]{0.82,0.25,0.23}{##1}}}
\expandafter\def\csname PY@tok@nv\endcsname{\def\PY@tc##1{\textcolor[rgb]{0.10,0.09,0.49}{##1}}}
\expandafter\def\csname PY@tok@no\endcsname{\def\PY@tc##1{\textcolor[rgb]{0.53,0.00,0.00}{##1}}}
\expandafter\def\csname PY@tok@nl\endcsname{\def\PY@tc##1{\textcolor[rgb]{0.63,0.63,0.00}{##1}}}
\expandafter\def\csname PY@tok@ni\endcsname{\let\PY@bf=\textbf\def\PY@tc##1{\textcolor[rgb]{0.60,0.60,0.60}{##1}}}
\expandafter\def\csname PY@tok@na\endcsname{\def\PY@tc##1{\textcolor[rgb]{0.49,0.56,0.16}{##1}}}
\expandafter\def\csname PY@tok@nt\endcsname{\let\PY@bf=\textbf\def\PY@tc##1{\textcolor[rgb]{0.00,0.50,0.00}{##1}}}
\expandafter\def\csname PY@tok@nd\endcsname{\def\PY@tc##1{\textcolor[rgb]{0.67,0.13,1.00}{##1}}}
\expandafter\def\csname PY@tok@s\endcsname{\def\PY@tc##1{\textcolor[rgb]{0.73,0.13,0.13}{##1}}}
\expandafter\def\csname PY@tok@sd\endcsname{\let\PY@it=\textit\def\PY@tc##1{\textcolor[rgb]{0.73,0.13,0.13}{##1}}}
\expandafter\def\csname PY@tok@si\endcsname{\let\PY@bf=\textbf\def\PY@tc##1{\textcolor[rgb]{0.73,0.40,0.53}{##1}}}
\expandafter\def\csname PY@tok@se\endcsname{\let\PY@bf=\textbf\def\PY@tc##1{\textcolor[rgb]{0.73,0.40,0.13}{##1}}}
\expandafter\def\csname PY@tok@sr\endcsname{\def\PY@tc##1{\textcolor[rgb]{0.73,0.40,0.53}{##1}}}
\expandafter\def\csname PY@tok@ss\endcsname{\def\PY@tc##1{\textcolor[rgb]{0.10,0.09,0.49}{##1}}}
\expandafter\def\csname PY@tok@sx\endcsname{\def\PY@tc##1{\textcolor[rgb]{0.00,0.50,0.00}{##1}}}
\expandafter\def\csname PY@tok@m\endcsname{\def\PY@tc##1{\textcolor[rgb]{0.40,0.40,0.40}{##1}}}
\expandafter\def\csname PY@tok@gh\endcsname{\let\PY@bf=\textbf\def\PY@tc##1{\textcolor[rgb]{0.00,0.00,0.50}{##1}}}
\expandafter\def\csname PY@tok@gu\endcsname{\let\PY@bf=\textbf\def\PY@tc##1{\textcolor[rgb]{0.50,0.00,0.50}{##1}}}
\expandafter\def\csname PY@tok@gd\endcsname{\def\PY@tc##1{\textcolor[rgb]{0.63,0.00,0.00}{##1}}}
\expandafter\def\csname PY@tok@gi\endcsname{\def\PY@tc##1{\textcolor[rgb]{0.00,0.63,0.00}{##1}}}
\expandafter\def\csname PY@tok@gr\endcsname{\def\PY@tc##1{\textcolor[rgb]{1.00,0.00,0.00}{##1}}}
\expandafter\def\csname PY@tok@ge\endcsname{\let\PY@it=\textit}
\expandafter\def\csname PY@tok@gs\endcsname{\let\PY@bf=\textbf}
\expandafter\def\csname PY@tok@gp\endcsname{\let\PY@bf=\textbf\def\PY@tc##1{\textcolor[rgb]{0.00,0.00,0.50}{##1}}}
\expandafter\def\csname PY@tok@go\endcsname{\def\PY@tc##1{\textcolor[rgb]{0.53,0.53,0.53}{##1}}}
\expandafter\def\csname PY@tok@gt\endcsname{\def\PY@tc##1{\textcolor[rgb]{0.00,0.27,0.87}{##1}}}
\expandafter\def\csname PY@tok@err\endcsname{\def\PY@bc##1{\setlength{\fboxsep}{0pt}\fcolorbox[rgb]{1.00,0.00,0.00}{1,1,1}{\strut ##1}}}
\expandafter\def\csname PY@tok@kc\endcsname{\let\PY@bf=\textbf\def\PY@tc##1{\textcolor[rgb]{0.00,0.50,0.00}{##1}}}
\expandafter\def\csname PY@tok@kd\endcsname{\let\PY@bf=\textbf\def\PY@tc##1{\textcolor[rgb]{0.00,0.50,0.00}{##1}}}
\expandafter\def\csname PY@tok@kn\endcsname{\let\PY@bf=\textbf\def\PY@tc##1{\textcolor[rgb]{0.00,0.50,0.00}{##1}}}
\expandafter\def\csname PY@tok@kr\endcsname{\let\PY@bf=\textbf\def\PY@tc##1{\textcolor[rgb]{0.00,0.50,0.00}{##1}}}
\expandafter\def\csname PY@tok@bp\endcsname{\def\PY@tc##1{\textcolor[rgb]{0.00,0.50,0.00}{##1}}}
\expandafter\def\csname PY@tok@fm\endcsname{\def\PY@tc##1{\textcolor[rgb]{0.00,0.00,1.00}{##1}}}
\expandafter\def\csname PY@tok@vc\endcsname{\def\PY@tc##1{\textcolor[rgb]{0.10,0.09,0.49}{##1}}}
\expandafter\def\csname PY@tok@vg\endcsname{\def\PY@tc##1{\textcolor[rgb]{0.10,0.09,0.49}{##1}}}
\expandafter\def\csname PY@tok@vi\endcsname{\def\PY@tc##1{\textcolor[rgb]{0.10,0.09,0.49}{##1}}}
\expandafter\def\csname PY@tok@vm\endcsname{\def\PY@tc##1{\textcolor[rgb]{0.10,0.09,0.49}{##1}}}
\expandafter\def\csname PY@tok@sa\endcsname{\def\PY@tc##1{\textcolor[rgb]{0.73,0.13,0.13}{##1}}}
\expandafter\def\csname PY@tok@sb\endcsname{\def\PY@tc##1{\textcolor[rgb]{0.73,0.13,0.13}{##1}}}
\expandafter\def\csname PY@tok@sc\endcsname{\def\PY@tc##1{\textcolor[rgb]{0.73,0.13,0.13}{##1}}}
\expandafter\def\csname PY@tok@dl\endcsname{\def\PY@tc##1{\textcolor[rgb]{0.73,0.13,0.13}{##1}}}
\expandafter\def\csname PY@tok@s2\endcsname{\def\PY@tc##1{\textcolor[rgb]{0.73,0.13,0.13}{##1}}}
\expandafter\def\csname PY@tok@sh\endcsname{\def\PY@tc##1{\textcolor[rgb]{0.73,0.13,0.13}{##1}}}
\expandafter\def\csname PY@tok@s1\endcsname{\def\PY@tc##1{\textcolor[rgb]{0.73,0.13,0.13}{##1}}}
\expandafter\def\csname PY@tok@mb\endcsname{\def\PY@tc##1{\textcolor[rgb]{0.40,0.40,0.40}{##1}}}
\expandafter\def\csname PY@tok@mf\endcsname{\def\PY@tc##1{\textcolor[rgb]{0.40,0.40,0.40}{##1}}}
\expandafter\def\csname PY@tok@mh\endcsname{\def\PY@tc##1{\textcolor[rgb]{0.40,0.40,0.40}{##1}}}
\expandafter\def\csname PY@tok@mi\endcsname{\def\PY@tc##1{\textcolor[rgb]{0.40,0.40,0.40}{##1}}}
\expandafter\def\csname PY@tok@il\endcsname{\def\PY@tc##1{\textcolor[rgb]{0.40,0.40,0.40}{##1}}}
\expandafter\def\csname PY@tok@mo\endcsname{\def\PY@tc##1{\textcolor[rgb]{0.40,0.40,0.40}{##1}}}
\expandafter\def\csname PY@tok@ch\endcsname{\let\PY@it=\textit\def\PY@tc##1{\textcolor[rgb]{0.25,0.50,0.50}{##1}}}
\expandafter\def\csname PY@tok@cm\endcsname{\let\PY@it=\textit\def\PY@tc##1{\textcolor[rgb]{0.25,0.50,0.50}{##1}}}
\expandafter\def\csname PY@tok@cpf\endcsname{\let\PY@it=\textit\def\PY@tc##1{\textcolor[rgb]{0.25,0.50,0.50}{##1}}}
\expandafter\def\csname PY@tok@c1\endcsname{\let\PY@it=\textit\def\PY@tc##1{\textcolor[rgb]{0.25,0.50,0.50}{##1}}}
\expandafter\def\csname PY@tok@cs\endcsname{\let\PY@it=\textit\def\PY@tc##1{\textcolor[rgb]{0.25,0.50,0.50}{##1}}}

\def\PYZbs{\char`\\}
\def\PYZus{\char`\_}
\def\PYZob{\char`\{}
\def\PYZcb{\char`\}}
\def\PYZca{\char`\^}
\def\PYZam{\char`\&}
\def\PYZlt{\char`\<}
\def\PYZgt{\char`\>}
\def\PYZsh{\char`\#}
\def\PYZpc{\char`\%}
\def\PYZdl{\char`\$}
\def\PYZhy{\char`\-}
\def\PYZsq{\char`\'}
\def\PYZdq{\char`\"}
\def\PYZti{\char`\~}
% for compatibility with earlier versions
\def\PYZat{@}
\def\PYZlb{[}
\def\PYZrb{]}
\makeatother

% For linebreaks inside Verbatim environment from package fancyvrb. 
\makeatletter
	\newbox\Wrappedcontinuationbox 
	\newbox\Wrappedvisiblespacebox 
	\newcommand*\Wrappedvisiblespace {\textcolor{red}{\textvisiblespace}} 
	\newcommand*\Wrappedcontinuationsymbol {\textcolor{red}{\llap{\tiny$\m@th\hookrightarrow$}}} 
	\newcommand*\Wrappedcontinuationindent {3ex } 
	\newcommand*\Wrappedafterbreak {\kern\Wrappedcontinuationindent\copy\Wrappedcontinuationbox} 
	% Take advantage of the already applied Pygments mark-up to insert 
	% potential linebreaks for TeX processing. 
	%        {, <, #, %, $, ' and ": go to next line. 
	%        _, }, ^, &, >, - and ~: stay at end of broken line. 
	% Use of \textquotesingle for straight quote. 
	\newcommand*\Wrappedbreaksatspecials {% 
		\def\PYGZus{\discretionary{\char`\_}{\Wrappedafterbreak}{\char`\_}}% 
		\def\PYGZob{\discretionary{}{\Wrappedafterbreak\char`\{}{\char`\{}}% 
		\def\PYGZcb{\discretionary{\char`\}}{\Wrappedafterbreak}{\char`\}}}% 
		\def\PYGZca{\discretionary{\char`\^}{\Wrappedafterbreak}{\char`\^}}% 
		\def\PYGZam{\discretionary{\char`\&}{\Wrappedafterbreak}{\char`\&}}% 
		\def\PYGZlt{\discretionary{}{\Wrappedafterbreak\char`\<}{\char`\<}}% 
		\def\PYGZgt{\discretionary{\char`\>}{\Wrappedafterbreak}{\char`\>}}% 
		\def\PYGZsh{\discretionary{}{\Wrappedafterbreak\char`\#}{\char`\#}}% 
		\def\PYGZpc{\discretionary{}{\Wrappedafterbreak\char`\%}{\char`\%}}% 
		\def\PYGZdl{\discretionary{}{\Wrappedafterbreak\char`\$}{\char`\$}}% 
		\def\PYGZhy{\discretionary{\char`\-}{\Wrappedafterbreak}{\char`\-}}% 
		\def\PYGZsq{\discretionary{}{\Wrappedafterbreak\textquotesingle}{\textquotesingle}}% 
		\def\PYGZdq{\discretionary{}{\Wrappedafterbreak\char`\"}{\char`\"}}% 
		\def\PYGZti{\discretionary{\char`\~}{\Wrappedafterbreak}{\char`\~}}% 
	} 
	% Some characters . , ; ? ! / are not pygmentized. 
	% This macro makes them "active" and they will insert potential linebreaks 
	\newcommand*\Wrappedbreaksatpunct {% 
		\lccode`\~`\.\lowercase{\def~}{\discretionary{\hbox{\char`\.}}{\Wrappedafterbreak}{\hbox{\char`\.}}}% 
		\lccode`\~`\,\lowercase{\def~}{\discretionary{\hbox{\char`\,}}{\Wrappedafterbreak}{\hbox{\char`\,}}}% 
		\lccode`\~`\;\lowercase{\def~}{\discretionary{\hbox{\char`\;}}{\Wrappedafterbreak}{\hbox{\char`\;}}}% 
		\lccode`\~`\:\lowercase{\def~}{\discretionary{\hbox{\char`\:}}{\Wrappedafterbreak}{\hbox{\char`\:}}}% 
		\lccode`\~`\?\lowercase{\def~}{\discretionary{\hbox{\char`\?}}{\Wrappedafterbreak}{\hbox{\char`\?}}}% 
		\lccode`\~`\!\lowercase{\def~}{\discretionary{\hbox{\char`\!}}{\Wrappedafterbreak}{\hbox{\char`\!}}}% 
		\lccode`\~`\/\lowercase{\def~}{\discretionary{\hbox{\char`\/}}{\Wrappedafterbreak}{\hbox{\char`\/}}}% 
		\catcode`\.\active
		\catcode`\,\active 
		\catcode`\;\active
		\catcode`\:\active
		\catcode`\?\active
		\catcode`\!\active
		\catcode`\/\active 
		\lccode`\~`\~ 	
	}
\makeatother

\let\OriginalVerbatim=\Verbatim
\makeatletter
\renewcommand{\Verbatim}[1][1]{%
	%\parskip\z@skip
	\sbox\Wrappedcontinuationbox {\Wrappedcontinuationsymbol}%
	\sbox\Wrappedvisiblespacebox {\FV@SetupFont\Wrappedvisiblespace}%
	\def\FancyVerbFormatLine ##1{\hsize\linewidth
		\vtop{\raggedright\hyphenpenalty\z@\exhyphenpenalty\z@
			\doublehyphendemerits\z@\finalhyphendemerits\z@
			\strut ##1\strut}%
	}%
	% If the linebreak is at a space, the latter will be displayed as visible
	% space at end of first line, and a continuation symbol starts next line.
	% Stretch/shrink are however usually zero for typewriter font.
	\def\FV@Space {%
		\nobreak\hskip\z@ plus\fontdimen3\font minus\fontdimen4\font
		\discretionary{\copy\Wrappedvisiblespacebox}{\Wrappedafterbreak}
		{\kern\fontdimen2\font}%
	}%
	
	% Allow breaks at special characters using \PYG... macros.
	\Wrappedbreaksatspecials
	% Breaks at punctuation characters . , ; ? ! and / need catcode=\active 	
	\OriginalVerbatim[#1,codes*=\Wrappedbreaksatpunct]%
}
\makeatother


% prompt
\makeatletter
\newcommand{\boxspacing}{\kern\kvtcb@left@rule\kern\kvtcb@boxsep}
\makeatother
\newcommand{\prompt}[4]{
	\ttfamily\llap{{\color{#2}[#3]:\hspace{3pt}#4}}\vspace{-\baselineskip}
}
    

% Prevent overflowing lines due to hard-to-break entities
\sloppy 

% Setup hyperref package
\hypersetup{
  breaklinks=true,  % so long urls are correctly broken across lines
  colorlinks=true,
  urlcolor=urlcolor,
  linkcolor=linkcolor,
  citecolor=citecolor,
  pdfauthor={John D. Baker},
  pdftitle={Analyze the Data not the Drivel},
  pdfsubject={Blog},
  pdfcreator={MikTeX+LaTeXe},
  pdfkeywords={blog,wordpress},
  }
  
% Slightly bigger margins than the latex defaults
% \geometry{verbose,tmargin=1in,bmargin=1in,lmargin=1in,rmargin=1in}  

%\usepackage{wrapfig}

% source code listings
\usepackage{listings}

\lstdefinelanguage{bat}
{morekeywords={echo,title,pushd,popd,setlocal,endlocal,off,if,not,exist,set,goto,pause},
sensitive=True,
morecomment=[l]{rem}
}

\lstdefinelanguage{jdoc}
{
morekeywords={},
otherkeywords={assert.,break.,continue.,for.,do.,if.,else.,elseif.,return.,select.,end.
,while.,whilst.,throw.,catch.,catchd.,catcht.,try.,case.,fcase.},
sensitive=True,
morecomment=[l]{NB.},
morestring=[b]',
morestring=[d]',
}

% latex size ordering - can never remember it
% \tiny
% \scriptsize
% \footnotesize
% \small
% \normalsize
% \large
% \Large
% \LARGE
% \huge
% \Huge
 
% listings package settings  
\lstset{%
  language=jdoc,                                % j document settings
  basicstyle=\ttfamily\footnotesize,            
  keywordstyle=\bfseries\color{keywcolor}\footnotesize,
  identifierstyle=\color{black},
  commentstyle=\slshape\color{CodeComment},     % colored slanted comments
  stringstyle=\color{red}\ttfamily,
  showstringspaces=false,                       
  %backgroundcolor=\color{CodeBackGround},       
  frame=single,                                
  framesep=1pt,                                 
  framerule=0.8pt,                             
  rulecolor=\color{CodeBackGround},   
  showspaces=false,
  %columns=fullflexible,
  %numbers=left,
  %numberstyle=\footnotesize,
  %numbersep=9pt,
  tabsize=2,
  showtabs=false,
  captionpos=b
  breaklines=true,                              
  breakindent=5pt                              
}

\lstdefinelanguage{JavaScript}{
  keywords={typeof, new, true, false, catch, function, return, null, catch, switch, var, if, in, while, do, else, case, break},
  ndkeywords={class, export, boolean, throw, implements, import, this},
  ndkeywordstyle=\color{darkgray}\bfseries,
  sensitive=false,
  comment=[l]{//},
  morecomment=[s]{/*}{*/},
  morestring=[b]',
  morestring=[b]"
}

% C# settings
\lstdefinestyle{sharpc}{
language=[Sharp]C,
basicstyle=\ttfamily\scriptsize, 
keywordstyle=\bfseries\color{keywcolor}\scriptsize,
framerule=0pt
}

% for source code listing longer than two use smaller font
\lstdefinestyle{smallersource}{
basicstyle=\ttfamily\scriptsize, 
keywordstyle=\bfseries\color{keywcolor}\scriptsize,
framerule=0pt
}

\lstdefinestyle{resetdefaults}{
language=jdoc,
basicstyle=\ttfamily\footnotesize,  
keywordstyle=\bfseries\color{keywcolor}\footnotesize,                                                               
framerule=0.8pt 
}

% APL UTF8 code points listed for lstlisting processing
\makeatletter
\lst@InputCatcodes
\def\lst@DefEC{%
 \lst@CCECUse \lst@ProcessLetter
  ^^80^^81^^82^^83^^84^^85^^86^^87^^88^^89^^8a^^8b^^8c^^8d^^8e^^8f%
  ^^90^^91^^92^^93^^94^^95^^96^^97^^98^^99^^9a^^9b^^9c^^9d^^9e^^9f%
  ^^a0^^a1^^a2^^a3^^a4^^a5^^a6^^a7^^a8^^a9^^aa^^ab^^ac^^ad^^ae^^af%
  ^^b0^^b1^^b2^^b3^^b4^^b5^^b6^^b7^^b8^^b9^^ba^^bb^^bc^^bd^^be^^bf%
  ^^c0^^c1^^c2^^c3^^c4^^c5^^c6^^c7^^c8^^c9^^ca^^cb^^cc^^cd^^ce^^cf%
  ^^d0^^d1^^d2^^d3^^d4^^d5^^d6^^d7^^d8^^d9^^da^^db^^dc^^dd^^de^^df%
  ^^e0^^e1^^e2^^e3^^e4^^e5^^e6^^e7^^e8^^e9^^ea^^eb^^ec^^ed^^ee^^ef%
  ^^f0^^f1^^f2^^f3^^f4^^f5^^f6^^f7^^f8^^f9^^fa^^fb^^fc^^fd^^fe^^ff%
  ^^^^20ac^^^^0153^^^^0152%
  ^^^^20a7^^^^2190^^^^2191^^^^2192^^^^2193^^^^2206^^^^2207^^^^220a%
  ^^^^2218^^^^2228^^^^2229^^^^222a^^^^2235^^^^223c^^^^2260^^^^2261%
  ^^^^2262^^^^2264^^^^2265^^^^2282^^^^2283^^^^2296^^^^22a2^^^^22a3%
  ^^^^22a4^^^^22a5^^^^22c4^^^^2308^^^^230a^^^^2336^^^^2337^^^^2339%
  ^^^^233b^^^^233d^^^^233f^^^^2340^^^^2342^^^^2347^^^^2348^^^^2349%
  ^^^^234b^^^^234e^^^^2350^^^^2352^^^^2355^^^^2357^^^^2359^^^^235d%
  ^^^^235e^^^^235f^^^^2361^^^^2362^^^^2363^^^^2364^^^^2365^^^^2368%
  ^^^^236a^^^^236b^^^^236c^^^^2371^^^^2372^^^^2373^^^^2374^^^^2375%
  ^^^^2377^^^^2378^^^^237a^^^^2395^^^^25af^^^^25ca^^^^25cb%  
  ^^00}
\lst@RestoreCatcodes
\makeatother

% custom lengths used within minipages
\newcommand{\minindent}{17pt}

\makeindex

\begin{document}

\subsection*{\href{http://analyzethedatanotthedrivel.org/2025/03/16/gonggone-gone-parts-7-8/}{Gonggone Gone --- Parts 7 \& 8}}
\addcontentsline{toc}{subsection}{Gonggone Gone --- Parts 7 \& 8}


\noindent\emph{Posted: 16 Mar 2025 19:21:12}
\vspace{6pt}

%\subsection{mine routine}\label{mine-routine}
\begin{center}\large\textbf{-- \emph{mine routine} --}\normalsize\end{center}

Their days quickly turned into a routine. Every ``morning,'' they
crawled out of their sleeping bags when Alex's alarm clock went off at
midnight. Alex and Doug lived medieval monastery hours; they got up in
the middle of the night and called it morning. Monastery hours began
when Doug saw their stove exhaust forming towering white plumes outside
in the cold daylight air. The plumes could be seen from miles away.
Staying hidden remained a top priority, so they shifted their hours to
run ``hot'' (even though it never got hot) at night. During daylight
hours, they tamped down the stoves to avoid plumes.

After squirming out of their bags, they flipped on camping lanterns and
quickly changed their base layer underwear. Alex and Doug constantly
rotated through sets of base-layer underwear. Wearing, changing, and
drying them over clotheslines rigged above the stoves. After hanging
yesterday's underwear above the stove, they plugged the small air vent
on the wall above the tent with thick insulation. Air trapped in the box
was limited; they worried about suffocating when sleeping. Next, Alex
studied his daily task list for the umpteenth time. His list had many
things to do. He couldn't depend on remembering everything without
constant review.

``How many times are you going to read that old man? You'd think you'd
know it by now.''

``It's what you think you know that gets you in trouble.''

Alex's first task, as always, was starting a daily log notebook entry
and noting the powerpack charge levels and the box's interior
temperature and humidity. It was warm this morning, about two degrees
Celsius. Alex tracked their wake-up temperature with a spreadsheet. He
estimated they could tolerate outdoor temperatures of minus 200 degrees
Celsius. Below -200C, even the inner box would become too cold. One of
his spreadsheet trend lines estimated \emph{when} they'd hit -200C. He
looked at it frequently but didn't discuss their \emph{doom date} with
Doug.

Next, Alex turned on his laptop and checked the exterior webcam views.
The night vision view on the north-facing webcam showed snow had drifted
over the mine entrance again. They'd have to dig out again. The
thermometer on the north antenna pole always read -60 Fahrenheit. It was
too cold for the thermometer. Using copper spool resistance, he
estimated an outside temperature of -78 Celsius. Cold even for the
middle of old Antarctica. He didn't see anything on the south-facing
webcam. It was probably overcast. Satisfied with the webcam views, Alex
shut down his laptop.

While Alex checked the webcams, Doug listened to the radio. For the last
month, he had dutifully compared radio-reported temperatures to their
copper spool resistance estimates. They corresponded until about a week
before Earth crossed Mar's orbit. From then, the radio temperatures were
significantly warmer than their estimates.

The discrepancies infuriated Doug, ``Government shits won't even stop
lying now! What's the goddamn point?''

Alex didn't engage. His son didn't used to share his cynicism but being
lied to every day about easily checked facts --- the fricking
temperature --- was red-pilling him.

Satellite FM constantly advised people to stay home and wait for
openings in shelter tunnels. Mythical government tunnels being excavated
(trust us) as we speak. Between repeated admonishments to shelter in
place and weather reports, satellite FM filled its broadcast hours with
``interviews'' of carefree shelter dwellers discussing the good life in
the warm tunnels. Just obediently wait, and you will soon be warm and
safe. Everything was fine on satellite propaganda FM.

Things differed on the ham and pirate short-wave bands. Tens of millions
had already died of exposure, and people frantically begged for help.

While listening, they flipped on the duct fans. They always turned on
the duct fans before stoking up or relighting the stainless-steel stove.
It had proved necessary to use small, regularly spaced fans in the
ducts. The fans initially ran on batteries, but they had to rewire them
to run off the powerpacks. With the fans running, Alex fitted their
detachable flexible PVC air intake tube to a duct hole they had cut in
the box floor. The floor hole exposed a duct running under the box and
connecting to the 8-inch air intake duct, which ran to the mine's
exterior. To limit heat loss from pulling freezing outside air into the
box, the detachable intake tube was thickly wrapped with rockwool and
bubble wrap held in place with duct tape. Alex didn't like cutting
through the floor insulation layers to rig the air intake, but it had to
be done. When not in use, they plugged the floor hole with cut pool
noodles and covered it with thick rockwool and yoga mats. The PVC tube
attached to a metal T-joint hanging on the stove's air intake vent. With
practice, Doug and Alex learned how to burn small amounts of coal and
wood without filling the box with smoke. Once going, the flame burned
steadily and vented without problems.

After getting the stove going, they poured purified and filtered water
from large four-liter thermoses into smaller steel water bottles, which
they set on the stove to heat up. When the water in the bottles boiled,
Doug screwed on their caps and stored them in a drink cooler he had
upgraded by lining its interior with additional Styrofoam insulation.
The hot water bottles would remain hot in the jacked-up cooler for a
day. Doug used the water to make coffee. And, sometimes, he wrapped the
hot bottles in thick socks and used them to warm up sleeping bags.

As the box heated up, Alex checked the silica gel canisters. They had
indicators that turned pink when saturated. Pink meant it was time to
bake them to drive off water. Today, none of the canisters needed
baking.

Next, they ate breakfast. They did most of their cooking in the airlock
box, but they started the day with easily prepared breakfast foods.
Today, they stove-heated oatmeal, canned ham, and instant coffee
whitened with powdered milk. As they consumed food, Alex recorded the
items. Weeks ago, he had numbered every food item in their stores and
saved the numbers in a spreadsheet. He ticked them off daily; they would
probably freeze before starving at current consumption rates.

With breakfast done, they carefully licked the metal camping bowls and
cups clean, and then, using as little water as possible, they wiped them
down with rags, which they wrung over a pail. Some days, they boiled
rags in the airlock and hung them up to dry beside the warm exhaust vent
running along the mine shaft ceiling. Doing end-of-the-world dishes is
still a pain.

With breakfast done, they sorted trash. Easily burnable materials like
cardboard or wrapping paper went into one plastic garbage bag.
Unburnable items, like opened cans and glass bottles, went in another
bag. Nasty burnables, like plastic wrapping, went in a third bag. By
carefully licking plates and bowls, they almost eliminated food waste,
but sometimes, they couldn't hold their bladders and had to pee in a
large clear plastic jug. So far, they had managed to hold number two and
only took dumps in the ``garage'' on Doug's elegantly cut toilet lid. It
was necessary to keep the toilet box lid in the airlock and only use it
after warming it up. With garage temperatures around -50C, your bare ass
would stick to the freezing lid otherwise.

Every time Alex exposed his butt to freezing air, he thought of the old
racist joke. ``What do boogs want? Loose shoes, a tight pussy, and a
warm place to shit.'' He'd never appreciated the warm place to shit bit
until now.

After breakfast and trash sorting, Alex and Doug split up. Alex always
went to the airlock while Doug stayed in the inner box. They both needed
some time by themselves. Months of ``incarceration,'' as Doug called it,
wore on them. In their time apart, Alex would either set up his
astrograph outside the mine entrance and take CCD images of the sky or
spend his mornings melting ice and filtering water, cooking lunch, or
baking. Hot bread was their biggest treat. Alex experimented with baking
on the airlock stove but found using a propane camping stove worked
better. He didn't like burning the extra propane; they tried to use
propane only to run the generators and recharge the powerpacks, but the
hot bread always made them feel better.

Doug used his inner box alone time to sew. He patched tears in their
clothes and reworked blankets and comforters to augment their parkas,
boots, and mittens. Currently, he's making boots for boots. It was
already too cold for their Scheels winter boots. Their toes went numb if
they spent more than half an hour outside. By wrapping their boots in
thick layered recut nylon down comforter material with soles made of
rockwool wrapped in cut bedsheets, Doug hoped to keep their feet warm as
temperatures plunged.

To leave the box and give Doug some time alone, Alex squirmed into his
second base layer of red Santa Claus underwear, put on his graphene
jacket, and checked his jacket battery. He also plugged in the
astrograph electric blanket. To control electric devices, they ran
extension cords and other wires through the inner box's walls. The wire
port was carefully sealed with rockwool, Styrofoam, and duct tape. The
cables connected to propane generators, webcams, motion detectors, vent
fans, shaft lights, and the radio antenna. They could all be turned on
and off from banks of power bars.

``It's probably overcast out there. I'll dump our shit pail and get
ice.''

Getting out of the box was a chore. The front of the box was packed with
food items, primarily cans, which might be damaged by freezing. They
kept many cans in suitcases. Suitcases were easier to move around than
piles of loose cans; the remaining cans they put on rude shelves built
into the front of the box. They used Alex's wheeled carry-on as moveable
insulation. It contained nonperishable food; they rolled it up against
the box door when inside the box.

Alex pushed the carry-on out of the way and crawled through a narrow
opening to the small, heavily insulated door. Before opening the door,
Doug handed him today's trash bags, Alex's log notebook, the night
vision monocular, and a trail camera. What they hauled out of the box
varied from day to day. Some days, they took empty water thermoses or
jugs of pee, and on others, like today, trail cameras. They were down to
two trail cameras. One had stopped working when temperatures dropped
below -50C, and they had rewired the others to run off powerpacks. Also,
it was too cold to fidget with the cameras outdoors. It was easier to
take them inside and change memory cards when they warmed up.

``Don't forget the walkie-talkie old man. What's all that list checking
for?''

Doug handed him a recharged walkie-talkie. They used walkie-talkies for
a mine shaft intercom. Months ago, they'd argued about \emph{Maxwell
Caging} the mine's outer door. Alex wanted to cover the door with
aluminum foil to block stray walkie-talkie signals from ``escaping'' the
mine and alerting the deep state. This was too crazy for Doug.

``Are you hearing yourself? You're one step away from \emph{literal} tin
foil hats.''

Doug won the argument. From the inner box, walkie-talkie signals could
barely reach the catchment pond but no further. Still, Alex sparingly
used the walkie-talkies; he only sent short, necessary messages.

Alex put the walkie-talkie in one of his graphene jacket's pockets and
then turned off the motion detectors in the mine shaft. To exit, he
peeled back the Velcro seals on the door, pushed it open, and eased
today's items out of the box. Moving them aside, Alex crawled out of the
box onto a small porch of yoga mats. Doug pulled the door shut behind
him.

It was always a scramble to get out of the box and get into his parka,
ski pants, and winter boots hung outside. As he put on his outerwear, he
did calisthenics to warm himself and his freezing clothes. Eventually,
they would have to drag their outerwear into the box. After dressing, he
checked the thermometer mounted on a shaft support near the inner box.
This morning, it was -15 degrees Celsius: essentially tropical.

Today was a powerpack recharging day; Alex opened the Styrofoam-lined
plywood box beside the yoga mat porch they had built to house their
propane-powered generators and started one generator. The generators
were the only way they could charge their powerpacks. When the propane
ran out, their only light would be stove flames. The stoves had nice
flameproof glass portals; sometimes, they used stove light to conserve
powerpack/propane energy. They weren't looking forward to when the
propane ran out. At current consumption rates, they would be empty in
two years. Maybe three if they spent more time in the dark.

They ran one generator at a time, usually the one attached to a 30-pound
tank. The generator connected to a 100-pound tank was sparingly used.
They held it in reserve in case they had to stay in the mine for
extended periods. Both generators connected to duct vents and extension
cords running to the powerpacks.

Once the generator was humming, he went to the front of the mine,
stooping to avoid knocking his head on the shaft shoring and hanging
vent ducts. On the way, he took a quick census of their freeze-proof
foods. Fifty-pound bean bags were stacked on a narrow platform they made
from wood salvaged from Grampa's ruined house. They were getting sick of
beans.

Before reaching the airlock, he dumped today's unburnable trash in a
large garbage bag. He did the same for the nasty burnables. He left the
empty garbage bags on the bean bags to reuse later. Plastic garbage bags
are surprisingly durable. Four large metal pails of mine tailings and a
cooler filled with mined coal were lined up outside the airlock. It was
from yesterday's mining. After lunch, Alex and Doug worked together on
the coal seam.

Stepping around the tailing pails, he unzipped the Velcro seal on the
airlock's back door and went inside. He did the same for the front door.
With both doors open, he ferried the tailing pails through the airlock
and dumped their contents in the wheelbarrow. He then hauled the empty
pails back through the airlock for today's mining session. With the
pails sorted, he carried the cooler full of coal into the box and sealed
the doors. Then, using some of the newly mined coal, he lit the stove in
the front box.

When the flame burned smoothly, he placed one of the turkey baster pans
filled with pond ice on the stove. Their Walmart water was running low;
they now collected ice from the catchment pond, which they melted and
ran through water purification filters. They collected about two big
four-liter steel thermos of purified water daily. Alex tracked how much
water they used, like everything else.

With the ice melted and filtering into a large thermos, Alex prepared to
open the outer door. Before swinging the outer door open, he took a few
multimeter resistance readings. The temperature outside measured about
-78 degrees Celsius. Earlier, when he checked the webcams, it didn't
look windy, but he'd been fooled before. He opened the front door of the
heat box and let warm air flow into the garage. After fifteen minutes,
he grabbed Doug's toilet lid, which they kept near the airlock stove.
The toilet lid felt toasty warm. He stepped into the garage, placed the
toilet lid on Doug's shit box, and defecated into a prepared pail inside
the box. Yesterday, he had shoveled a thin layer of mine waste and burnt
coal cinders into the pail. If they didn't do this, their crap would
freeze and stick to the metal pail. Banging frozen shit out of pails in
-80C weather is not recommended! Today, his stool appeared soft, and his
urine looked clear. As a treat, he used some of their dwindling supply
of toilet paper to wipe himself. Most days, they used printer paper.

With his business complete, he pulled up his double underwear and thick
ski pants, secured his face covering over a disposable face mask, and
then turned on the graphene jacket under his oversized parka; it was
amazingly warm, even in -80C weather. His last chore before opening the
mine was closing the front door of the airlock. This kept frigid air out
of the airlock. It also blocked airlock lights from spilling out. Alex
didn't want to be seen. As the graphene jacket warmed his core, he swung
the outer door up and tied it open.

A blast of cold assaulted him.

The entrance lay half buried in drifted snow. It was the same every
goddamn day. Before stepping outside, he dug the night vision monocular
out of his parka and scanned the view to the south for any lights.
Nothing appeared: no lights, no stars, nothing. The Earth was getting
closer and closer to Jupiter. At its closest approach, Jupiter's disk
would be visible to the naked eye. Jupiter now shone painfully bright in
the east, more brilliant than Venus used to be. When he couldn't see it,
the sky was overcast. Overcast skies sucked for sky-watching, but they
had a big plus on runaway Earth; they kept the air just a shade warmer.

Tonight, he wouldn't get out the telescope. On clear nights, he set up
the astrograph and took images of Jupiter and Saturn. By measuring their
apparent diameters with his laptop sky chart program, Alex could
estimate how far Earth was from each planet. Weeks ago, he had
calculated how close Earth would come to the outer planets and
discovered that it would fly so close to Saturn that the planet's disk
would span about two degrees: four times an old full Moon, and the
majestic rings even more. It would be awesome: almost worth freezing
everyone on Earth for.

As observing was out tonight, Alex called Doug on the walkie-talkie.
``You can unplug the astrograph. It's overcast.''

Electric blankets drained powerpacks; they carefully limited their use.
With the telescope blanket off, he turned to this morning's chores.
First up, shovel the frigging opening. They had \emph{inexplicably}
forgotten to get a snow shovel when apocalypse shopping. Luckily, super
cold, newly blown snow is always light and crystalline; it's easily
pushed aside with makeshift shovels. Setting his night vision preserving
red headlamp to its lowest setting (the dim red light would be enough to
see but too dim to attract attention), he used a plow hacked together
from plywood and two-by-fours, to push enough snow aside to finish the
job with their largest shovel.

Using the walkie-talkie again, he told Doug, ``I'm going out now to
switch cameras and get ice.''

Grabbing the trail camera, Alex followed a polypropylene guide rope to
the southern antenna pole. Guide ropes made finding things on dark,
moonless nights easy. They had three guide ropes. One went to the
antenna, another to the active latrine pit, and a third to the frozen
catchment pond.

At the antenna pole, he swapped trail cameras. The trail camera
redundantly pointed southeast, overlapping with the webcam view, but it
ran 24x7, while the webcam only worked when he plugged it into his
laptop.

With the camera secure, he followed the guide rope back to the garage,
put the frozen trail camera on the astrograph box, and then dragged
their makeshift ice sled, loaded with ice-breaking tools, to the frozen
solid catchment pond. Using a shovel, Alex exposed the surface of the
ice adjacent to where they had previously mined. Working quickly by the
dim red headlamp, he sledgehammered ice blocks out of the pond and
loaded them on the sled. When he had enough ice to fill two melting
pans, he dragged the sled back to the garage.

Before closing the outer mine door, Alex opened their toilet box and
fetched the shit pail. Grabbing a hammer, he went outside one more time
carrying the pail. Alex followed the guide rope to the active latrine
pit. Drifting snow had filled the pit again; they had given up shoveling
shit snow, so he dumped the pail and banged it to knock out residual
crap. Retreating to the mine's garage, he replaced the pail in the
toilet box and lined its bottom with coal cinders and crushed rock.
Making it ready for the next shit. This was not how Alex had imagined
the end of the world.

%\subsection{light show}\label{light-show}
\begin{center}\large\textbf{-- \emph{light show} --}\normalsize\end{center}

Twenty-five weeks after runaway, as Earth crossed the main asteroid
belt, the weather took another turn. For weeks before, relentless
streams of high-altitude clouds blocked the skies above the mine. It
annoyed Alex; he didn't appreciate his last observing session being
clouded out.

``Would it be too much to ask for a few clear nights before we all
die?''

Overcast skies came with ice-crystal snow. Too cold for regular
snowstorms, small amounts wafted down as microscopic ice splinters.
While dropping, the ice crystals scattered light, erecting
perpendicular, hazy glowing pillars. When Jupiter, now so bright it cast
shadows, rose in the east, a leading shaft of light bright enough to see
on the webcams heralded its rise. If Alex didn't see it, he didn't
bother heating the astrograph. Why waste precious powerpack and propane
energy?

During overcast nights and windy days, they listened to the radio.
Government satellite propaganda FM still reported global temperatures
but little else. They had given up gaslighting, opting for clandestine
silence. Short-wave stations still reported interesting developments,
like, for the first time in human history, the Nile River froze. People
in equatorial regions couldn't deal with -45 degrees Celsius cold. Their
houses lacked central heating and decent insulation. Public utilities,
like water systems, assumed warmth. Water mains broke, flooding streets
that the merciless cold turned into ice rinks. Many died of exposure and
now starvation. In the past, such misery triggered massive refugee
surges, but now, moving north or south only led to colder weather.
People tore cities apart to fuel millions of fires. Governments gave up
policing and focused their dwindling resources on protecting food
supplies. Troops surrounded granaries, rice depots, and corn silos; they
shot trespassers.

Ham radio stations started dropping off. In the first runaway months,
they found hundreds of hams but now had difficulty finding dozens. The
remaining hams either broadcast rambling stream of conscious monologues
or attempted to relay news and messages. Relaying emergency information
is a longstanding ham tradition, but this wasn't a typical emergency.
Nobody is coming to your aid. Everyone is in distress. Why even bother
asking for help? Alex gravitated to sky-watching hams while Doug
searched for sources of interesting news.

After weeks of overcast conditions, the skies cleared, and local
temperatures dropped to new lows. Copper spool resistance indicated -92
Celsius, colder than old Antarctica record lows.

Radio weather reports claimed the clearing mid-latitude skies resulted
from expanding polar ice packs. The northern pack had reached as far
south as Astoria, Oregon, and the southern pack marched up the coasts of
Chile and Argentina and approached South Africa and Australia. The rate
at which the pack ice spread surprised meteorologists. Their models had
it moving at about half the observed rate.

Alex couldn't resist smirking, ``Weren't these the same guys forecasting
global warming in tenths of a degree? I'm enjoying the settled science
vibe.''

Doug gave his dad a \emph{give-it-up} look, but Alex let it slide. After
half a year in the mine, they both had shaggy beards. Wasting fuel,
water, and time on shaving didn't make sense. Doug's ragged blond beard
and six-foot-five-inch stature made him look like Thor after a bender in
Valhalla. According to Doug, Alex's gray whiskers gave him a wanted
poster look. Their daily mining sessions had hardened their bodies. Coal
mining with simple hand tools is hard physical work. Doug had ripped
arms, and Alex's soft, pasty, middle-aged body was lean and tight. Both
were in better shape than ever and couldn't resist admiring themselves
in tiny first-aid kit mirrors.

As pack ice covered the oceans, the water entering the atmosphere
decreased. Seawater became the sole source of atmospheric moisture when
other sources like rivers, lakes, and forests froze solid. Soon, the ice
packs would meet at the equator, sealing the Earth's liquid water from
the air. As the planet continued to cool, any moisture remaining
airborne would freeze out, yielding cloud-free skies. Additionally, with
the Sun shrinking and pumping less and less energy into the atmosphere
the winds would also quiet down. Earth would soon have perpetual dark,
clear, windless skies: an amateur astronomers paradise.

With the clouds gone, Alex's nights became superbly dark and clear. He
looked forward to crawling out of the box. Even Doug, on occasion, would
join him observing. The Moon, Alex's lifetime deep sky nemesis, no
longer tormented him, and as the Earth raced out of the inner solar
system, auroras also diminished. Increased distance diluted charged
particles from the Sun before they hit Earth's magnetosphere. It would
be interesting to see how much longer auroras would be visible. Night
after night, Alex witnessed the best dark skies he had ever seen. Human
light pollution had vanished. Utility grids couldn't take extreme cold.
Most failed, blacking out vast swathes of the countryside. Being too
cold to fix electrical grids, when the lights went out, they stayed out.

It was lovely except for the punishing cold. At current temperatures
their best cold weather gear limited their time outside to a few hours.
This interval would shrink as temperatures dropped, and they could do
little to cope. They had already beefed up their boots and mittens.
Doug's boots in boots kept their feet warm, and they both wore his
altered fitted nylon comforters over their thick parkas. Alex spent the
most time outside the mine. The electric graphene jacket still worked
beautifully under his oversized parka. On some observing nights, he
almost forgot about the cold, but they both knew their outside time
would continue to shrink.

Two monastery mornings after recording a new low of -94 Celsius, Alex's
clock woke him, and he robotically worked through his morning checklist.
Nothing caught his attention until he plugged the webcams into his
laptop. The north and south cam views dazzled with blazing spears of
light. It must be a bright meteor shower!

He shook Doug awake. Doug worked harder at the coal seam than Alex and
was often exhausted at the end of their days. Alex let him sleep in
until he left the box. They had agreed to always keep in touch when
doing dangerous things, like going outside or working the seam.

``Wake up, look at this!''

Annoyed at first, Doug turned and gave Alex a big grin after looking at
the laptop screen.

``Damn!''

Rushing through their chores, they crawled out of the inner box, put on
their outerwear, and hurried to the airlock. They didn't bother lighting
the stove. Instead, they squeezed through the airlock and sealed the
doors. In the garage, they pulled up and tied the outer mine door open.

Intense meteors filled the entire sky, flashing from the east like line
lighting and arc strobing the landscape. Timing streaks with his diving
watch, Alex estimated they fell at about two hundred per minute. Some
glowed painfully bright, leaving afterimages burnt in your eyes, and he
swore he could hear some.

``Did you hear that?'' he asked Doug.

``Yeah. How?''

``Sonic booms. They're all traveling much faster than sound. Must be a
Mach cone background.''

The spectacle seized their attention until the bitter cold interrupted.
Retreating to the front box, they fetched the electric blanket draped
around the astrograph. The blanket felt warm. Alex had flipped it on
before leaving the box. Wrapping themselves in the blanket, they
returned outside and huddled under the meteor-sliced skies. This show
was worth draining a powerpack. The electric blanket provided enough
warmth to endure another hour. The shower tapered off. Streak counts
dropped to a dozen per minute, but the few coming were just as bright.
Then, like a staged fireworks grand finale, a huge, exceptionally bright
meteor flashed from the east. It blinded them and grew brighter as it
raced over them. Its white glare blasted the entire snow-covered valley.
Seconds later, it detonated in the west with a blast so bright it looked
like staring into the old Sun.

Stunned, Alex forgot to count down. Soon, an eardrum-shattering
shockwave punched through the cold air.

``What the hell, Dad?''

``A large bolide, a meteor exploding in the atmosphere. Cool!''

After the blast, they continued watching, enduring the cold as long as
possible, but the -90C air drove them back into the mine. Before
crawling into the box, they started up a generator. Their time outside
under an electric blanket had depleted a powerpack. Neither of them had
the energy to screw around lighting finicky coal fires. So, they turned
on Doug's little electric heater hanging over the tunnel tent. They only
ran the heater while the generators were going. Feeling celebratory,
they rummaged through their supplies and picked out some favorites. Alex
had some canned pork and trail mix bars, and Doug ate his prized canned
cheese and freeze-dried apricots. They washed everything down with
canned cider and beer.

As Alex sipped his cider, he reiterated, ``Good call picking up booze;
dry apocalypses are the worst.''

``You know, until the world ended, I didn't know canned cheese was a
thing.''

``You're getting some \emph{queso} on your mustache, Thor.''

Finishing their cold dinner, they made hot cocoa with a stashed hot
bottle and powdered milk and then snuggled into their sleeping bags at
midnight (solar noon) and listened to the radio.

The meteor shower put on a global show. Even propaganda satellite FM
spent time on it. There were bolide airbursts worldwide and some surface
impacts, with the biggest taking out half a city block in downtown Lima.
The Lima meteorite measured only three or four meters in diameter but
slammed into the city at eighty thousand kilometers per hour. Do the
kinetic energy math. Kaboom!

Sky-watching radio hams weren't surprised by the shower: many expected
showers. After all, the Earth was crossing the main asteroid belt. Even
God Radio had learned a few astronomical tricks. God Radio prophesied
the Earth would slam into a big belt asteroid. Freezing our asses wasn't
enough. God was going to pound our bungholes like a fisting dominatrix
with a baseball bat. Yeah, God Radio still wasn't pussyfooting around.

A few wondered why so many meteors flashed through Earth's skies when a
dozen spacecraft had safely navigated the main asteroid belt without
hitting anything. Others pointed out the big difference between the
Earth and tiny spacecraft. A few hundred meteors per minute over a
terrestrial hemisphere needs only a few randomly scattered
baseball-sized asteroids every ten kilometers. A tiny Voyager or a
larger Juno spacecraft would likely traverse such a region of space
without hitting anything, while the Earth would plow into thousands. The
collision statistics resemble the Battleship game played on vast grids.

Meteor showers continued sporadically for the next three weeks but
feebly gave way to pure dark skies, with the main attraction being
Jupiter. Their naked eyes resolved Jupiter's disk and the Galilean
Moons. In binoculars, Jupiter's equatorial and mid-latitude cloud belts
prominently stood out. His astrograph digital images looked like space
telescope shots. If it wasn't for the intense cold, now sometimes
dipping below -100C, Alex would have snapped hundreds of Jupiter images
and stacked the best of them with his laptop astro imaging software to
produce legitimate masterpieces, but he couldn't stay outside long
enough. Still, the few shots he processed were the best images of
Jupiter he had ever taken.

``You're really good at this, Dad.''

And as the Earth approached, the pictures kept getting better.

Thirty-eight weeks after runaway, the Earth made its closest approach to
Jupiter. Jupiter's disk gleamed as a tiny, banded ball of brilliance to
the naked eye. Astrograph images were spectacular! Only orbiting
spacecraft shots were better. Alex was so grateful Doug was with him.
Sharing his superb Jupiter images with his son almost made him cry. He
wanted to share them on social media. A year ago, he did. It didn't seem
important then, but now, nobody would ever see or like anything he ever
did again.

At forty-two weeks, Earth crossed Jupiter's orbit. Crossing revealed
another never-before-seen spectacle, Jupiter as a crescent in Earth's
skies. For all of history, Jupiter has been an outer planet, always a
disk in our scopes. Now, Jupiter glowed like a gigantic Venus, a new
inner planet. It would fade as Earth moved further and further away.
Eventually, the great gas giant would be a faint star, then nothing to
the eye. No more king of the gods in our skies. No more observers on
Earth to care.

Passing Jupiter marked a Rubicon. With the great giant shrinking in
Earth's skies, whatever hope survivors on runaway Earth harbored
vanished; a year of relentless, ever-accelerating cooling had rendered
the planet unrecognizable. There would be no return to normal. Warm
winds would never blow on Earth again. Global food stores were near
exhaustion, with no way to replenish them. They heard some of this on
the radio but inferred the rest. Radio stations of all types, including
government, short-wave, and ham stations, started disappearing. On some
days, they only heard static. A few hams started broadcasting repeating
messages in Morse code. They decoded some of these messages; many
pleaded for help or served as radio memorials. As station operators
died, some rigged their transmitters to broadcast coded epitaphs. The
messages cycled until the power failed.

Life kept getting harder and harder. Mineshaft air temperatures kept
falling as the residual heat in the surrounding rock bled away. The mine
wasn't deep enough to benefit from the Earth's heat. Deep mines, a
kilometer or deeper, would be toasty and warm, but the rock around them
only served as a thick layer of insulation. It would eventually reach
thermal equilibrium with the outside. Outside temperatures hung around
-97 Celsius and continued to fall. Soon, the cold would penetrate the
mine. He figured in two years, it would be -90C in the mine and -160C
outside. It's unlikely they could survive such temperatures.

To trap as much stove heat as possible they rerouted the exhaust vent of
the airlock stove to loop back into the mine shaft before turning and
venting outside. Using more aluminum foil, they wrapped the looped
exhaust and intake pipes. The foil functioned as a radiator, bleeding
exhaust heat into the shaft. It also helped warm intake air. In a final
touch, they mounted their remaining vent fans outside the pipes. The
fans directed cold shaft air over the warm exhaust pipe and pushed the
slightly warmed air deeper into the mine. If they maintained fires in
both stoves for twelve hours, their vent hack warmed shaft air about 10
degrees Celsius. Instead of working in -30C temperatures, they worked in
-20C. Burning two fires for longer periods required more and more
mining. Every day, Alex logged their coal consumption in a spreadsheet.
In six months, mining every waking hour wouldn't be enough.

``It was not sustainable, Duh!''


%\end{document}
 

