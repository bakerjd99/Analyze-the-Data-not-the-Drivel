%% uncomment to list all files in log
%\listfiles

\documentclass[12pt]{report}

\usepackage{fontspec}

%\setmainfont[Scale=MatchLowercase]{Lucida Bright}
%\setmonofont{FreeMono}
%\setmonofont{Source Code Pro}
\setmonofont[Scale=MatchLowercase]{Ubuntu Mono}

\usepackage[headings]{fullpage}

% national use characters 
%\usepackage{inputenc}

% ams mathematical symbols
\usepackage{amsmath,amssymb}

% added to support pandoc highlighting
\usepackage{microtype}

\usepackage{makeidx}

% add index and bibliographies to table of contents
\usepackage[nottoc]{tocbibind}

% postscript courier and times in place of cm fonts
%\usepackage{courier}
%\usepackage{times}

% extended coloring
\usepackage{color}
\usepackage[table,dvipsnames]{xcolor}
\usepackage{colortbl}

% advanced date formating
\usepackage{datetime}

%support pandoc code highlighting
\usepackage{fancyvrb}
\DefineShortVerb[commandchars=\\\{\}]{\|}
\DefineVerbatimEnvironment{Highlighting}{Verbatim}{commandchars=\\\{\}}
% Add ',fontsize=\small' for more characters per line

%tango style colors
% \usepackage{framed}
% \definecolor{shadecolor}{RGB}{255,255,255}
% \newenvironment{Shaded}{\begin{snugshade}}{\end{snugshade}}
% \newcommand{\KeywordTok}[1]{\textcolor[rgb]{0.13,0.29,0.53}{\textbf{{#1}}}}
% \newcommand{\DataTypeTok}[1]{\textcolor[rgb]{0.13,0.29,0.53}{{#1}}}
% \newcommand{\DecValTok}[1]{\textcolor[rgb]{0.00,0.00,0.81}{{#1}}}
% \newcommand{\BaseNTok}[1]{\textcolor[rgb]{0.00,0.00,0.81}{{#1}}}
% \newcommand{\FloatTok}[1]{\textcolor[rgb]{0.00,0.00,0.81}{{#1}}}
% \newcommand{\CharTok}[1]{\textcolor[rgb]{0.31,0.60,0.02}{{#1}}}
% \newcommand{\StringTok}[1]{\textcolor[rgb]{0.31,0.60,0.02}{{#1}}}
% \newcommand{\CommentTok}[1]{\textcolor[rgb]{0.56,0.35,0.01}{\textit{{#1}}}}
% \newcommand{\OtherTok}[1]{\textcolor[rgb]{0.56,0.35,0.01}{{#1}}}
% \newcommand{\AlertTok}[1]{\textcolor[rgb]{0.94,0.16,0.16}{{#1}}}
% \newcommand{\FunctionTok}[1]{\textcolor[rgb]{0.00,0.00,0.00}{{#1}}}
% \newcommand{\RegionMarkerTok}[1]{{#1}}
% \newcommand{\ErrorTok}[1]{\textbf{{#1}}}
% \newcommand{\NormalTok}[1]{{#1}}

%espresso style colors
% \usepackage{framed}
% \definecolor{shadecolor}{RGB}{42,33,28}
% \newenvironment{Shaded}{\begin{snugshade}}{\end{snugshade}}
% \newcommand{\KeywordTok}[1]{\textcolor[rgb]{0.26,0.66,0.93}{\textbf{{#1}}}}
% \newcommand{\DataTypeTok}[1]{\textcolor[rgb]{0.74,0.68,0.62}{\underline{{#1}}}}
% \newcommand{\DecValTok}[1]{\textcolor[rgb]{0.27,0.67,0.26}{{#1}}}
% \newcommand{\BaseNTok}[1]{\textcolor[rgb]{0.27,0.67,0.26}{{#1}}}
% \newcommand{\FloatTok}[1]{\textcolor[rgb]{0.27,0.67,0.26}{{#1}}}
% \newcommand{\CharTok}[1]{\textcolor[rgb]{0.02,0.61,0.04}{{#1}}}
% \newcommand{\StringTok}[1]{\textcolor[rgb]{0.02,0.61,0.04}{{#1}}}
% \newcommand{\CommentTok}[1]{\textcolor[rgb]{0.00,0.40,1.00}{\textit{{#1}}}}
% \newcommand{\OtherTok}[1]{\textcolor[rgb]{0.74,0.68,0.62}{{#1}}}
% \newcommand{\AlertTok}[1]{\textcolor[rgb]{1.00,1.00,0.00}{{#1}}}
% \newcommand{\FunctionTok}[1]{\textcolor[rgb]{1.00,0.58,0.35}{\textbf{{#1}}}}
% \newcommand{\RegionMarkerTok}[1]{\textcolor[rgb]{0.74,0.68,0.62}{{#1}}}
% \newcommand{\ErrorTok}[1]{\textcolor[rgb]{0.74,0.68,0.62}{\textbf{{#1}}}}
% \newcommand{\NormalTok}[1]{\textcolor[rgb]{0.74,0.68,0.62}{{#1}}}

%kete style colors
% \newenvironment{Shaded}{}{}
% \newcommand{\KeywordTok}[1]{\textbf{{#1}}}
% \newcommand{\DataTypeTok}[1]{\textcolor[rgb]{0.50,0.00,0.00}{{#1}}}
% \newcommand{\DecValTok}[1]{\textcolor[rgb]{0.00,0.00,1.00}{{#1}}}
% \newcommand{\BaseNTok}[1]{\textcolor[rgb]{0.00,0.00,1.00}{{#1}}}
% \newcommand{\FloatTok}[1]{\textcolor[rgb]{0.50,0.00,0.50}{{#1}}}
% \newcommand{\CharTok}[1]{\textcolor[rgb]{1.00,0.00,1.00}{{#1}}}
% \newcommand{\StringTok}[1]{\textcolor[rgb]{0.87,0.00,0.00}{{#1}}}
% \newcommand{\CommentTok}[1]{\textcolor[rgb]{0.50,0.50,0.50}{\textit{{#1}}}}
% \newcommand{\OtherTok}[1]{{#1}}
% \newcommand{\AlertTok}[1]{\textcolor[rgb]{0.00,1.00,0.00}{\textbf{{#1}}}}
% \newcommand{\FunctionTok}[1]{\textcolor[rgb]{0.00,0.00,0.50}{{#1}}}
% \newcommand{\RegionMarkerTok}[1]{{#1}}
% \newcommand{\ErrorTok}[1]{\textcolor[rgb]{1.00,0.00,0.00}{\textbf{{#1}}}}
% \newcommand{\NormalTok}[1]{{#1}}
%end pandoc code hacks

% jodliterate colors
\usepackage{color}
\definecolor{shadecolor}{RGB}{248,248,248}
% j control structures 
\definecolor{keywcolor}{rgb}{0.13,0.29,0.53}
% j explicit arguments x y m n u v
\definecolor{datacolor}{rgb}{0.13,0.29,0.53}
% j numbers - all types see j.xml
\definecolor{decvcolor}{rgb}{0.00,0.00,0.81}
\definecolor{basencolor}{rgb}{0.00,0.00,0.81}
\definecolor{floatcolor}{rgb}{0.00,0.00,0.81}
% j local assignments
\definecolor{charcolor}{rgb}{0.31,0.60,0.02}
\definecolor{stringcolor}{rgb}{0.31,0.60,0.02}
\definecolor{commentcolor}{rgb}{0.56,0.35,0.01}
% primitive adverbs and conjunctions
%\definecolor{othercolor}{rgb}{0.56,0.35,0.01}   
\definecolor{othercolor}{RGB}{0,0,255}
% global assignments
\definecolor{alertcolor}{rgb}{0.94,0.16,0.16}
% primitive J verbs and noun names
\definecolor{funccolor}{rgb}{0.00,0.00,0.00}    

\usepackage{framed}
\newenvironment{Shaded}{}{}
\newcommand{\KeywordTok}[1]{\textcolor{keywcolor}{\textbf{{#1}}}}
\newcommand{\DataTypeTok}[1]{\textcolor{datacolor}{{#1}}}
%\newcommand{\DecValTok}[1]{\textcolor{decvcolor}{{#1}}}
\newcommand{\DecValTok}[1]{{#1}} 
\newcommand{\BaseNTok}[1]{\textcolor{basencolor}{{#1}}}
\newcommand{\FloatTok}[1]{\textcolor{floatcolor}{{#1}}}
\newcommand{\CharTok}[1]{\textcolor{charcolor}{\textbf{{#1}}}}
\newcommand{\StringTok}[1]{\textcolor{stringcolor}{{#1}}}
\newcommand{\CommentTok}[1]{\textcolor{commentcolor}{\textit{{#1}}}}
\newcommand{\OtherTok}[1]{\textcolor{othercolor}{{#1}}} 
\newcommand{\AlertTok}[1]{\textcolor{alertcolor}{\textbf{{#1}}}}
%\newcommand{\FunctionTok}[1]{\textcolor{funccolor}{{#1}}}
\newcommand{\FunctionTok}[1]{{#1}}
\newcommand{\RegionMarkerTok}[1]{{#1}}
\newcommand{\ErrorTok}[1]{\textbf{{#1}}}
\newcommand{\NormalTok}[1]{{#1}}

% headers and footers
\usepackage{fancyhdr}
\pagestyle{fancy}

\fancyhead{}
\fancyfoot{}

%\fancyhead[LE,RO]{\slshape \rightmark}
%\fancyhead[LO,RE]{\slshape \leftmark}
\fancyfoot[C]{\thepage}
%\headrulewidth 0.4pt
%\footrulewidth 0 pt

%\addtolength{\headheight}{\baselineskip}

%\lfoot{\emph{Analyze the Data not the Drivel}}
%\rfoot{\emph{\today}}

% subfigure handles figures that contain subfigures
%\usepackage{color,graphicx,subfigure,sidecap}
\usepackage{graphicx,sidecap}
\usepackage{subfigure}
\graphicspath{{./inclusions/}}

% floatflt provides for text wrapping around small figures and tables
\usepackage{floatflt}

% tweak caption formats 
\usepackage{caption} 
\usepackage{sidecap}
%\usepackage{subcaption} % not compatible with subfigure

\usepackage{rotating} % flip tables sideways

% complex footnotes
%\usepackage{bigfoot}

% weird logos \XeLaTeX
\usepackage{metalogo}

% source code listings
\usepackage{listings}

% long tables
% \usepackage{longtable}

\newcommand{\HRule}{\rule{\linewidth}{0.5mm}}

% map LaTeX cross references into PDF cross references
\usepackage[
            %dvips,
            colorlinks,
            linkcolor=blue,
            citecolor=blue,
            urlcolor=blue,   % magenta, cyan default        
            pdfauthor={John D. Baker},
            pdftitle={Analyze the Data not the Drivel},
            pdfsubject={Blog},
            pdfcreator={MikTeX+LaTeXe with hyperref package},
            pdfkeywords={blog,wordpress},
            ]{hyperref}
           
% custom colors
\definecolor{CodeBackGround}{cmyk}{0.0,0.0,0,0.05}    % light gray
\definecolor{CodeComment}{rgb}{0,0.50,0.00}           % dark green {0,0.45,0.08}
\definecolor{TableStripes}{gray}{0.9}                 % odd/even background in tables

\lstdefinelanguage{bat}
{morekeywords={echo,title,pushd,popd,setlocal,endlocal,off,if,not,exist,set,goto,pause},
sensitive=True,
morecomment=[l]{rem}
}

\lstdefinelanguage{jdoc}
{
morekeywords={},
otherkeywords={assert.,break.,continue.,for.,do.,if.,else.,elseif.,return.,select.,end.
,while.,whilst.,throw.,catch.,catchd.,catcht.,try.,case.,fcase.},
sensitive=True,
morecomment=[l]{NB.},
morestring=[b]',
morestring=[d]',
}

% latex size ordering - can never remember it
% \tiny
% \scriptsize
% \footnotesize
% \small
% \normalsize
% \large
% \Large
% \LARGE
% \huge
% \Huge
 
% listings package settings  
\lstset{%
  language=jdoc,                                % j document settings
  basicstyle=\ttfamily\footnotesize,            
  keywordstyle=\bfseries\color{keywcolor}\footnotesize,
  identifierstyle=\color{black},
  commentstyle=\slshape\color{CodeComment},     % colored slanted comments
  stringstyle=\color{red}\ttfamily,
  showstringspaces=false,                       
  %backgroundcolor=\color{CodeBackGround},       
  frame=single,                                
  framesep=1pt,                                 
  framerule=0.8pt,                             
  rulecolor=\color{CodeBackGround},   
  showspaces=false,
  %columns=fullflexible,
  %numbers=left,
  %numberstyle=\footnotesize,
  %numbersep=9pt,
  tabsize=2,
  showtabs=false,
  captionpos=b
  breaklines=true,                              
  breakindent=5pt                              
}

\lstdefinelanguage{JavaScript}{
  keywords={typeof, new, true, false, catch, function, return, null, catch, switch, var, if, in, while, do, else, case, break},
  ndkeywords={class, export, boolean, throw, implements, import, this},
  ndkeywordstyle=\color{darkgray}\bfseries,
  sensitive=false,
  comment=[l]{//},
  morecomment=[s]{/*}{*/},
  morestring=[b]',
  morestring=[b]"
}

% C# settings
\lstdefinestyle{sharpc}{
language=[Sharp]C,
basicstyle=\ttfamily\scriptsize, 
keywordstyle=\bfseries\color{keywcolor}\scriptsize,
framerule=0pt
}

% for source code listing longer than two use smaller font
\lstdefinestyle{smallersource}{
basicstyle=\ttfamily\scriptsize, 
keywordstyle=\bfseries\color{keywcolor}\scriptsize,
framerule=0pt
}

\lstdefinestyle{resetdefaults}{
language=jdoc,
basicstyle=\ttfamily\footnotesize,  
keywordstyle=\bfseries\color{keywcolor}\footnotesize,                                                               
framerule=0.8pt 
}

% APL UTF8 code points listed for lstlisting processing
\makeatletter
\lst@InputCatcodes
\def\lst@DefEC{%
 \lst@CCECUse \lst@ProcessLetter
  ^^80^^81^^82^^83^^84^^85^^86^^87^^88^^89^^8a^^8b^^8c^^8d^^8e^^8f%
  ^^90^^91^^92^^93^^94^^95^^96^^97^^98^^99^^9a^^9b^^9c^^9d^^9e^^9f%
  ^^a0^^a1^^a2^^a3^^a4^^a5^^a6^^a7^^a8^^a9^^aa^^ab^^ac^^ad^^ae^^af%
  ^^b0^^b1^^b2^^b3^^b4^^b5^^b6^^b7^^b8^^b9^^ba^^bb^^bc^^bd^^be^^bf%
  ^^c0^^c1^^c2^^c3^^c4^^c5^^c6^^c7^^c8^^c9^^ca^^cb^^cc^^cd^^ce^^cf%
  ^^d0^^d1^^d2^^d3^^d4^^d5^^d6^^d7^^d8^^d9^^da^^db^^dc^^dd^^de^^df%
  ^^e0^^e1^^e2^^e3^^e4^^e5^^e6^^e7^^e8^^e9^^ea^^eb^^ec^^ed^^ee^^ef%
  ^^f0^^f1^^f2^^f3^^f4^^f5^^f6^^f7^^f8^^f9^^fa^^fb^^fc^^fd^^fe^^ff%
  ^^^^20ac^^^^0153^^^^0152%
  ^^^^20a7^^^^2190^^^^2191^^^^2192^^^^2193^^^^2206^^^^2207^^^^220a%
  ^^^^2218^^^^2228^^^^2229^^^^222a^^^^2235^^^^223c^^^^2260^^^^2261%
  ^^^^2262^^^^2264^^^^2265^^^^2282^^^^2283^^^^2296^^^^22a2^^^^22a3%
  ^^^^22a4^^^^22a5^^^^22c4^^^^2308^^^^230a^^^^2336^^^^2337^^^^2339%
  ^^^^233b^^^^233d^^^^233f^^^^2340^^^^2342^^^^2347^^^^2348^^^^2349%
  ^^^^234b^^^^234e^^^^2350^^^^2352^^^^2355^^^^2357^^^^2359^^^^235d%
  ^^^^235e^^^^235f^^^^2361^^^^2362^^^^2363^^^^2364^^^^2365^^^^2368%
  ^^^^236a^^^^236b^^^^236c^^^^2371^^^^2372^^^^2373^^^^2374^^^^2375%
  ^^^^2377^^^^2378^^^^237a^^^^2395^^^^25af^^^^25ca^^^^25cb%  
  ^^00}
\lst@RestoreCatcodes
\makeatother

% custom lengths used within minipages
\newcommand{\minindent}{17pt}


\makeindex

\begin{document}

\subsection*{\href{https://bakerjd99.wordpress.com/2014/09/01/pandoras-star-a-grand-sprawling-entertainment/}{Pandora's Star: a Grand Sprawling Entertainment}}
\addcontentsline{toc}{subsection}{Pandora's Star: a Grand Sprawling Entertainment}


\noindent\emph{Posted: 01 Sep 2014 21:00:56}
\vspace{6pt}

%\href{http://www.barnesandnoble.com/w/pandoras-star-peter-f-hamilton/1100293432?ean=9780345479211}{\includegraphics{pandoras-star.jpg}}

\captionsetup[floatingfigure]{labelformat=empty}
\begin{floatingfigure}[l]{0.24\textwidth}
\centering
\href{http://www.barnesandnoble.com/w/pandoras-star-peter-f-hamilton/1100293432?ean=9780345479211}{\includegraphics[width=0.22\textwidth]{pandoras-star.jpg}}
\label{fig:4817X0}
\end{floatingfigure} In
my fevered youth I was an avid fan of science fiction but as I crossed
the Rubicon of middle age I read less and less of the genre. For years I
preferred nonfiction: mostly science with a smattering of history and
biography. Then, about five years ago, I started reading science fiction
again.

What kept me away? Most of the authors of my youth had died: Heinlein,
Clarke, Asimov, Bradbury, Anderson, Herbert and Dick -- all gone! I had
to find new -- to me -- authors. I knew and loved Neal Stephenson, the
author of \emph{Cryptonomicon, Diamond Age, Snow Crash}, and
\emph{Anathem}, but after four or five thousand pages of Neal it was
time to move on. My first post Stephenson, new to me, author was Iain M.
Banks.

Banks specialized in what's often called alien infested space opera. His
universes are overflowing with life. Aliens are everywhere, inhabiting
niches that most biologists would poo-poo as impossible. I prefer more
empty and serene universes but Banks' books like \emph{The Algebraist},
\emph{Surface Detail}, and \emph{Consider Phelbas} whet my appetite for
his crowded milieus. I was looking forward to following Banks for years
but it wasn't to be. Iain M. Banks died of cancer, at the ridiculously
premature age of 59, leaving fans all over the word wanting. There is no
greater outrage than mortality!

After Banks' death I looked around for other operatic authors; it didn't
take me long to find \href{http://www.peterfhamilton.co.uk/}{Peter F.
Hamilton} and
\href{http://www.barnesandnoble.com/w/pandoras-star-peter-f-hamilton/1100293432?ean=9780345479211}{\emph{Pandora's
Star}}. \emph{Pandora's Star} is a huge, highly entertaining, example of
what I call \emph{restrained} science fiction.

Restraint is what separates science fiction from fantasy. Fantasy
tolerates an anything goes mishmash of logical inconsistencies.
Literature has a term for this: \emph{Deus ex machina.} Modern fantasy is
a veritable high-tech \emph{Deux ex machina} factory churning out
\href{http://theredpillroom.blogspot.com/p/masculexicon.html}{beta-male}
vampires that take implausible romantic interests in their food,
prepubescent wizards jerking off in boarding school, (Oh it happened),
fireproof maximum babes with pet dragons, and armies of oxymoronic
brain-dead brain eating zombies. \emph{Only scripture piles on more
logical nonsense than fantasy.}

I enjoy fantasy as much as the next nerd but it's not science fiction.
Proper restrained science fiction admits a small number of ``magic
suppositions'' but otherwise rigorously adheres to what we know about
physical reality. You need some damn science in your science fiction
people. The universe of \emph{Pandora's Star~}presumes a few
impossibilities; it assumes wormholes and faster than light (FTL)
travel. FTL is a standard plot enabling device. Civilizations spanning
thousands of light years simply cannot exist, on human time scales,
without it. \emph{Pandora's Star} makes three more ``impossible''
assumptions which I will not divulge because ruining good books should
be a capital crime. Aside from these allowed departures from reality the
universe of \emph{Pandora's Star} sticks to scientific bricks and unfolds
with lovely consistency.

Most science fiction writers make impossible assumptions but great ones
take them in unexpected directions. Consider wormholes. Wormholes have
been a staple of science fiction forever. Three, not entirely
restrained, \href{http://stargate.mgm.com/}{TV series} had contemporary
soldiers marching through them every week for years. They've popped up
in every two-bit tale that needed quick point A to B plumbing. Wormholes
are a cliché and their presence often signals unimaginative hackery. If
you're going to confront me with wormholes you better damn well show me
something new or I'm outta your lame book. The opening chapter of
\emph{Pandora's Star} is one of the most humorous and imaginative use of
wormholes in science fiction. A few pages later Hamilton sends trains
through wormholes.
It's~\href{http://www.cbs.com/shows/big\_bang\_theory/}{Sheldon
Cooper's} wet dream: \emph{trains in space.} I had to smile and keep on
reading. \emph{Pandora's Star} is a big book, almost one thousand pages,
but like all great sprawling books it's too damn short. Fortunately,
there's a second book, \emph{Judas Unchained,~} that keeps the story
rolling. I haven't had this much fun with science fiction since
\emph{Dune}. It's that good.

%\captionsetup[floatingfigure]{labelformat=empty}
%\begin{figure}[htbp]
%\begin{floatingfigure}[l]{0.25\textwidth}
%\centering
%\includegraphics[width=0.23\textwidth]{pandoras-star-e1409603743667.jpg}
%\caption{~~~IMCAPTION~~~}
%\label{fig:4817X0}
%\end{floatingfigure}
%\end{figure}



%\end{document}