%% bm.pdf preamble - material merged from previous preamble and current pandoc preamable output
% NOTE: float placement required changes to the source files referenced by bm.tex
% May 28, 2020
%
% Use lualatex to compile - test with MiKTeX 2.9

% uncomment to list all files in log
%\listfiles

\documentclass[12pt]{report}


\usepackage{fontspec}

%\setmainfont[Scale=MatchLowercase]{Lucida Bright}
%\setmonofont{FreeMono}
%\setmonofont{Source Code Pro}
\setmonofont[Scale=MatchLowercase]{Ubuntu Mono}

% short snippets of asian languages
\newfontfamily\myAsian{Noto Serif TC Medium}

\usepackage[headings]{fullpage}

% national use characters 
%\usepackage{inputenc}

% ams mathematical symbols
\usepackage{amsmath,amssymb}

% added to support pandoc highlighting
\usepackage{microtype}

\usepackage{makeidx}

% add index and bibliographies to table of contents
\usepackage[nottoc]{tocbibind}

% postscript courier and times in place of cm fonts
%\usepackage{courier}
%\usepackage{times}

% extended coloring
\usepackage{color}
\usepackage[table,dvipsnames]{xcolor}
\usepackage{colortbl}

% advanced date formating
\usepackage{datetime}

%support pandoc code highlighting
\usepackage{fancyvrb}

% \DefineShortVerb[commandchars=\\\{\}]{\|}
% \DefineVerbatimEnvironment{Highlighting}{Verbatim}{commandchars=\\\{\}}
% % Add ',fontsize=\small' for more characters per line

% tango style colors
% \usepackage{framed}
% \definecolor{shadecolor}{RGB}{255,255,255}
% \newenvironment{Shaded}{\begin{snugshade}}{\end{snugshade}}
% \newcommand{\KeywordTok}[1]{\textcolor[rgb]{0.13,0.29,0.53}{\textbf{{#1}}}}
% \newcommand{\DataTypeTok}[1]{\textcolor[rgb]{0.13,0.29,0.53}{{#1}}}
% \newcommand{\DecValTok}[1]{\textcolor[rgb]{0.00,0.00,0.81}{{#1}}}
% \newcommand{\BaseNTok}[1]{\textcolor[rgb]{0.00,0.00,0.81}{{#1}}}
% \newcommand{\FloatTok}[1]{\textcolor[rgb]{0.00,0.00,0.81}{{#1}}}
% \newcommand{\CharTok}[1]{\textcolor[rgb]{0.31,0.60,0.02}{{#1}}}
% \newcommand{\StringTok}[1]{\textcolor[rgb]{0.31,0.60,0.02}{{#1}}}
% \newcommand{\CommentTok}[1]{\textcolor[rgb]{0.56,0.35,0.01}{\textit{{#1}}}}
% \newcommand{\OtherTok}[1]{\textcolor[rgb]{0.56,0.35,0.01}{{#1}}}
% \newcommand{\AlertTok}[1]{\textcolor[rgb]{0.94,0.16,0.16}{{#1}}}
% \newcommand{\FunctionTok}[1]{\textcolor[rgb]{0.00,0.00,0.00}{{#1}}}
% \newcommand{\RegionMarkerTok}[1]{{#1}}
% \newcommand{\ErrorTok}[1]{\textbf{{#1}}}
% \newcommand{\NormalTok}[1]{{#1}}

% %espresso style colors
% \usepackage{framed}
% \definecolor{shadecolor}{RGB}{42,33,28}
% \newenvironment{Shaded}{\begin{snugshade}}{\end{snugshade}}
% \newcommand{\KeywordTok}[1]{\textcolor[rgb]{0.26,0.66,0.93}{\textbf{{#1}}}}
% \newcommand{\DataTypeTok}[1]{\textcolor[rgb]{0.74,0.68,0.62}{\underline{{#1}}}}
% \newcommand{\DecValTok}[1]{\textcolor[rgb]{0.27,0.67,0.26}{{#1}}}
% \newcommand{\BaseNTok}[1]{\textcolor[rgb]{0.27,0.67,0.26}{{#1}}}
% \newcommand{\FloatTok}[1]{\textcolor[rgb]{0.27,0.67,0.26}{{#1}}}
% \newcommand{\CharTok}[1]{\textcolor[rgb]{0.02,0.61,0.04}{{#1}}}
% \newcommand{\StringTok}[1]{\textcolor[rgb]{0.02,0.61,0.04}{{#1}}}
% \newcommand{\CommentTok}[1]{\textcolor[rgb]{0.00,0.40,1.00}{\textit{{#1}}}}
% \newcommand{\OtherTok}[1]{\textcolor[rgb]{0.74,0.68,0.62}{{#1}}}
% \newcommand{\AlertTok}[1]{\textcolor[rgb]{1.00,1.00,0.00}{{#1}}}
% \newcommand{\FunctionTok}[1]{\textcolor[rgb]{1.00,0.58,0.35}{\textbf{{#1}}}}
% \newcommand{\RegionMarkerTok}[1]{\textcolor[rgb]{0.74,0.68,0.62}{{#1}}}
% \newcommand{\ErrorTok}[1]{\textcolor[rgb]{0.74,0.68,0.62}{\textbf{{#1}}}}
% \newcommand{\NormalTok}[1]{\textcolor[rgb]{0.74,0.68,0.62}{{#1}}}

% %kete style colors
% \newenvironment{Shaded}{}{}
% \newcommand{\KeywordTok}[1]{\textbf{{#1}}}
% \newcommand{\DataTypeTok}[1]{\textcolor[rgb]{0.50,0.00,0.00}{{#1}}}
% \newcommand{\DecValTok}[1]{\textcolor[rgb]{0.00,0.00,1.00}{{#1}}}
% \newcommand{\BaseNTok}[1]{\textcolor[rgb]{0.00,0.00,1.00}{{#1}}}
% \newcommand{\FloatTok}[1]{\textcolor[rgb]{0.50,0.00,0.50}{{#1}}}
% \newcommand{\CharTok}[1]{\textcolor[rgb]{1.00,0.00,1.00}{{#1}}}
% \newcommand{\StringTok}[1]{\textcolor[rgb]{0.87,0.00,0.00}{{#1}}}
% \newcommand{\CommentTok}[1]{\textcolor[rgb]{0.50,0.50,0.50}{\textit{{#1}}}}
% \newcommand{\OtherTok}[1]{{#1}}
% \newcommand{\AlertTok}[1]{\textcolor[rgb]{0.00,1.00,0.00}{\textbf{{#1}}}}
% \newcommand{\FunctionTok}[1]{\textcolor[rgb]{0.00,0.00,0.50}{{#1}}}
% \newcommand{\RegionMarkerTok}[1]{{#1}}
% \newcommand{\ErrorTok}[1]{\textcolor[rgb]{1.00,0.00,0.00}{\textbf{{#1}}}}
% \newcommand{\NormalTok}[1]{{#1}}
% %end pandoc code hacks

% jodliterate colors
\usepackage{color}
\definecolor{shadecolor}{RGB}{248,248,248}
% j control structures 
\definecolor{keywcolor}{rgb}{0.13,0.29,0.53}
% j explicit arguments x y m n u v
\definecolor{datacolor}{rgb}{0.13,0.29,0.53}
% j numbers - all types see j.xml
\definecolor{decvcolor}{rgb}{0.00,0.00,0.81}
\definecolor{basencolor}{rgb}{0.00,0.00,0.81}
\definecolor{floatcolor}{rgb}{0.00,0.00,0.81}
% j local assignments
\definecolor{charcolor}{rgb}{0.31,0.60,0.02}
\definecolor{stringcolor}{rgb}{0.31,0.60,0.02}
\definecolor{commentcolor}{rgb}{0.56,0.35,0.01}
% primitive adverbs and conjunctions
%\definecolor{othercolor}{rgb}{0.56,0.35,0.01}   
\definecolor{othercolor}{RGB}{0,0,255}
% global assignments
\definecolor{alertcolor}{rgb}{0.94,0.16,0.16}
% primitive J verbs and noun names
\definecolor{funccolor}{rgb}{0.00,0.00,0.00}

% custom colors
\definecolor{CodeBackGround}{cmyk}{0.0,0.0,0,0.05}    % light gray
\definecolor{CodeComment}{rgb}{0,0.50,0.00}           % dark green {0,0.45,0.08}
\definecolor{TableStripes}{gray}{0.9}                 % odd/even background in tables

% Colors for the hyperref package
\definecolor{urlcolor}{rgb}{0,.145,.698}
\definecolor{linkcolor}{rgb}{.71,0.21,0.01}
\definecolor{citecolor}{rgb}{.12,.54,.11}

% % Exact colors from NB
\definecolor{incolor}{HTML}{303F9F}
\definecolor{outcolor}{HTML}{D84315}
\definecolor{cellborder}{HTML}{CFCFCF}
\definecolor{cellbackground}{HTML}{F7F7F7}

% % ANSI colors
\definecolor{ansi-black}{HTML}{3E424D}
\definecolor{ansi-black-intense}{HTML}{282C36}
\definecolor{ansi-red}{HTML}{E75C58}
\definecolor{ansi-red-intense}{HTML}{B22B31}
\definecolor{ansi-green}{HTML}{00A250}
\definecolor{ansi-green-intense}{HTML}{007427}
\definecolor{ansi-yellow}{HTML}{DDB62B}
\definecolor{ansi-yellow-intense}{HTML}{B27D12}
\definecolor{ansi-blue}{HTML}{208FFB}
\definecolor{ansi-blue-intense}{HTML}{0065CA}
\definecolor{ansi-magenta}{HTML}{D160C4}
\definecolor{ansi-magenta-intense}{HTML}{A03196}
\definecolor{ansi-cyan}{HTML}{60C6C8}
\definecolor{ansi-cyan-intense}{HTML}{258F8F}
\definecolor{ansi-white}{HTML}{C5C1B4}
\definecolor{ansi-white-intense}{HTML}{A1A6B2}
\definecolor{ansi-default-inverse-fg}{HTML}{FFFFFF}
\definecolor{ansi-default-inverse-bg}{HTML}{000000}
    

% \usepackage{framed}
% \newenvironment{Shaded}{}{}
% \newcommand{\KeywordTok}[1]{\textcolor{keywcolor}{\textbf{{#1}}}}
% \newcommand{\DataTypeTok}[1]{\textcolor{datacolor}{{#1}}}
% %\newcommand{\DecValTok}[1]{\textcolor{decvcolor}{{#1}}}
% \newcommand{\DecValTok}[1]{{#1}} 
% \newcommand{\BaseNTok}[1]{\textcolor{basencolor}{{#1}}}
% \newcommand{\FloatTok}[1]{\textcolor{floatcolor}{{#1}}}
% \newcommand{\CharTok}[1]{\textcolor{charcolor}{\textbf{{#1}}}}
% \newcommand{\StringTok}[1]{\textcolor{stringcolor}{{#1}}}
% \newcommand{\CommentTok}[1]{\textcolor{commentcolor}{\textit{{#1}}}}
% \newcommand{\OtherTok}[1]{\textcolor{othercolor}{{#1}}} 
% \newcommand{\AlertTok}[1]{\textcolor{alertcolor}{\textbf{{#1}}}}
% %\newcommand{\FunctionTok}[1]{\textcolor{funccolor}{{#1}}}
% \newcommand{\FunctionTok}[1]{{#1}}
% \newcommand{\RegionMarkerTok}[1]{{#1}}
% \newcommand{\ErrorTok}[1]{\textbf{{#1}}}
% \newcommand{\NormalTok}[1]{{#1}}

% The default LaTeX title has an obnoxious amount of whitespace. By default,
% titling removes some of it. It also provides customization options.
\usepackage{titling}

% headers and footers
\usepackage{fancyhdr}
%\pagestyle{fancy}
\pagestyle{plain}

\fancyhead{}
\fancyfoot{}

%\fancyhead[LE,RO]{\slshape \rightmark}
%\fancyhead[LO,RE]{\slshape \leftmark}
\fancyfoot[C]{\thepage}
%\headrulewidth 0.4pt
%\footrulewidth 0 pt

%\addtolength{\headheight}{\baselineskip}

%\lfoot{\emph{Analyze the Data not the Drivel}}
%\rfoot{\emph{\today}}

% subfigure handles figures that contain subfigures
%\usepackage{color,graphicx,subfigure,sidecap}
\usepackage{graphicx,sidecap}
\usepackage{subfigure}
\graphicspath{{./inclusions/}}

% floatflt provides for text wrapping around small figures and tables
\usepackage{floatflt}

% tweak caption formats 
\usepackage{caption} 
\usepackage{sidecap}
%\usepackage{subcaption} % not compatible with subfigure

\usepackage{rotating} % flip tables sideways

% complex footnotes
%\usepackage{bigfoot}

% weird logos \XeLaTeX
\usepackage{metalogo}

\newcommand{\HRule}{\rule{\linewidth}{0.5mm}}

\usepackage[breakable]{tcolorbox}

\usepackage{parskip} % Stop auto-indenting (to mimic markdown behaviour)
    
% Basic figure setup, for now with no caption control since it's done
% automatically by Pandoc (which extracts ![](path) syntax from Markdown).
\usepackage{graphicx}

%\DeclareCaptionFormat{nocaption}{}
%\captionsetup{format=nocaption,aboveskip=0pt,belowskip=0pt}

\usepackage[Export]{adjustbox} % Used to constrain images to a maximum size
\adjustboxset{max size={0.9\linewidth}{0.9\paperheight}}
\usepackage{float}

%\floatplacement{figure}{H} % forces figures to be placed at the correct location

\usepackage{xcolor} % Allow colors to be defined
\usepackage{enumerate} % Needed for markdown enumerations to work
\usepackage{geometry} % Used to adjust the document margins

%\usepackage{amsmath} % Equations
%\usepackage{amssymb} % Equations

\usepackage{textcomp} % defines textquotesingle

% Hack from http://tex.stackexchange.com/a/47451/13684:
\AtBeginDocument{%
	\def\PYZsq{\textquotesingle}% Upright quotes in Pygmentized code
}

\usepackage{upquote} % Upright quotes for verbatim code
\usepackage{eurosym} % defines \euro
\usepackage[mathletters]{ucs} % Extended unicode (utf-8) support

%\usepackage{fancyvrb} % verbatim replacement that allows latex

\usepackage{grffile} % extends the file name processing of package graphics 
					 % to support a larger range
					 
\makeatletter % fix for grffile with XeLaTeX
\def\Gread@@xetex#1{%
  \IfFileExists{"\Gin@base".bb}%
  {\Gread@eps{\Gin@base.bb}}%
  {\Gread@@xetex@aux#1}%
}
\makeatother

% The hyperref package gives us a pdf with properly built
% internal navigation ('pdf bookmarks' for the table of contents,
% internal cross-reference links, web links for URLs, etc.)
\usepackage{hyperref}
% The default LaTeX title has an obnoxious amount of whitespace. By default,
% titling removes some of it. It also provides customization options.
\usepackage{titling}
\usepackage{longtable} % longtable support required by pandoc >1.10
\usepackage{booktabs}  % table support for pandoc > 1.12.2
\usepackage[inline]{enumitem} % IRkernel/repr support (it uses the enumerate* environment)
\usepackage[normalem]{ulem} % ulem is needed to support strikethroughs (\sout)
							% normalem makes italics be italics, not underlines
\usepackage{mathrsfs}

% commands and environments needed by pandoc snippets
% extracted from the output of `pandoc -s`
\providecommand{\tightlist}{%
  \setlength{\itemsep}{0pt}\setlength{\parskip}{0pt}}
  
\DefineVerbatimEnvironment{Highlighting}{Verbatim}{commandchars=\\\{\}}
% Add ',fontsize=\small' for more characters per line
\newenvironment{Shaded}{}{}
\newcommand{\KeywordTok}[1]{\textcolor[rgb]{0.00,0.44,0.13}{\textbf{{#1}}}}
\newcommand{\DataTypeTok}[1]{\textcolor[rgb]{0.56,0.13,0.00}{{#1}}}
\newcommand{\DecValTok}[1]{\textcolor[rgb]{0.25,0.63,0.44}{{#1}}}
\newcommand{\BaseNTok}[1]{\textcolor[rgb]{0.25,0.63,0.44}{{#1}}}
\newcommand{\FloatTok}[1]{\textcolor[rgb]{0.25,0.63,0.44}{{#1}}}
\newcommand{\CharTok}[1]{\textcolor[rgb]{0.25,0.44,0.63}{{#1}}}
\newcommand{\StringTok}[1]{\textcolor[rgb]{0.25,0.44,0.63}{{#1}}}
\newcommand{\CommentTok}[1]{\textcolor[rgb]{0.38,0.63,0.69}{\textit{{#1}}}}
\newcommand{\OtherTok}[1]{\textcolor[rgb]{0.00,0.44,0.13}{{#1}}}
\newcommand{\AlertTok}[1]{\textcolor[rgb]{1.00,0.00,0.00}{\textbf{{#1}}}}
\newcommand{\FunctionTok}[1]{\textcolor[rgb]{0.02,0.16,0.49}{{#1}}}
\newcommand{\RegionMarkerTok}[1]{{#1}}
\newcommand{\ErrorTok}[1]{\textcolor[rgb]{1.00,0.00,0.00}{\textbf{{#1}}}}
\newcommand{\NormalTok}[1]{{#1}}

% Additional commands for more recent versions of Pandoc
\newcommand{\ConstantTok}[1]{\textcolor[rgb]{0.53,0.00,0.00}{{#1}}}
\newcommand{\SpecialCharTok}[1]{\textcolor[rgb]{0.25,0.44,0.63}{{#1}}}
\newcommand{\VerbatimStringTok}[1]{\textcolor[rgb]{0.25,0.44,0.63}{{#1}}}
\newcommand{\SpecialStringTok}[1]{\textcolor[rgb]{0.73,0.40,0.53}{{#1}}}
\newcommand{\ImportTok}[1]{{#1}}
\newcommand{\DocumentationTok}[1]{\textcolor[rgb]{0.73,0.13,0.13}{\textit{{#1}}}}
\newcommand{\AnnotationTok}[1]{\textcolor[rgb]{0.38,0.63,0.69}{\textbf{\textit{{#1}}}}}
\newcommand{\CommentVarTok}[1]{\textcolor[rgb]{0.38,0.63,0.69}{\textbf{\textit{{#1}}}}}
\newcommand{\VariableTok}[1]{\textcolor[rgb]{0.10,0.09,0.49}{{#1}}}
\newcommand{\ControlFlowTok}[1]{\textcolor[rgb]{0.00,0.44,0.13}{\textbf{{#1}}}}
\newcommand{\OperatorTok}[1]{\textcolor[rgb]{0.40,0.40,0.40}{{#1}}}
\newcommand{\BuiltInTok}[1]{{#1}}
\newcommand{\ExtensionTok}[1]{{#1}}
\newcommand{\PreprocessorTok}[1]{\textcolor[rgb]{0.74,0.48,0.00}{{#1}}}
\newcommand{\AttributeTok}[1]{\textcolor[rgb]{0.49,0.56,0.16}{{#1}}}
\newcommand{\InformationTok}[1]{\textcolor[rgb]{0.38,0.63,0.69}{\textbf{\textit{{#1}}}}}
\newcommand{\WarningTok}[1]{\textcolor[rgb]{0.38,0.63,0.69}{\textbf{\textit{{#1}}}}}

% Define a nice break command that doesn't care if a line doesn't already exist.
\def\br{\hspace*{\fill} \\* }
% Math Jax compatibility definitions
\def\gt{>}
\def\lt{<}
\let\Oldtex\TeX
\let\Oldlatex\LaTeX
\renewcommand{\TeX}{\textrm{\Oldtex}}
\renewcommand{\LaTeX}{\textrm{\Oldlatex}}
 
% Pygments definitions
\makeatletter
\def\PY@reset{\let\PY@it=\relax \let\PY@bf=\relax%
    \let\PY@ul=\relax \let\PY@tc=\relax%
    \let\PY@bc=\relax \let\PY@ff=\relax}
\def\PY@tok#1{\csname PY@tok@#1\endcsname}
\def\PY@toks#1+{\ifx\relax#1\empty\else%
    \PY@tok{#1}\expandafter\PY@toks\fi}
\def\PY@do#1{\PY@bc{\PY@tc{\PY@ul{%
    \PY@it{\PY@bf{\PY@ff{#1}}}}}}}
\def\PY#1#2{\PY@reset\PY@toks#1+\relax+\PY@do{#2}}

\expandafter\def\csname PY@tok@w\endcsname{\def\PY@tc##1{\textcolor[rgb]{0.73,0.73,0.73}{##1}}}
\expandafter\def\csname PY@tok@c\endcsname{\let\PY@it=\textit\def\PY@tc##1{\textcolor[rgb]{0.25,0.50,0.50}{##1}}}
\expandafter\def\csname PY@tok@cp\endcsname{\def\PY@tc##1{\textcolor[rgb]{0.74,0.48,0.00}{##1}}}
\expandafter\def\csname PY@tok@k\endcsname{\let\PY@bf=\textbf\def\PY@tc##1{\textcolor[rgb]{0.00,0.50,0.00}{##1}}}
\expandafter\def\csname PY@tok@kp\endcsname{\def\PY@tc##1{\textcolor[rgb]{0.00,0.50,0.00}{##1}}}
\expandafter\def\csname PY@tok@kt\endcsname{\def\PY@tc##1{\textcolor[rgb]{0.69,0.00,0.25}{##1}}}
\expandafter\def\csname PY@tok@o\endcsname{\def\PY@tc##1{\textcolor[rgb]{0.40,0.40,0.40}{##1}}}
\expandafter\def\csname PY@tok@ow\endcsname{\let\PY@bf=\textbf\def\PY@tc##1{\textcolor[rgb]{0.67,0.13,1.00}{##1}}}
\expandafter\def\csname PY@tok@nb\endcsname{\def\PY@tc##1{\textcolor[rgb]{0.00,0.50,0.00}{##1}}}
\expandafter\def\csname PY@tok@nf\endcsname{\def\PY@tc##1{\textcolor[rgb]{0.00,0.00,1.00}{##1}}}
\expandafter\def\csname PY@tok@nc\endcsname{\let\PY@bf=\textbf\def\PY@tc##1{\textcolor[rgb]{0.00,0.00,1.00}{##1}}}
\expandafter\def\csname PY@tok@nn\endcsname{\let\PY@bf=\textbf\def\PY@tc##1{\textcolor[rgb]{0.00,0.00,1.00}{##1}}}
\expandafter\def\csname PY@tok@ne\endcsname{\let\PY@bf=\textbf\def\PY@tc##1{\textcolor[rgb]{0.82,0.25,0.23}{##1}}}
\expandafter\def\csname PY@tok@nv\endcsname{\def\PY@tc##1{\textcolor[rgb]{0.10,0.09,0.49}{##1}}}
\expandafter\def\csname PY@tok@no\endcsname{\def\PY@tc##1{\textcolor[rgb]{0.53,0.00,0.00}{##1}}}
\expandafter\def\csname PY@tok@nl\endcsname{\def\PY@tc##1{\textcolor[rgb]{0.63,0.63,0.00}{##1}}}
\expandafter\def\csname PY@tok@ni\endcsname{\let\PY@bf=\textbf\def\PY@tc##1{\textcolor[rgb]{0.60,0.60,0.60}{##1}}}
\expandafter\def\csname PY@tok@na\endcsname{\def\PY@tc##1{\textcolor[rgb]{0.49,0.56,0.16}{##1}}}
\expandafter\def\csname PY@tok@nt\endcsname{\let\PY@bf=\textbf\def\PY@tc##1{\textcolor[rgb]{0.00,0.50,0.00}{##1}}}
\expandafter\def\csname PY@tok@nd\endcsname{\def\PY@tc##1{\textcolor[rgb]{0.67,0.13,1.00}{##1}}}
\expandafter\def\csname PY@tok@s\endcsname{\def\PY@tc##1{\textcolor[rgb]{0.73,0.13,0.13}{##1}}}
\expandafter\def\csname PY@tok@sd\endcsname{\let\PY@it=\textit\def\PY@tc##1{\textcolor[rgb]{0.73,0.13,0.13}{##1}}}
\expandafter\def\csname PY@tok@si\endcsname{\let\PY@bf=\textbf\def\PY@tc##1{\textcolor[rgb]{0.73,0.40,0.53}{##1}}}
\expandafter\def\csname PY@tok@se\endcsname{\let\PY@bf=\textbf\def\PY@tc##1{\textcolor[rgb]{0.73,0.40,0.13}{##1}}}
\expandafter\def\csname PY@tok@sr\endcsname{\def\PY@tc##1{\textcolor[rgb]{0.73,0.40,0.53}{##1}}}
\expandafter\def\csname PY@tok@ss\endcsname{\def\PY@tc##1{\textcolor[rgb]{0.10,0.09,0.49}{##1}}}
\expandafter\def\csname PY@tok@sx\endcsname{\def\PY@tc##1{\textcolor[rgb]{0.00,0.50,0.00}{##1}}}
\expandafter\def\csname PY@tok@m\endcsname{\def\PY@tc##1{\textcolor[rgb]{0.40,0.40,0.40}{##1}}}
\expandafter\def\csname PY@tok@gh\endcsname{\let\PY@bf=\textbf\def\PY@tc##1{\textcolor[rgb]{0.00,0.00,0.50}{##1}}}
\expandafter\def\csname PY@tok@gu\endcsname{\let\PY@bf=\textbf\def\PY@tc##1{\textcolor[rgb]{0.50,0.00,0.50}{##1}}}
\expandafter\def\csname PY@tok@gd\endcsname{\def\PY@tc##1{\textcolor[rgb]{0.63,0.00,0.00}{##1}}}
\expandafter\def\csname PY@tok@gi\endcsname{\def\PY@tc##1{\textcolor[rgb]{0.00,0.63,0.00}{##1}}}
\expandafter\def\csname PY@tok@gr\endcsname{\def\PY@tc##1{\textcolor[rgb]{1.00,0.00,0.00}{##1}}}
\expandafter\def\csname PY@tok@ge\endcsname{\let\PY@it=\textit}
\expandafter\def\csname PY@tok@gs\endcsname{\let\PY@bf=\textbf}
\expandafter\def\csname PY@tok@gp\endcsname{\let\PY@bf=\textbf\def\PY@tc##1{\textcolor[rgb]{0.00,0.00,0.50}{##1}}}
\expandafter\def\csname PY@tok@go\endcsname{\def\PY@tc##1{\textcolor[rgb]{0.53,0.53,0.53}{##1}}}
\expandafter\def\csname PY@tok@gt\endcsname{\def\PY@tc##1{\textcolor[rgb]{0.00,0.27,0.87}{##1}}}
\expandafter\def\csname PY@tok@err\endcsname{\def\PY@bc##1{\setlength{\fboxsep}{0pt}\fcolorbox[rgb]{1.00,0.00,0.00}{1,1,1}{\strut ##1}}}
\expandafter\def\csname PY@tok@kc\endcsname{\let\PY@bf=\textbf\def\PY@tc##1{\textcolor[rgb]{0.00,0.50,0.00}{##1}}}
\expandafter\def\csname PY@tok@kd\endcsname{\let\PY@bf=\textbf\def\PY@tc##1{\textcolor[rgb]{0.00,0.50,0.00}{##1}}}
\expandafter\def\csname PY@tok@kn\endcsname{\let\PY@bf=\textbf\def\PY@tc##1{\textcolor[rgb]{0.00,0.50,0.00}{##1}}}
\expandafter\def\csname PY@tok@kr\endcsname{\let\PY@bf=\textbf\def\PY@tc##1{\textcolor[rgb]{0.00,0.50,0.00}{##1}}}
\expandafter\def\csname PY@tok@bp\endcsname{\def\PY@tc##1{\textcolor[rgb]{0.00,0.50,0.00}{##1}}}
\expandafter\def\csname PY@tok@fm\endcsname{\def\PY@tc##1{\textcolor[rgb]{0.00,0.00,1.00}{##1}}}
\expandafter\def\csname PY@tok@vc\endcsname{\def\PY@tc##1{\textcolor[rgb]{0.10,0.09,0.49}{##1}}}
\expandafter\def\csname PY@tok@vg\endcsname{\def\PY@tc##1{\textcolor[rgb]{0.10,0.09,0.49}{##1}}}
\expandafter\def\csname PY@tok@vi\endcsname{\def\PY@tc##1{\textcolor[rgb]{0.10,0.09,0.49}{##1}}}
\expandafter\def\csname PY@tok@vm\endcsname{\def\PY@tc##1{\textcolor[rgb]{0.10,0.09,0.49}{##1}}}
\expandafter\def\csname PY@tok@sa\endcsname{\def\PY@tc##1{\textcolor[rgb]{0.73,0.13,0.13}{##1}}}
\expandafter\def\csname PY@tok@sb\endcsname{\def\PY@tc##1{\textcolor[rgb]{0.73,0.13,0.13}{##1}}}
\expandafter\def\csname PY@tok@sc\endcsname{\def\PY@tc##1{\textcolor[rgb]{0.73,0.13,0.13}{##1}}}
\expandafter\def\csname PY@tok@dl\endcsname{\def\PY@tc##1{\textcolor[rgb]{0.73,0.13,0.13}{##1}}}
\expandafter\def\csname PY@tok@s2\endcsname{\def\PY@tc##1{\textcolor[rgb]{0.73,0.13,0.13}{##1}}}
\expandafter\def\csname PY@tok@sh\endcsname{\def\PY@tc##1{\textcolor[rgb]{0.73,0.13,0.13}{##1}}}
\expandafter\def\csname PY@tok@s1\endcsname{\def\PY@tc##1{\textcolor[rgb]{0.73,0.13,0.13}{##1}}}
\expandafter\def\csname PY@tok@mb\endcsname{\def\PY@tc##1{\textcolor[rgb]{0.40,0.40,0.40}{##1}}}
\expandafter\def\csname PY@tok@mf\endcsname{\def\PY@tc##1{\textcolor[rgb]{0.40,0.40,0.40}{##1}}}
\expandafter\def\csname PY@tok@mh\endcsname{\def\PY@tc##1{\textcolor[rgb]{0.40,0.40,0.40}{##1}}}
\expandafter\def\csname PY@tok@mi\endcsname{\def\PY@tc##1{\textcolor[rgb]{0.40,0.40,0.40}{##1}}}
\expandafter\def\csname PY@tok@il\endcsname{\def\PY@tc##1{\textcolor[rgb]{0.40,0.40,0.40}{##1}}}
\expandafter\def\csname PY@tok@mo\endcsname{\def\PY@tc##1{\textcolor[rgb]{0.40,0.40,0.40}{##1}}}
\expandafter\def\csname PY@tok@ch\endcsname{\let\PY@it=\textit\def\PY@tc##1{\textcolor[rgb]{0.25,0.50,0.50}{##1}}}
\expandafter\def\csname PY@tok@cm\endcsname{\let\PY@it=\textit\def\PY@tc##1{\textcolor[rgb]{0.25,0.50,0.50}{##1}}}
\expandafter\def\csname PY@tok@cpf\endcsname{\let\PY@it=\textit\def\PY@tc##1{\textcolor[rgb]{0.25,0.50,0.50}{##1}}}
\expandafter\def\csname PY@tok@c1\endcsname{\let\PY@it=\textit\def\PY@tc##1{\textcolor[rgb]{0.25,0.50,0.50}{##1}}}
\expandafter\def\csname PY@tok@cs\endcsname{\let\PY@it=\textit\def\PY@tc##1{\textcolor[rgb]{0.25,0.50,0.50}{##1}}}

\def\PYZbs{\char`\\}
\def\PYZus{\char`\_}
\def\PYZob{\char`\{}
\def\PYZcb{\char`\}}
\def\PYZca{\char`\^}
\def\PYZam{\char`\&}
\def\PYZlt{\char`\<}
\def\PYZgt{\char`\>}
\def\PYZsh{\char`\#}
\def\PYZpc{\char`\%}
\def\PYZdl{\char`\$}
\def\PYZhy{\char`\-}
\def\PYZsq{\char`\'}
\def\PYZdq{\char`\"}
\def\PYZti{\char`\~}
% for compatibility with earlier versions
\def\PYZat{@}
\def\PYZlb{[}
\def\PYZrb{]}
\makeatother

% For linebreaks inside Verbatim environment from package fancyvrb. 
\makeatletter
	\newbox\Wrappedcontinuationbox 
	\newbox\Wrappedvisiblespacebox 
	\newcommand*\Wrappedvisiblespace {\textcolor{red}{\textvisiblespace}} 
	\newcommand*\Wrappedcontinuationsymbol {\textcolor{red}{\llap{\tiny$\m@th\hookrightarrow$}}} 
	\newcommand*\Wrappedcontinuationindent {3ex } 
	\newcommand*\Wrappedafterbreak {\kern\Wrappedcontinuationindent\copy\Wrappedcontinuationbox} 
	% Take advantage of the already applied Pygments mark-up to insert 
	% potential linebreaks for TeX processing. 
	%        {, <, #, %, $, ' and ": go to next line. 
	%        _, }, ^, &, >, - and ~: stay at end of broken line. 
	% Use of \textquotesingle for straight quote. 
	\newcommand*\Wrappedbreaksatspecials {% 
		\def\PYGZus{\discretionary{\char`\_}{\Wrappedafterbreak}{\char`\_}}% 
		\def\PYGZob{\discretionary{}{\Wrappedafterbreak\char`\{}{\char`\{}}% 
		\def\PYGZcb{\discretionary{\char`\}}{\Wrappedafterbreak}{\char`\}}}% 
		\def\PYGZca{\discretionary{\char`\^}{\Wrappedafterbreak}{\char`\^}}% 
		\def\PYGZam{\discretionary{\char`\&}{\Wrappedafterbreak}{\char`\&}}% 
		\def\PYGZlt{\discretionary{}{\Wrappedafterbreak\char`\<}{\char`\<}}% 
		\def\PYGZgt{\discretionary{\char`\>}{\Wrappedafterbreak}{\char`\>}}% 
		\def\PYGZsh{\discretionary{}{\Wrappedafterbreak\char`\#}{\char`\#}}% 
		\def\PYGZpc{\discretionary{}{\Wrappedafterbreak\char`\%}{\char`\%}}% 
		\def\PYGZdl{\discretionary{}{\Wrappedafterbreak\char`\$}{\char`\$}}% 
		\def\PYGZhy{\discretionary{\char`\-}{\Wrappedafterbreak}{\char`\-}}% 
		\def\PYGZsq{\discretionary{}{\Wrappedafterbreak\textquotesingle}{\textquotesingle}}% 
		\def\PYGZdq{\discretionary{}{\Wrappedafterbreak\char`\"}{\char`\"}}% 
		\def\PYGZti{\discretionary{\char`\~}{\Wrappedafterbreak}{\char`\~}}% 
	} 
	% Some characters . , ; ? ! / are not pygmentized. 
	% This macro makes them "active" and they will insert potential linebreaks 
	\newcommand*\Wrappedbreaksatpunct {% 
		\lccode`\~`\.\lowercase{\def~}{\discretionary{\hbox{\char`\.}}{\Wrappedafterbreak}{\hbox{\char`\.}}}% 
		\lccode`\~`\,\lowercase{\def~}{\discretionary{\hbox{\char`\,}}{\Wrappedafterbreak}{\hbox{\char`\,}}}% 
		\lccode`\~`\;\lowercase{\def~}{\discretionary{\hbox{\char`\;}}{\Wrappedafterbreak}{\hbox{\char`\;}}}% 
		\lccode`\~`\:\lowercase{\def~}{\discretionary{\hbox{\char`\:}}{\Wrappedafterbreak}{\hbox{\char`\:}}}% 
		\lccode`\~`\?\lowercase{\def~}{\discretionary{\hbox{\char`\?}}{\Wrappedafterbreak}{\hbox{\char`\?}}}% 
		\lccode`\~`\!\lowercase{\def~}{\discretionary{\hbox{\char`\!}}{\Wrappedafterbreak}{\hbox{\char`\!}}}% 
		\lccode`\~`\/\lowercase{\def~}{\discretionary{\hbox{\char`\/}}{\Wrappedafterbreak}{\hbox{\char`\/}}}% 
		\catcode`\.\active
		\catcode`\,\active 
		\catcode`\;\active
		\catcode`\:\active
		\catcode`\?\active
		\catcode`\!\active
		\catcode`\/\active 
		\lccode`\~`\~ 	
	}
\makeatother

\let\OriginalVerbatim=\Verbatim
\makeatletter
\renewcommand{\Verbatim}[1][1]{%
	%\parskip\z@skip
	\sbox\Wrappedcontinuationbox {\Wrappedcontinuationsymbol}%
	\sbox\Wrappedvisiblespacebox {\FV@SetupFont\Wrappedvisiblespace}%
	\def\FancyVerbFormatLine ##1{\hsize\linewidth
		\vtop{\raggedright\hyphenpenalty\z@\exhyphenpenalty\z@
			\doublehyphendemerits\z@\finalhyphendemerits\z@
			\strut ##1\strut}%
	}%
	% If the linebreak is at a space, the latter will be displayed as visible
	% space at end of first line, and a continuation symbol starts next line.
	% Stretch/shrink are however usually zero for typewriter font.
	\def\FV@Space {%
		\nobreak\hskip\z@ plus\fontdimen3\font minus\fontdimen4\font
		\discretionary{\copy\Wrappedvisiblespacebox}{\Wrappedafterbreak}
		{\kern\fontdimen2\font}%
	}%
	
	% Allow breaks at special characters using \PYG... macros.
	\Wrappedbreaksatspecials
	% Breaks at punctuation characters . , ; ? ! and / need catcode=\active 	
	\OriginalVerbatim[#1,codes*=\Wrappedbreaksatpunct]%
}
\makeatother


% prompt
\makeatletter
\newcommand{\boxspacing}{\kern\kvtcb@left@rule\kern\kvtcb@boxsep}
\makeatother
\newcommand{\prompt}[4]{
	\ttfamily\llap{{\color{#2}[#3]:\hspace{3pt}#4}}\vspace{-\baselineskip}
}
    

% Prevent overflowing lines due to hard-to-break entities
\sloppy 

% Setup hyperref package
\hypersetup{
  breaklinks=true,  % so long urls are correctly broken across lines
  colorlinks=true,
  urlcolor=urlcolor,
  linkcolor=linkcolor,
  citecolor=citecolor,
  pdfauthor={John D. Baker},
  pdftitle={Analyze the Data not the Drivel},
  pdfsubject={Blog},
  pdfcreator={MikTeX+LaTeXe},
  pdfkeywords={blog,wordpress},
  }
  
% Slightly bigger margins than the latex defaults
% \geometry{verbose,tmargin=1in,bmargin=1in,lmargin=1in,rmargin=1in}  

%\usepackage{wrapfig}

% source code listings
\usepackage{listings}

\lstdefinelanguage{bat}
{morekeywords={echo,title,pushd,popd,setlocal,endlocal,off,if,not,exist,set,goto,pause},
sensitive=True,
morecomment=[l]{rem}
}

\lstdefinelanguage{jdoc}
{
morekeywords={},
otherkeywords={assert.,break.,continue.,for.,do.,if.,else.,elseif.,return.,select.,end.
,while.,whilst.,throw.,catch.,catchd.,catcht.,try.,case.,fcase.},
sensitive=True,
morecomment=[l]{NB.},
morestring=[b]',
morestring=[d]',
}

% latex size ordering - can never remember it
% \tiny
% \scriptsize
% \footnotesize
% \small
% \normalsize
% \large
% \Large
% \LARGE
% \huge
% \Huge
 
% listings package settings  
\lstset{%
  language=jdoc,                                % j document settings
  basicstyle=\ttfamily\footnotesize,            
  keywordstyle=\bfseries\color{keywcolor}\footnotesize,
  identifierstyle=\color{black},
  commentstyle=\slshape\color{CodeComment},     % colored slanted comments
  stringstyle=\color{red}\ttfamily,
  showstringspaces=false,                       
  %backgroundcolor=\color{CodeBackGround},       
  frame=single,                                
  framesep=1pt,                                 
  framerule=0.8pt,                             
  rulecolor=\color{CodeBackGround},   
  showspaces=false,
  %columns=fullflexible,
  %numbers=left,
  %numberstyle=\footnotesize,
  %numbersep=9pt,
  tabsize=2,
  showtabs=false,
  captionpos=b
  breaklines=true,                              
  breakindent=5pt                              
}

\lstdefinelanguage{JavaScript}{
  keywords={typeof, new, true, false, catch, function, return, null, catch, switch, var, if, in, while, do, else, case, break},
  ndkeywords={class, export, boolean, throw, implements, import, this},
  ndkeywordstyle=\color{darkgray}\bfseries,
  sensitive=false,
  comment=[l]{//},
  morecomment=[s]{/*}{*/},
  morestring=[b]',
  morestring=[b]"
}

% C# settings
\lstdefinestyle{sharpc}{
language=[Sharp]C,
basicstyle=\ttfamily\scriptsize, 
keywordstyle=\bfseries\color{keywcolor}\scriptsize,
framerule=0pt
}

% for source code listing longer than two use smaller font
\lstdefinestyle{smallersource}{
basicstyle=\ttfamily\scriptsize, 
keywordstyle=\bfseries\color{keywcolor}\scriptsize,
framerule=0pt
}

\lstdefinestyle{resetdefaults}{
language=jdoc,
basicstyle=\ttfamily\footnotesize,  
keywordstyle=\bfseries\color{keywcolor}\footnotesize,                                                               
framerule=0.8pt 
}

% APL UTF8 code points listed for lstlisting processing
\makeatletter
\lst@InputCatcodes
\def\lst@DefEC{%
 \lst@CCECUse \lst@ProcessLetter
  ^^80^^81^^82^^83^^84^^85^^86^^87^^88^^89^^8a^^8b^^8c^^8d^^8e^^8f%
  ^^90^^91^^92^^93^^94^^95^^96^^97^^98^^99^^9a^^9b^^9c^^9d^^9e^^9f%
  ^^a0^^a1^^a2^^a3^^a4^^a5^^a6^^a7^^a8^^a9^^aa^^ab^^ac^^ad^^ae^^af%
  ^^b0^^b1^^b2^^b3^^b4^^b5^^b6^^b7^^b8^^b9^^ba^^bb^^bc^^bd^^be^^bf%
  ^^c0^^c1^^c2^^c3^^c4^^c5^^c6^^c7^^c8^^c9^^ca^^cb^^cc^^cd^^ce^^cf%
  ^^d0^^d1^^d2^^d3^^d4^^d5^^d6^^d7^^d8^^d9^^da^^db^^dc^^dd^^de^^df%
  ^^e0^^e1^^e2^^e3^^e4^^e5^^e6^^e7^^e8^^e9^^ea^^eb^^ec^^ed^^ee^^ef%
  ^^f0^^f1^^f2^^f3^^f4^^f5^^f6^^f7^^f8^^f9^^fa^^fb^^fc^^fd^^fe^^ff%
  ^^^^20ac^^^^0153^^^^0152%
  ^^^^20a7^^^^2190^^^^2191^^^^2192^^^^2193^^^^2206^^^^2207^^^^220a%
  ^^^^2218^^^^2228^^^^2229^^^^222a^^^^2235^^^^223c^^^^2260^^^^2261%
  ^^^^2262^^^^2264^^^^2265^^^^2282^^^^2283^^^^2296^^^^22a2^^^^22a3%
  ^^^^22a4^^^^22a5^^^^22c4^^^^2308^^^^230a^^^^2336^^^^2337^^^^2339%
  ^^^^233b^^^^233d^^^^233f^^^^2340^^^^2342^^^^2347^^^^2348^^^^2349%
  ^^^^234b^^^^234e^^^^2350^^^^2352^^^^2355^^^^2357^^^^2359^^^^235d%
  ^^^^235e^^^^235f^^^^2361^^^^2362^^^^2363^^^^2364^^^^2365^^^^2368%
  ^^^^236a^^^^236b^^^^236c^^^^2371^^^^2372^^^^2373^^^^2374^^^^2375%
  ^^^^2377^^^^2378^^^^237a^^^^2395^^^^25af^^^^25ca^^^^25cb%  
  ^^00}
\lst@RestoreCatcodes
\makeatother

% custom lengths used within minipages
\newcommand{\minindent}{17pt}

\makeindex

\begin{document}

\subsection*{\href{http://analyzethedatanotthedrivel.org/2025/03/16/gonggone-gone-parts-3-4/}{Gonggone Gone --- Parts 3 \& 4}}
\addcontentsline{toc}{subsection}{Gonggone Gone --- Parts 3 \& 4}


\noindent\emph{Posted: 16 Mar 2025 18:51:43}
\vspace{6pt}

\begin{center}\large\textbf{-- \emph{a road trip} --}\normalsize\end{center}

By late afternoon, as the stores began filling with panicked shoppers,
Doug and Alex topped up the truck at a gas station on Eagle and Pine.
Gas pump traffic was heavier than usual. It was sinking in. This could
not be ignored or blamed on people you don't like.

Handing Doug the truck keys, Alex said, ``Why don't you drive?''

They pulled onto Interstate 84 and headed east. The highway was busy but
not much worse than a typical rush hour. As they drove out of town, Alex
reset his iPhone to its factory default.

``Give me your phone.'' Doug handed over his phone. ``What's the
passcode?''

He quickly reset Doug's phone and tossed both phones out the truck
window between Boise and Mountain Home.

``What the hell, Dad?''

``We must stay completely offline. I don't want government goons
tracking us with cell towers.''

``Aren't they busy with --- I don't know --- the end of the fucking
world.''

``When, in your life, has the goddamn government personally helped
you?''

It was Alex's standard; \emph{reduce it to yourself}, rejoinder. Thorny
policy questions evaporate when reduced to \emph{yourself}. Simply ask,
does action X help or hurt me? Forget about the rest of humanity. What
does it do for you? Just you, right now! Thomas Hobbes said we are in a
constant war of all against all, and when we forget, we become marks,
suckers, drones, tools: meat robots for others.

Doug, used to his dad's rants, shrugged and said, ``Try the radio.''

Alex turned on the radio and tuned in NPR, or \emph{Nitwit and Puerile
Radio,} as he mockingly called it. The NPR ladies, both female and male,
were more hysterical than usual. The Earth's sudden departure from
gravity as we know it stumped the NPR experts, and the vibes were all
bad. Naturally, NPR trotted out a parade of celebrity physicists, but
they couldn't explain what had happened. Nobody wanted to ask the
pretentious NPR \emph{what does this all mean questions}.

Alex smirked, ``I wonder why?''

When one young Hawaii-based astronomer started explaining what might
happen if Earth stayed on this course, they abruptly cut him off.

``Had enough?'' Alex asked. ``Let's check God Radio.''

The NPR ladies were hysterical, but God Radio was flat-out
\emph{end-timing it}. The rapture was nigh. God had untied the bonds of
heaven to punish the wicked and sinful, and you better repent your sorry
ass. Maybe this time, God wouldn't pull a Noah and flood the Earth.
Perhaps he preferred to cast it into the freezing dark. God Radio didn't
pussyfoot around. We're all doomed, and for once, they were right.

Doug laughed, ``So the `consensus' is we're all fucked, up, down, and
sideways.''

Expecting roadblocks at any moment, they kept driving east. They passed
Twin Falls and followed I-86 to the I-15 junction at Pocatello. While
gassing up in Idaho Falls, they joined a group of truckers and other
drivers discussing the sudden closure of the Targhee Route into
Yellowstone. The cops showed up beyond St.~Anthony and blocked Highway
20. They let traffic leave the park but blocked vehicles from entering.
The usual louts tried to go around the roadblocks but quickly ran into
recently deployed National Guardsmen. Some truckers said they'd been
messaging other drivers who reported the northern, eastern, and Jackson
entrances were also blocked. All the roads in and out of Yellowstone
were blocked.

Back on the road, Alex remarked, ``I'm not surprised. Yellowstone is a
major geothermal region. Geothermal is now the world's most precious
resource. Yellowstone's underground heat will keep pouring forth as the
Earth cools. If you dig deep tunnels near geothermal features fast
enough, it may be possible to survive in them for a while. Long-term
prospects are still bleak.''

``I bet governments will seize geothermal sites everywhere. Hell, NATO
or the Russians will probably invade Iceland. They better do it before
the oceans freeze.''

Heading north, they kept on I-15 until it hit I-90 at Butte. At Butte,
Alex took the wheel and turned east on I-90 to start the long drive
across Montana. They hoped to reach Miles City via I-94 and turn east on
State Road 12 to Baker. Over the border in North Dakota, they would turn
south and follow secondary paved and rough dirt ranch roads to Grampa's
Valley. Grampa's Valley was near the North and South Dakota borders on
the Montana side. It was an unremarkable patch of hilly and barren
badlands.

Halfway to Miles City, Doug checked the radio for news. The government
had just declared a national emergency, and the talking heads were
exploding and spewing predictable rants like, ``They're always looking
for an excuse to declare an emergency and infringe on civil liberties''
and ``How do we know Earth is running away?'' One polite NSF astronomer
explained how ``people at home'' could check it out.

He said, ``Hammer a pole, like a broom handle, into your lawn. Make sure
it's straight. Watch the pole's shadow. At solar noon, the shadow will
be pointing dead north or south. Mark the clock time. Now, check the
shadow at the same clock time every day. The shadow will start leaning.
The angle will rapidly grow. When we pass Mars, it will be over 45
degrees. At Gonggong, it will be almost 90 degrees. Fun fact: the secant
of the shadow angle is the distance to the Sun in astronomical units.''

Alex elaborated. ``That's only true if distances are in astronomical
units. See, even on runaway Earth, you cannot escape trigonometry. Good
idea, though, we can use this.''

``What's a Gonggong?''

``It's a minor Kuiper belt object---a small body orbiting beyond
Neptune. It has the coolest Kuiper belt object name. It's named after a
Chinese water god.''

``We need to keep going. I don't like the sound of `national
emergency.'\,''

They pulled into a Super 8 in Miles City at 2 AM. It had taken longer to
cross Montana than expected. There were no road closures, but highway
repair crews had abandoned their posts along a few interstate stretches.
Golly gee, giving up on fixing highways in the middle of the crisis to
end all crises. What a surprise. The missing road crews led to confusion
on a few single-lane sections, which led to car accidents, traffic jams,
and long delays. At one traffic jam, a few four-wheel drive vehicles
crossed the median to drive on the wrong side of the highway. This led
to more road jams and even longer delays.

Alex wasn't expecting to find any motel rooms, but to his surprise, the
Miles City Super 8 parking lot was almost empty. You know your town
sucks when, even at the end of things, people would still rather be
elsewhere. Entering the lobby, they pressed the front desk service
button a few times, and a sleepy, fat teenage girl with green hair
highlights emerged from an office cubby behind the front desk. Alex
quickly paid with cash. The girl looked at the bills like she had never
seen them before. Taking advantage of the empty motel, they picked a
ground-floor room beside the parked truck and trailer. They wanted to
keep an eye on their supplies.

Retiring to their room, Alex adjusted the room's window blinds. Usually,
he made motel rooms as dark as possible, but tonight, he only closed the
translucent inner blind and tolerated the glaring streetlight outside.

Even though they were both exhausted, they did a few chores. Alex went
outside, rummaged in the truck, and extracted the two large powerpacks,
his newly acquired power tools, and the light amplification monocular.
Usually, new packs, tools, and gadgets need charging. He plugged them
into the motel outlets to charge.

Doug flipped on the room's TV to catch up on any \emph{relevant} news
they missed driving. The national emergency imposed a country-wide
curfew, forcibly closed the borders, and deployed National Guardsmen and
Army Reservists to ``maintain order.'' People were to stay in their
homes, and all nonessential stores, basically most of them, were to
remain closed. Travel was restricted to within a few miles of your
domicile. Food rationing was now in effect. You now had to register at
several major tech companies and upload images of driver's licenses,
passports, or other valid forms of ID, plus self-administered mugshots,
to get a ration hash: a unique, impossible-to-forge QR code needed to
buy food.

Even cynical Alex was dismayed. ``Looks like the deep state is creaming
all over us!''

``What are they supposed to do?''

``Check the local news. See if there are any roadblocks around here.''

Local news proved useless. Face it: the news hasn't been \emph{useful}
for ages. Alex dug out his laptop and availed himself of the motel's
Wi-Fi. He quickly found the local \emph{Nextdoor} and other neighborhood
websites. By reading through recent posts, Alex found some local live
streams. People around town were pointing cellphones and house security
cameras at road intersections and live-streaming traffic.

``It looks like the Gestapo hasn't reached Miles City yet. We'll have to
go if we see any move to block streets. Doug, try and get some sleep.
I'll keep an eye on the streams and wake you up.''

Doug wasn't happy with stopping. ``Why did we stop? You've been going on
about roadblocks. Why take chances?''

``Without GPS, I'm not sure I could find the turnoff to Grampa's Valley
in the dark.''

``Well, maybe you shouldn't have tossed our phones.''

``Try and get some sleep.''

Taking his dad's advice, Doug plopped on the bed by the motel room's
bathroom and pulled the covers over his large body. In minutes, he fell
asleep.

Alex kept an eye on the traffic streams and indulged himself with what
he now realized would be his last bit of web browsing. It reminded him
of his earlier reading material conundrum. What do you do in your last
Internet session? He checked for astronomy software updates. An
emergency runaway Earth sky map update had been released. It projected
Earth's new path through the solar system.

``That was fast?''

He downloaded and installed the update and then gathered PDF manuals for
the tools and gadgets they had bought, and then, thinking they would
help, he painstakingly collected high-resolution Google Maps images of
the lands within 150 kilometers of Grampa's Valley.

The high-resolution image files were large; it took about an hour to get
them all. He lost track of the traffic streams while downloading.
Checking stream browser tabs, he was alarmed to see police cars with
spinning lights blocking the on and off-ramps of I-94. A few cops were
standing beside their cars, directing cars to turn around if they were
attempting to pull onto the highway. Cars exiting I-94 were directed
into town.

Alex shook Doug, ``We have to go ---now.''

%\subsection{grampa's valley}\label{grampas-valley}
\begin{center}\large\textbf{-- \emph{grampa's valley} --}\normalsize\end{center}

They quickly gathered the items they hauled into the motel and put them
in the back seat of the king cab. By the time they had loaded the truck,
two more police cars had blocked Highway 59 before the interstate
underpass.

Doug pointed out the new roadblock. ``We can't get to Highway 12.''

``How about south?''

``I don't see anything.''

``If we can get past Cemetery Road, we can take 59 south down to 212 and
then come back north to 12 through Albion.''

``That adds over two hundred miles.''

``It's way off Interstates. They'll be blocking main roads first. We
don't have a lot of options.''

Alex pulled out of the Super 8 parking lot and headed south. He
carefully obeyed speed limits while Doug watched the road behind them in
one of the side mirrors. The cops did nothing to stop them. Within
minutes, they put Miles City behind them. They saw no cars on the road.

Alex relaxed, ``You know what I like about southeastern Montana? No
goddamn people.''

Alex was tempted to put on the night vision monocular, turn off the
truck headlights, and drive in the dark but decided against it. Cops
would stop anyone driving without headlights. They went down to 212 and
turned north as dawn light spilled over the flat landscape.

An hour later, bright morning sunlight cast long shadows when they
gassed up in Ekalaka, Montana. Doug generously bribed the station
attendant. He wouldn't turn on the pump otherwise. Gas needed hash
ration QR codes now. It didn't take the deep state long to snuff civil
liberties. The corrupt fuckers always exploited crises; runaway Earth
suited them.

Continuing north, Alex said, ``I'm worried about Baker. It's big enough
to enforce this stay-home crap. I'm going to get off at 7 and go through
Webster. It connects with the secondary we need to get to Grampa's
Valley.''

They headed east on dirt roads and crossed into North Dakota, where they
turned south on a ranch road. After driving almost to the South Dakota
border, they headed west back into Montana. In Montana, they crept over
a rough ranch road before turning off on an even rougher double-wheeled
rut \emph{path}. After another hour of cautious off-road driving, with
the trailer bottoming out a few times, Alex and Doug descended a gentle
grade into Grampa's Valley. It wasn't much of a valley. It was a small
badland laced with eroded washes and surrounded by flat, barren steppes.

Many years ago, Alex's Grampa and his second wife made a go of sheep
ranching here. It didn't work out because Grampa hated sheep. According
to Alex's dad, the feeling was mutual. His wife managed without his
help, but she died of diabetes in the 1960s. After her death, Grampa
stayed in the valley, going deeper and deeper into debt. He sold off all
the surrounding lands to neighboring ranches but kept a small plot
around a catchment pond and reserved a right of way. He remained in his
small wood frame house, now a collapsed ruin, until Alex's dad took him
in and looked after him in the years before his death.

Alex's dad held onto the valley after Grampa's death, and Alex had fond
memories of camping with his dad in the little valley and watching the
dark, star-filled eastern Montana skies. Alex caught the amateur
astronomy bug in this valley.

Doug pointed at the ridge overlooking the small pond. ``There's Star
Hill. Remember when we watched the sky there?''

When Doug was growing up, Alex took him camping here. Alex would set up
his telescopes on Star Hill, and the two of them would delight in
Saturn's rings, the bands of Jupiter, and Doug's favorite, the Hercules
cluster.

After Alex's dad died, the valley passed to him. As part of his nuptial
assets, he'd have to sell it after his divorce. Alice started nagging
him to sell right after his dad's death, arguing the valley was too far
away for his ``useless hobby.'' Well, screw her. Let the succubus get
her half when the liquid oxygen rains start.

Creeping around the last wash curve, they spotted the derelict ruins of
Grampa's old frame house. The tiny frame house had collapsed years ago
and had been steadily picked over for campfire wood ever since. It sat
above a small, dirty catchment pond. Pulling in beside Grampa's rubbled
house, they got out of the truck with binoculars and carefully scanned
the landscape for others. They didn't want to be seen. Before doing
anything else, they hid the trail cameras: one overlooking the catchment
pond, another in Grampa's ruined house, and the third on an old run of
barbed wire fencing, overlooking the rutted truck path they had driven
in on.

With the cameras set up, they searched the south-facing hill above the
pond for Grampa's ``illegal'' mine.

Fifty years ago, when Alex's dad moved Grampa out of his valley, he
boarded up and buried the entrance of Grampa's ``illegal'' coal mine.
Grampa didn't start the mine. It probably started years before when
rogue miners crossed over from North Dakota. Even today, if you Google
abandoned coal mines, you'll find hundreds on the North Dakota side but
none across the border in Montana. Geology doesn't care about borders.
Coal seams don't check maps. After watching Dr.~Rebecca's YouTube video,
Alex realized governments would start ``securing and rationing'' coal
mines worldwide, and lucky for them, Grampa's mine didn't appear on
Google or USGS geological survey maps.

Alex wasn't sure why everyone called the mine ``illegal.'' Alex's dad
blamed it on ``rancher ecology\emph{.'' If it's not a cow or a rancher,
shoot it.} Shooting mines is stupid, even for cowboys, but he could
appreciate keeping livestock, including the idiot two-legged variety,
out of open mines.

It didn't take them long to find the hillside depression marking the
mine entrance. They quickly dug out the loose fill covering a thick,
oil-stained wooden door. The door wasn't locked, but opening it required
crowbars. The door didn't have hinges, so they dragged it forward and
set it to one side. Before entering the mine, Alex fetched a carbon
monoxide detector and a handful of head strap lights from the trailer.

``Where are the bike helmets? I'm not banging my head.'' Digging the
helmets out of the trailer took a few more minutes.

Affixing LED head strap lights to their foreheads, they carefully
entered the mine. The shaft receding into the hillside measured under
two meters high and more than one meter wide. Alex had to duck to avoid
hitting his head, and poor Doug had to lean way forward. Seven meters
into the shaft, the tunnel branched to the right. The secondary shaft
went about three meters and ended in a small pit filled with junk and
crushed rock. It looked like miners dumped waste rock and trash in this
side pit. It didn't make sense. Maybe ``illegal'' miners used the waste
pit to stay out of sight.

The mine shaft was dry and sat above the water table of the nearby pond.
They were impressed with the nice shaft support work. Somebody had
invested a lot of time neatly shoring up the shaft with rail ties. In
roughly thirty meters, the mine widened into an oval-shaped room about
four meters wide. Further on, the shaft bent slightly to the right and
narrowed again. They followed the re-narrowed shaft to its end some
thirty meters on. It ended in the coal seam. The seam wasn't big, about
a meter thick. An old metal pan with half a dozen dull steel pikes and
chisels sat beneath the working face.

To reassure himself, Alex fetched a sledgehammer and spent a few minutes
pounding pikes into the seam. The coal was hard, but it broke quickly.
Freeing enough to keep warm would be hard work, and he had no idea how
much further the seam ran. It already seemed to be petering out. Mining
would be their end-times exercise program.

After checking the mine, Doug backed the truck and trailer past the
derelict house as close to the mine as possible. For the rest of the
day, they methodically emptied the trailer and truck with the
wheelbarrow. They put everything on plastic tarps in the mine, sorting
items as they unloaded. Building materials went close to where they
planned to use them. Food items went nearby. Tools, propane tanks, and
generators also went to handy spots. They had no problems fitting things
in. Mine shafts have lots of storage space.

After unloading, they set up a small backpack tent in the secondary side
shaft. It was wide enough for the small tent. Exhausted, they squeezed
into the tent and fell asleep within minutes of zipping up their
sleeping bags.

The following day, they went through the mine, mounting LED lanterns,
thermometers, carbon monoxide detectors, and fire alarms on the mine's
support shoring. The thermometer nearest the coal seam face read 49
degrees Fahrenheit: the year-round average temperature of this spot in
Montana. That value would drop as the outside temperatures plunged, and
the rate at which it dropped would determine how long they could hold
on. Fetching his log notebook, Alex recorded the seam temperature. He'd
track it, day by day, for as long as he could.

With everything unpacked and the mine prepped for work, they dealt with
their truck problem.

``We can't leave the truck parked here. This will only work if we stay
out of sight.''

Doug agreed, ``Yeah, parking a big ass truck and trailer beside your
secret lair is lame-ass Bond villainy.''

The rutted road they drove in on connected with a ranch road about seven
kilometers to the north. Three or four kilometers further to the east,
past a few rut road turnoffs, they had driven by a small chain and snow
removal turnout. They decided to leave the truck and trailer there.
They'd remove the vehicle's license plates and ensure nothing in the
truck led back to them. It wasn't an optimal solution, but anyone
inspecting the truck faced a puzzle. Competent deep-state operatives
with access to motor vehicle VIN databases and land deed registries
would easily infer their probable presence in the valley, but average
roadside dumbasses and local cops probably wouldn't. Besides, the
authorities had bigger problems, like keeping their own fat rears alive.

``I'll dump the truck if you stay here and keep watch,'' Alex said.

Alex waited until an hour before sunset to drive the truck and trailer
out of the valley to the turnout. At the turnout, he removed license
plates, checked the truck and trailer for rental receipts, and calmly
watched the half-sized Moon set. The Moon hung twice its normal distance
from Earth. It looked strange and beautiful. When he could no longer
make out landscape details, he turned on his night vision monocular and
returned to the mine. Without the Moon, it became too dark for anyone
without light amplification to see him. It was Alex's first time using a
high-quality light amplification device. His walk back to the mine took
most of the night because he couldn't help looking at the sky with the
monocular. The amplified star-filled sky overwhelmed him.


%\end{document}

