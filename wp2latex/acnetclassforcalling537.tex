%% uncomment to list all files in log
%\listfiles

\documentclass[12pt]{report}

\usepackage{fontspec}

%\setmainfont[Scale=MatchLowercase]{Lucida Bright}
%\setmonofont{FreeMono}
%\setmonofont{Source Code Pro}
\setmonofont[Scale=MatchLowercase]{Ubuntu Mono}

\usepackage[headings]{fullpage}

% national use characters 
%\usepackage{inputenc}

% ams mathematical symbols
\usepackage{amsmath,amssymb}

% added to support pandoc highlighting
\usepackage{microtype}

\usepackage{makeidx}

% add index and bibliographies to table of contents
\usepackage[nottoc]{tocbibind}

% postscript courier and times in place of cm fonts
%\usepackage{courier}
%\usepackage{times}

% extended coloring
\usepackage{color}
\usepackage[table,dvipsnames]{xcolor}
\usepackage{colortbl}

% advanced date formating
\usepackage{datetime}

%support pandoc code highlighting
\usepackage{fancyvrb}
\DefineShortVerb[commandchars=\\\{\}]{\|}
\DefineVerbatimEnvironment{Highlighting}{Verbatim}{commandchars=\\\{\}}
% Add ',fontsize=\small' for more characters per line

%tango style colors
% \usepackage{framed}
% \definecolor{shadecolor}{RGB}{255,255,255}
% \newenvironment{Shaded}{\begin{snugshade}}{\end{snugshade}}
% \newcommand{\KeywordTok}[1]{\textcolor[rgb]{0.13,0.29,0.53}{\textbf{{#1}}}}
% \newcommand{\DataTypeTok}[1]{\textcolor[rgb]{0.13,0.29,0.53}{{#1}}}
% \newcommand{\DecValTok}[1]{\textcolor[rgb]{0.00,0.00,0.81}{{#1}}}
% \newcommand{\BaseNTok}[1]{\textcolor[rgb]{0.00,0.00,0.81}{{#1}}}
% \newcommand{\FloatTok}[1]{\textcolor[rgb]{0.00,0.00,0.81}{{#1}}}
% \newcommand{\CharTok}[1]{\textcolor[rgb]{0.31,0.60,0.02}{{#1}}}
% \newcommand{\StringTok}[1]{\textcolor[rgb]{0.31,0.60,0.02}{{#1}}}
% \newcommand{\CommentTok}[1]{\textcolor[rgb]{0.56,0.35,0.01}{\textit{{#1}}}}
% \newcommand{\OtherTok}[1]{\textcolor[rgb]{0.56,0.35,0.01}{{#1}}}
% \newcommand{\AlertTok}[1]{\textcolor[rgb]{0.94,0.16,0.16}{{#1}}}
% \newcommand{\FunctionTok}[1]{\textcolor[rgb]{0.00,0.00,0.00}{{#1}}}
% \newcommand{\RegionMarkerTok}[1]{{#1}}
% \newcommand{\ErrorTok}[1]{\textbf{{#1}}}
% \newcommand{\NormalTok}[1]{{#1}}

%espresso style colors
% \usepackage{framed}
% \definecolor{shadecolor}{RGB}{42,33,28}
% \newenvironment{Shaded}{\begin{snugshade}}{\end{snugshade}}
% \newcommand{\KeywordTok}[1]{\textcolor[rgb]{0.26,0.66,0.93}{\textbf{{#1}}}}
% \newcommand{\DataTypeTok}[1]{\textcolor[rgb]{0.74,0.68,0.62}{\underline{{#1}}}}
% \newcommand{\DecValTok}[1]{\textcolor[rgb]{0.27,0.67,0.26}{{#1}}}
% \newcommand{\BaseNTok}[1]{\textcolor[rgb]{0.27,0.67,0.26}{{#1}}}
% \newcommand{\FloatTok}[1]{\textcolor[rgb]{0.27,0.67,0.26}{{#1}}}
% \newcommand{\CharTok}[1]{\textcolor[rgb]{0.02,0.61,0.04}{{#1}}}
% \newcommand{\StringTok}[1]{\textcolor[rgb]{0.02,0.61,0.04}{{#1}}}
% \newcommand{\CommentTok}[1]{\textcolor[rgb]{0.00,0.40,1.00}{\textit{{#1}}}}
% \newcommand{\OtherTok}[1]{\textcolor[rgb]{0.74,0.68,0.62}{{#1}}}
% \newcommand{\AlertTok}[1]{\textcolor[rgb]{1.00,1.00,0.00}{{#1}}}
% \newcommand{\FunctionTok}[1]{\textcolor[rgb]{1.00,0.58,0.35}{\textbf{{#1}}}}
% \newcommand{\RegionMarkerTok}[1]{\textcolor[rgb]{0.74,0.68,0.62}{{#1}}}
% \newcommand{\ErrorTok}[1]{\textcolor[rgb]{0.74,0.68,0.62}{\textbf{{#1}}}}
% \newcommand{\NormalTok}[1]{\textcolor[rgb]{0.74,0.68,0.62}{{#1}}}

%kete style colors
% \newenvironment{Shaded}{}{}
% \newcommand{\KeywordTok}[1]{\textbf{{#1}}}
% \newcommand{\DataTypeTok}[1]{\textcolor[rgb]{0.50,0.00,0.00}{{#1}}}
% \newcommand{\DecValTok}[1]{\textcolor[rgb]{0.00,0.00,1.00}{{#1}}}
% \newcommand{\BaseNTok}[1]{\textcolor[rgb]{0.00,0.00,1.00}{{#1}}}
% \newcommand{\FloatTok}[1]{\textcolor[rgb]{0.50,0.00,0.50}{{#1}}}
% \newcommand{\CharTok}[1]{\textcolor[rgb]{1.00,0.00,1.00}{{#1}}}
% \newcommand{\StringTok}[1]{\textcolor[rgb]{0.87,0.00,0.00}{{#1}}}
% \newcommand{\CommentTok}[1]{\textcolor[rgb]{0.50,0.50,0.50}{\textit{{#1}}}}
% \newcommand{\OtherTok}[1]{{#1}}
% \newcommand{\AlertTok}[1]{\textcolor[rgb]{0.00,1.00,0.00}{\textbf{{#1}}}}
% \newcommand{\FunctionTok}[1]{\textcolor[rgb]{0.00,0.00,0.50}{{#1}}}
% \newcommand{\RegionMarkerTok}[1]{{#1}}
% \newcommand{\ErrorTok}[1]{\textcolor[rgb]{1.00,0.00,0.00}{\textbf{{#1}}}}
% \newcommand{\NormalTok}[1]{{#1}}
%end pandoc code hacks

% jodliterate colors
\usepackage{color}
\definecolor{shadecolor}{RGB}{248,248,248}
% j control structures 
\definecolor{keywcolor}{rgb}{0.13,0.29,0.53}
% j explicit arguments x y m n u v
\definecolor{datacolor}{rgb}{0.13,0.29,0.53}
% j numbers - all types see j.xml
\definecolor{decvcolor}{rgb}{0.00,0.00,0.81}
\definecolor{basencolor}{rgb}{0.00,0.00,0.81}
\definecolor{floatcolor}{rgb}{0.00,0.00,0.81}
% j local assignments
\definecolor{charcolor}{rgb}{0.31,0.60,0.02}
\definecolor{stringcolor}{rgb}{0.31,0.60,0.02}
\definecolor{commentcolor}{rgb}{0.56,0.35,0.01}
% primitive adverbs and conjunctions
%\definecolor{othercolor}{rgb}{0.56,0.35,0.01}   
\definecolor{othercolor}{RGB}{0,0,255}
% global assignments
\definecolor{alertcolor}{rgb}{0.94,0.16,0.16}
% primitive J verbs and noun names
\definecolor{funccolor}{rgb}{0.00,0.00,0.00}    

\usepackage{framed}
\newenvironment{Shaded}{}{}
\newcommand{\KeywordTok}[1]{\textcolor{keywcolor}{\textbf{{#1}}}}
\newcommand{\DataTypeTok}[1]{\textcolor{datacolor}{{#1}}}
%\newcommand{\DecValTok}[1]{\textcolor{decvcolor}{{#1}}}
\newcommand{\DecValTok}[1]{{#1}} 
\newcommand{\BaseNTok}[1]{\textcolor{basencolor}{{#1}}}
\newcommand{\FloatTok}[1]{\textcolor{floatcolor}{{#1}}}
\newcommand{\CharTok}[1]{\textcolor{charcolor}{\textbf{{#1}}}}
\newcommand{\StringTok}[1]{\textcolor{stringcolor}{{#1}}}
\newcommand{\CommentTok}[1]{\textcolor{commentcolor}{\textit{{#1}}}}
\newcommand{\OtherTok}[1]{\textcolor{othercolor}{{#1}}} 
\newcommand{\AlertTok}[1]{\textcolor{alertcolor}{\textbf{{#1}}}}
%\newcommand{\FunctionTok}[1]{\textcolor{funccolor}{{#1}}}
\newcommand{\FunctionTok}[1]{{#1}}
\newcommand{\RegionMarkerTok}[1]{{#1}}
\newcommand{\ErrorTok}[1]{\textbf{{#1}}}
\newcommand{\NormalTok}[1]{{#1}}

% headers and footers
\usepackage{fancyhdr}
\pagestyle{fancy}

\fancyhead{}
\fancyfoot{}

%\fancyhead[LE,RO]{\slshape \rightmark}
%\fancyhead[LO,RE]{\slshape \leftmark}
\fancyfoot[C]{\thepage}
%\headrulewidth 0.4pt
%\footrulewidth 0 pt

%\addtolength{\headheight}{\baselineskip}

%\lfoot{\emph{Analyze the Data not the Drivel}}
%\rfoot{\emph{\today}}

% subfigure handles figures that contain subfigures
%\usepackage{color,graphicx,subfigure,sidecap}
\usepackage{graphicx,sidecap}
\usepackage{subfigure}
\graphicspath{{./inclusions/}}

% floatflt provides for text wrapping around small figures and tables
\usepackage{floatflt}

% tweak caption formats 
\usepackage{caption} 
\usepackage{sidecap}
%\usepackage{subcaption} % not compatible with subfigure

\usepackage{rotating} % flip tables sideways

% complex footnotes
%\usepackage{bigfoot}

% weird logos \XeLaTeX
\usepackage{metalogo}

% source code listings
\usepackage{listings}

% long tables
% \usepackage{longtable}

\newcommand{\HRule}{\rule{\linewidth}{0.5mm}}

% map LaTeX cross references into PDF cross references
\usepackage[
            %dvips,
            colorlinks,
            linkcolor=blue,
            citecolor=blue,
            urlcolor=blue,   % magenta, cyan default        
            pdfauthor={John D. Baker},
            pdftitle={Analyze the Data not the Drivel},
            pdfsubject={Blog},
            pdfcreator={MikTeX+LaTeXe with hyperref package},
            pdfkeywords={blog,wordpress},
            ]{hyperref}
           
% custom colors
\definecolor{CodeBackGround}{cmyk}{0.0,0.0,0,0.05}    % light gray
\definecolor{CodeComment}{rgb}{0,0.50,0.00}           % dark green {0,0.45,0.08}
\definecolor{TableStripes}{gray}{0.9}                 % odd/even background in tables

\lstdefinelanguage{bat}
{morekeywords={echo,title,pushd,popd,setlocal,endlocal,off,if,not,exist,set,goto,pause},
sensitive=True,
morecomment=[l]{rem}
}

\lstdefinelanguage{jdoc}
{
morekeywords={},
otherkeywords={assert.,break.,continue.,for.,do.,if.,else.,elseif.,return.,select.,end.
,while.,whilst.,throw.,catch.,catchd.,catcht.,try.,case.,fcase.},
sensitive=True,
morecomment=[l]{NB.},
morestring=[b]',
morestring=[d]',
}

% latex size ordering - can never remember it
% \tiny
% \scriptsize
% \footnotesize
% \small
% \normalsize
% \large
% \Large
% \LARGE
% \huge
% \Huge
 
% listings package settings  
\lstset{%
  language=jdoc,                                % j document settings
  basicstyle=\ttfamily\footnotesize,            
  keywordstyle=\bfseries\color{keywcolor}\footnotesize,
  identifierstyle=\color{black},
  commentstyle=\slshape\color{CodeComment},     % colored slanted comments
  stringstyle=\color{red}\ttfamily,
  showstringspaces=false,                       
  %backgroundcolor=\color{CodeBackGround},       
  frame=single,                                
  framesep=1pt,                                 
  framerule=0.8pt,                             
  rulecolor=\color{CodeBackGround},   
  showspaces=false,
  %columns=fullflexible,
  %numbers=left,
  %numberstyle=\footnotesize,
  %numbersep=9pt,
  tabsize=2,
  showtabs=false,
  captionpos=b
  breaklines=true,                              
  breakindent=5pt                              
}

\lstdefinelanguage{JavaScript}{
  keywords={typeof, new, true, false, catch, function, return, null, catch, switch, var, if, in, while, do, else, case, break},
  ndkeywords={class, export, boolean, throw, implements, import, this},
  ndkeywordstyle=\color{darkgray}\bfseries,
  sensitive=false,
  comment=[l]{//},
  morecomment=[s]{/*}{*/},
  morestring=[b]',
  morestring=[b]"
}

% C# settings
\lstdefinestyle{sharpc}{
language=[Sharp]C,
basicstyle=\ttfamily\scriptsize, 
keywordstyle=\bfseries\color{keywcolor}\scriptsize,
framerule=0pt
}

% for source code listing longer than two use smaller font
\lstdefinestyle{smallersource}{
basicstyle=\ttfamily\scriptsize, 
keywordstyle=\bfseries\color{keywcolor}\scriptsize,
framerule=0pt
}

\lstdefinestyle{resetdefaults}{
language=jdoc,
basicstyle=\ttfamily\footnotesize,  
keywordstyle=\bfseries\color{keywcolor}\footnotesize,                                                               
framerule=0.8pt 
}

% APL UTF8 code points listed for lstlisting processing
\makeatletter
\lst@InputCatcodes
\def\lst@DefEC{%
 \lst@CCECUse \lst@ProcessLetter
  ^^80^^81^^82^^83^^84^^85^^86^^87^^88^^89^^8a^^8b^^8c^^8d^^8e^^8f%
  ^^90^^91^^92^^93^^94^^95^^96^^97^^98^^99^^9a^^9b^^9c^^9d^^9e^^9f%
  ^^a0^^a1^^a2^^a3^^a4^^a5^^a6^^a7^^a8^^a9^^aa^^ab^^ac^^ad^^ae^^af%
  ^^b0^^b1^^b2^^b3^^b4^^b5^^b6^^b7^^b8^^b9^^ba^^bb^^bc^^bd^^be^^bf%
  ^^c0^^c1^^c2^^c3^^c4^^c5^^c6^^c7^^c8^^c9^^ca^^cb^^cc^^cd^^ce^^cf%
  ^^d0^^d1^^d2^^d3^^d4^^d5^^d6^^d7^^d8^^d9^^da^^db^^dc^^dd^^de^^df%
  ^^e0^^e1^^e2^^e3^^e4^^e5^^e6^^e7^^e8^^e9^^ea^^eb^^ec^^ed^^ee^^ef%
  ^^f0^^f1^^f2^^f3^^f4^^f5^^f6^^f7^^f8^^f9^^fa^^fb^^fc^^fd^^fe^^ff%
  ^^^^20ac^^^^0153^^^^0152%
  ^^^^20a7^^^^2190^^^^2191^^^^2192^^^^2193^^^^2206^^^^2207^^^^220a%
  ^^^^2218^^^^2228^^^^2229^^^^222a^^^^2235^^^^223c^^^^2260^^^^2261%
  ^^^^2262^^^^2264^^^^2265^^^^2282^^^^2283^^^^2296^^^^22a2^^^^22a3%
  ^^^^22a4^^^^22a5^^^^22c4^^^^2308^^^^230a^^^^2336^^^^2337^^^^2339%
  ^^^^233b^^^^233d^^^^233f^^^^2340^^^^2342^^^^2347^^^^2348^^^^2349%
  ^^^^234b^^^^234e^^^^2350^^^^2352^^^^2355^^^^2357^^^^2359^^^^235d%
  ^^^^235e^^^^235f^^^^2361^^^^2362^^^^2363^^^^2364^^^^2365^^^^2368%
  ^^^^236a^^^^236b^^^^236c^^^^2371^^^^2372^^^^2373^^^^2374^^^^2375%
  ^^^^2377^^^^2378^^^^237a^^^^2395^^^^25af^^^^25ca^^^^25cb%  
  ^^00}
\lst@RestoreCatcodes
\makeatother

% custom lengths used within minipages
\newcommand{\minindent}{17pt}


\makeindex

\begin{document}

\subsection*{\href{https://bakerjd99.wordpress.com/2010/05/28/a-c-net-class-for-calling-j/}{A C\# .Net Class for calling J}}
\addcontentsline{toc}{subsection}{A C\# .Net Class for calling J}


\noindent\emph{Posted: 28 May 2010 14:48:18}
\vspace{6pt}

%\href{http://www.jsoftware.com/}{\includegraphics{bitmap_thumb.png}}
\captionsetup[floatingfigure]{labelformat=empty}
\begin{floatingfigure}[l]{0.12\textwidth}
\centering
\href{http://www.jsoftware.com/}{\includegraphics[width=0.10\textwidth]{bitmap_thumb.png}}
\label{fig:537X0}
\end{floatingfigure}One of my \href{http://www.jsoftware.com/}{favorite programming tools is
J}. In skilled hands \emph{J is a spear in a world of bent spoons.} In my
day job I rarely encounter programming problems that cannot be brutally
dispatched with a few dozen lines of J. Most accomplished
\href{http://www.lulu.com/product/paperback/j-for-c-programmers/4669553}{J
programmers} laud the
\href{http://portal.acm.org/citation.cfm?id=508562}{elegance and power
of the language} and frequently remark on how learning J changed the way
the way they think about programming. If you are intrigued please take a
look but a word of caution. Learning J is like learning Calculus. Don't
expect to progress beyond the trivial without a
\href{http://norvig.com/21-days.html}{substantial intellectual effort}
on your behalf.

J has many strengths but current implementations also have some serious
shortcomings.

\begin{enumerate}
\item
  J's GUI user interface tools are primitive compared to what you find
  in Microsoft Visual Studio or Java Eclipse environments.
\item
  It is difficult to use J in mixed language projects. J can make C
  style API calls and the Windows version sports a COM interface. Both
  of these call mechanisms are solid and work well but the C API
  struggles with many C++ libraries and COM is now considered a legacy
  technology in Microsoft .Net circles.
\item
  .Net executables can call J but J cannot \emph{easily} call .Net
  executables.
\item
  There are very few useful J libraries.
  \href{http://pypi.python.org/pypi/}{Python} programmers often find
  complete solutions to their problems in libraries. With J you often
  end up writing your own libraries This fosters an independent frame of
  mind at the expense of productivity.
\item
  Packaging J solutions is largely \emph{ad hoc} and platform dependent.
  It's not like C\# or Java where you get a nice self-contained install
  package.
\end{enumerate}
To deal with J's deficiencies I cheat and use other languages and tools.
This is getting the best of both worlds or
\href{http://www.stlyrics.com/lyrics/hannahmontana/bestofbothworlds.htm}{Miley
Cryrus}'ing it! Miley Cryus'ing in Windows environments leads straight
to .Net and the premier .Net programming language C\#. J is not a .Net
language but J can be called from C\# by COM or by C style API calls.
This
\href{http://cid-f964330e36001519.skydrive.live.com/self.aspx/Public/cs/JServer10may27.zip}{JServer}
class uses COM. JServer was inspired by Alex Rufon's
\href{http://202.67.223.49/jwiki/Guides/J\%20CSharp}{J Wiki essay} but
differs in that all JServer calls are strongly typed. There is no point
in using strongly typed languages like C\# if you are constantly casting
objects. Use the type checking Luke!

The following
\href{http://cid-f964330e36001519.skydrive.live.com/self.aspx/Public/cs/JServerTest10may27.zip}{JServerTest}
code snippet shows JServer calls.

%[sourcecode language="csharp" wraplines="false" gutter="false" autolinks="false"]
\lstset{style=sharpc, label=lst:scr537X0}
\begin{lstlisting}
using System;
using System.Collections.Generic;
using System.Text;
using System.Data;
using JServerClass;  // add reference to JServer.exe

namespace JServerTest
{
  class Program
  {
    static void Main(string[] args)
    {
      // create new j exe server - load only the j profile
      JServer js = new JServer(JServer.JScriptType.OnlyProfile);

      // make server visible/invisible/visible
      js.jShowServer = true;
      System.Threading.Thread.Sleep(200);
      js.jShowServer = false;
      System.Threading.Thread.Sleep(200);
      js.jShowServer = true;

      // do tests - create j nouns that interface can fetch

      js.jDo("18!:5 ''"); // should be in base locale

      // atoms - rank 0
      js.jDo("byteAtom=. 'A'");
      js.jDo("boolAtom=. 1");
      js.jDo("intAtom=. 42");
      js.jDo("doubleAtom=. 1x1"); // e in j notation

      // arrays of rank 1 and 2 - higher rank arrays are not
      // explicitly supported by the C# interface
      js.jDo("boolArray=. ?50#2");
      js.jDo("intArray=. 10 10$?100#10");
      js.jDo("doubleArray=. 5 10$(?50#50) % ?50#50");
      js.jDo("byteArray=. 20 30$'goaheadbyteme'");
      js.jDo("stringArray=. ;:'not by the hair of my chinny chin chin'");
      js.jDo("stringArray2=. 11 7$stringArray");

      // get tests - fetch j nouns - get and set are C# overloads

      // rank 0 gets
      byte byteAtom;
      js.jGet("byteAtom", out byteAtom);
      bool boolAtom;
      js.jGet("boolAtom", out boolAtom);
      int intAtom;
      js.jGet("intAtom", out intAtom);
      double doubleAtom;
      js.jGet("doubleAtom", out doubleAtom);

      // rank 1 and/or 2 gets
      bool[] boolArray;
      js.jGet("boolArray", out boolArray);
      int[,] intArray;
      js.jGet("intArray", out intArray);
      double[,] doubleArray;
      js.jGet("doubleArray", out doubleArray);
      byte[,] byteArray;
      js.jGet("byteArray", out byteArray);
      string[] stringArray;
      js.jGet("stringArray", out stringArray);
      string[,] stringArray2;
      js.jGet("stringArray2", out stringArray2);

      // set tests - set copies of fetched nouns in j and test
      js.jSet("byteAtomC", byteAtom);
      js.jDo("byteAtom -: byteAtomC");   // should be identical - result 1
      js.jSet("boolAtomC", boolAtom);
      js.jDo("boolAtomC -: boolAtomC");
      js.jSet("intAtomC", intAtom);
      js.jDo("intAtomC -: intAtom");
      js.jSet("doubleAtomC", doubleAtom);
      js.jDo("doubleAtomC -: doubleAtom");

      js.jSet("boolArrayC", boolArray);
      js.jDo("boolArrayC -: boolArray");
      js.jSet("intArrayC", intArray);
      js.jDo("intArrayC -: intArray");
      js.jSet("doubleArrayC", doubleArray);
      js.jDo("doubleArrayC -: doubleArray");
      js.jSet("byteArrayC", byteArray);
      js.jDo("byteArrayC -: byteArray");
      js.jSet("stringArrayC", stringArray);
      js.jDo("stringArrayC -: stringArray");

      // no overload for this case - it's not
      // as important as the rank 1 case
      //js.jSet("stringArray2C", stringArray2);

      // Datatable's are supported by the interface
      // as they can be quickly displayed and manipulated
      // in DataGridView objects
      DataTable dt = new DataTable();
      dt.Clear();

      // generate test j datatable representation - the interface
      // loads a support locale CSsrv that contains the necessary
      // j verbs to support these representations
      js.jDo("DTTEST=: testDataTable_CSsrv_ >:?100 10");

      // get the datatable
      dt = js.jGet("DTTEST");

      // set a copy of the datatable back in j and test equivalence
      // slight differences in floating number character formats
      // are reconciled with (testDataTableMatch)
      js.jSet("DTTESTC", dt);
      js.jDo("DTTESTC testDataTableMatch_CSsrv_ DTTEST");

      // wait five seconds before shutting
      // down so user can view the j exe server
      System.Threading.Thread.Sleep(5000);
    }
  }
}
\end{lstlisting}
\lstset{style=resetdefaults}

%\captionsetup[floatingfigure]{labelformat=empty}
%\begin{figure}[htbp]
%\begin{floatingfigure}[l]{0.25\textwidth}
%\centering
%\includegraphics[width=0.23\textwidth]{bitmap_thumb.png}
%\caption{~~~IMCAPTION~~~}
%\label{fig:537X0}
%\end{floatingfigure}
%\end{figure}



%\end{document}