%% bm.pdf preamble - material merged from previous preamble and current pandoc preamable output
% NOTE: float placement required changes to the source files referenced by bm.tex
% May 28, 2020
%
% Use lualatex to compile - test with MiKTeX 2.9

% uncomment to list all files in log
%\listfiles

\documentclass[12pt]{report}


\usepackage{fontspec}

%\setmainfont[Scale=MatchLowercase]{Lucida Bright}
%\setmonofont{FreeMono}
%\setmonofont{Source Code Pro}
\setmonofont[Scale=MatchLowercase]{Ubuntu Mono}

% short snippets of asian languages
\newfontfamily\myAsian{Noto Serif TC Medium}

\usepackage[headings]{fullpage}

% national use characters 
%\usepackage{inputenc}

% ams mathematical symbols
\usepackage{amsmath,amssymb}

% added to support pandoc highlighting
\usepackage{microtype}

\usepackage{makeidx}

% add index and bibliographies to table of contents
\usepackage[nottoc]{tocbibind}

% postscript courier and times in place of cm fonts
%\usepackage{courier}
%\usepackage{times}

% extended coloring
\usepackage{color}
\usepackage[table,dvipsnames]{xcolor}
\usepackage{colortbl}

% advanced date formating
\usepackage{datetime}

%support pandoc code highlighting
\usepackage{fancyvrb}

% \DefineShortVerb[commandchars=\\\{\}]{\|}
% \DefineVerbatimEnvironment{Highlighting}{Verbatim}{commandchars=\\\{\}}
% % Add ',fontsize=\small' for more characters per line

% tango style colors
% \usepackage{framed}
% \definecolor{shadecolor}{RGB}{255,255,255}
% \newenvironment{Shaded}{\begin{snugshade}}{\end{snugshade}}
% \newcommand{\KeywordTok}[1]{\textcolor[rgb]{0.13,0.29,0.53}{\textbf{{#1}}}}
% \newcommand{\DataTypeTok}[1]{\textcolor[rgb]{0.13,0.29,0.53}{{#1}}}
% \newcommand{\DecValTok}[1]{\textcolor[rgb]{0.00,0.00,0.81}{{#1}}}
% \newcommand{\BaseNTok}[1]{\textcolor[rgb]{0.00,0.00,0.81}{{#1}}}
% \newcommand{\FloatTok}[1]{\textcolor[rgb]{0.00,0.00,0.81}{{#1}}}
% \newcommand{\CharTok}[1]{\textcolor[rgb]{0.31,0.60,0.02}{{#1}}}
% \newcommand{\StringTok}[1]{\textcolor[rgb]{0.31,0.60,0.02}{{#1}}}
% \newcommand{\CommentTok}[1]{\textcolor[rgb]{0.56,0.35,0.01}{\textit{{#1}}}}
% \newcommand{\OtherTok}[1]{\textcolor[rgb]{0.56,0.35,0.01}{{#1}}}
% \newcommand{\AlertTok}[1]{\textcolor[rgb]{0.94,0.16,0.16}{{#1}}}
% \newcommand{\FunctionTok}[1]{\textcolor[rgb]{0.00,0.00,0.00}{{#1}}}
% \newcommand{\RegionMarkerTok}[1]{{#1}}
% \newcommand{\ErrorTok}[1]{\textbf{{#1}}}
% \newcommand{\NormalTok}[1]{{#1}}

% %espresso style colors
% \usepackage{framed}
% \definecolor{shadecolor}{RGB}{42,33,28}
% \newenvironment{Shaded}{\begin{snugshade}}{\end{snugshade}}
% \newcommand{\KeywordTok}[1]{\textcolor[rgb]{0.26,0.66,0.93}{\textbf{{#1}}}}
% \newcommand{\DataTypeTok}[1]{\textcolor[rgb]{0.74,0.68,0.62}{\underline{{#1}}}}
% \newcommand{\DecValTok}[1]{\textcolor[rgb]{0.27,0.67,0.26}{{#1}}}
% \newcommand{\BaseNTok}[1]{\textcolor[rgb]{0.27,0.67,0.26}{{#1}}}
% \newcommand{\FloatTok}[1]{\textcolor[rgb]{0.27,0.67,0.26}{{#1}}}
% \newcommand{\CharTok}[1]{\textcolor[rgb]{0.02,0.61,0.04}{{#1}}}
% \newcommand{\StringTok}[1]{\textcolor[rgb]{0.02,0.61,0.04}{{#1}}}
% \newcommand{\CommentTok}[1]{\textcolor[rgb]{0.00,0.40,1.00}{\textit{{#1}}}}
% \newcommand{\OtherTok}[1]{\textcolor[rgb]{0.74,0.68,0.62}{{#1}}}
% \newcommand{\AlertTok}[1]{\textcolor[rgb]{1.00,1.00,0.00}{{#1}}}
% \newcommand{\FunctionTok}[1]{\textcolor[rgb]{1.00,0.58,0.35}{\textbf{{#1}}}}
% \newcommand{\RegionMarkerTok}[1]{\textcolor[rgb]{0.74,0.68,0.62}{{#1}}}
% \newcommand{\ErrorTok}[1]{\textcolor[rgb]{0.74,0.68,0.62}{\textbf{{#1}}}}
% \newcommand{\NormalTok}[1]{\textcolor[rgb]{0.74,0.68,0.62}{{#1}}}

% %kete style colors
% \newenvironment{Shaded}{}{}
% \newcommand{\KeywordTok}[1]{\textbf{{#1}}}
% \newcommand{\DataTypeTok}[1]{\textcolor[rgb]{0.50,0.00,0.00}{{#1}}}
% \newcommand{\DecValTok}[1]{\textcolor[rgb]{0.00,0.00,1.00}{{#1}}}
% \newcommand{\BaseNTok}[1]{\textcolor[rgb]{0.00,0.00,1.00}{{#1}}}
% \newcommand{\FloatTok}[1]{\textcolor[rgb]{0.50,0.00,0.50}{{#1}}}
% \newcommand{\CharTok}[1]{\textcolor[rgb]{1.00,0.00,1.00}{{#1}}}
% \newcommand{\StringTok}[1]{\textcolor[rgb]{0.87,0.00,0.00}{{#1}}}
% \newcommand{\CommentTok}[1]{\textcolor[rgb]{0.50,0.50,0.50}{\textit{{#1}}}}
% \newcommand{\OtherTok}[1]{{#1}}
% \newcommand{\AlertTok}[1]{\textcolor[rgb]{0.00,1.00,0.00}{\textbf{{#1}}}}
% \newcommand{\FunctionTok}[1]{\textcolor[rgb]{0.00,0.00,0.50}{{#1}}}
% \newcommand{\RegionMarkerTok}[1]{{#1}}
% \newcommand{\ErrorTok}[1]{\textcolor[rgb]{1.00,0.00,0.00}{\textbf{{#1}}}}
% \newcommand{\NormalTok}[1]{{#1}}
% %end pandoc code hacks

% jodliterate colors
\usepackage{color}
\definecolor{shadecolor}{RGB}{248,248,248}
% j control structures 
\definecolor{keywcolor}{rgb}{0.13,0.29,0.53}
% j explicit arguments x y m n u v
\definecolor{datacolor}{rgb}{0.13,0.29,0.53}
% j numbers - all types see j.xml
\definecolor{decvcolor}{rgb}{0.00,0.00,0.81}
\definecolor{basencolor}{rgb}{0.00,0.00,0.81}
\definecolor{floatcolor}{rgb}{0.00,0.00,0.81}
% j local assignments
\definecolor{charcolor}{rgb}{0.31,0.60,0.02}
\definecolor{stringcolor}{rgb}{0.31,0.60,0.02}
\definecolor{commentcolor}{rgb}{0.56,0.35,0.01}
% primitive adverbs and conjunctions
%\definecolor{othercolor}{rgb}{0.56,0.35,0.01}   
\definecolor{othercolor}{RGB}{0,0,255}
% global assignments
\definecolor{alertcolor}{rgb}{0.94,0.16,0.16}
% primitive J verbs and noun names
\definecolor{funccolor}{rgb}{0.00,0.00,0.00}

% custom colors
\definecolor{CodeBackGround}{cmyk}{0.0,0.0,0,0.05}    % light gray
\definecolor{CodeComment}{rgb}{0,0.50,0.00}           % dark green {0,0.45,0.08}
\definecolor{TableStripes}{gray}{0.9}                 % odd/even background in tables

% Colors for the hyperref package
\definecolor{urlcolor}{rgb}{0,.145,.698}
\definecolor{linkcolor}{rgb}{.71,0.21,0.01}
\definecolor{citecolor}{rgb}{.12,.54,.11}

% % Exact colors from NB
\definecolor{incolor}{HTML}{303F9F}
\definecolor{outcolor}{HTML}{D84315}
\definecolor{cellborder}{HTML}{CFCFCF}
\definecolor{cellbackground}{HTML}{F7F7F7}

% % ANSI colors
\definecolor{ansi-black}{HTML}{3E424D}
\definecolor{ansi-black-intense}{HTML}{282C36}
\definecolor{ansi-red}{HTML}{E75C58}
\definecolor{ansi-red-intense}{HTML}{B22B31}
\definecolor{ansi-green}{HTML}{00A250}
\definecolor{ansi-green-intense}{HTML}{007427}
\definecolor{ansi-yellow}{HTML}{DDB62B}
\definecolor{ansi-yellow-intense}{HTML}{B27D12}
\definecolor{ansi-blue}{HTML}{208FFB}
\definecolor{ansi-blue-intense}{HTML}{0065CA}
\definecolor{ansi-magenta}{HTML}{D160C4}
\definecolor{ansi-magenta-intense}{HTML}{A03196}
\definecolor{ansi-cyan}{HTML}{60C6C8}
\definecolor{ansi-cyan-intense}{HTML}{258F8F}
\definecolor{ansi-white}{HTML}{C5C1B4}
\definecolor{ansi-white-intense}{HTML}{A1A6B2}
\definecolor{ansi-default-inverse-fg}{HTML}{FFFFFF}
\definecolor{ansi-default-inverse-bg}{HTML}{000000}
    

% \usepackage{framed}
% \newenvironment{Shaded}{}{}
% \newcommand{\KeywordTok}[1]{\textcolor{keywcolor}{\textbf{{#1}}}}
% \newcommand{\DataTypeTok}[1]{\textcolor{datacolor}{{#1}}}
% %\newcommand{\DecValTok}[1]{\textcolor{decvcolor}{{#1}}}
% \newcommand{\DecValTok}[1]{{#1}} 
% \newcommand{\BaseNTok}[1]{\textcolor{basencolor}{{#1}}}
% \newcommand{\FloatTok}[1]{\textcolor{floatcolor}{{#1}}}
% \newcommand{\CharTok}[1]{\textcolor{charcolor}{\textbf{{#1}}}}
% \newcommand{\StringTok}[1]{\textcolor{stringcolor}{{#1}}}
% \newcommand{\CommentTok}[1]{\textcolor{commentcolor}{\textit{{#1}}}}
% \newcommand{\OtherTok}[1]{\textcolor{othercolor}{{#1}}} 
% \newcommand{\AlertTok}[1]{\textcolor{alertcolor}{\textbf{{#1}}}}
% %\newcommand{\FunctionTok}[1]{\textcolor{funccolor}{{#1}}}
% \newcommand{\FunctionTok}[1]{{#1}}
% \newcommand{\RegionMarkerTok}[1]{{#1}}
% \newcommand{\ErrorTok}[1]{\textbf{{#1}}}
% \newcommand{\NormalTok}[1]{{#1}}

% The default LaTeX title has an obnoxious amount of whitespace. By default,
% titling removes some of it. It also provides customization options.
\usepackage{titling}

% headers and footers
\usepackage{fancyhdr}
%\pagestyle{fancy}
\pagestyle{plain}

\fancyhead{}
\fancyfoot{}

%\fancyhead[LE,RO]{\slshape \rightmark}
%\fancyhead[LO,RE]{\slshape \leftmark}
\fancyfoot[C]{\thepage}
%\headrulewidth 0.4pt
%\footrulewidth 0 pt

%\addtolength{\headheight}{\baselineskip}

%\lfoot{\emph{Analyze the Data not the Drivel}}
%\rfoot{\emph{\today}}

% subfigure handles figures that contain subfigures
%\usepackage{color,graphicx,subfigure,sidecap}
\usepackage{graphicx,sidecap}
\usepackage{subfigure}
\graphicspath{{./inclusions/}}

% floatflt provides for text wrapping around small figures and tables
\usepackage{floatflt}

% tweak caption formats 
\usepackage{caption} 
\usepackage{sidecap}
%\usepackage{subcaption} % not compatible with subfigure

\usepackage{rotating} % flip tables sideways

% complex footnotes
%\usepackage{bigfoot}

% weird logos \XeLaTeX
\usepackage{metalogo}

\newcommand{\HRule}{\rule{\linewidth}{0.5mm}}

\usepackage[breakable]{tcolorbox}

\usepackage{parskip} % Stop auto-indenting (to mimic markdown behaviour)
    
% Basic figure setup, for now with no caption control since it's done
% automatically by Pandoc (which extracts ![](path) syntax from Markdown).
\usepackage{graphicx}

%\DeclareCaptionFormat{nocaption}{}
%\captionsetup{format=nocaption,aboveskip=0pt,belowskip=0pt}

\usepackage[Export]{adjustbox} % Used to constrain images to a maximum size
\adjustboxset{max size={0.9\linewidth}{0.9\paperheight}}
\usepackage{float}

%\floatplacement{figure}{H} % forces figures to be placed at the correct location

\usepackage{xcolor} % Allow colors to be defined
\usepackage{enumerate} % Needed for markdown enumerations to work
\usepackage{geometry} % Used to adjust the document margins

%\usepackage{amsmath} % Equations
%\usepackage{amssymb} % Equations

\usepackage{textcomp} % defines textquotesingle

% Hack from http://tex.stackexchange.com/a/47451/13684:
\AtBeginDocument{%
	\def\PYZsq{\textquotesingle}% Upright quotes in Pygmentized code
}

\usepackage{upquote} % Upright quotes for verbatim code
\usepackage{eurosym} % defines \euro
\usepackage[mathletters]{ucs} % Extended unicode (utf-8) support

%\usepackage{fancyvrb} % verbatim replacement that allows latex

\usepackage{grffile} % extends the file name processing of package graphics 
					 % to support a larger range
					 
\makeatletter % fix for grffile with XeLaTeX
\def\Gread@@xetex#1{%
  \IfFileExists{"\Gin@base".bb}%
  {\Gread@eps{\Gin@base.bb}}%
  {\Gread@@xetex@aux#1}%
}
\makeatother

% The hyperref package gives us a pdf with properly built
% internal navigation ('pdf bookmarks' for the table of contents,
% internal cross-reference links, web links for URLs, etc.)
\usepackage{hyperref}
% The default LaTeX title has an obnoxious amount of whitespace. By default,
% titling removes some of it. It also provides customization options.
\usepackage{titling}
\usepackage{longtable} % longtable support required by pandoc >1.10
\usepackage{booktabs}  % table support for pandoc > 1.12.2
\usepackage[inline]{enumitem} % IRkernel/repr support (it uses the enumerate* environment)
\usepackage[normalem]{ulem} % ulem is needed to support strikethroughs (\sout)
							% normalem makes italics be italics, not underlines
\usepackage{mathrsfs}

% commands and environments needed by pandoc snippets
% extracted from the output of `pandoc -s`
\providecommand{\tightlist}{%
  \setlength{\itemsep}{0pt}\setlength{\parskip}{0pt}}
  
\DefineVerbatimEnvironment{Highlighting}{Verbatim}{commandchars=\\\{\}}
% Add ',fontsize=\small' for more characters per line
\newenvironment{Shaded}{}{}
\newcommand{\KeywordTok}[1]{\textcolor[rgb]{0.00,0.44,0.13}{\textbf{{#1}}}}
\newcommand{\DataTypeTok}[1]{\textcolor[rgb]{0.56,0.13,0.00}{{#1}}}
\newcommand{\DecValTok}[1]{\textcolor[rgb]{0.25,0.63,0.44}{{#1}}}
\newcommand{\BaseNTok}[1]{\textcolor[rgb]{0.25,0.63,0.44}{{#1}}}
\newcommand{\FloatTok}[1]{\textcolor[rgb]{0.25,0.63,0.44}{{#1}}}
\newcommand{\CharTok}[1]{\textcolor[rgb]{0.25,0.44,0.63}{{#1}}}
\newcommand{\StringTok}[1]{\textcolor[rgb]{0.25,0.44,0.63}{{#1}}}
\newcommand{\CommentTok}[1]{\textcolor[rgb]{0.38,0.63,0.69}{\textit{{#1}}}}
\newcommand{\OtherTok}[1]{\textcolor[rgb]{0.00,0.44,0.13}{{#1}}}
\newcommand{\AlertTok}[1]{\textcolor[rgb]{1.00,0.00,0.00}{\textbf{{#1}}}}
\newcommand{\FunctionTok}[1]{\textcolor[rgb]{0.02,0.16,0.49}{{#1}}}
\newcommand{\RegionMarkerTok}[1]{{#1}}
\newcommand{\ErrorTok}[1]{\textcolor[rgb]{1.00,0.00,0.00}{\textbf{{#1}}}}
\newcommand{\NormalTok}[1]{{#1}}

% Additional commands for more recent versions of Pandoc
\newcommand{\ConstantTok}[1]{\textcolor[rgb]{0.53,0.00,0.00}{{#1}}}
\newcommand{\SpecialCharTok}[1]{\textcolor[rgb]{0.25,0.44,0.63}{{#1}}}
\newcommand{\VerbatimStringTok}[1]{\textcolor[rgb]{0.25,0.44,0.63}{{#1}}}
\newcommand{\SpecialStringTok}[1]{\textcolor[rgb]{0.73,0.40,0.53}{{#1}}}
\newcommand{\ImportTok}[1]{{#1}}
\newcommand{\DocumentationTok}[1]{\textcolor[rgb]{0.73,0.13,0.13}{\textit{{#1}}}}
\newcommand{\AnnotationTok}[1]{\textcolor[rgb]{0.38,0.63,0.69}{\textbf{\textit{{#1}}}}}
\newcommand{\CommentVarTok}[1]{\textcolor[rgb]{0.38,0.63,0.69}{\textbf{\textit{{#1}}}}}
\newcommand{\VariableTok}[1]{\textcolor[rgb]{0.10,0.09,0.49}{{#1}}}
\newcommand{\ControlFlowTok}[1]{\textcolor[rgb]{0.00,0.44,0.13}{\textbf{{#1}}}}
\newcommand{\OperatorTok}[1]{\textcolor[rgb]{0.40,0.40,0.40}{{#1}}}
\newcommand{\BuiltInTok}[1]{{#1}}
\newcommand{\ExtensionTok}[1]{{#1}}
\newcommand{\PreprocessorTok}[1]{\textcolor[rgb]{0.74,0.48,0.00}{{#1}}}
\newcommand{\AttributeTok}[1]{\textcolor[rgb]{0.49,0.56,0.16}{{#1}}}
\newcommand{\InformationTok}[1]{\textcolor[rgb]{0.38,0.63,0.69}{\textbf{\textit{{#1}}}}}
\newcommand{\WarningTok}[1]{\textcolor[rgb]{0.38,0.63,0.69}{\textbf{\textit{{#1}}}}}

% Define a nice break command that doesn't care if a line doesn't already exist.
\def\br{\hspace*{\fill} \\* }
% Math Jax compatibility definitions
\def\gt{>}
\def\lt{<}
\let\Oldtex\TeX
\let\Oldlatex\LaTeX
\renewcommand{\TeX}{\textrm{\Oldtex}}
\renewcommand{\LaTeX}{\textrm{\Oldlatex}}
 
% Pygments definitions
\makeatletter
\def\PY@reset{\let\PY@it=\relax \let\PY@bf=\relax%
    \let\PY@ul=\relax \let\PY@tc=\relax%
    \let\PY@bc=\relax \let\PY@ff=\relax}
\def\PY@tok#1{\csname PY@tok@#1\endcsname}
\def\PY@toks#1+{\ifx\relax#1\empty\else%
    \PY@tok{#1}\expandafter\PY@toks\fi}
\def\PY@do#1{\PY@bc{\PY@tc{\PY@ul{%
    \PY@it{\PY@bf{\PY@ff{#1}}}}}}}
\def\PY#1#2{\PY@reset\PY@toks#1+\relax+\PY@do{#2}}

\expandafter\def\csname PY@tok@w\endcsname{\def\PY@tc##1{\textcolor[rgb]{0.73,0.73,0.73}{##1}}}
\expandafter\def\csname PY@tok@c\endcsname{\let\PY@it=\textit\def\PY@tc##1{\textcolor[rgb]{0.25,0.50,0.50}{##1}}}
\expandafter\def\csname PY@tok@cp\endcsname{\def\PY@tc##1{\textcolor[rgb]{0.74,0.48,0.00}{##1}}}
\expandafter\def\csname PY@tok@k\endcsname{\let\PY@bf=\textbf\def\PY@tc##1{\textcolor[rgb]{0.00,0.50,0.00}{##1}}}
\expandafter\def\csname PY@tok@kp\endcsname{\def\PY@tc##1{\textcolor[rgb]{0.00,0.50,0.00}{##1}}}
\expandafter\def\csname PY@tok@kt\endcsname{\def\PY@tc##1{\textcolor[rgb]{0.69,0.00,0.25}{##1}}}
\expandafter\def\csname PY@tok@o\endcsname{\def\PY@tc##1{\textcolor[rgb]{0.40,0.40,0.40}{##1}}}
\expandafter\def\csname PY@tok@ow\endcsname{\let\PY@bf=\textbf\def\PY@tc##1{\textcolor[rgb]{0.67,0.13,1.00}{##1}}}
\expandafter\def\csname PY@tok@nb\endcsname{\def\PY@tc##1{\textcolor[rgb]{0.00,0.50,0.00}{##1}}}
\expandafter\def\csname PY@tok@nf\endcsname{\def\PY@tc##1{\textcolor[rgb]{0.00,0.00,1.00}{##1}}}
\expandafter\def\csname PY@tok@nc\endcsname{\let\PY@bf=\textbf\def\PY@tc##1{\textcolor[rgb]{0.00,0.00,1.00}{##1}}}
\expandafter\def\csname PY@tok@nn\endcsname{\let\PY@bf=\textbf\def\PY@tc##1{\textcolor[rgb]{0.00,0.00,1.00}{##1}}}
\expandafter\def\csname PY@tok@ne\endcsname{\let\PY@bf=\textbf\def\PY@tc##1{\textcolor[rgb]{0.82,0.25,0.23}{##1}}}
\expandafter\def\csname PY@tok@nv\endcsname{\def\PY@tc##1{\textcolor[rgb]{0.10,0.09,0.49}{##1}}}
\expandafter\def\csname PY@tok@no\endcsname{\def\PY@tc##1{\textcolor[rgb]{0.53,0.00,0.00}{##1}}}
\expandafter\def\csname PY@tok@nl\endcsname{\def\PY@tc##1{\textcolor[rgb]{0.63,0.63,0.00}{##1}}}
\expandafter\def\csname PY@tok@ni\endcsname{\let\PY@bf=\textbf\def\PY@tc##1{\textcolor[rgb]{0.60,0.60,0.60}{##1}}}
\expandafter\def\csname PY@tok@na\endcsname{\def\PY@tc##1{\textcolor[rgb]{0.49,0.56,0.16}{##1}}}
\expandafter\def\csname PY@tok@nt\endcsname{\let\PY@bf=\textbf\def\PY@tc##1{\textcolor[rgb]{0.00,0.50,0.00}{##1}}}
\expandafter\def\csname PY@tok@nd\endcsname{\def\PY@tc##1{\textcolor[rgb]{0.67,0.13,1.00}{##1}}}
\expandafter\def\csname PY@tok@s\endcsname{\def\PY@tc##1{\textcolor[rgb]{0.73,0.13,0.13}{##1}}}
\expandafter\def\csname PY@tok@sd\endcsname{\let\PY@it=\textit\def\PY@tc##1{\textcolor[rgb]{0.73,0.13,0.13}{##1}}}
\expandafter\def\csname PY@tok@si\endcsname{\let\PY@bf=\textbf\def\PY@tc##1{\textcolor[rgb]{0.73,0.40,0.53}{##1}}}
\expandafter\def\csname PY@tok@se\endcsname{\let\PY@bf=\textbf\def\PY@tc##1{\textcolor[rgb]{0.73,0.40,0.13}{##1}}}
\expandafter\def\csname PY@tok@sr\endcsname{\def\PY@tc##1{\textcolor[rgb]{0.73,0.40,0.53}{##1}}}
\expandafter\def\csname PY@tok@ss\endcsname{\def\PY@tc##1{\textcolor[rgb]{0.10,0.09,0.49}{##1}}}
\expandafter\def\csname PY@tok@sx\endcsname{\def\PY@tc##1{\textcolor[rgb]{0.00,0.50,0.00}{##1}}}
\expandafter\def\csname PY@tok@m\endcsname{\def\PY@tc##1{\textcolor[rgb]{0.40,0.40,0.40}{##1}}}
\expandafter\def\csname PY@tok@gh\endcsname{\let\PY@bf=\textbf\def\PY@tc##1{\textcolor[rgb]{0.00,0.00,0.50}{##1}}}
\expandafter\def\csname PY@tok@gu\endcsname{\let\PY@bf=\textbf\def\PY@tc##1{\textcolor[rgb]{0.50,0.00,0.50}{##1}}}
\expandafter\def\csname PY@tok@gd\endcsname{\def\PY@tc##1{\textcolor[rgb]{0.63,0.00,0.00}{##1}}}
\expandafter\def\csname PY@tok@gi\endcsname{\def\PY@tc##1{\textcolor[rgb]{0.00,0.63,0.00}{##1}}}
\expandafter\def\csname PY@tok@gr\endcsname{\def\PY@tc##1{\textcolor[rgb]{1.00,0.00,0.00}{##1}}}
\expandafter\def\csname PY@tok@ge\endcsname{\let\PY@it=\textit}
\expandafter\def\csname PY@tok@gs\endcsname{\let\PY@bf=\textbf}
\expandafter\def\csname PY@tok@gp\endcsname{\let\PY@bf=\textbf\def\PY@tc##1{\textcolor[rgb]{0.00,0.00,0.50}{##1}}}
\expandafter\def\csname PY@tok@go\endcsname{\def\PY@tc##1{\textcolor[rgb]{0.53,0.53,0.53}{##1}}}
\expandafter\def\csname PY@tok@gt\endcsname{\def\PY@tc##1{\textcolor[rgb]{0.00,0.27,0.87}{##1}}}
\expandafter\def\csname PY@tok@err\endcsname{\def\PY@bc##1{\setlength{\fboxsep}{0pt}\fcolorbox[rgb]{1.00,0.00,0.00}{1,1,1}{\strut ##1}}}
\expandafter\def\csname PY@tok@kc\endcsname{\let\PY@bf=\textbf\def\PY@tc##1{\textcolor[rgb]{0.00,0.50,0.00}{##1}}}
\expandafter\def\csname PY@tok@kd\endcsname{\let\PY@bf=\textbf\def\PY@tc##1{\textcolor[rgb]{0.00,0.50,0.00}{##1}}}
\expandafter\def\csname PY@tok@kn\endcsname{\let\PY@bf=\textbf\def\PY@tc##1{\textcolor[rgb]{0.00,0.50,0.00}{##1}}}
\expandafter\def\csname PY@tok@kr\endcsname{\let\PY@bf=\textbf\def\PY@tc##1{\textcolor[rgb]{0.00,0.50,0.00}{##1}}}
\expandafter\def\csname PY@tok@bp\endcsname{\def\PY@tc##1{\textcolor[rgb]{0.00,0.50,0.00}{##1}}}
\expandafter\def\csname PY@tok@fm\endcsname{\def\PY@tc##1{\textcolor[rgb]{0.00,0.00,1.00}{##1}}}
\expandafter\def\csname PY@tok@vc\endcsname{\def\PY@tc##1{\textcolor[rgb]{0.10,0.09,0.49}{##1}}}
\expandafter\def\csname PY@tok@vg\endcsname{\def\PY@tc##1{\textcolor[rgb]{0.10,0.09,0.49}{##1}}}
\expandafter\def\csname PY@tok@vi\endcsname{\def\PY@tc##1{\textcolor[rgb]{0.10,0.09,0.49}{##1}}}
\expandafter\def\csname PY@tok@vm\endcsname{\def\PY@tc##1{\textcolor[rgb]{0.10,0.09,0.49}{##1}}}
\expandafter\def\csname PY@tok@sa\endcsname{\def\PY@tc##1{\textcolor[rgb]{0.73,0.13,0.13}{##1}}}
\expandafter\def\csname PY@tok@sb\endcsname{\def\PY@tc##1{\textcolor[rgb]{0.73,0.13,0.13}{##1}}}
\expandafter\def\csname PY@tok@sc\endcsname{\def\PY@tc##1{\textcolor[rgb]{0.73,0.13,0.13}{##1}}}
\expandafter\def\csname PY@tok@dl\endcsname{\def\PY@tc##1{\textcolor[rgb]{0.73,0.13,0.13}{##1}}}
\expandafter\def\csname PY@tok@s2\endcsname{\def\PY@tc##1{\textcolor[rgb]{0.73,0.13,0.13}{##1}}}
\expandafter\def\csname PY@tok@sh\endcsname{\def\PY@tc##1{\textcolor[rgb]{0.73,0.13,0.13}{##1}}}
\expandafter\def\csname PY@tok@s1\endcsname{\def\PY@tc##1{\textcolor[rgb]{0.73,0.13,0.13}{##1}}}
\expandafter\def\csname PY@tok@mb\endcsname{\def\PY@tc##1{\textcolor[rgb]{0.40,0.40,0.40}{##1}}}
\expandafter\def\csname PY@tok@mf\endcsname{\def\PY@tc##1{\textcolor[rgb]{0.40,0.40,0.40}{##1}}}
\expandafter\def\csname PY@tok@mh\endcsname{\def\PY@tc##1{\textcolor[rgb]{0.40,0.40,0.40}{##1}}}
\expandafter\def\csname PY@tok@mi\endcsname{\def\PY@tc##1{\textcolor[rgb]{0.40,0.40,0.40}{##1}}}
\expandafter\def\csname PY@tok@il\endcsname{\def\PY@tc##1{\textcolor[rgb]{0.40,0.40,0.40}{##1}}}
\expandafter\def\csname PY@tok@mo\endcsname{\def\PY@tc##1{\textcolor[rgb]{0.40,0.40,0.40}{##1}}}
\expandafter\def\csname PY@tok@ch\endcsname{\let\PY@it=\textit\def\PY@tc##1{\textcolor[rgb]{0.25,0.50,0.50}{##1}}}
\expandafter\def\csname PY@tok@cm\endcsname{\let\PY@it=\textit\def\PY@tc##1{\textcolor[rgb]{0.25,0.50,0.50}{##1}}}
\expandafter\def\csname PY@tok@cpf\endcsname{\let\PY@it=\textit\def\PY@tc##1{\textcolor[rgb]{0.25,0.50,0.50}{##1}}}
\expandafter\def\csname PY@tok@c1\endcsname{\let\PY@it=\textit\def\PY@tc##1{\textcolor[rgb]{0.25,0.50,0.50}{##1}}}
\expandafter\def\csname PY@tok@cs\endcsname{\let\PY@it=\textit\def\PY@tc##1{\textcolor[rgb]{0.25,0.50,0.50}{##1}}}

\def\PYZbs{\char`\\}
\def\PYZus{\char`\_}
\def\PYZob{\char`\{}
\def\PYZcb{\char`\}}
\def\PYZca{\char`\^}
\def\PYZam{\char`\&}
\def\PYZlt{\char`\<}
\def\PYZgt{\char`\>}
\def\PYZsh{\char`\#}
\def\PYZpc{\char`\%}
\def\PYZdl{\char`\$}
\def\PYZhy{\char`\-}
\def\PYZsq{\char`\'}
\def\PYZdq{\char`\"}
\def\PYZti{\char`\~}
% for compatibility with earlier versions
\def\PYZat{@}
\def\PYZlb{[}
\def\PYZrb{]}
\makeatother

% For linebreaks inside Verbatim environment from package fancyvrb. 
\makeatletter
	\newbox\Wrappedcontinuationbox 
	\newbox\Wrappedvisiblespacebox 
	\newcommand*\Wrappedvisiblespace {\textcolor{red}{\textvisiblespace}} 
	\newcommand*\Wrappedcontinuationsymbol {\textcolor{red}{\llap{\tiny$\m@th\hookrightarrow$}}} 
	\newcommand*\Wrappedcontinuationindent {3ex } 
	\newcommand*\Wrappedafterbreak {\kern\Wrappedcontinuationindent\copy\Wrappedcontinuationbox} 
	% Take advantage of the already applied Pygments mark-up to insert 
	% potential linebreaks for TeX processing. 
	%        {, <, #, %, $, ' and ": go to next line. 
	%        _, }, ^, &, >, - and ~: stay at end of broken line. 
	% Use of \textquotesingle for straight quote. 
	\newcommand*\Wrappedbreaksatspecials {% 
		\def\PYGZus{\discretionary{\char`\_}{\Wrappedafterbreak}{\char`\_}}% 
		\def\PYGZob{\discretionary{}{\Wrappedafterbreak\char`\{}{\char`\{}}% 
		\def\PYGZcb{\discretionary{\char`\}}{\Wrappedafterbreak}{\char`\}}}% 
		\def\PYGZca{\discretionary{\char`\^}{\Wrappedafterbreak}{\char`\^}}% 
		\def\PYGZam{\discretionary{\char`\&}{\Wrappedafterbreak}{\char`\&}}% 
		\def\PYGZlt{\discretionary{}{\Wrappedafterbreak\char`\<}{\char`\<}}% 
		\def\PYGZgt{\discretionary{\char`\>}{\Wrappedafterbreak}{\char`\>}}% 
		\def\PYGZsh{\discretionary{}{\Wrappedafterbreak\char`\#}{\char`\#}}% 
		\def\PYGZpc{\discretionary{}{\Wrappedafterbreak\char`\%}{\char`\%}}% 
		\def\PYGZdl{\discretionary{}{\Wrappedafterbreak\char`\$}{\char`\$}}% 
		\def\PYGZhy{\discretionary{\char`\-}{\Wrappedafterbreak}{\char`\-}}% 
		\def\PYGZsq{\discretionary{}{\Wrappedafterbreak\textquotesingle}{\textquotesingle}}% 
		\def\PYGZdq{\discretionary{}{\Wrappedafterbreak\char`\"}{\char`\"}}% 
		\def\PYGZti{\discretionary{\char`\~}{\Wrappedafterbreak}{\char`\~}}% 
	} 
	% Some characters . , ; ? ! / are not pygmentized. 
	% This macro makes them "active" and they will insert potential linebreaks 
	\newcommand*\Wrappedbreaksatpunct {% 
		\lccode`\~`\.\lowercase{\def~}{\discretionary{\hbox{\char`\.}}{\Wrappedafterbreak}{\hbox{\char`\.}}}% 
		\lccode`\~`\,\lowercase{\def~}{\discretionary{\hbox{\char`\,}}{\Wrappedafterbreak}{\hbox{\char`\,}}}% 
		\lccode`\~`\;\lowercase{\def~}{\discretionary{\hbox{\char`\;}}{\Wrappedafterbreak}{\hbox{\char`\;}}}% 
		\lccode`\~`\:\lowercase{\def~}{\discretionary{\hbox{\char`\:}}{\Wrappedafterbreak}{\hbox{\char`\:}}}% 
		\lccode`\~`\?\lowercase{\def~}{\discretionary{\hbox{\char`\?}}{\Wrappedafterbreak}{\hbox{\char`\?}}}% 
		\lccode`\~`\!\lowercase{\def~}{\discretionary{\hbox{\char`\!}}{\Wrappedafterbreak}{\hbox{\char`\!}}}% 
		\lccode`\~`\/\lowercase{\def~}{\discretionary{\hbox{\char`\/}}{\Wrappedafterbreak}{\hbox{\char`\/}}}% 
		\catcode`\.\active
		\catcode`\,\active 
		\catcode`\;\active
		\catcode`\:\active
		\catcode`\?\active
		\catcode`\!\active
		\catcode`\/\active 
		\lccode`\~`\~ 	
	}
\makeatother

\let\OriginalVerbatim=\Verbatim
\makeatletter
\renewcommand{\Verbatim}[1][1]{%
	%\parskip\z@skip
	\sbox\Wrappedcontinuationbox {\Wrappedcontinuationsymbol}%
	\sbox\Wrappedvisiblespacebox {\FV@SetupFont\Wrappedvisiblespace}%
	\def\FancyVerbFormatLine ##1{\hsize\linewidth
		\vtop{\raggedright\hyphenpenalty\z@\exhyphenpenalty\z@
			\doublehyphendemerits\z@\finalhyphendemerits\z@
			\strut ##1\strut}%
	}%
	% If the linebreak is at a space, the latter will be displayed as visible
	% space at end of first line, and a continuation symbol starts next line.
	% Stretch/shrink are however usually zero for typewriter font.
	\def\FV@Space {%
		\nobreak\hskip\z@ plus\fontdimen3\font minus\fontdimen4\font
		\discretionary{\copy\Wrappedvisiblespacebox}{\Wrappedafterbreak}
		{\kern\fontdimen2\font}%
	}%
	
	% Allow breaks at special characters using \PYG... macros.
	\Wrappedbreaksatspecials
	% Breaks at punctuation characters . , ; ? ! and / need catcode=\active 	
	\OriginalVerbatim[#1,codes*=\Wrappedbreaksatpunct]%
}
\makeatother


% prompt
\makeatletter
\newcommand{\boxspacing}{\kern\kvtcb@left@rule\kern\kvtcb@boxsep}
\makeatother
\newcommand{\prompt}[4]{
	\ttfamily\llap{{\color{#2}[#3]:\hspace{3pt}#4}}\vspace{-\baselineskip}
}
    

% Prevent overflowing lines due to hard-to-break entities
\sloppy 

% Setup hyperref package
\hypersetup{
  breaklinks=true,  % so long urls are correctly broken across lines
  colorlinks=true,
  urlcolor=urlcolor,
  linkcolor=linkcolor,
  citecolor=citecolor,
  pdfauthor={John D. Baker},
  pdftitle={Analyze the Data not the Drivel},
  pdfsubject={Blog},
  pdfcreator={MikTeX+LaTeXe},
  pdfkeywords={blog,wordpress},
  }
  
% Slightly bigger margins than the latex defaults
% \geometry{verbose,tmargin=1in,bmargin=1in,lmargin=1in,rmargin=1in}  

%\usepackage{wrapfig}

% source code listings
\usepackage{listings}

\lstdefinelanguage{bat}
{morekeywords={echo,title,pushd,popd,setlocal,endlocal,off,if,not,exist,set,goto,pause},
sensitive=True,
morecomment=[l]{rem}
}

\lstdefinelanguage{jdoc}
{
morekeywords={},
otherkeywords={assert.,break.,continue.,for.,do.,if.,else.,elseif.,return.,select.,end.
,while.,whilst.,throw.,catch.,catchd.,catcht.,try.,case.,fcase.},
sensitive=True,
morecomment=[l]{NB.},
morestring=[b]',
morestring=[d]',
}

% latex size ordering - can never remember it
% \tiny
% \scriptsize
% \footnotesize
% \small
% \normalsize
% \large
% \Large
% \LARGE
% \huge
% \Huge
 
% listings package settings  
\lstset{%
  language=jdoc,                                % j document settings
  basicstyle=\ttfamily\footnotesize,            
  keywordstyle=\bfseries\color{keywcolor}\footnotesize,
  identifierstyle=\color{black},
  commentstyle=\slshape\color{CodeComment},     % colored slanted comments
  stringstyle=\color{red}\ttfamily,
  showstringspaces=false,                       
  %backgroundcolor=\color{CodeBackGround},       
  frame=single,                                
  framesep=1pt,                                 
  framerule=0.8pt,                             
  rulecolor=\color{CodeBackGround},   
  showspaces=false,
  %columns=fullflexible,
  %numbers=left,
  %numberstyle=\footnotesize,
  %numbersep=9pt,
  tabsize=2,
  showtabs=false,
  captionpos=b
  breaklines=true,                              
  breakindent=5pt                              
}

\lstdefinelanguage{JavaScript}{
  keywords={typeof, new, true, false, catch, function, return, null, catch, switch, var, if, in, while, do, else, case, break},
  ndkeywords={class, export, boolean, throw, implements, import, this},
  ndkeywordstyle=\color{darkgray}\bfseries,
  sensitive=false,
  comment=[l]{//},
  morecomment=[s]{/*}{*/},
  morestring=[b]',
  morestring=[b]"
}

% C# settings
\lstdefinestyle{sharpc}{
language=[Sharp]C,
basicstyle=\ttfamily\scriptsize, 
keywordstyle=\bfseries\color{keywcolor}\scriptsize,
framerule=0pt
}

% for source code listing longer than two use smaller font
\lstdefinestyle{smallersource}{
basicstyle=\ttfamily\scriptsize, 
keywordstyle=\bfseries\color{keywcolor}\scriptsize,
framerule=0pt
}

\lstdefinestyle{resetdefaults}{
language=jdoc,
basicstyle=\ttfamily\footnotesize,  
keywordstyle=\bfseries\color{keywcolor}\footnotesize,                                                               
framerule=0.8pt 
}

% APL UTF8 code points listed for lstlisting processing
\makeatletter
\lst@InputCatcodes
\def\lst@DefEC{%
 \lst@CCECUse \lst@ProcessLetter
  ^^80^^81^^82^^83^^84^^85^^86^^87^^88^^89^^8a^^8b^^8c^^8d^^8e^^8f%
  ^^90^^91^^92^^93^^94^^95^^96^^97^^98^^99^^9a^^9b^^9c^^9d^^9e^^9f%
  ^^a0^^a1^^a2^^a3^^a4^^a5^^a6^^a7^^a8^^a9^^aa^^ab^^ac^^ad^^ae^^af%
  ^^b0^^b1^^b2^^b3^^b4^^b5^^b6^^b7^^b8^^b9^^ba^^bb^^bc^^bd^^be^^bf%
  ^^c0^^c1^^c2^^c3^^c4^^c5^^c6^^c7^^c8^^c9^^ca^^cb^^cc^^cd^^ce^^cf%
  ^^d0^^d1^^d2^^d3^^d4^^d5^^d6^^d7^^d8^^d9^^da^^db^^dc^^dd^^de^^df%
  ^^e0^^e1^^e2^^e3^^e4^^e5^^e6^^e7^^e8^^e9^^ea^^eb^^ec^^ed^^ee^^ef%
  ^^f0^^f1^^f2^^f3^^f4^^f5^^f6^^f7^^f8^^f9^^fa^^fb^^fc^^fd^^fe^^ff%
  ^^^^20ac^^^^0153^^^^0152%
  ^^^^20a7^^^^2190^^^^2191^^^^2192^^^^2193^^^^2206^^^^2207^^^^220a%
  ^^^^2218^^^^2228^^^^2229^^^^222a^^^^2235^^^^223c^^^^2260^^^^2261%
  ^^^^2262^^^^2264^^^^2265^^^^2282^^^^2283^^^^2296^^^^22a2^^^^22a3%
  ^^^^22a4^^^^22a5^^^^22c4^^^^2308^^^^230a^^^^2336^^^^2337^^^^2339%
  ^^^^233b^^^^233d^^^^233f^^^^2340^^^^2342^^^^2347^^^^2348^^^^2349%
  ^^^^234b^^^^234e^^^^2350^^^^2352^^^^2355^^^^2357^^^^2359^^^^235d%
  ^^^^235e^^^^235f^^^^2361^^^^2362^^^^2363^^^^2364^^^^2365^^^^2368%
  ^^^^236a^^^^236b^^^^236c^^^^2371^^^^2372^^^^2373^^^^2374^^^^2375%
  ^^^^2377^^^^2378^^^^237a^^^^2395^^^^25af^^^^25ca^^^^25cb%  
  ^^00}
\lst@RestoreCatcodes
\makeatother

% custom lengths used within minipages
\newcommand{\minindent}{17pt}

\makeindex

\begin{document}

\subsection*{\href{http://analyzethedatanotthedrivel.org/2025/03/16/gonggone-gone-parts-5-6/}{Gonggone Gone --- Parts 5 \& 6}}
\addcontentsline{toc}{subsection}{Gonggone Gone --- Parts 5 \& 6}


\noindent\emph{Posted: 16 Mar 2025 19:06:58}
\vspace{6pt}

%\subsection{heat boxes}\label{heat-boxes}
\begin{center}\large\textbf{-- \emph{heat boxes} --}\normalsize\end{center}

In the morning, they installed infrared battery-powered motion detectors
in the mine shaft. The alarms would warn them if anyone entered the mine
while they were inside working. As a further precaution, Alex carried
his little Scheel's 9mm pistol in his snap-up overall pockets.

``Paranoid much?'' Doug teased.

``Hey, it's a statistical fact that mine shafts are catnip to
end-of-the-world zombies.''

Before doing any work, Alex fetched his logbook and quickly sketched
what he had in mind. Doug approved with some caveats.

``That's a very long exhaust line.''

``That's what the little fans are for.''

``Ah, I wondered about them. Well, let's get started.''

They set up one propane-powered portable generator near the mine's
entrance. Using a power saw, they fashioned two twelve-foot poles by
cutting and screwing some two-by-fours together. Lying the poles on
their side about four meters apart, Alex attached one of the short-wave
radio's external dipole antennas and two webcams. One cam faced north
and the other south. On the northern pole, Alex nailed in plastic
streamers to gauge windspeeds. Then, they gently raised the poles and
dropped them in anchor holes, aligning them north and south on the solar
noon meridian.

Alex said as they raised the poles, ``I'm glad we heard about the sun
shadow yesterday. We had to set up the antenna anyway. This kills two
birds with one stone.''

By timing when the southern pole's shadow hit the northern pole, they'd
use the Earth's rotation to measure the astronomical unit secant angle.
By simply timing a shadow, they'd know the Earth-Sun distance.

At solar noon, they checked pole alignment, packed the anchor pits with
crushed rock, and reinforced the poles with guide ropes. Using insulated
wiring, Alex attached the pole's antenna to one of the short-wave
radios. The radio worked beautifully. Relieved to hear the outside
world, they diligently listened to the radio from them on. To listen
inside the mine, they ran a cable along the mine shoring and connected
it to the outside antenna.

After checking the radio, Alex attached the pole webcams to his laptop
with active USB extension cables. The webcams also worked beautifully.

To finish the Sun angle and antenna poles, Alex fastened an entire spool
of insulated copper wiring to the northern pole. Above the wire spool,
he securely attached an outdoor thermometer. After checking he could see
the temperature gauge from the webcam view on his laptop, he unraveled
enough wire to lead back into the mine. On their apocalypse shopping
run, they had picked up a variety of wall-mounted thermometers, but the
coldest gauges they could find only went to -60 degrees Fahrenheit: not
cold enough! How do you measure temperatures of -100 degrees Celsius and
colder? Then Alex remembered a bit of physics. Electrical resistance
decreases with temperature. They would use the multimeter to measure the
resistance of the mounted copper spool. With some calibration, the
resistance would yield an estimate of outdoor temperatures. Even better,
pure copper metal does not exhibit superconductivity at cryogenic
temperatures. They could measure extremely low temperatures as long as
they stayed alive.

``We'll have to do a little temperature and Ohm curve fitting as things
cool down.''

Next, they dug a series of meter-deep latrine pits west of the mine
entrance. They would slowly fill the pits with feces, urine, and mine
tailings. They arranged the pits in a north-south line so they wouldn't
have to walk past potentially snow-filled holes when disposing of waste.
They planned to fill the pits furthest from the mine first and slowly
move closer and closer to the entrance as it got colder and colder. To
find the active latrine pit, even when buried in deep snow, they marked
it with a two-by-four pole.

After digging the pits, Doug cut plywood and two-by-fours to build a
small, sturdy, open-ended box. Exhibiting saw skills Alex didn't know he
had, Doug cut a beautiful toilet lid-shaped hole in the box lid.

``I not going to spend my apocalypse squatting over freezing holes.''
They both laughed.

For the next two weeks, they worked nonstop. After measuring the oval
bend about thirty meters into the mine, Alex figured it was big enough
to build an oblong, heavily insulated box large enough to hold the
tunnel tent, one of the stainless-steel hot tent stoves, and a small
crawl-through storage pantry. By keeping the box narrow, they could
squeeze around it and get to the coal seam deeper in the mine. The gap
would also let air flow in the mine shaft.

They laid out cinder blocks around the oval. The \emph{inner heat box},
as they now called it, would sit on the blocks, leaving an air gap under
the box. The mine floor was uneven. They did their best to level things,
but it was hopeless. The floor was going to be up and down. Cutting
two-by-fours, they quickly put together a frame on the cinder blocks.
Using six-by-one fence slabs, they fashioned a rude floor on which they
stapled plastic vapor barrier wrap. Flipping the floor frame, they
covered the vapor barrier with bubble wrap. Then they stuffed the frame
with rockwool, which they stapled down with more fence slats. With one
layer done, they built another frame on top and repeated the process.
This resulted in a floor insulated by over a foot of rockwool and a few
more inches of bubble wrap. They finished the floor by cutting plywood
panels and stapling them down. Finally, they covered the plywood with
yoga mats. Yoga mats make excellent soft insulation.

They did the same for the walls and ceiling. It was the first time they
had worked with quality power tools. Power staplers and screwdrivers
speed things up.

``I should have done this for my day job.''

``What day job? \emph{Call of Duty.''}

``Be nice. Or you'll be end-timing alone.''

They finished their crude oblong box in less time than expected and
covered the outside with tarps, which they stapled in place. To control
possible moisture problems, Alex made basic wall vents. The vents were
small, well-insulated boxes they could put in or out of the walls. In
the LED lantern light, the oblong box looked like an underground
homeless encampment.

The hardest bit of the entire build was installing the little
stainless-steel stove and its ducting. Alex ran the exhaust pipe inside
the top of the box to capture as much heat as possible. Fresh air would
be a problem. Normally, hot tent stoves draw air from inside large
tents, which can be opened to let in outside air. This wouldn't work
underground, so they ran two 8-inch metal ducts, intake and exhaust,
from the inner heat box along the top of the shaft to the mine entrance.
The long ducting run lowered the already low shaft ceiling.

Doug complained, ``I'll be stooping for the rest of my life.''

Alex almost reminded him he wouldn't have this problem for long, but
they had stopped talking about the inevitable future.

They finished the intake and exhaust ducts by wrapping the first fifteen
meters from the box with aluminum foil. The foil served as a radiator.
It transferred heat from the exhaust pipe to the intake pipe and the
surrounding mine shaft air. Alex and Doug aimed to keep as much stove
heat in the mine as possible. Having anticipated air flow problems back
in his Meridian bedroom, Alex installed T-joint ducts every five meters.
They sealed the T-joints with aluminum foil. If needed, they could peel
back the foil and insert small battery-powered ventilation fans.

When they finished their heat box, they set up the tunnel tent inside
it. There wasn't a lot of room in the box. The floor and ceiling were
about one and a half meters apart, enough space to comfortably sit up
while leaning against a wall but not enough to stand. To move around,
they'd crawl.

The box was longer and broader than tall: about eleven and two meters,
respectively.

There was just enough room to leave an air gap between the tent and box
walls. On one side of the tunnel tent, the gap was large enough to build
foot-wide shelves from the floor to the ceiling, which they packed with
canned foods and other freeze-sensitive items.

They covered the outside of the tent with one of the king-size electric
blankets. Inside, they covered the tent's floor with multiple high
R-factor sleeping mats. On top of the high R-factor mats, they set up
the folding camping cots. The cots opened a six-inch air gap between the
floor mats and the cot frame. They packed the gaps with cheap sleeping
bags and placed their highest quality cold weather sleeping bags on the
cots. With the cots in place, they were left with about a meter of free
space at the end of the tent. Alex stored notebooks, Doug's Manga, his
laptop, their books, two eight-plug power bars, and the two
1500-watt-hour powerpacks there.

The powerpacks took up more space than he liked but keeping the
powerpack batteries warm was almost as important as keeping themselves
warm. They finished decorating the tent by hanging four rechargeable
camping LED lanterns from its support struts. The tent was advertised as
three-person, but when packed, there wasn't a lot of space.

After furnishing the tent, they screwed dozens of hooks into the walls
beside the stove and in the gap between the curve of the tent and the
ceiling. On the hooks, they hung pans, towels, knives, extra LED
headlamps, the ceramic heater, and other odds and ends. Finishing up
with their cozy box, Alex deployed all thirty-kilogram silica gel
canisters. Putting some in the tent and some outside on the yoga mat
floor. The canisters would control humidity in the box.

With the inner heat box done, Alex wanted to start building the
``airlock'' box near the mine entrance, but Doug objected.

``Once the airlock box is built, it will be a pain to haul things into
the mine. We need to collect as much wood as possible from Grampa's
house. We can use the big planks for shoring and burn the rest.''

Alex agreed.

For the next four days, they sledgehammered and pulled apart what
remained of Grampa's old frame house. Before raising the house ruins,
Alex moved the house trail camera. He put it on the southern antenna
pole. The webcams already covered the south, but the webcams only worked
when plugged into his laptop. The trail cameras were always on. Most of
the wood in Grampa's house was rotten and covered with asbestos
shingles. Hammering raised so much dust they both wore COVID masks while
bashing. Some floorboard planks were in good shape; they dragged those
back to the mine and stored them in the little waste pit side shaft they
were now calling their ``closet.'' Smaller broken boards were stacked in
loose piles in the waste pit. There were many nails in the wood. At
first, they tried to yank the nails out but soon gave up.

As Doug unloaded a wood-filled wheelbarrow in the closet, he said,
``We'll have to be careful when cutting this stuff. Wouldn't want
tetanus to spoil the end of the world.''

After salvaging what they could from Grampa's house, they leveled what
remained. As the last partially upright wall fell, Alex said, ``I've
been worrying about Grampa's house attracting people looking for things
to burn. Now, with a good snowstorm, nobody will ever see it.''

After demolishing Grampa's house, they started working on the airlock
box.

The airlock box was not as well insulated and much smaller. It sat about
four meters back from the mine entrance door, which they had reworked.
The original heavy wooden door was hard to handle. So, they broke it
apart. They removed most of the planks but left a few to frame plywood,
which they stapled down. The new door was much lighter. They opened it
from the inside by swinging it inwards with ropes and suspending it on
hooks drilled into the first shaft rail tie. Air gaps around the outer
door and the mine shaft walls were sealed with Velcro and cut-up yoga
mats. Alex called the space between the outer shaft door and the airlock
heat box their ``garage.'' They kept the wheelbarrow and other tools
they needed for outside work in the garage.

The airlock box behind the outer door measured about a meter wide and
two and a half meters long. It had insulated front and back doors with
pool noodles filling the gaps between the box and mine shaft walls.
Opening the outer mine door and the airlock box doors allowed fresh air
into the mine. They could limit the loss of warmer air when going
outside with one door open and the other closed. It pained Alex to think
of losing precious heat.

They set up the second stainless steel hot tent stove inside. The second
stove connected to the same ducts used to vent the first stove via
jury-rigged T-joint valves. It was a squeeze getting past the stove to
where they fashioned bare floor-to-ceiling shelves to hold canned foods,
pots and pans, water purification kits, towels, and other utensils. The
airlock box was taller than the inner box, but neither could stand up
without ducking. From now on, they would be stooping and ducking
everywhere in the mine. Alex hoped to do most of their cooking, water
filtration, and other ``hot'' activities in the airlock box. It would
also serve as an observer warm-up room.

In front of the airlock box, in the ``garage,'' Alex built a small
plywood Styrofoam-lined box to store his astrograph. At the bottom of
the scope box, he placed two thick sheets of Styrofoam to set the little
telescope with its tripod on. Alex covered the instrument in the box
with an electric blanket and a cheap sleeping bag. On freezing nights,
he would turn on the electric scope blanket to warm it up before use.
He'd spent many freezing nights fiddling with painful-to-touch telescope
controls, and the coming days and nights would be beyond cold.

%\subsection{radio and video}\label{radio-and-video}
\begin{center}\large\textbf{-- \emph{radio and video} --}\normalsize\end{center}

While Alex and Doug worked on the boxes, they listened to the radio,
paying close attention to nearby FM and AM stations. The first week
after runaway, as everyone called it, people panicked. They looted
stores. Rioted over food, fuel, and travel restrictions. Governments
imposed stern curfews, warned about ``hoarding,'' and implemented
rationing schemes. Troops backed up local police. Governments closed
borders. Banks closed and reopened with limited hours and strictly
enforced withdrawal limits to curtail black market transactions.

There was much bad news but also, inconceivably, good news. Many
pointless wars around the world just stopped. It's hard to motivate
fighters when everyone is going to die anyway. Long-sought and
previously impossible-to-negotiate ceasefires just spontaneously took
effect. When the shooting stopped, many combatants just walked away,
sometimes behind the very commanders who had been threatening to shoot
them days before. It was wild.

On the short-wave bands, it was more of the same. About two weeks after
runaway, things calmed down. The Sun appeared only slightly smaller in
the sky. You needed some astronomical expertise to see the difference.
And the weather, while cooler than usual, hardly reached cryogenic
levels. Maybe this whole thing was just another World Economic
Forum-style COVID conspiracy to strip people of their rights and
implement the repressive measures global commies are always seeking. Of
course, the tides were still absent, and the Moon was so far away it
looked like a bigger Venus. How the global deep state managed that was
glossed over on conspiracy radio, and naturally, gun sales reached
all-time highs.

``What are they going to shoot when it's 200 below?'' Doug asked.

``I'll take each other for whatever the fuck I can eat, Alex.''

They also amused themselves by watching trail camera videos. Every few
days, they checked the cameras and swapped memory cards. Before going to
sleep, Alex copied video files onto his laptop, which they both watched.
Doug paid close attention to the north-facing camera overlooking the
rutted road, the most likely route strangers would take to enter the
valley. Some nights, the camera recorded truck headlights turning on a
ranch road about seven kilometers away. Alex noted this traffic in his
logbook. So far, nothing has moved toward them.

The other cameras were equally interesting. Every night, coyotes prowled
beside Grampa's demolished house. Alex counted all animal sightings. He
expected the counts to drop as animals, fleeing the ever-increasing
cold, headed south. One day, the catchment pond camera caught a massive
flock of ravens: thousands of birds, vibrating like an oil slick on the
surf while frantically taking sips from the pond. They took flight when
Doug and Alex emerged from the mine. It bothered them that innocent
animals would freeze.

Four weeks into runaway, about halfway to Mars's orbit, the weather
shifted. The poles cooled faster than ever, and it started snowing at
mid-latitudes months early. Even the tropics saw frost. The end of tides
had unexpected knock-on effects on global weather. Tides flush vast
volumes of seawater from one place to another. The heat transferred by
tides easily dwarfs all human emissions. It was a huge,
taken-for-granted heat transfer mechanism that had stopped. The impact
was greatest in the tropics. Wind patterns shifted worldwide. Snow
started falling outside the mine, and it kept falling. Within days, an
entire winter's worth of snow fell around them.

Alex welcomed the snow; his biggest worry---other people. As the
inevitable sunk in, he expected mobs to scour the countryside, looking
for shelter and things to eat.

Doug disagreed, ``You've expected the worst for your entire life.''

Alex didn't respond to Doug's \emph{shade}; they were taking care not to
irritate each other.

But it was still too early for mobs. Four weeks was hardly enough time
to starve your average American porker. Hell, a little starvation would
improve the health of many lard butts, but in another month, things
would be different. Windblown snow around Grampa's Valley and unplowed
ranch roads would deter vagrants, but skis or snowmobiles could easily
reach their location, so they stayed out of sight in the mine.

``If we get caught in the mine, it would be easy to smoke us out. Just
light a fire; we'd be helpless.''

Until temperatures dropped below -50 degrees Celsius, they would be at
extreme risk of discovery. They'd be safer once things \emph{really}
froze: -100 Celsius or colder. Snowmobiling at -100C is extremely
difficult, and skiing is almost impossible. Nobody would be pushing
shopping carts down the roads of a -100 degrees Celsius world. People
would freeze before they had to resort to cannibalism, which, oddly,
would make it easier for the remaining cannibals. They'd find lots of
frozen meat; no need to hunt the living.

Radio news slowly pivoted from histrionic partisan bullshit to somber
\emph{useful} news.

Doug noted the change in tone. ``Everyone's coming to grips with the
idea that \emph{they} can't escape.''

As the polar ice caps started expanding, many governments, realizing
everyone was equally screwed, made some intelligent moves. Even
flatworms will turn from pain. Icebreakers in the Arctic were ordered
south before the pack ice would trap them in northern seas. Russia and
the US allowed all such ships to pass through the Bering Strait. They
redeployed along the advancing ice front and planned to move south with
the expanding ice pack and report on the rate of ocean freezing. Around
this time, governments took over satellite radio and decrypted
transmissions. You could pick up crystal-clear FM satellite broadcasts
with a regular radio. It only took a global catastrophe to fix satellite
radio.

Alex welcomed the clear FM signal; he even enjoyed the tranquilizing
elevator music played between bouts of government propaganda and weather
reports. The weather reports were the main reason people listened to the
radio. Every few hours, a long list of temperatures, wind speeds, and
precipitation levels were read out for locations worldwide. Alex
recorded temperatures for nearby locales in his log notebook. Local
temperatures helped calibrate their copper spool resistance thermometer.

Government FM was clear but insipid. However, the ham radio bands were
an absolute delight. While installing ducts, they listened to a novel
solution to the \emph{Fermi Paradox}. Maybe the universe is a living
thing, a cosmic Gaia, with an immune system. Look at the mess humans
have made on Earth. Imagine us infecting the galaxy. Perhaps Earth's
runaway is a cosmic white cell ingesting a noxious pathogen. Maybe the
universe is littered with quadrillions of runaway planets, all harboring
the frozen crystal remains of unworthy aliens.

Yes, the FCC had lost control of broadcasting and would never reclaim
it. Everyone would die entertained.


%\end{document}

