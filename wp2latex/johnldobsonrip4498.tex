%% uncomment to list all files in log
%\listfiles

\documentclass[12pt]{report}


\usepackage{fontspec}

%\setmainfont[Scale=MatchLowercase]{Lucida Bright}
%\setmonofont{FreeMono}
%\setmonofont{Source Code Pro}
\setmonofont[Scale=MatchLowercase]{Ubuntu Mono}

\usepackage[headings]{fullpage}

% national use characters 
%\usepackage{inputenc}

% ams mathematical symbols
\usepackage{amsmath,amssymb}

% added to support pandoc highlighting
\usepackage{microtype}

\usepackage{makeidx}

% add index and bibliographies to table of contents
\usepackage[nottoc]{tocbibind}

% postscript courier and times in place of cm fonts
%\usepackage{courier}
%\usepackage{times}

% extended coloring
\usepackage{color}
\usepackage[table,dvipsnames]{xcolor}
\usepackage{colortbl}

% advanced date formating
\usepackage{datetime}

%support pandoc code highlighting
\usepackage{fancyvrb}
\DefineShortVerb[commandchars=\\\{\}]{\|}
\DefineVerbatimEnvironment{Highlighting}{Verbatim}{commandchars=\\\{\}}
% Add ',fontsize=\small' for more characters per line

%tango style colors
% \usepackage{framed}
% \definecolor{shadecolor}{RGB}{255,255,255}
% \newenvironment{Shaded}{\begin{snugshade}}{\end{snugshade}}
% \newcommand{\KeywordTok}[1]{\textcolor[rgb]{0.13,0.29,0.53}{\textbf{{#1}}}}
% \newcommand{\DataTypeTok}[1]{\textcolor[rgb]{0.13,0.29,0.53}{{#1}}}
% \newcommand{\DecValTok}[1]{\textcolor[rgb]{0.00,0.00,0.81}{{#1}}}
% \newcommand{\BaseNTok}[1]{\textcolor[rgb]{0.00,0.00,0.81}{{#1}}}
% \newcommand{\FloatTok}[1]{\textcolor[rgb]{0.00,0.00,0.81}{{#1}}}
% \newcommand{\CharTok}[1]{\textcolor[rgb]{0.31,0.60,0.02}{{#1}}}
% \newcommand{\StringTok}[1]{\textcolor[rgb]{0.31,0.60,0.02}{{#1}}}
% \newcommand{\CommentTok}[1]{\textcolor[rgb]{0.56,0.35,0.01}{\textit{{#1}}}}
% \newcommand{\OtherTok}[1]{\textcolor[rgb]{0.56,0.35,0.01}{{#1}}}
% \newcommand{\AlertTok}[1]{\textcolor[rgb]{0.94,0.16,0.16}{{#1}}}
% \newcommand{\FunctionTok}[1]{\textcolor[rgb]{0.00,0.00,0.00}{{#1}}}
% \newcommand{\RegionMarkerTok}[1]{{#1}}
% \newcommand{\ErrorTok}[1]{\textbf{{#1}}}
% \newcommand{\NormalTok}[1]{{#1}}

%espresso style colors
% \usepackage{framed}
% \definecolor{shadecolor}{RGB}{42,33,28}
% \newenvironment{Shaded}{\begin{snugshade}}{\end{snugshade}}
% \newcommand{\KeywordTok}[1]{\textcolor[rgb]{0.26,0.66,0.93}{\textbf{{#1}}}}
% \newcommand{\DataTypeTok}[1]{\textcolor[rgb]{0.74,0.68,0.62}{\underline{{#1}}}}
% \newcommand{\DecValTok}[1]{\textcolor[rgb]{0.27,0.67,0.26}{{#1}}}
% \newcommand{\BaseNTok}[1]{\textcolor[rgb]{0.27,0.67,0.26}{{#1}}}
% \newcommand{\FloatTok}[1]{\textcolor[rgb]{0.27,0.67,0.26}{{#1}}}
% \newcommand{\CharTok}[1]{\textcolor[rgb]{0.02,0.61,0.04}{{#1}}}
% \newcommand{\StringTok}[1]{\textcolor[rgb]{0.02,0.61,0.04}{{#1}}}
% \newcommand{\CommentTok}[1]{\textcolor[rgb]{0.00,0.40,1.00}{\textit{{#1}}}}
% \newcommand{\OtherTok}[1]{\textcolor[rgb]{0.74,0.68,0.62}{{#1}}}
% \newcommand{\AlertTok}[1]{\textcolor[rgb]{1.00,1.00,0.00}{{#1}}}
% \newcommand{\FunctionTok}[1]{\textcolor[rgb]{1.00,0.58,0.35}{\textbf{{#1}}}}
% \newcommand{\RegionMarkerTok}[1]{\textcolor[rgb]{0.74,0.68,0.62}{{#1}}}
% \newcommand{\ErrorTok}[1]{\textcolor[rgb]{0.74,0.68,0.62}{\textbf{{#1}}}}
% \newcommand{\NormalTok}[1]{\textcolor[rgb]{0.74,0.68,0.62}{{#1}}}

%kete style colors
% \newenvironment{Shaded}{}{}
% \newcommand{\KeywordTok}[1]{\textbf{{#1}}}
% \newcommand{\DataTypeTok}[1]{\textcolor[rgb]{0.50,0.00,0.00}{{#1}}}
% \newcommand{\DecValTok}[1]{\textcolor[rgb]{0.00,0.00,1.00}{{#1}}}
% \newcommand{\BaseNTok}[1]{\textcolor[rgb]{0.00,0.00,1.00}{{#1}}}
% \newcommand{\FloatTok}[1]{\textcolor[rgb]{0.50,0.00,0.50}{{#1}}}
% \newcommand{\CharTok}[1]{\textcolor[rgb]{1.00,0.00,1.00}{{#1}}}
% \newcommand{\StringTok}[1]{\textcolor[rgb]{0.87,0.00,0.00}{{#1}}}
% \newcommand{\CommentTok}[1]{\textcolor[rgb]{0.50,0.50,0.50}{\textit{{#1}}}}
% \newcommand{\OtherTok}[1]{{#1}}
% \newcommand{\AlertTok}[1]{\textcolor[rgb]{0.00,1.00,0.00}{\textbf{{#1}}}}
% \newcommand{\FunctionTok}[1]{\textcolor[rgb]{0.00,0.00,0.50}{{#1}}}
% \newcommand{\RegionMarkerTok}[1]{{#1}}
% \newcommand{\ErrorTok}[1]{\textcolor[rgb]{1.00,0.00,0.00}{\textbf{{#1}}}}
% \newcommand{\NormalTok}[1]{{#1}}
%end pandoc code hacks

% jodliterate colors
\usepackage{color}
\definecolor{shadecolor}{RGB}{248,248,248}
% j control structures 
\definecolor{keywcolor}{rgb}{0.13,0.29,0.53}
% j explicit arguments x y m n u v
\definecolor{datacolor}{rgb}{0.13,0.29,0.53}
% j numbers - all types see j.xml
\definecolor{decvcolor}{rgb}{0.00,0.00,0.81}
\definecolor{basencolor}{rgb}{0.00,0.00,0.81}
\definecolor{floatcolor}{rgb}{0.00,0.00,0.81}
% j local assignments
\definecolor{charcolor}{rgb}{0.31,0.60,0.02}
\definecolor{stringcolor}{rgb}{0.31,0.60,0.02}
\definecolor{commentcolor}{rgb}{0.56,0.35,0.01}
% primitive adverbs and conjunctions
%\definecolor{othercolor}{rgb}{0.56,0.35,0.01}   
\definecolor{othercolor}{RGB}{0,0,255}
% global assignments
\definecolor{alertcolor}{rgb}{0.94,0.16,0.16}
% primitive J verbs and noun names
\definecolor{funccolor}{rgb}{0.00,0.00,0.00}    

\usepackage{framed}
\newenvironment{Shaded}{}{}
\newcommand{\KeywordTok}[1]{\textcolor{keywcolor}{\textbf{{#1}}}}
\newcommand{\DataTypeTok}[1]{\textcolor{datacolor}{{#1}}}
%\newcommand{\DecValTok}[1]{\textcolor{decvcolor}{{#1}}}
\newcommand{\DecValTok}[1]{{#1}} 
\newcommand{\BaseNTok}[1]{\textcolor{basencolor}{{#1}}}
\newcommand{\FloatTok}[1]{\textcolor{floatcolor}{{#1}}}
\newcommand{\CharTok}[1]{\textcolor{charcolor}{\textbf{{#1}}}}
\newcommand{\StringTok}[1]{\textcolor{stringcolor}{{#1}}}
\newcommand{\CommentTok}[1]{\textcolor{commentcolor}{\textit{{#1}}}}
\newcommand{\OtherTok}[1]{\textcolor{othercolor}{{#1}}} 
\newcommand{\AlertTok}[1]{\textcolor{alertcolor}{\textbf{{#1}}}}
%\newcommand{\FunctionTok}[1]{\textcolor{funccolor}{{#1}}}
\newcommand{\FunctionTok}[1]{{#1}}
\newcommand{\RegionMarkerTok}[1]{{#1}}
\newcommand{\ErrorTok}[1]{\textbf{{#1}}}
\newcommand{\NormalTok}[1]{{#1}}

% headers and footers
\usepackage{fancyhdr}
\pagestyle{fancy}

\fancyhead{}
\fancyfoot{}

%\fancyhead[LE,RO]{\slshape \rightmark}
%\fancyhead[LO,RE]{\slshape \leftmark}
\fancyfoot[C]{\thepage}
%\headrulewidth 0.4pt
%\footrulewidth 0 pt

%\addtolength{\headheight}{\baselineskip}

%\lfoot{\emph{Analyze the Data not the Drivel}}
%\rfoot{\emph{\today}}

% subfigure handles figures that contain subfigures
%\usepackage{color,graphicx,subfigure,sidecap}
\usepackage{graphicx,sidecap}
\usepackage{subfigure}
\graphicspath{{./inclusions/}}

% floatflt provides for text wrapping around small figures and tables
\usepackage{floatflt}

% tweak caption formats 
\usepackage{caption} 
\usepackage{sidecap}
%\usepackage{subcaption} % not compatible with subfigure

\usepackage{rotating} % flip tables sideways

% complex footnotes
%\usepackage{bigfoot}

% weird logos \XeLaTeX
\usepackage{metalogo}

% source code listings
\usepackage{listings}

% long tables
% \usepackage{longtable}

\newcommand{\HRule}{\rule{\linewidth}{0.5mm}}

% map LaTeX cross references into PDF cross references
\usepackage[
            %dvips,
            colorlinks,
            linkcolor=blue,
            citecolor=blue,
            urlcolor=blue,   % magenta, cyan default        
            pdfauthor={John D. Baker},
            pdftitle={Analyze the Data not the Drivel},
            pdfsubject={Blog},
            pdfcreator={MikTeX+LaTeXe with hyperref package},
            pdfkeywords={blog,wordpress},
            ]{hyperref}
           
% custom colors
\definecolor{CodeBackGround}{cmyk}{0.0,0.0,0,0.05}    % light gray
\definecolor{CodeComment}{rgb}{0,0.50,0.00}           % dark green {0,0.45,0.08}
\definecolor{TableStripes}{gray}{0.9}                 % odd/even background in tables

\lstdefinelanguage{bat}
{morekeywords={echo,title,pushd,popd,setlocal,endlocal,off,if,not,exist,set,goto,pause},
sensitive=True,
morecomment=[l]{rem}
}

\lstdefinelanguage{jdoc}
{
morekeywords={},
otherkeywords={assert.,break.,continue.,for.,do.,if.,else.,elseif.,return.,select.,end.
,while.,whilst.,throw.,catch.,catchd.,catcht.,try.,case.,fcase.},
sensitive=True,
morecomment=[l]{NB.},
morestring=[b]',
morestring=[d]',
}

% latex size ordering - can never remember it
% \tiny
% \scriptsize
% \footnotesize
% \small
% \normalsize
% \large
% \Large
% \LARGE
% \huge
% \Huge
 
% listings package settings  
\lstset{%
  language=jdoc,                                % j document settings
  basicstyle=\ttfamily\footnotesize,            
  keywordstyle=\bfseries\color{keywcolor}\footnotesize,
  identifierstyle=\color{black},
  commentstyle=\slshape\color{CodeComment},     % colored slanted comments
  stringstyle=\color{red}\ttfamily,
  showstringspaces=false,                       
  %backgroundcolor=\color{CodeBackGround},       
  frame=single,                                
  framesep=1pt,                                 
  framerule=0.8pt,                             
  rulecolor=\color{CodeBackGround},   
  showspaces=false,
  %columns=fullflexible,
  %numbers=left,
  %numberstyle=\footnotesize,
  %numbersep=9pt,
  tabsize=2,
  showtabs=false,
  captionpos=b
  breaklines=true,                              
  breakindent=5pt                              
}

\lstdefinelanguage{JavaScript}{
  keywords={typeof, new, true, false, catch, function, return, null, catch, switch, var, if, in, while, do, else, case, break},
  ndkeywords={class, export, boolean, throw, implements, import, this},
  ndkeywordstyle=\color{darkgray}\bfseries,
  sensitive=false,
  comment=[l]{//},
  morecomment=[s]{/*}{*/},
  morestring=[b]',
  morestring=[b]"
}

% C# settings
\lstdefinestyle{sharpc}{
language=[Sharp]C,
basicstyle=\ttfamily\scriptsize, 
keywordstyle=\bfseries\color{keywcolor}\scriptsize,
framerule=0pt
}

% for source code listing longer than two use smaller font
\lstdefinestyle{smallersource}{
basicstyle=\ttfamily\scriptsize, 
keywordstyle=\bfseries\color{keywcolor}\scriptsize,
framerule=0pt
}

\lstdefinestyle{resetdefaults}{
language=jdoc,
basicstyle=\ttfamily\footnotesize,  
keywordstyle=\bfseries\color{keywcolor}\footnotesize,                                                               
framerule=0.8pt 
}

% APL UTF8 code points listed for lstlisting processing
\makeatletter
\lst@InputCatcodes
\def\lst@DefEC{%
 \lst@CCECUse \lst@ProcessLetter
  ^^80^^81^^82^^83^^84^^85^^86^^87^^88^^89^^8a^^8b^^8c^^8d^^8e^^8f%
  ^^90^^91^^92^^93^^94^^95^^96^^97^^98^^99^^9a^^9b^^9c^^9d^^9e^^9f%
  ^^a0^^a1^^a2^^a3^^a4^^a5^^a6^^a7^^a8^^a9^^aa^^ab^^ac^^ad^^ae^^af%
  ^^b0^^b1^^b2^^b3^^b4^^b5^^b6^^b7^^b8^^b9^^ba^^bb^^bc^^bd^^be^^bf%
  ^^c0^^c1^^c2^^c3^^c4^^c5^^c6^^c7^^c8^^c9^^ca^^cb^^cc^^cd^^ce^^cf%
  ^^d0^^d1^^d2^^d3^^d4^^d5^^d6^^d7^^d8^^d9^^da^^db^^dc^^dd^^de^^df%
  ^^e0^^e1^^e2^^e3^^e4^^e5^^e6^^e7^^e8^^e9^^ea^^eb^^ec^^ed^^ee^^ef%
  ^^f0^^f1^^f2^^f3^^f4^^f5^^f6^^f7^^f8^^f9^^fa^^fb^^fc^^fd^^fe^^ff%
  ^^^^20ac^^^^0153^^^^0152%
  ^^^^20a7^^^^2190^^^^2191^^^^2192^^^^2193^^^^2206^^^^2207^^^^220a%
  ^^^^2218^^^^2228^^^^2229^^^^222a^^^^2235^^^^223c^^^^2260^^^^2261%
  ^^^^2262^^^^2264^^^^2265^^^^2282^^^^2283^^^^2296^^^^22a2^^^^22a3%
  ^^^^22a4^^^^22a5^^^^22c4^^^^2308^^^^230a^^^^2336^^^^2337^^^^2339%
  ^^^^233b^^^^233d^^^^233f^^^^2340^^^^2342^^^^2347^^^^2348^^^^2349%
  ^^^^234b^^^^234e^^^^2350^^^^2352^^^^2355^^^^2357^^^^2359^^^^235d%
  ^^^^235e^^^^235f^^^^2361^^^^2362^^^^2363^^^^2364^^^^2365^^^^2368%
  ^^^^236a^^^^236b^^^^236c^^^^2371^^^^2372^^^^2373^^^^2374^^^^2375%
  ^^^^2377^^^^2378^^^^237a^^^^2395^^^^25af^^^^25ca^^^^25cb%  
  ^^00}
\lst@RestoreCatcodes
\makeatother

% custom lengths used within minipages
\newcommand{\minindent}{17pt}


\makeindex

\begin{document}

\subsection*{\href{https://bakerjd99.wordpress.com/2014/01/18/john-l-dobson-r-i-p/}{John L. Dobson R.I.P.}}
\addcontentsline{toc}{subsection}{John L. Dobson R.I.P.}


\noindent\emph{Posted: 18 Jan 2014 12:37:10}
\vspace{6pt}

At tonight's meeting of the \href{http://www.slasonline.org/}{St. Louis
Astronomical Society} I learned of
\href{http://www.universetoday.com/108150/john-dobson-inventor-of-the-popular-dobsonian-telescope-dead-at-98/}{John
Dobson's} recent death. John Dobson was widely known as the inventor of
the homemade ``Dobsonian'' telescope and the co-founder of the
\href{http://www.sidewalkastronomers.us/}{Sidewalk Astronomers}: perhaps
the most famous and effective amateur astronomy outreach group in modern
times. ``Big Dob'' light buckets are a staple at
\href{http://stardate.org/nightsky/star\_parties}{star parties} around
the world and most of them derive from~John's original designs. John
lived a long life and touched many people including myself.

I briefly met John at a star party in central Texas in the late 1990's.
I cannot remember exactly where we were but it was about two hundred
miles west of Fort Worth and was~one of the best ``dark sky'' sites
within easy driving distance of the Fort Worth Dallas light pollution
wasteland.\footnote{
The best dark sky sites in
Texas are in the Davis Mountains near the McDonald Observatory. The
\href{http://texasstarparty.org/}{Texas Star Party} is held nearby every year.
} Amateur astronomers
abhor, detest, loathe and constantly rage, rage against --- street
lights. When I see the sky sodomized by one ill placed security light or
a hideous blinking radio tower I have to suppress Homeric, (Simpson),
urges to kill. Light Pollution is an assault on one of the most
beautiful things the human eye can behold, a glorious night sky, and
most people are completely and utterly oblivious to it.

Because our central Texas star party was far from the maddening crowds,
just the way hard-core dark sky connoisseurs ~like it, ~there weren't
very many people present and most in attendance where armed with state
of the art telescopic gadgetry. This did not quite suit John. I remember
he remarked that this was an ``astronomer's gathering.'' Meaning this
was a gathering for people who already knew the majesty of night sky. It
was John's passion to introduce neophytes to the that glory and judging
by the accolades coming in from people who caught the astronomy bug at
one of John's sidewalk star parties it's a passion that will outlive
him.

John felt it was vitally important for people to see,\emph{with their
own eyes,} the craters of the moon, the moons of Jupiter, Saturn's
rings, sunspots, the Andromeda galaxy and thousands of other sky
wonders. He apparently never tired of watching someone look through a
telescope for the first time and until tonight I must confess I didn't
really appreciate just how crucial such
``\href{https://en.wikipedia.org/wiki/First\_light\_(astronomy)}{first
lights}'' are. ~As I drove home I started thinking about my top
astronomical experiences and soon realized they are some of my top
experiences --- period.~Here's my top ten ``first lights'' in no
particular order. Only the last stands above the others.

\medskip

\textbf{1. First look at the moon through a telescope}:~ \textbf{Redwash
Utah, United States}. When I was in grade school my parents got me a
60mm Tasco refractor. It was a simple, and surprisingly good, little
telescope. I've talked to many amateur astronomers over the years and
many fondly remember getting started with a 60mm Tasco. Shortly after I
got that telescope I set it up and waited for the sky to darken. The
moon was not full when it rose;~I didn't care. I lined up the scope,
fiddled with the focus, and then suddenly, the moon's craters appeared:
the sharpness and clarity almost hurt. I've been hooked~on amateur
astronomy ever since. In retrospect the moon's craters made a bigger
impression on me than my first kiss. I don't remember my first kiss,~but
I'll never forget my first telescopic glimpse of the moon. John never
tired of introducing strangers to the moon.

\medskip

\textbf{2. Seeing the crescent of Venus for the first time: Redwash
Utah, United States.} It took me awhile to learn how to properly focus my
telescope. The moon was easy but for some reason I never got Venus
dialed in until one evening when I turned the knob far enough to
condense the unfocused blob of Venus into a sharp little crescent. It
was mind-blowing. The seeing on the high Uintah plateau was superb. I
have seldom seen Venus so steady, sharp and clear. I felt like Galileo
--- hell for a brief instant I was Galileo! ~John loved showing off the
bright planets. Venus, Jupiter, Saturn, even Mars and Mercury are all
easily visible from the middle of light polluted cities. ~Sidewalk
Astronomers turned these ancient wanderers into modern celebrities.

\medskip

\textbf{3. Seeing Jupiter and its four Galilean moons: Agha Jari
Iran.} Shortly after getting my first refractor my family moved to Iran.
I lugged my little telescope half way around the world. The telescope
was my hand baggage. In those days airlines were more tolerant of large
carry-ons and being a kid I enjoyed~extra latitude. ~After we settled in
Agha Jari I set up the telescope in our front yard. I wanted to check
out a bright star that was hanging about twenty degrees above the hills
to the southwest. I knew that looking at stars in the 60mm was, with the
exception of binaries, kind of dull. Stars appear as points of light in
even the largest of telescopes. Only in modern times, by using
\href{http://isi.ssl.berkeley.edu/}{long baseline interferometry}, has
it become possible to resolve details on distant stars. ~Not expecting
much I aimed the scope at the bright star and focused. Suddenly a new
``solar system'' sprang into view. I could see a tiny ball surrounded by
four bright spots. For a few moments I thought I might be seeing
something new. How could people have missed this? Then I realized it
was~Jupiter. Big J is still my favorite planet.~ In John's Big Dobs
Jupiter is a super-planet. You can easily see four or more atmospheric
bands, shadows of major moons and the Great Red Spot. Even better, it's
all visible from city sidewalks.

\medskip

\textbf{4. Catching Halley's Comet: Edmonton Alberta Canada.} Once you
catch the observer's bug it never really leaves you. It may go dormant
but something always wakes it up. The return of Halley's Comet roused my
inner observer.
\href{https://en.wikipedia.org/wiki/Halley\%27s\_Comet}{Halley's Comet
is the most famous of all comets.} It takes roughly one human lifetime
for it to complete its orbit so catching Halley's Comet is something
most of us will only do once. The last time Halley's Comet zoomed by in
1910 it put on a spectacular show. Astronomers warned that the 1986
passing would be ``disappointing.'' The comet was further~away than it
was in 1910. They were right but I was still delighted by what I saw. It
was a freezing -30C Edmonton winter night when we drove to the city's
southern outskirts to see the comet. Despite the blistering cold dozens
of people were parking along the road and getting out of their cars to
look for the comet. I knew exactly where to look and it only took me a
few seconds to find the fuzzy ball known as Halley's Comet. It wasn't
spectacular, but it was historically satisfying. John spent a lot of
time educating people about what they would see in the sky. If you
understand, even the faintest of objects can thrill.

\medskip

\textbf{5. All sky aurora Edmonton Alberta Canada.} Amateur astronomers
have mixed feelings about auroras. Amateurs that live in aurora zones,
Canadians, Norwegians, Argentines and others often bitch about ``natural
light pollution'' until they witness a full-blown all sky aurora.~ It
was another cold Edmonton winter night and I was up late watching the
idiot box when a local news alert interrupted programming to report an
impressive aurora was underway. Northern lights in Edmonton are common
and seldom newsworthy. I had to see what the fuss was about so I bundled
up, stepped outside and was immediately transfixed. Auroras are usually
silent slithering green sheens. Tonight the sky was blazing green, then
red, hinting at purple, then back to green, red again, rippling and
roaring, from the north to south, east and west: all ablaze. I had never
seen a display of such magnitude. Giant auroras are not only beautiful
they can \href{http://www.solarstorms.org/SWChapter1.html}{shut down
power grids.} ~I love it: natural light pollution shutting down man-made
light pollution. I don't know if John saw great auroras but I know he
would have loved them.

\medskip

\textbf{6. Comet Bennett near Canmore Alberta Canada.} I wasn't looking
for \href{https://en.wikipedia.org/wiki/Comet\_Bennett}{Bennett's Comet}
when I saw it. I was driving back to Calgary from Vancouver with
friends. We made a road side stop in the Canadian Rockies near Canmore
to relieve ourselves. I trudged out into the snow, unzipped my fly,
~looked up and saw, just above the dark jagged outline of the mountains,
the most exquisite comet. I wasn't sure~what it was. It was so striking
that we stopped peeing and admired it. I remember saying, ``It looks
exactly like the comets you see in textbooks.'' I was right.~ That
passing of Bennett's Comet was canonical. I've seen some great comets
since Bennett:
\href{https://en.wikipedia.org/wiki/Comet\_Hale\%E2\%80\%93Bopp}{Hale-Bopp}
and~\href{https://en.wikipedia.org/wiki/Comet\_Hyakutake}{Hyakutake}~were
both spectacular but unexpected Bennett is still my favorite. John often
reminded people that you don't always need a telescope: just keep
looking up.

\medskip

\textbf{7. Total Eclipse of the Moon: Tamale Ghana Africa.} I taught
mathematics for two years in a northern Ghanaian boarding school after
graduating from university. The best total lunar eclipse I have ever
seen occurred during my Ghanaian years. I reckoned the eclipse would be
a great teaching opportunity for my students. ~The school had and old
telescope; it was a small reflector that a previous teacher had donated.
The scope didn't have an eye piece so we adapted a microscope eye piece.
It worked better than expected. ~On the night of the eclipse we set up
the scope on a second story veranda with a nice southern view. Lunar
eclipses are leisurely events. It takes a long time for the Earth's
shadow to cover the moon. Before the eclipse began students started
peaking at the moon through the telescope. Many of them where as
delighted as I was with my first telescopic glimpse of the moon. I
remember some of the older, and cooler, students had to reign in their
obvious excitement. As the Earth's shadow touched the moon people in
small villages around the school started pounding drums. As the shadow
crept further and further the drumming got louder and louder and
bonfires started popping up all around the school. ~I didn't expect this
reaction. It was the best damn star party I've ever attended. The
eclipse was a good one too. At totality the moon was a deep dark
blood-red. John took advantage of eclipses, nature's astronomical
advertising, to show even more people the greatest show off earth.

\medskip

\textbf{8. Annular Solar Eclipse: Syracuse New York United States.} 1994
was the year of
\href{https://en.wikipedia.org/wiki/Comet\_Shoemaker\%E2\%80\%93Levy\_9}{Shoemaker--Levy
9}: the fragmented comet that smashed into Jupiter with such awesome
energy that it blinded sensors on large telescopes and left massive
bruises that could be seen in small telescopes. Shoemaker-Levy 9 was a
rare major event. We were lucky to see such an impact in our lifetimes,
but the 1994 event that sticks in my head was the annular eclipse of
that year. Annular eclipses are, according to eclipse snobs, failed
total eclipses. The moon is too far away to precisely match the angular
size the sun so at totality an annular looks like perfect super bright
ring in the sky. It's a rare celestial event that fits into your work
day but the 1994 annular eclipse did just that. Totality occurred~during
lunch hour and many of my coworkers and strangers on Syracuse sidewalks
paused to look through eclipse shades and welding glass at the one ring
to bind them all. It was Dobsonian sidewalk astronomy at its finest.

\medskip

\textbf{9. Glimpsing the Gegenschein: Grand Teton National Park, United
States.} Seeing the
\href{https://en.wikipedia.org/wiki/Gegenschein}{gegenschein} requires
very dark and clear skies. The tiniest hint of light pollution will wash
it out. I'd been observing for years before I saw it. I was south of
Yellowstone Park looking directly north. There are no large cities
directly north of Yellowstone for hundreds of miles and the park is
blissfully black. At 3:00am I noticed a slight glowing that steady
increased in brightness. Seeing was superb. I could see
7\textsuperscript{th} magnitude stars with averted naked eye vision. I
was alone, in the cold, in the dark. Not John's style but this was a
personal first. Sometimes the sky is all you need.

\medskip

\textbf{10. Total Solar Eclipse: Zambia Africa}. \emph{Total solar
eclipses are beyond awesome.~}They utterly wowed our ancient ancestors
and still blow away the most jaded~and media saturated people today. You
have to put yourself in the moon's shadow and give yourself to the
spectacle. It's worth spending thousands of dollars and going to the
ends of the Earth to stand in the moon's shadow. Of all the wonderful
and spectacular things I have seen I'd rate my first total solar eclipse
above them all. Of all the planets, moons and other bodies in our solar
system only the Earth enjoys pure total solar eclipses. By some freak
cosmic accident the moon and sun are almost exactly the same angular
size in our sky. \emph{It's only when I'm standing in the moon's shadow,
during a total solar eclipse that I don't mind being trapped on Earth.}

\medskip

Looking over my list it's pretty clear that John Dobson had the right
idea. The things that stuck were first glances through telescopes and
watching special things in the sky with my own eyes. I have spent many
long nights tracking objects down in telescopes but after a few years
such ``serious'' sessions blend together. It's those fleeting first
lights that dig in and change how you feel. John Dobson changed how many
people feel about the sky. Nice work John.

%\begin{center}\rule{3in}{0.4pt}\end{center}

%\href{/fit/dft/jdm.docx\#\_ftnref1}{
%%{[}1{]}
%} The best dark sky sites in
%Texas are in the Davis Mountains near the McDonald Observatory. The
%\href{http://texasstarparty.org/}{Texas Star Party} is held nearby every
%year.



%\end{document}